\section{Ecological Mechanisms}

A key improvement to the model is the addition of a positive feedback between coral abundance and Crown-of-Thorns Starfish (COTS) recruitment. The original model treated COTS recruitment (immigration) as independent of local conditions. However, research shows that COTS larval survival and recruitment success increases with coral abundance due to:

1. Greater availability of coral prey for newly settled juveniles
2. Enhanced habitat complexity providing shelter from predation
3. Improved chemical settlement cues from coral-associated organisms

This feedback is implemented through a coral-dependent multiplier on the base recruitment rate:

recruitment = base\_recruitment * (1 + recruit\_effect * total\_coral)

where recruit\_effect controls the strength of this positive feedback. This mechanism helps explain the rapid population growth during outbreaks when coral is abundant, followed by population crashes when coral becomes depleted - a pattern visible in the historical data but not well-captured by the original model.

This addition represents an important indirect pathway that can generate more realistic boom-bust dynamics in the COTS-coral system. The feedback creates a mechanism for both rapid outbreak development when conditions are favorable (high coral cover) and natural outbreak termination when coral resources become depleted.
