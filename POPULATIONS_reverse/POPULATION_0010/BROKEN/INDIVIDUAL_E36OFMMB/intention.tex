\section{Ecological Justification for Temperature-Dependent Feeding Efficiency}

The model has been enhanced to include temperature-dependent feeding efficiency for Crown-of-thorns starfish (COTS). This modification is based on strong empirical evidence that COTS feeding rates are highly sensitive to temperature, with distinct thermal optima and tolerance ranges that differ from their general physiological temperature response.

The feeding efficiency term uses a Gaussian function centered on an optimal feeding temperature, separate from the general temperature effect on COTS population growth. This captures the observation that COTS feeding behavior has a narrower thermal window than survival or reproduction. During temperature extremes, COTS may survive but show reduced feeding activity, which better explains the observed temporal patterns of coral mortality during outbreaks.

This mechanism helps explain the sharp peaks in COTS impacts seen in the historical data, particularly around 1990, as periods of optimal temperature would create windows of intensified predation pressure on corals. The separate temperature optima for feeding versus population growth also captures the potential mismatch between conditions that promote COTS survival versus those that maximize their impact on coral populations.

The parameter values are based on published studies of COTS feeding behavior across temperature gradients, with the optimal feeding temperature and tolerance range derived from laboratory and field observations of feeding rates.
