\section{Temperature-Dependent Predation}

The model has been enhanced to include temperature-dependent attack rates for Crown-of-thorns Starfish (COTS) predation on corals. This modification is based on the ecological understanding that COTS, like many marine ectotherms, exhibit temperature-dependent metabolic and feeding rates.

The attack rates ($\alpha_{slow}$ and $\alpha_{fast}$) are now modified by a Q10 temperature coefficient, which describes how biological rates change with temperature:

\[ \alpha_{i}(T) = \alpha_{i} \cdot Q10^{(T-T_{opt})/10} \]

where:
\begin{itemize}
    \item $\alpha_{i}(T)$ is the temperature-adjusted attack rate
    \item $T$ is the current temperature
    \item $T_{opt}$ is the optimal temperature
    \item $Q10$ is the temperature coefficient (typically 2-3 for biological rates)
\end{itemize}

This modification allows the model to capture:
\begin{itemize}
    \item Increased predation pressure during warmer periods
    \item Reduced feeding activity during cooler periods
    \item More realistic temporal dynamics in COTS-coral interactions
\end{itemize}

The Q10 relationship is well-established in physiological ecology and provides a mechanistic basis for temperature-dependent predation rates, potentially improving the model's ability to capture observed population dynamics.

The temperature effects are implemented in two ways:
\begin{itemize}
    \item A Gaussian survival effect that peaks at the optimal temperature
    \item A Q10-based metabolic effect on attack rates, bounded to prevent extreme values
\end{itemize}

Both effects are carefully bounded to maintain numerical stability while preserving the essential biological relationships. The survival effect is bounded between 0.1 and 1.0, while the Q10 effect is limited to a 2-fold change in either direction.
