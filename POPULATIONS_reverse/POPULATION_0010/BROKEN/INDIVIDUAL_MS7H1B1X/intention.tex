\section{Temperature-Dependent Predation Vulnerability}

The model has been enhanced to incorporate temperature stress effects on coral vulnerability to COTS predation. While the existing temperature effect ($temp\_effect$) modulates coral growth rates, we now also include a stress multiplier ($stress\_mult$) that increases predation efficiency when temperatures deviate significantly from optimal conditions.

The stress multiplier follows a bounded quadratic form:
\[ stress\_mult = \min(1 + 0.2(\frac{T - T_{opt}}{T_{threshold}})^2, 1.5) \]

This formulation captures several key ecological mechanisms:
\begin{itemize}
    \item Coral polyps become less effective at defending against predation when thermally stressed
    \item The effect accelerates as temperature deviates further from optimal
    \item The baseline multiplier is 1 (no effect) when temperature is optimal
    \item The threshold parameter allows calibration to observed stress responses
\end{itemize}

This mechanism provides a more realistic representation of how temperature extremes can indirectly increase coral mortality through enhanced vulnerability to predation, rather than just through direct growth effects.
