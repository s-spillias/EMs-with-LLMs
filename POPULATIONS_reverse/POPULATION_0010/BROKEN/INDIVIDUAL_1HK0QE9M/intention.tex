\section{Ecological Mechanisms}

The model incorporates a key positive feedback mechanism between coral abundance and COTS feeding efficiency. When coral cover is high, COTS feeding becomes more efficient due to:

1. Reduced search time between feeding events
2. Better nutrition leading to increased reproduction
3. Enhanced survival of juvenile COTS in coral-rich areas

This is implemented through a coral-dependent feeding efficiency term:

\[ E(C) = e_{base} + e_{max}\frac{C}{C + h_{eff}} \]

where:
- $E(C)$ is the feeding efficiency as a function of total coral cover $C$
- $e_{base}$ is the baseline feeding efficiency
- $e_{max}$ is the maximum additional efficiency
- $h_{eff}$ is the half-saturation constant for this effect

This creates a positive feedback that can help explain COTS outbreak dynamics - as coral cover increases, COTS feeding becomes more efficient, leading to faster population growth and potentially triggering outbreaks when conditions are favorable.
