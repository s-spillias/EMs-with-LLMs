The model has been modified to include density-dependent feeding efficiency of Crown-of-thorns Starfish (COTS). This change reflects the ecological observation that COTS become more efficient predators at higher densities due to:

1. Aggregation behavior during outbreaks leading to more effective coral tissue consumption
2. Overwhelmed coral defense mechanisms when multiple COTS attack simultaneously
3. Chemical cues from feeding that attract other COTS, creating positive feedback

The modification is implemented through a density-dependent attack rate multiplier that increases linearly with COTS density relative to their carrying capacity. This creates a positive feedback during outbreaks where higher COTS densities lead to more efficient coral predation, potentially better capturing the sharp declines in coral cover observed in the historical data.

This mechanism is particularly important because it represents a key aspect of COTS outbreak dynamics - their ability to rapidly consume coral tissue when at high densities. The linear scaling with K_cots provides a natural upper bound on this effect while maintaining model stability.
