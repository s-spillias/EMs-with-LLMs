\section{Model Description}

The model incorporates temperature-dependent efficiency in coral growth to better represent how thermal stress affects coral population dynamics. While the existing temperature scaling term ($e^{-\frac{(T-T_{opt})^2}{2T_{range}^2}}$) captures the overall metabolic response to temperature, we add efficiency terms ($\eta_{slow}$ and $\eta_{fast}$) that specifically modify how effectively corals can convert available resources into growth.

These efficiency terms decrease quadratically as temperature deviates from optimal, but with different sensitivities for slow and fast-growing corals. This reflects the ecological understanding that slow-growing corals are generally more sensitive to thermal stress, while fast-growing corals often have better thermal tolerance.

The modified coral growth equations now include these efficiency terms, allowing for reduced growth even when space is available if temperature conditions are suboptimal. This better captures the observed variability in coral cover, particularly for slow-growing corals which show more pronounced fluctuations in the historical data.
