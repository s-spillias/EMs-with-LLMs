This model incorporates a density-dependent modification to COTS attack rates that better represents the ecological reality of predator-prey interactions in coral reef systems. The attack rates are scaled by a factor that depends on total coral cover:

cover_effect = 1 - exp(-q * total_cover)

where q is a scaling parameter and total_cover is the sum of slow and fast-growing coral cover.

This modification captures several important ecological mechanisms:

1. Search efficiency: When coral cover is low, COTS must expend more energy searching for prey, reducing their effective attack rate.

2. Predator behavior: COTS may modify their feeding behavior or leave areas with low coral cover, creating a natural feedback mechanism.

3. Refuge effects: Low coral cover may provide better hiding places for remaining coral colonies, making them harder for COTS to find.

This density-dependent predation efficiency creates a stabilizing feedback that could help explain the observed patterns of coral recovery after COTS outbreaks. The exponential form ensures a smooth transition between low and high efficiency states, with diminishing returns at high coral cover.
