\section{Ecological Mechanisms}

The model has been enhanced to better represent coral bleaching mortality during temperature extremes. While the original model included temperature effects on growth rates through a Gaussian response curve, it didn't capture the asymmetric impacts of high temperature stress that can cause rapid coral mortality through bleaching.

The new formulation adds a threshold-based bleaching mortality term that:
\begin{itemize}
    \item Only activates when temperatures exceed a critical threshold above optimal conditions
    \item Increases quadratically with temperature deviation beyond the threshold
    \item Directly reduces coral population proportional to current abundance
\end{itemize}

This better represents the ecological reality that:
\begin{itemize}
    \item Coral bleaching is triggered by sustained temperatures above normal maxima
    \item Mortality increases non-linearly with temperature stress
    \item Both slow and fast-growing corals are susceptible to bleaching mortality
\end{itemize}

The separate handling of growth reduction versus mortality allows the model to capture both the gradual effects of sub-optimal temperatures and the acute impacts of thermal stress events, which should improve its ability to reproduce the observed variability in coral populations.
