This model incorporates coral thermal adaptation, recognizing that coral populations can adjust their thermal tolerance over time. The adaptation mechanism is implemented through a dynamic optimal temperature that gradually shifts based on recently experienced temperatures:

1. A new state variable adapted_temp_opt tracks the evolving thermal optimum
2. The adaptation rate parameter (adapt_rate) controls how quickly corals adjust
3. The optimal temperature moves toward recently experienced temperatures using a simple tracking equation

This better represents coral plasticity and adaptation potential, which is especially important given:
- Historical evidence of coral thermal acclimation
- The role of adaptive capacity in coral reef resilience
- The need to capture recovery dynamics after thermal stress events

The adaptation mechanism allows corals to partially compensate for temperature changes, potentially explaining some of the higher coral cover values seen in the historical data that the original model underestimated. This provides a more realistic representation of coral thermal biology while maintaining model parsimony.
