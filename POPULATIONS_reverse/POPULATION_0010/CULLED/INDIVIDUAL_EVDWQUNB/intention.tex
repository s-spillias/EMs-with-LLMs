This model modification introduces differential temperature stress effects on fast-growing versus slow-growing corals. The key ecological reasoning is:

1. Fast-growing branching corals are typically more susceptible to thermal stress than slow-growing massive corals
2. This is implemented through separate temperature scaling functions with different sensitivity coefficients (2.0 vs 0.5)
3. The temperature stress directly modifies the growth rates, representing reduced photosynthesis and calcification under thermal stress
4. This creates an indirect competitive advantage for slow-growing corals during periods of environmental stress
5. The mechanism helps explain the observed pattern where fast coral cover declined more dramatically after 1995 while slow corals maintained better persistence

This addition better captures the fundamental tradeoff between growth rate and stress tolerance in coral life history strategies. It provides a mechanistic basis for temporal shifts in coral community composition under varying environmental conditions.
