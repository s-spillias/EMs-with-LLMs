The model has been enhanced to include asymmetric competition between fast and slow-growing corals. This modification better represents the ecological reality of coral reef systems, where different growth strategies lead to different competitive abilities.

Fast-growing corals typically have stronger competitive effects (higher c_fast_on_slow) as they can quickly overtop and shade slower-growing species. However, slow-growing corals often have lower but still significant competitive effects (lower c_slow_on_fast) through mechanisms like chemical deterrence and physical space occupation.

The competition coefficients modify how each coral type experiences space limitation, replacing the simple total cover term with separate competition effect terms for each coral type. This allows for:

1. Asymmetric competition intensities between coral types
2. More realistic representation of space competition dynamics
3. Better capture of the differential responses seen in the historical data

This change is supported by literature showing that competitive hierarchies are important in structuring coral communities, with fast-growing species often dominating in favorable conditions but potentially being more vulnerable to environmental stress.
