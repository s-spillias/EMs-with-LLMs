The model has been enhanced to include coral recruitment facilitation - a critical ecological feedback where existing coral cover promotes successful recruitment and survival of new corals. This is implemented through a Holling Type III-like function that reduces recruitment when coral cover is very low.

The recruitment facilitation term (slow_pred/(slow_pred + R50_slow)) approaches zero when coral cover is very low, representing the difficulty of recovery in severely degraded reefs due to lost habitat complexity, reduced larval settlement cues, and harsher environmental conditions. As coral cover increases, the facilitation effect strengthens, capturing how established corals create favorable microhabitats and environmental conditions that enhance recruitment success.

This mechanism better represents the ecological reality of coral reef recovery dynamics, particularly the observed pattern of slow recovery from very low coral cover followed by accelerating recovery once sufficient cover is established. The R50_slow parameter represents the coral cover at which recruitment facilitation reaches 50% of its maximum effect, providing a biologically meaningful threshold for reef recovery potential.
