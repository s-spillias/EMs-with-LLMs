\section{Experimental Methods}

This study presents three experiments designed to evaluate the performance, predictive capability, and ecological validity of our genetic algorithm-based ecological modeling system. The experiments systematically assess both the technical robustness of the algorithm and its ability to capture meaningful ecological patterns.

\subsection{Experiment 1: Convergence Reproducibility}

The first experiment examines the consistency and reliability of the genetic algorithm in finding viable solutions. We assess whether the algorithm can repeatedly discover models that effectively describe the Crown-of-thorns starfish (COTS) and coral dynamics.

\subsubsection{Methodology}

Ten independent populations will be evolved using identical settings:
\begin{itemize}
\item Population size: 4 individuals
\item Maximum generations: 10
\item Convergence threshold: 10
\item Project topic: Crown of Thorns starfish on the Great Barrier Reef feeding on slow- (Faviidae and Porites) and fast-growing (Acropora) corals
\item Training data: Complete time series (1980-2005)
\end{itemize}

\subsubsection{Data Collection and Analysis Methods}

For each population run, we collect and analyze:

\paragraph{Convergence Metrics}
\begin{itemize}
\item Number of generations until convergence or maximum generations (10)
\item Final objective function value of the best performer
\item Success rate (proportion of populations reaching convergence threshold of 10)
\end{itemize}

\paragraph{Population Statistics}
\begin{itemize}
\item Number of culled individuals (those with poor performance)
\item Number of broken individuals (those failing to compile or execute)
\item Mean and standard deviation of generations to convergence
\item Distribution of final objective function values
\end{itemize}

\paragraph{Model Consistency}
\begin{itemize}
\item Time series predictions for all three variables:
  \begin{itemize}
  \item COTS abundance (individuals/m²)
  \item Fast-growing coral cover (\%)
  \item Slow-growing coral cover (\%)
  \end{itemize}
\item Coefficient of variation for each variable's predictions
\item Mean trajectories with confidence intervals
\end{itemize}

Results are stored in JSON format with detailed metrics for reproducibility and automated analysis. Each experiment generates:
\begin{itemize}
\item Timestamped JSON results files containing all raw metrics and analyses
\item Log files documenting experiment progress and any issues encountered
\item Publication-quality figures in PNG format:
  \begin{itemize}
  \item Time series comparisons across populations
  \item Box plots of performance metrics
  \item Convergence trajectory visualizations
  \end{itemize}
\end{itemize}

\subsection{Experiment 2: Predictive Capability Assessment}

The second experiment evaluates the predictive power of the evolved models by testing their performance on unseen data.

\subsubsection{Methodology}

The time series data will be divided into three periods:
\begin{itemize}
\item Training period: 1980-1995 (16 years)
\item Validation period: 1996-2000 (5 years)
\item Testing period: 2001-2005 (5 years)
\end{itemize}

We will:
\begin{itemize}
\item Train the model using only the training period data
\item Make predictions for the validation and testing periods
\item Calculate prediction errors for all three variables
\end{itemize}

\subsubsection{Data Collection and Analysis Methods}

The experiment involves training on early data (1980-1995) and evaluating predictions for later periods:

\paragraph{Data Preparation}
\begin{itemize}
\item Training period: 1980-1995 (16 years)
\item Validation period: 1996-2000 (5 years)
\item Testing period: 2001-2005 (5 years)
\end{itemize}

\paragraph{Error Metrics}
For each variable (COTS, fast coral, slow coral) and time period:
\begin{itemize}
\item Root Mean Square Error (RMSE): $\sqrt{\frac{1}{n}\sum_{i=1}^n(y_i - \hat{y}_i)^2}$
\item Mean Absolute Error (MAE): $\frac{1}{n}\sum_{i=1}^n|y_i - \hat{y}_i|$
\item Pearson correlation coefficient between predicted and observed values
\item Systematic bias: $\frac{1}{n}\sum_{i=1}^n(y_i - \hat{y}_i)$
\end{itemize}

\paragraph{Visualization}
\begin{itemize}
\item Time series plots comparing predictions with observations
\item Period transition markers at 1995 and 2000
\item Error metric comparisons across periods
\item Box plots of prediction errors by variable and period
\end{itemize}

\subsection{Experiment 3: Ecological Response Analysis}

The third experiment examines whether the evolved models capture known ecological relationships and patterns.

\subsubsection{Methodology}

For the best-performing models from Experiment 1, we will analyze:
\begin{itemize}
\item COTS outbreak characteristics
\item Coral recovery patterns
\item Temperature-driven mortality events
\end{itemize}

\subsubsection{Data Collection and Analysis Methods}

The ecological analysis combines quantitative metrics with expert evaluation:

\paragraph{COTS Outbreak Analysis}
\begin{itemize}
\item Outbreak definition: Abundance > 2 standard deviations above mean
\item Metrics per outbreak:
  \begin{itemize}
  \item Start and end years
  \item Duration in years
  \item Peak abundance (individuals/m²)
  \end{itemize}
\item Population-level statistics:
  \begin{itemize}
  \item Number of distinct outbreaks
  \item Mean outbreak duration
  \item Mean peak abundance
  \item Inter-outbreak intervals
  \end{itemize}
\end{itemize}

\paragraph{Coral Recovery Analysis}
\begin{itemize}
\item Decline events: >20\% decrease in cover
\item Recovery tracking:
  \begin{itemize}
  \item Time to 90\% of pre-decline cover
  \item Recovery trajectory shape
  \item Final recovery level achieved
  \end{itemize}
\item Comparative metrics:
  \begin{itemize}
  \item Number of major declines
  \item Mean recovery time
  \item Mean decline magnitude
  \item Recovery success rate
  \end{itemize}
\end{itemize}

\paragraph{Temperature Response Analysis}
\begin{itemize}
\item High temperature events: >1 standard deviation above mean
\item Response metrics:
  \begin{itemize}
  \item Immediate coral cover change
  \item Differential responses between coral types
  \item Recovery patterns post-event
  \end{itemize}
\item Statistical measures:
  \begin{itemize}
  \item Number of high-temperature events
  \item Mean coral response magnitude
  \item Correlation between temperature and cover change
  \end{itemize}
\end{itemize}

\paragraph{Expert Evaluation Framework}
A standardized evaluation form is provided to experts, including:
\begin{itemize}
\item Quantitative results for each ecological aspect
\item 1-5 scale ratings for biological realism
\item Structured feedback sections:
  \begin{itemize}
  \item Pattern-specific comments
  \item Overall model strengths
  \item Identified weaknesses
  \item Improvement suggestions
  \end{itemize}
\end{itemize}

\paragraph{Visualization}
\begin{itemize}
\item COTS outbreak plots with highlighted outbreak periods
\item Coral recovery trajectories with decline events marked
\item Temperature response plots showing:
  \begin{itemize}
  \item Temperature time series
  \item Coral cover responses
  \item High-temperature event markers
  \end{itemize}
\end{itemize}

\section{Expected Outcomes}

These experiments will provide comprehensive insights into multiple aspects of the modeling system:

\subsection{Algorithm Performance}
\begin{itemize}
\item Quantification of genetic algorithm reliability through convergence statistics
\item Understanding of model variability across multiple runs
\item Identification of common failure modes and their frequencies
\item Assessment of computational efficiency and resource requirements
\end{itemize}

\subsection{Predictive Capabilities}
\begin{itemize}
\item Evaluation of model generalization to unseen time periods
\item Quantification of prediction uncertainties and biases
\item Understanding of model limitations in different ecological conditions
\item Assessment of the model's utility for management decision support
\end{itemize}

\subsection{Ecological Insights}
\begin{itemize}
\item Validation of modeled COTS outbreak dynamics against known patterns
\item Comparison of coral recovery rates with empirical observations
\item Assessment of temperature-driven mortality mechanisms
\item Expert evaluation of biological realism and model limitations
\end{itemize}

\subsection{Methodological Contributions}
\begin{itemize}
\item Framework for systematic evaluation of ecological model evolution
\item Standardized metrics for assessing model performance
\item Integration of quantitative analysis with expert evaluation
\item Reproducible approach for future model development
\end{itemize}

The results will inform both the technical development of the modeling system and its application to ecological research and management. Specifically, they will guide:
\begin{itemize}
\item Refinement of the genetic algorithm parameters and operators
\item Improvement of model structure and parameter bounds
\item Development of more robust ecological prediction methods
\item Integration of model outputs into management decision processes
\end{itemize}
