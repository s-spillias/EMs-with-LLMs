\subsection{Best Performing NPZ Model}

\subsubsection{Model Description}
The following model represents our framework's attempt to recover the NPZ dynamics from \cite{edwards1999zooplankton}. The model aims to capture three key components:
\begin{itemize}
\item Nutrient uptake and recycling
\item Phytoplankton growth and mortality
\item Zooplankton predation and dynamics
\end{itemize}

\subsubsection{Model Intention}
\begin{lstlisting}
\section{Ecological Intention}

A key modification was made to incorporate direct nutrient recycling from zooplankton grazing activity. In marine systems, zooplankton feeding is often inefficient, with a significant portion of consumed phytoplankton being released as dissolved nutrients rather than being assimilated into biomass or entering the detritus pool. This "sloppy feeding" process creates an important feedback loop where grazing can stimulate new primary production through rapid nutrient recycling.

The recycling efficiency is temperature-dependent, reflecting how metabolic rates and feeding mechanics vary with temperature. This creates an adaptive feedback where warmer conditions lead to both increased grazing pressure and faster nutrient recycling, better capturing the coupled nature of predator-prey interactions in planktonic systems.

The modification introduces a direct pathway from grazing to dissolved nutrients, complementing the slower recycling through the detritus pool. This better represents the multiple timescales of nutrient cycling in marine food webs and helps explain how high productivity can be maintained even under intense grazing pressure.
\end{lstlisting}

\subsubsection{Model Implementation}
\newpage
\section*{NPZ Model: Parameter and Equation Tables}

\begin{landscape}
\subsection*{Parameter summary}

\begin{table}[ht]
\centering
\scriptsize
\setlength{\tabcolsep}{4pt}
\begin{tabular}{l p{3cm} p{10cm} c l l l}
\toprule
Symbol & Units & Meaning & Init.\ value & Bounds & Source & Literature\\
\midrule
log\_mu\_max & day$^{-1}$ (log scale) & Log of maximum phytoplankton growth rate at reference conditions (day$^{-1}$). & -0.02 & [-0.22, 0.18] & literature & Yes\\
log\_K\_N & g C m$^{-3}$ (log scale) & Log of half-saturation constant for nutrient uptake (g C m$^{-3}$). & -3.00 & [-6.91, 0.00] & literature & Yes\\
I & W m$^{-2}$ & Mean photosynthetically active irradiance proxy over the modeled period. & 150.00 & [0.00, 500.00] & initial estimate & No \\
log\_K\_I & W m$^{-2}$ (log scale) & Log of light half-saturation constant for photosynthesis (W m$^{-2}$). & 4.32 & [0.00, 5.70] & literature & Yes\\
log\_g\_max & day$^{-1}$ (log scale) & Log of maximum zooplankton grazing rate per unit Z biomass (day$^{-1}$). & -0.69 & [-3.00, 0.69] & literature & Yes\\
log\_K\_G & g C m$^{-3}$ (log scale) & Log of P half-saturation constant for grazing functional response (g C m$^{-3}$). & -2.30 & [-6.91, 0.00] & literature & Yes\\
h\_grazing & dimensionless & Holling type III shape exponent (h $\ge$ 1). & 2.00 & [1.00, 3.00] & literature & Yes\\
logit\_e\_Z & dimensionless (logit scale) & Logit of zooplankton assimilation efficiency ($e_Z \in (0,1)$); $e_Z = 0.5$ at value 0. & 0.00 & \textemdash & literature & Yes\\
log\_m\_P & day$^{-1}$ (log scale) & Log of phytoplankton linear mortality rate (day$^{-1}$). & -3.00 & [-6.91, -1.20] & literature & Yes\\
log\_m\_Z & day$^{-1}$ (log scale) & Log of zooplankton linear mortality rate (day$^{-1}$). & -3.51 & [-6.91, -1.20] & literature & Yes\\
log\_gamma\_Z & (g C m$^{-3}$)$^{-1}$ day$^{-1}$ (log scale) & Log of zooplankton quadratic self-limitation coefficient ((g C m$^{-3}$)$^{-1}$ day$^{-1}$). & -4.61 & [-9.21, -1.61] & initial estimate & No \\
logit\_r\_P & dimensionless (logit scale) & Logit of fraction of P mortality that is remineralized to N (0..1). & 0.85 & \textemdash & literature & Yes\\
logit\_r\_Z & dimensionless (logit scale) & Logit of fraction of Z mortality that is remineralized to N (0..1). & 0.85 & \textemdash & literature & Yes\\
log\_ex\_Z & day$^{-1}$ (log scale) & Log of zooplankton excretion rate to nutrients (day$^{-1}$). & -4.61 & [-13.82, -1.61] & initial estimate & No \\
log\_k\_mix & day$^{-1}$ (log scale) & Log of vertical mixing rate driving nutrients toward $N_\star$ (day$^{-1}$). & -3.91 & [-13.82, -0.69] & initial estimate & No \\
$N_\star$ & g C m$^{-3}$ & Deep/source nutrient concentration towards which mixing relaxes the system. & 0.30 & [0.00, 2.00] & initial estimate & No \\
log\_q10 & dimensionless (log scale) & Log of Q10 temperature scaling factor (dimensionless), typical $Q10 \approx 2$. & 0.66 & [0.61, 0.71] & literature & Yes\\
T\_C & deg C & Ambient temperature used for Q10 scaling (deg C). & 15.00 & [0.00, 35.00] & initial estimate & No \\
T\_ref & deg C & Reference temperature for Q10 scaling (deg C). & 15.00 & [0.00, 35.00] & literature & Yes\\
log\_k\_rem & day$^{-1}$ (log scale) & Log of detritus remineralization rate to nutrients (day$^{-1}$). & -2.30 & [-4.61, 0.00] & conceptual addition & No \\
log\_k\_sink & day$^{-1}$ (log scale) & Log of detritus sinking/export rate out of mixed layer (day$^{-1}$). & -4.61 & [-13.82, 0.00] & conceptual addition & No \\
log\_sigma\_N & log-scale SD & Log of observation SD for N on the log scale. & -2.30 & [-5.00, 2.00] & initial estimate & No \\
log\_sigma\_P & log-scale SD & Log of observation SD for P on the log scale. & -2.30 & [-5.00, 2.00] & initial estimate & No \\
log\_sigma\_Z & log-scale SD & Log of observation SD for Z on the log scale. & -2.30 & [-5.00, 2.00] & initial estimate & No \\
\bottomrule
\end{tabular}
\end{table}

\end{landscape}