
\section{Comparative Analysis of Best-Performing Models}
\label{sec:model_comparison}

Before presenting the full code for each model, we analyze the key differences between the best-performing models to understand their ecological approaches and mathematical structures.


\subsection{Key Parameter Comparison}
\label{subsec:parameter_comparison}

Table \ref{tab:parameter_comparison} presents a detailed comparison of key parameters across the five best-performing models and the human-generated model. These parameters represent fundamental ecological processes and reveal different modelling approaches to COTS-coral dynamics.

\begin{landscape}
\begin{table}[H]
\centering
\begin{footnotesize}
\caption{Comparison of key parameters across best-performing models}
\label{tab:parameter_comparison}
\begin{tabular}{p{3.2cm}p{2.8cm}p{2.8cm}p{2.8cm}p{2.8cm}p{2.8cm}p{2.8cm}}
 
\textbf{Parameter} & \textbf{Human Model} & \textbf{o3 mini} & \textbf{Claude Sonnet 3.6} & \textbf{Claude Sonnet 3.7} & \textbf{o4 mini} & \textbf{gpt 4.1} \\
 
COTS growth rate (yr$^{-1}$) & Beverton-Holt (h=0.5) & exp(log\_growth\_rate) & 0.8 & 0.8 & 0.5 & 0.5 \\
 
COTS mortality (yr$^{-1}$) & Mcots = 2.3 & exp(log\_decline\_rate) & -- & 0.4 & 0.3 & 0.37 \\
 
COTS carrying capacity & Derived from R0=1.0 & -- & 2.0 & 2.5 & 50 & 0.61 \\
 
Slow coral growth (yr$^{-1}$) & rm = 0.1 & 0.1 (fixed) & 0.2 & 0.1 & 0.05 (fixed) & 0.37 \\
 
Fast coral growth (yr$^{-1}$) & rf = 0.5 & 0.2 (fixed) & 0.4 & 0.3 & 0.1 (fixed) & 0.61 \\
 
Coral carrying capacity & K = 3000 (shared) & -- & 0.8 & K\_slow = 30, K\_fast = 50 & -- & K\_slow = 20.1, K\_fast = 33.1 \\
 
Fast coral optimal temp (°C) & SST0\_f = 26 & -- & -- & -- & -- & -- \\
 
Slow coral optimal temp (°C) & SST0\_m = 27 & -- & -- & -- & -- & -- \\
 
COTS optimal temp (°C) & Implicit & -- & 28 & 28 & -- & -- \\
 
Attack rate (fast coral) & p1f = 0.15 & 0.4 & 0.1 & 0.2 & 0.05 & 0.14 \\
 
Attack rate (slow coral) & p1m = 0.06 & 0.6 & 0.05 & 0.05 & 0.05 & 0.08 \\
 
Predation switching & switchSlope = 5 & -- & -- & -- & -- & -- \\
 
Functional response & Sigmoid switching & Logistic with quadratic adjustment & Type II & Type II & Type III & Type II \\
 
\end{tabular}
\end{footnotesize}
\end{table}
\end{landscape}

\subsection{Model Structure Comparison}
\label{subsec:structure_comparison}

Table \ref{tab:equation_comparison} presents a detailed comparison of the key equations used in each model, highlighting the different mathematical approaches to representing COTS-coral dynamics.


\begin{longtable}{p{2cm}p{13cm}}
\caption{Comparison of key equations across models}\label{tab:equation_comparison} \\

\textbf{Model} & \textbf{Key Equations} \\
 
\endfirsthead

\multicolumn{2}{c}%
{{\bfseries \tablename\ \thetable{} -- continued from previous page}} \\

\textbf{Model} & \textbf{Key Equations} \\
 
\endhead

\multicolumn{2}{r}{{Continued on next page}} \\
\endfoot


\endlastfoot
Human Model &
\begin{tabular}[t]{p{12.5cm}}
\textbf{COTS dynamics:} \\
Age-structured model with three age classes (0, 1, 2+) \\
$N(yr+1,1) = N(yr,0) \cdot \exp(-1 \cdot M\_CoTS\_age(0))$ \\
$N(yr+1,2) = N(yr,1) \cdot \exp(-f \cdot M\_CoTS\_age(1)) + N(yr,2) \cdot \exp(-f \cdot M\_CoTS\_age(2))$ \\
$Rcots(yr+1) = \frac{\alpha \cdot (N(yr+1,2)/Kots\_sp)}{\beta + (N(yr+1,2)/Kots\_sp)}$ \\
$N(yr+1,0) = (Rcots(yr+1) + Imm\_CoTS) \cdot \exp(Imm\_res(yr+1) + \sigma_{CoTS}^2/2)$ \\
\\
\textbf{Coral dynamics:} \\
$Cf(yr+1) = Cf(yr) \cdot (1.0 + \rho_{SST\_F} \cdot rf \cdot (1-(Cf(yr) + Cm(yr))/K)) - Qf - M\_ble\_f$ \\
$Cm(yr+1) = Cm(yr) \cdot (1.0 + \rho_{SST\_M} \cdot rm \cdot (1-(Cf(yr) + Cm(yr))/K)) - Qm - M\_ble\_m$ \\
\\
\textbf{Predation:} \\
$\rho = \exp(-switchSlope \cdot Cf(yr)/K)$ \\
$Qf = Cf(yr) \cdot (1.0-\rho) \cdot p1f \cdot \frac{N(yr,1)+N(yr,2)}{1.0+\exp(-(N(yr,1)+N(yr,2))/p2f)}$ \\
$Qm = Cm(yr) \cdot \rho \cdot p1m \cdot \frac{N(yr,1)+N(yr,2)}{1.0+\exp(-(N(yr,1)+N(yr,2))/p2m)}$ \\
\\
\textbf{Temperature effects:} \\
$\rho_{SST\_F} = \exp(-\frac{(SST-SST0\_f)^2}{2 \cdot SST\_sig\_f^2})$ \\
$\rho_{SST\_M} = \exp(-\frac{(SST-SST0\_m)^2}{2 \cdot SST\_sig\_m^2})$ \\
$M\_ble\_f = Cf(yr) \cdot \frac{1.0}{1.0 + \exp(-Eta\_f \cdot (SST-M\_SST50\_f))}$ \\
$M\_ble\_m = Cm(yr) \cdot \frac{1.0}{1.0 + \exp(-Eta\_m \cdot (SST-M\_SST50\_m))}$
\end{tabular} \\
 
o3 mini &
\begin{tabular}[t]{p{12.5cm}}
\textbf{COTS dynamics:} \\
$logistic\_factor = \frac{1}{1 + \exp(-outbreak\_steepness \cdot (resource\_limitation - threshold))}$ \\
$quadratic\_adjustment = \begin{cases}
1 + poly\_coeff \cdot (resource\_limitation - threshold)^2 & \text{if } resource\_limitation > threshold \\
1 & \text{otherwise}
\end{cases}$ \\
$outbreak\_factor = logistic\_factor \cdot quadratic\_adjustment$ \\
$temperature\_factor = 1 + effect\_sst \cdot sst\_dat(t-1)$ \\
$cots\_pred[t] = cots\_pred[t-1] + (growth\_rate \cdot cots\_pred[t-1] \cdot outbreak\_factor \cdot temperature\_factor - decline\_rate \cdot cots\_pred[t-1]) \cdot dt$ \\
\\
\textbf{Coral dynamics:} \\
$fast\_pred[t] = fast\_pred[t-1] + dt \cdot (fast\_growth\_rate \cdot fast\_pred[t-1] \cdot (1 - fast\_pred[t-1] / fast\_cap) - efficiency\_fast \cdot cots\_pred[t-1] \cdot fast\_pred[t-1])$ \\
$mod\_eff\_slow = efficiency\_slow \cdot (1 + temp\_mod\_eff\_slow \cdot sst\_dat(t-1))$ \\
$slow\_pred[t] = slow\_pred[t-1] + dt \cdot (slow\_growth\_rate \cdot slow\_pred[t-1] \cdot (1 - slow\_pred[t-1] / slow\_cap) - mod\_eff\_slow \cdot cots\_pred[t-1] \cdot slow\_pred[t-1])$
\end{tabular} \\
 
Claude Sonnet 3.6 &
\begin{tabular}[t]{p{12.5cm}}
\textbf{COTS dynamics:} \\
$temp\_effect = \exp(-0.5 \cdot \frac{(sst\_dat(t-1) - temp\_opt)^2}{temp\_range^2})$ \\
$resource\_limit = \frac{total\_coral}{total\_coral + \epsilon}$ \\
$recruitment = cotsimm\_dat(t-1) \cdot temp\_effect$ \\
$cots\_pred(t) = cots\_pred(t-1) \cdot (1 + r\_cots \cdot resource\_limit \cdot (1 - \frac{cots\_pred(t-1)}{K\_cots})) + recruitment$ \\
\\
\textbf{Coral dynamics:} \\
$coral\_space = \max(0, 1 - \frac{fast\_pred(t-1) + slow\_pred(t-1)}{100 \cdot coral\_limit})$ \\
$fast\_growth = r\_fast \cdot fast\_pred(t-1) \cdot coral\_space$ \\
$fast\_pred\_loss = grazing\_fast \cdot cots\_pred(t-1) \cdot fast\_pred(t-1)$ \\
$fast\_pred(t) = fast\_pred(t-1) + fast\_growth - fast\_pred\_loss$ \\
$slow\_growth = r\_slow \cdot slow\_pred(t-1) \cdot coral\_space$ \\
$slow\_pred\_loss = grazing\_slow \cdot cots\_pred(t-1) \cdot slow\_pred(t-1)$ \\
$slow\_pred(t) = slow\_pred(t-1) + slow\_growth - slow\_pred\_loss$
\end{tabular} \\
 
Claude Sonnet 3.7 &
\begin{tabular}[t]{p{12.5cm}}
\textbf{COTS dynamics:} \\
$temp\_effect = \exp(-0.5 \cdot \frac{(sst - temp\_opt)^2}{temp\_width^2})$ \\
$pred\_fast = \frac{a\_fast \cdot fast\_t0 \cdot cots\_t0}{1.0 + a\_fast \cdot h\_fast \cdot fast\_t0 + a\_slow \cdot h\_slow \cdot slow\_t0 + \epsilon}$ \\
$pred\_slow = \frac{a\_slow \cdot slow\_t0 \cdot cots\_t0}{1.0 + a\_fast \cdot h\_fast \cdot fast\_t0 + a\_slow \cdot h\_slow \cdot slow\_t0 + \epsilon}$ \\
$bleach\_effect = \frac{1.0}{1.0 + \exp(-2.0 \cdot (sst - bleach\_threshold))}$ \\
$cots\_growth = r\_cots \cdot cots\_t0 \cdot (1.0 - \frac{cots\_t0}{K\_cots}) \cdot temp\_effect$ \\
$imm\_term = \frac{imm\_effect \cdot cotsimm}{1.0 + cotsimm + \epsilon}$ \\
$food\_limitation = m\_cots \cdot (1.0 + \frac{1.0}{fast\_t0 + slow\_t0 + \epsilon})$ \\
$cots\_pred(t) = cots\_t0 + cots\_growth - food\_limitation \cdot cots\_t0 + imm\_term$ \\
\\
\textbf{Coral dynamics:} \\
$fast\_growth = r\_fast \cdot fast\_t0 \cdot (1.0 - \frac{fast\_t0 + competition \cdot slow\_t0}{K\_fast})$ \\
$fast\_bleaching = bleach\_mortality\_fast \cdot bleach\_effect \cdot fast\_t0$ \\
$fast\_pred(t) = fast\_t0 + fast\_growth - pred\_fast - fast\_bleaching$ \\
$slow\_growth = r\_slow \cdot slow\_t0 \cdot (1.0 - \frac{slow\_t0 + competition \cdot fast\_t0}{K\_slow})$ \\
$slow\_bleaching = bleach\_mortality\_slow \cdot bleach\_effect \cdot slow\_t0$ \\
$slow\_pred(t) = slow\_t0 + slow\_growth - pred\_slow - slow\_bleaching$
\end{tabular} \\
 
o4 mini &
\begin{tabular}[t]{p{12.5cm}}
\textbf{COTS dynamics:} \\
$coral\_availability = \frac{fast\_pred[t-1] + slow\_pred[t-1]}{200}$ \\
$resource\_factor = \frac{coral\_availability + coral\_saturation\_coefficient \cdot coral\_availability^2}{0.5 + coral\_availability + coral\_saturation\_coefficient \cdot coral\_availability^2}$ \\
$growth = growth\_rate\_cots \cdot cots\_pred[t-1] \cdot (1 - \frac{cots\_pred[t-1]}{carrying\_capacity + \epsilon}) \cdot (1 + resource\_limitation\_strength \cdot (resource\_factor - 0.5))$ \\
$effective\_sharpness = outbreak\_sharpness \cdot environmental\_modifier \cdot (1 + extreme\_outbreak\_modifier \cdot (environmental\_modifier - 1))$ \\
$raw\_trigger = \frac{1}{1 + \exp(- effective\_sharpness \cdot (cots\_pred[t-1]^{outbreak\_shape} + outbreak\_nonlinearity \cdot cots\_pred[t-1]^2 - (outbreak\_threshold \cdot carrying\_capacity)^{outbreak\_shape}))}$ \\
$outbreak\_trigger = raw\_trigger + outbreak\_hysteresis \cdot raw\_trigger \cdot (1 - raw\_trigger)$ \\
$decline = decay\_rate\_cots \cdot cots\_pred[t-1]^{outbreak\_decline\_exponent} \cdot outbreak\_trigger$ \\
$cots\_pred[t] = cots\_pred[t-1] + growth - decline$ \\
\\
\textbf{Coral dynamics:} \\
$fast\_pred[t] = fast\_pred[t-1] + 0.1 \cdot coral\_recovery\_modifier \cdot coral\_recovery\_environmental\_modifier \cdot (100 - fast\_pred[t-1]) \cdot (1 - coral\_recovery\_inhibition \cdot \frac{cots\_pred[t-1]}{carrying\_capacity + \epsilon}) - \frac{cots\_pred[t-1] \cdot coral\_predation\_efficiency \cdot fast\_pred[t-1] \cdot (\frac{fast\_pred[t-1]}{fast\_pred[t-1] + predation\_scaler})^{predation\_efficiency\_exponent}}{1 + handling\_time \cdot fast\_pred[t-1]}$ \\
$slow\_pred[t] = slow\_pred[t-1] + 0.05 \cdot coral\_recovery\_environmental\_modifier \cdot (100 - slow\_pred[t-1]) \cdot (1 - coral\_recovery\_inhibition \cdot \frac{cots\_pred[t-1]}{carrying\_capacity + \epsilon}) - \frac{cots\_pred[t-1] \cdot coral\_predation\_efficiency \cdot slow\_pred[t-1] \cdot (\frac{slow\_pred[t-1]}{slow\_pred[t-1] + predation\_scaler})^{predation\_efficiency\_exponent}}{1 + handling\_time \cdot slow\_pred[t-1]}$
\end{tabular} \\

gpt 4.1 &
\begin{tabular}[t]{p{12.5cm}}
\textbf{COTS dynamics:} \\
$coral\_effect = \frac{fast\_prev \cdot e\_fast + slow\_prev \cdot e\_slow}{K\_fast \cdot e\_fast + K\_slow \cdot e\_slow + \epsilon}$ \\
$sst\_effect = 1.0 + theta\_sst \cdot (sst\_dat(t) - 27.0)$ \\
$immig\_effect = immig\_scale \cdot cotsimm\_dat(t)$ \\
$outbreak\_boost = 1.0 + phi\_outbreak \cdot (coral\_effect - 0.5)$ \\
$cots\_growth = r\_cots \cdot cots\_prev \cdot (1.0 - \frac{cots\_prev}{K\_cots + \epsilon}) \cdot coral\_effect \cdot sst\_effect \cdot outbreak\_boost$ \\
$cots\_mortality = m\_cots \cdot cots\_prev$ \\
$cots\_next = cots\_prev + cots\_growth - cots\_mortality + immig\_effect$ \\
\\
\textbf{Coral dynamics:} \\
$pred\_fast = \frac{\alpha\_fast \cdot cots\_prev \cdot fast\_prev}{fast\_prev + slow\_prev + \epsilon}$ \\
$pred\_slow = \frac{\alpha\_slow \cdot cots\_prev \cdot slow\_prev}{fast\_prev + slow\_prev + \epsilon}$ \\
$fast\_growth = r\_fast \cdot fast\_prev \cdot (1.0 - \frac{fast\_prev}{K\_fast + \epsilon})$ \\
$fast\_mortality = m\_fast \cdot fast\_prev$ \\
$fast\_next = fast\_prev + fast\_growth - pred\_fast - fast\_mortality$ \\
$slow\_growth = r\_slow \cdot slow\_prev \cdot (1.0 - \frac{slow\_prev}{K\_slow + \epsilon})$ \\
$slow\_mortality = m\_slow \cdot slow\_prev$ \\
$slow\_next = slow\_prev + slow\_growth - pred\_slow - slow\_mortality$
\end{tabular}
\end{longtable}

\subsection{Detailed Ecological Mechanisms}
\label{subsec:ecological_mechanisms}

The models employ distinctly different approaches to represent key ecological processes:

\subsubsection{Temperature Dependency}
\label{subsubsec:temperature_dependency}

\paragraph{Human Model:} Implements temperature effects through two distinct mechanisms: (1) Gaussian functions modifying coral growth rates and (2) a logistic bleaching mortality function with explicit temperature thresholds (M\_SST50\_f, M\_SST50\_m).

\paragraph{o3 mini:} Employs an asymmetric Gaussian temperature response with a skew parameter, allowing for non-symmetric responses to temperature deviations.

\paragraph{Claude Sonnet 3.6:} Uses a standard Gaussian temperature effect (temp\_opt = 28°C) similar to the human model but without the explicit bleaching threshold.

\paragraph{Claude Sonnet 3.7:} Implements a Gaussian temperature response with temp\_opt = 28°C and temp\_width = 2°C, affecting COTS recruitment. Also includes a bleaching effect with a threshold of 30°C.

\paragraph{o4 mini:} Does not include explicit temperature dependency for COTS in its core equations, focusing instead on resource limitation and outbreak dynamics.

\paragraph{gpt 4.1:} Implements a linear temperature effect where SST modifies growth (centered at 27°C) through the parameter theta\_sst.

\subsubsection{Predation Formulations}
\label{subsubsec:predation_formulations}

\paragraph{Human Model:} Features an explicit prey-switching function where COTS predation preference between fast and slow corals depends on the relative abundance of fast-growing coral, with separate parameters for predation intensity (p1f, p1m) and density-dependence (p2f, p2m).

\paragraph{o3 mini:} Implements direct predation with efficiency factors of 0.4 for fast coral and 0.6 for slow coral, with temperature modifying the predation efficiency on slow coral.

\paragraph{Claude Sonnet 3.6:} Uses a simple Type II functional response with grazing rates of 0.1 for fast coral and 0.05 for slow coral, creating a saturating predation effect at high prey densities.

\paragraph{Claude Sonnet 3.7:} Employs a Holling Type II functional response with attack rates of 0.2 for fast coral and 0.05 for slow coral, with handling times for both coral types.

\paragraph{o4 mini:} Implements a Type III functional response with predation efficiency of 0.05, creating a sigmoidal functional response that reduces predation at low prey densities.

\paragraph{gpt 4.1:} Uses a Type II functional response with attack rates of 0.14 for fast coral and 0.08 for slow coral, with predation partitioned by coral type.

\subsubsection{Population Structure}
\label{subsubsec:population_structure}

\paragraph{Human Model:} Implements an age-structured COTS population with explicit age classes (age-0, age-1, and age-2+), each with age-dependent mortality rates, and uses a Beverton-Holt stock-recruitment relationship.

\paragraph{AI Models:} Generally employ simpler, unstructured population approaches with single-state variables for COTS abundance. The models use various forms of logistic growth (o3 mini, Claude Sonnet 3.7, o4 mini, gpt 4.1) or temperature-modified reproduction functions (Claude Sonnet 3.6, Claude Sonnet 3.7, gpt 4.1).

\subsection{Comparison with Human Model}
\label{subsec:human_comparison}

The human-generated model provides an important reference point for evaluating the AI-generated models. This expert-developed model incorporates several ecological mechanisms that differ from the AI approaches.

\paragraph{Population structure:}
Unlike the AI models, the human-generated model implements an age-structured COTS population with explicit age classes (age-0, age-1, and age-2+), each with age-dependent mortality rates. This contrasts with the simpler, unstructured population approaches in the AI models, which generally use single-state variables for COTS abundance.

\paragraph{Stock-recruitment relationship:}
The human model employs a Beverton-Holt stock-recruitment relationship for COTS reproduction, with parameters derived from unexploited population characteristics. This mechanistic approach differs from the AI models, which typically use simpler logistic growth or temperature-modified reproduction functions.

\paragraph{Prey switching:}
A distinctive feature of the human model is its explicit prey-switching function, where COTS predation preference between fast and slow corals depends on the relative abundance of fast-growing coral. This creates a dynamic feedback mechanism not fully captured in most AI models, though the o3 mini model implements a somewhat similar approach with its complex feedback mechanisms. The gpt 4.1 model also implements a form of prey partitioning based on relative coral abundance.

\paragraph{Temperature effects:}
The human model implements temperature effects through two distinct mechanisms: (1) Gaussian functions modifying coral growth rates, similar to Claude Sonnet 3.6 and Claude Sonnet 3.7, and (2) a logistic bleaching mortality function with temperature thresholds, which is also implemented in Claude Sonnet 3.7 with its bleach\_threshold parameter of 30°C.

\paragraph{Parameter differences:}
The human model uses different parameterization approaches, including:
\begin{itemize}
\item Direct parameterization of carrying capacity (K = 3000) rather than log-transformed values used in Claude Sonnet 3.6, Claude Sonnet 3.7, and gpt 4.1
\item Separate parameters for predation intensity (p1f = 0.15, p1m = 0.06) and density-dependence (p2f, p2m)
\item Explicit bleaching threshold parameters (M\_SST50\_f, M\_SST50\_m) compared to the single bleach\_threshold in Claude Sonnet 3.7
\item Age-dependent mortality components for COTS (Mcots = 2.3) compared to simpler mortality formulations in the AI models (e.g., m\_cots = 0.4 in Claude Sonnet 3.7, 0.3 in o4 mini, and 0.37 in gpt 4.1)
\item Explicit optimal temperatures for both coral types (SST0\_f = 26°C, SST0\_m = 27°C) which are not specified in the AI models
\end{itemize}

\subsection{Carrying Capacity and Growth Rate Comparison}
\label{subsec:carrying_capacity_comparison}

The models show variation in their parameterization of carrying capacity and growth rates:

\paragraph{COTS carrying capacity:} Values range from 0.61 individuals/m² (gpt 4.1) to 50 individuals/m² (o4 mini), with Claude Sonnet 3.6 at 2.0 and Claude Sonnet 3.7 at 2.5. This order-of-magnitude difference reflects fundamentally different assumptions about ecosystem capacity.

\paragraph{COTS growth rate:} More consistency is observed in growth rates, with values of 0.8 per year (Claude Sonnet 3.6 and Claude Sonnet 3.7), 0.5 per year (o4 mini and gpt 4.1), compared to the Beverton-Holt approach in the human model.

\paragraph{Coral growth rates:} The models show variation in coral growth parameters, with fast coral growth rates ranging from 0.1 per year (o4 mini) to 0.61 per year (gpt 4.1), and slow coral growth rates from 0.05 per year (o4 mini) to 0.37 per year (gpt 4.1).

\paragraph{Coral carrying capacity:} The models use different approaches to coral carrying capacity, from the shared K = 3000 in the human model to separate values for fast and slow coral in Claude Sonnet 3.7 (K\_fast = 50, K\_slow = 30) and gpt 4.1 (K\_fast = 33.1, K\_slow = 20.1).