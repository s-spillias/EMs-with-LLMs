Ecosystem models provide invaluable information for managing complex interactions between nature and people \citep{McCarthy2004, Holden2016}, but their development traditionally requires significant time and expertise, creating a bottleneck in addressing urgent environmental challenges \citep{Dichmont2017,Holden2024b}, particularly as climate change demands rapid, adaptable approaches for ecosystem management \citep{weiskopf2020climate,malhi2020climate}.

Artificial Intelligence (AI) offers great promise as a part of the solution to these modelling challenges, with potential to accelerate model development and enhance adaptability \citep{Spillias2024}. While initial efforts to apply AI in ecological modelling focused on machine learning approaches that rely on black-box methods \citep{morales2024developing}, emerging techniques in equation discovery and automated scientific discovery show particular promise \citep{Huntingford_Nicoll_Klein_Ahmad_2024,floryan2022data}. These methods can derive interpretable mathematical relationships directly from data, offering advantages over statistical emulators when modelling novel environmental conditions \citep{Schaeffer_2017,chen2024constructing,karniadakis2021physics}. 

A recent development in AI is the rise of large language models (LLMs): neural networks trained on vast text corpora, including scientific literature, code, and web content. LLMs can generate and interpret language, code, and equations, encoding broad domain knowledge that can be leveraged for scientific modelling tasks \citep{mammides2024role,wills2024use}. Attempts to leverage LLMs for direct time-series prediction \citep{zhang2024large,su2024large,hassani2024predictions,gandhi2024generative,bylund2024chatgpt,cao2023tempo,li2024lite,garza2023timegpt}, though successful in other fields, are unsuited for producing reliable ecological insights or testing management interventions. For instance, work on multimodal LLMs for environmental prediction \citep{li2024lite} achieves impressive accuracy in forecasting physical variables like streamflow and water temperature, but does not address the mechanistic relationships needed for ecosystem management and applications.

Rather than using AI to replace traditional modelling approaches, recent advances in AI coding capabilities suggest a more promising direction \citep{Xu2021}. Recent demonstrations of LLMs automating scientific processes, from autonomous chemical experimentation \citep{burger2023autonomous} to biomedical research \citep{wang2024bioresearcher}, evidence synthesis \citep{spillias2024evaluating} and even fully automated scientific discovery \citep{kramer2023automated}, highlight their potential for systematic scientific work. The key challenge lies in developing frameworks that can systematically harness these capabilities while ensuring scientific rigor and maintaining human oversight in the discovery process \citep{kramer2023automated,Spillias2024}. To the best of our knowledge, such an approach has not been attempted yet in ecological modelling.

To address this challenge, we present ``LLM-Enabled Mechanistic Modelling for ecosystem Assessment'' (LEMMA), a novel framework that requires minimal inputs, only time-series data and research questions, and aims to produce ecologically sound mathematical models that explain the time-series data (Figure \ref{fig:modelling_intro}). LEMMA addresses the inverse problem of inferring ecologically meaningful mechanistic models and parameters that causally explain observed data. The inferred models can then be used to test management interventions and scenarios. When solving inverse problems through numerical optimization, practitioners usually predetermine the model and its allowable parameter ranges, which defines both the parameter space and its mapping to the observation space. Less commonly, the model itself is inferred directly from observations, as in Scientific Machine Learning (SciML) \citep{willard2022integrating}, System Identification \citep{chiuso2019system}, and Automated Algorithm Discovery \citep{blazek2024automated}. Successful model reconstruction in such scenarios requires extensive data to ensure statistical relations are represented in the observations. In addition, a common challenge across all inverse problem approaches is the need to impose constraints that prevent numerically accurate yet empirically unrealistic solutions.

LEMMA differs from traditional and SciML approaches by leveraging the broad scientific and ecological knowledge encoded in large language models (LLMs) through pretraining on diverse sources such as peer-reviewed literature, textbooks, and code repositories. This embedded knowledge is used to impose ecologically meaningful constraints on both model structures and parameter ranges. In addition, LEMMA incorporates a Retrieval-Augmented Generation (RAG) process to identify parameter values and plausible bounds from trusted local repositories and online scientific sources, ensuring biologically realistic parameterization. Within this framework, LEMMA employs Template Model Builder (TMB; an R/C++ library for efficient parameter estimation in complex nonlinear models) as its computational backbone, providing a rigorous statistical foundation for ecological modelling. The system operates iteratively: an LLM proposes candidate model structures as TMB-compatible equations, which are evaluated against time-series data using normalized objective functions (where lower values indicate better fit). These models then undergo evolutionary optimization across multiple generations, with high-performing structures (``individuals'') retained and refined while underperforming models are discarded. This evolutionary process systematically yields increasingly accurate representations of ecosystem dynamics and enables exploration of alternative mechanistic hypotheses with quantified support.

We test LEMMA's ability to construct plausible ecological models through two complementary case studies that test different aspects of ecological modelling, each involving both dependent variables (state variables predicted by the model) and forcing variables (external drivers affecting the system). First, we evaluate the framework's ability to recover fundamental ecological understanding using synthetic data generated from a well-established nutrient-phytoplankton-zooplankton (NPZ) model \citep{edwards1999zooplankton}.  This controlled experiment tests LEMMA's equation-learning capabilities by comparing discovered equations against known mathematical relationships that represent core ecological processes. For our NPZ case study, we additionally evaluate models using ecological accuracy scores (on a scale of 0-8), which measure how well the generated models recover known ecological mechanisms from the reference model. These scores assess specific components like nutrient uptake, phytoplankton growth, and zooplankton dynamics, with higher scores indicating closer alignment with established ecological understanding. Second, we assess LEMMA's ability to provide management-relevant predictions using synthetic data based on Crown-of-Thorns starfish (COTS) populations on the Great Barrier Reef, derived from existing MICE models \citep{Morello_Plaganyi_Babcock_Sweatman_Hillary_Punt_2014, Rogers_Plaganyi_2022,Plaganyi_Punt_Hillary_Morello_Thebaud_Hutton_Pillans_Thorson_Fulton_Smith_et_al_2014, Condie_Anthony_Babcock_Baird_Beeden_Fletcher_Gorton_Harrison_Hobday_Plaganyi_et_al_2021}. The COTS case study, with three dependent variables (COTS abundance, fast-growing coral cover, and slow-growing coral cover) and two forcing variables (temperature and COTS immigrants), tests the framework's robustness while focusing on a specific management challenge: quantifying outbreaks in a complex predator-prey system. Through systematic comparison of different LLMs (o3-mini, o4-mini, and GPT4.1 from OpenAI, Claude Sonnet 3.7 and 3.7 from Anthropic, and Gemini from Google), we evaluate how different AI capabilities affect model performance. This comparative approach demonstrates how our evolutionary framework can both recover theoretical ecological relationships and are potentially suitable for use in support of ecosystem management.

\begin{figure}[H]
    \centering
    \includegraphics[width=0.8\textwidth]{../Figures/modelling_steps_AI_&_human}
    \caption{Stylised representation of the iterative modelling process that LEMMA aims to support. Whilst human experts drive the majority of the process, we show that AI-driven processes could play an important role in the Model Development stage.}
    \label{fig:modelling_intro}
    \end{figure}