\setcounter{section}{0}
\renewcommand{\thesection}{S\arabic{section}}

\section{Curated Literature Collection}
\label{subsec:curated_literature}

The local document collection used in this case study was carefully curated to provide comprehensive coverage of marine ecosystem modeling approaches, with particular focus on COTS-coral dynamics and management interventions. The collection encompasses several key research areas:

\begin{itemize}
\item Ecosystem Modeling Frameworks: \cite{Plaganyi_2007} established foundational principles for ecosystem approaches to fisheries, while \cite{Plaganyi_Punt_Hillary_Morello_Thebaud_Hutton_Pillans_Thorson_Fulton_Smith_et_al_2014} introduced Models of Intermediate Complexity for Ecosystem assessments (MICE). \cite{Collie_Botsford_Hastings_Kaplan_Largier_Livingston_Plaganyi_Rose_Wells_Werner_2016} explored optimal model complexity levels.

\item COTS Management and Ecology: \cite{Pratchett_Caballes_Wilmes_Matthews_Mellin_Sweatman_Nadler_Brodie_Thompson_Hoey_et_al_2017} provided a comprehensive thirty-year review of COTS research. \cite{morello2014model} developed models for COTS outbreak management, while \cite{Rogers_Plaganyi_2022} analyzed corallivore culling impacts under bleaching scenarios.

\item Ecological Regime Shifts: \cite{Blamey_Plaganyi_Branch_2014} investigated predator-driven regime shifts in marine ecosystems. \cite{Plaganyi_Ellis_Blamey_Morello_Norman-Lopez_Robinson_Sporcic_Sweatman_2014} provided insights into ecological tipping points through ecosystem modeling.

\item Management Interventions: \cite{Condie_Anthony_Babcock_Baird_Beeden_Fletcher_Gorton_Harrison_Hobday_Plaganyi_et_al_2021} examined large-scale interventions on the Great Barrier Reef. \cite{Punt_MacCall_Essington_Francis_Hurtado-Ferro_Johnson_Kaplan_Koehn_Levin_Sydeman_2016} explored harvest control implications using MICE models.

\item Model Application Guidelines: \cite{Essington_Plaganyi_2014} provided critical guidelines for adapting ecosystem models to new applications. \cite{Gamble_Link_2009} demonstrated multispecies production model applications for analyzing ecological and fishing effects.

\item Integrated Systems: \cite{Hadley_Wild-Allen_Johnson_Macleod_2015} and \cite{Oca_Cremades_Jimenez_Pintado_Masalo_2019} explored integrated multi-trophic aquaculture modeling, providing insights into coupled biological systems. \cite{Spillias_Cottrell_2024} analyzed trade-offs in seaweed farming between food production, livelihoods, marine biodiversity, and carbon sequestration benefits.
\end{itemize}

These papers were selected based on their direct relevance to COTS population dynamics, coral reef ecology, and ecosystem modeling approaches. The collection provided both specific parameter values and broader ecological context for model development.

\section{RAG Architecture Implementation}
\label{subsec:rag_architecture}

The Retrieval-Augmented Generation (RAG) system facilitates parameter search and extraction from scientific literature. The system employs two primary search strategies: a local search of user-curated documents and a comprehensive web search. For local search, the system uses ChromaDB as a persistent vector store to maintain an indexed collection of scientific papers and technical documents specifically curated by research teams for their ecological systems. These documents are processed into semantic chunks of approximately 512 tokens with small overlaps to preserve context while enabling precise retrieval of relevant information.

The parameter search process begins with the generation of enhanced semantic descriptions for each parameter. These descriptions are crafted to improve search relevance by capturing the ecological and mathematical context in which the parameters are used. The system first searches the user-curated local documents using embeddings generated through Azure OpenAI's embedding service. When necessary, it extends to web-based sources through two channels: querying the Semantic Scholar database for highly-cited papers in biology, mathematics, and environmental science, and conducting broader literature searches through the Serper API to capture additional relevant sources.

The search results from both local and web sources are processed through an LLM to extract numerical values. The system applies consistent validation across both search pathways, identifying minimum and maximum bounds, ensuring unit consistency, and validating source reliability. When direct parameter values are not found in either the local collection or web sources, the system defaults to the initial estimates from the coding LLM. All extracted information, including parameter values, valid ranges, and complete citation details, is stored in a structured JSON database for reproducibility and future reference.

The RAG system implements automatic retry mechanisms when initial searches fail to yield usable results. Each retry attempt follows a structured progression: first accessing the curated local collection through ChromaDB queries, then expanding to Semantic Scholar for peer-reviewed literature, and finally utilizing Serper API for broader scientific content. This progressive broadening of scope, while maintaining focus on ecologically relevant sources, ensures robust parameter estimation even in cases where direct measurements are sparse in the literature.

\section{AI Prompts Used in Model Development}
\label{sec:ai_prompts}

The development of the model relied on several carefully crafted prompts to guide the artificial intelligence system. These prompts were designed to ensure numerical stability, proper likelihood calculation, and clear model structure. The following sections detail the exact prompts used at each stage of model development.

\subsection{Initial Model Creation}
\label{subsec:initial_model_prompt}

The initial model creation utilized a comprehensive prompt that emphasized three key aspects of model development. The prompt used for model initialization was:

\begin{lstlisting}
Please create a Template Model Builder model for the following topic:[PROJECT_TOPIC]. Start by writing intention.txt, in which you provide a concise summary of the ecological functioning of the model. In model.cpp, write your TMB model with the following important considerations:

1. NUMERICAL STABILITY:
- Always use small constants (e.g., Type(1e-8)) to prevent division by zero
- Use smooth transitions instead of hard cutoffs in equations
- Bound parameters within biologically meaningful ranges using smooth penalties rather than hard constraints

2. LIKELIHOOD CALCULATION:
- Always include observations in the likelihood calculation, don't skip any based on conditions
- Use fixed minimum standard deviations to prevent numerical issues when data values are small
- Consider log-transforming data if it spans multiple orders of magnitude
- Use appropriate error distributions (e.g., lognormal for strictly positive data)

3. MODEL STRUCTURE:
- Include comments after each line explaining the parameters (including their units and how to determine their values)
- Provide a numbered list of descriptions for the equations
- Ensure all important variables are included in the reporting section
- Use `_pred' suffix for model predictions corresponding to `_dat' observations
\end{lstlisting}

\subsection{Parameter Enhancement}
\label{subsec:parameter_enhancement_prompt}

To enhance parameter descriptions for improved semantic search capabilities, the following prompt was employed:

\begin{lstlisting}
Given a mathematical model about [PROJECT_TOPIC], enhance the semantic descriptions of these parameters to be more detailed and searchable. The model code shows these parameters are used in the following way:

[MODEL_CONTENT]

For each parameter below, create an enhanced semantic search, no longer than 10 words, that can be used for RAG search or semantic scholar search.
\end{lstlisting}

\subsection{Model Improvement}
\label{subsec:model_improvement_prompt}

For iterative model improvements, the system utilized this prompt:

\begin{lstlisting}
Improve the fit of the following ecological model by modifying the equations in this TMB script. Only make ONE discrete change most likely to improve the fit. Do not add stochasticity, but you may add other ecological relevant factors that may not be present here already.

You may add additional parameters if necessary, and if so, add them to parameters.json. Please concisely describe your ecological improvement in intention.txt and then provide the improved model.cpp and parameters.json content.

\end{lstlisting}

\subsection{Error Handling Prompts}
\label{subsec:error_handling_prompt}

For compilation errors, the system used this prompt:

\begin{lstlisting}
model.cpp failed to compile. Here's the error information:

[ERROR_INFO]

Do not suggest how to compile the script
\end{lstlisting}

For data leakage issues, the system employed this detailed prompt:

\begin{lstlisting}
Data leakage detected in model equations. The following response variables cannot be used to predict themselves:

To fix this:
1. Response variables ([RESPONSE_VARS]) must be predicted using only:
   - External forcing variables ([FORCING_VARS])
   - Other response variables' predictions (_pred variables)
   - Parameters and constants
2. Each response variable must have a corresponding prediction equation
3. Use ecological relationships to determine how variables affect each other

For example, instead of:
  slow_pred(i) = slow * growth_rate;
Use:
  slow_pred(i) = slow_pred(i-1) * growth_rate * (1 - impact_rate * cots_pred(i-1));

Please revise the model equations to avoid using response variables to predict themselves.
\end{lstlisting}

For numerical instabilities, the system used an adaptive prompt that became progressively more focused on simplification after multiple attempts:

\begin{lstlisting}
The model compiled but numerical instabilities occurred. Here's the error information:

[ERROR_INFO]

[After 2+ attempts: Consider making a much simpler model that we can iteratively improve later.]
Do not suggest how to compile the script
\end{lstlisting}

\subsection{NPZ Case Study - Recovering Equations}
\label{subsec:npz_evaluation_prompt}

The model implementation can be compared to the original NPZ equations from \cite{edwards1999zooplankton}:

\begin{align*}
    \frac{dN}{dt} &= \underbrace{-\frac{V_m N P}{k_s + N}}_{\text{nutrient uptake}}
                   + \underbrace{\gamma(1-\alpha)\frac{g P^2 Z}{k_g + P^2} + \mu_P P + \mu_Z Z^2}_{\text{recycling}}
                   + \underbrace{S(N_0 - N)}_{\text{mixing}} \\[6pt]
    \frac{dP}{dt} &= \underbrace{\frac{V_m N P}{k_s + N}}_{\text{growth}}
                   - \underbrace{\frac{g P^2 Z}{k_g + P^2}}_{\text{grazing loss}}
                   - \underbrace{\mu_P P}_{\text{mortality}}
                   - \underbrace{S P}_{\text{mixing}} \\[6pt]
    \frac{dZ}{dt} &= \underbrace{\alpha\frac{g P^2 Z}{k_g + P^2}}_{\text{growth (assimilation)}}
                   - \underbrace{\mu_Z Z^2}_{\text{mortality}}
                   - \underbrace{S Z}_{\text{mixing}}
    \end{align*}

Our generated model captures several key ecological processes from the original system:
\begin{enumerate}
\item Nutrient uptake by phytoplankton following Michaelis-Menten kinetics
\item Quadratic zooplankton mortality
\item Nutrient recycling through zooplankton excretion
\item Environmental mixing effects
\end{enumerate}

For evaluating the ecological characteristics of generated models against the NPZ reference model, the system employed a 4-level ordinal scoring system that compares each model component to both the ground truth equations and recognized alternate formulations from the ecological literature. The evaluation system assessed nine ecological characteristics organized by equation: nutrient uptake, recycling, and mixing (dN/dt); phytoplankton growth, grazing loss, mortality, and mixing (dP/dt); and zooplankton growth and mortality (dZ/dt).

The scoring rubric used for all evaluations was:

\begin{lstlisting}
Scoring rubric per characteristic (choose exactly one category):
- 3 = TRUTH_MATCH
    The mathematical structure is equivalent to the TRUTH model (modulo variable names,
    syntax, factor grouping, and coefficient naming). Quote the exact snippet that matches.
- 2 = ALTERNATE
    The implementation matches one of the alternates enumerated in the literature catalog,
    even if not identical to TRUTH. Name the family (e.g., "Michaelis-Menten uptake",
    "Ivlev grazing with threshold", "linear mortality", "Droop quota").
- 1 = SIMILAR_NOT_LISTED
    The implementation plays the same ecological role and is mathematically similar
    (e.g., another saturating curve or plausible closure) but is not represented in TRUTH
    or alternates list.
- 0 = NOT_PRESENT_OR_INCORRECT
    The ecological component is missing or cannot be identified.
\end{lstlisting}

The alternate formulations catalog was based on \cite{franks2002npz} and included:

\begin{itemize}
    \item Phytoplankton light response: linear, saturating (Michaelis-Menten, exponential, tanh), and photo-inhibiting forms
    \item Nutrient uptake: Michaelis-Menten, Liebig minimum limitation, Droop quota models
    \item Zooplankton grazing: linear, saturating with threshold, Holling/Ivlev type, acclimating forms
    \item Mortality terms: linear and quadratic (density-dependent) for both phytoplankton and zooplankton
\end{itemize}

Each characteristic was assigned a weight based on its contribution to its parent equation: the three nutrient equation components each had weight 0.333, the four phytoplankton components each had weight 0.25, and the two zooplankton components each had weight 0.5. The aggregate ecological score was calculated as the weighted sum of individual scores, then normalized to a 0-1 scale by dividing by the maximum possible score.

\subsubsection{Validation of Scoring System}

To validate the ecological characteristics scoring system, we tested it on the ground truth NPZ model itself (evaluating the model against its own equations). This test confirmed that the scoring system could correctly identify and score all nine ecological characteristics when they were present in their canonical forms.

The validation results demonstrated perfect performance:

\begin{itemize}
    \item All nine characteristics received scores of 3 (TRUTH\_MATCH)
    \item Raw total score: 8.997 (out of maximum 9.0, with small rounding due to floating point arithmetic)
    \item Normalized total score: 1.0000 (perfect score on 0-1 scale)
    \item Zero extra components identified (correctly recognized model contained only canonical NPZ processes)
\end{itemize}

The LLM evaluator correctly identified each ecological mechanism in the ground truth model, providing detailed explanations such as ``algebraically identical to the TRUTH NPZ model'' and specifically noting the presence of ``Michaelis-Menten style nutrient limitation multiplied by a light/self-shading term for phytoplankton growth'' and ``a saturating P\textsuperscript{2}/(µ\textsuperscript{2}+P\textsuperscript{2}) (Hill/Type-III-like) grazing formulation.'' This validation confirmed that the scoring system could reliably distinguish between different levels of ecological fidelity, from exact matches to the ground truth through recognized alternates to novel formulations, providing a robust framework for assessing LEMMA-generated models.

\section{NPZ Validation}
\label{sec:npz_validation}

% \begin{figure}[H]
% \centering
% \includegraphics[width=\textwidth]{../Figures/ecological_characteristics_vs_objective}
% \caption{Relationship between ecological accuracy scores and model performance for each NPZ model characteristic. Each panel shows how well models recovered a specific ecological mechanism (score from 0-1) versus their predictive accuracy (objective value). Lower objective values indicate better model fit. Two-sided Pearson's product-moment correlation coefficients (r) and their associated p-values are shown for each characteristic.}
% \label{fig:ecological_characteristics}
% \end{figure}

\subsection{Best Performing NPZ Model}

\subsubsection{Model Description}
The following model represents our framework's attempt to recover the NPZ dynamics from \cite{edwards1999zooplankton}. The model aims to capture three key components:
\begin{itemize}
\item Nutrient uptake and recycling
\item Phytoplankton growth and mortality
\item Zooplankton predation and dynamics
\end{itemize}

\subsubsection{Model Intention}
\begin{lstlisting}
\section{Ecological Intention}

A key modification was made to incorporate direct nutrient recycling from zooplankton grazing activity. In marine systems, zooplankton feeding is often inefficient, with a significant portion of consumed phytoplankton being released as dissolved nutrients rather than being assimilated into biomass or entering the detritus pool. This "sloppy feeding" process creates an important feedback loop where grazing can stimulate new primary production through rapid nutrient recycling.

The recycling efficiency is temperature-dependent, reflecting how metabolic rates and feeding mechanics vary with temperature. This creates an adaptive feedback where warmer conditions lead to both increased grazing pressure and faster nutrient recycling, better capturing the coupled nature of predator-prey interactions in planktonic systems.

The modification introduces a direct pathway from grazing to dissolved nutrients, complementing the slower recycling through the detritus pool. This better represents the multiple timescales of nutrient cycling in marine food webs and helps explain how high productivity can be maintained even under intense grazing pressure.
\end{lstlisting}

\subsubsection{Model Implementation}
\newpage
\section*{NPZ Model: Parameter and Equation Tables}

\begin{landscape}
\subsection*{Parameter summary}

\begin{table}[ht]
\centering
\scriptsize
\setlength{\tabcolsep}{4pt}
\begin{tabular}{l p{3cm} p{10cm} c l l l}
\toprule
Symbol & Units & Meaning & Init.\ value & Bounds & Source & Literature (citekey) \\
\midrule
log\_mu\_max & day$^{-1}$ (log scale) & Log of maximum phytoplankton growth rate at reference conditions (day$^{-1}$). & -0.02 & [-0.22, 0.18] & literature & Yes (LitNPZ\_log\_mu\_max) \\
log\_K\_N & g C m$^{-3}$ (log scale) & Log of half-saturation constant for nutrient uptake (g C m$^{-3}$). & -3.00 & [-6.91, 0.00] & literature & Yes (LitNPZ\_log\_K\_N) \\
I & W m$^{-2}$ & Mean photosynthetically active irradiance proxy over the modeled period. & 150.00 & [0.00, 500.00] & initial estimate & No \\
log\_K\_I & W m$^{-2}$ (log scale) & Log of light half-saturation constant for photosynthesis (W m$^{-2}$). & 4.32 & [0.00, 5.70] & literature & Yes (LitNPZ\_log\_K\_I) \\
log\_g\_max & day$^{-1}$ (log scale) & Log of maximum zooplankton grazing rate per unit Z biomass (day$^{-1}$). & -0.69 & [-3.00, 0.69] & literature & Yes (LitNPZ\_log\_g\_max) \\
log\_K\_G & g C m$^{-3}$ (log scale) & Log of P half-saturation constant for grazing functional response (g C m$^{-3}$). & -2.30 & [-6.91, 0.00] & literature & Yes (LitNPZ\_log\_K\_G) \\
h\_grazing & dimensionless & Holling type III shape exponent (h $\ge$ 1). & 2.00 & [1.00, 3.00] & literature & Yes (LitNPZ\_h\_grazing) \\
logit\_e\_Z & dimensionless (logit scale) & Logit of zooplankton assimilation efficiency ($e_Z \in (0,1)$); $e_Z = 0.5$ at value 0. & 0.00 & \textemdash & literature & Yes (LitNPZ\_logit\_e\_Z) \\
log\_m\_P & day$^{-1}$ (log scale) & Log of phytoplankton linear mortality rate (day$^{-1}$). & -3.00 & [-6.91, -1.20] & literature & Yes (LitNPZ\_log\_m\_P) \\
log\_m\_Z & day$^{-1}$ (log scale) & Log of zooplankton linear mortality rate (day$^{-1}$). & -3.51 & [-6.91, -1.20] & literature & Yes (LitNPZ\_log\_m\_Z) \\
log\_gamma\_Z & (g C m$^{-3}$)$^{-1}$ day$^{-1}$ (log scale) & Log of zooplankton quadratic self-limitation coefficient ((g C m$^{-3}$)$^{-1}$ day$^{-1}$). & -4.61 & [-9.21, -1.61] & initial estimate & No \\
logit\_r\_P & dimensionless (logit scale) & Logit of fraction of P mortality that is remineralized to N (0..1). & 0.85 & \textemdash & literature & Yes (LitNPZ\_logit\_r\_P) \\
logit\_r\_Z & dimensionless (logit scale) & Logit of fraction of Z mortality that is remineralized to N (0..1). & 0.85 & \textemdash & literature & Yes (LitNPZ\_logit\_r\_Z) \\
log\_ex\_Z & day$^{-1}$ (log scale) & Log of zooplankton excretion rate to nutrients (day$^{-1}$). & -4.61 & [-13.82, -1.61] & initial estimate & No \\
log\_k\_mix & day$^{-1}$ (log scale) & Log of vertical mixing rate driving nutrients toward $N_\star$ (day$^{-1}$). & -3.91 & [-13.82, -0.69] & initial estimate & No \\
$N_\star$ & g C m$^{-3}$ & Deep/source nutrient concentration towards which mixing relaxes the system. & 0.30 & [0.00, 2.00] & initial estimate & No \\
log\_q10 & dimensionless (log scale) & Log of Q10 temperature scaling factor (dimensionless), typical $Q10 \approx 2$. & 0.66 & [0.61, 0.71] & literature & Yes (LitNPZ\_log\_q10) \\
T\_C & deg C & Ambient temperature used for Q10 scaling (deg C). & 15.00 & [0.00, 35.00] & initial estimate & No \\
T\_ref & deg C & Reference temperature for Q10 scaling (deg C). & 15.00 & [0.00, 35.00] & literature & Yes (LitNPZ\_T\_ref) \\
log\_k\_rem & day$^{-1}$ (log scale) & Log of detritus remineralization rate to nutrients (day$^{-1}$). & -2.30 & [-4.61, 0.00] & conceptual addition & No \\
log\_k\_sink & day$^{-1}$ (log scale) & Log of detritus sinking/export rate out of mixed layer (day$^{-1}$). & -4.61 & [-13.82, 0.00] & conceptual addition & No \\
log\_sigma\_N & log-scale SD & Log of observation SD for N on the log scale. & -2.30 & [-5.00, 2.00] & initial estimate & No \\
log\_sigma\_P & log-scale SD & Log of observation SD for P on the log scale. & -2.30 & [-5.00, 2.00] & initial estimate & No \\
log\_sigma\_Z & log-scale SD & Log of observation SD for Z on the log scale. & -2.30 & [-5.00, 2.00] & initial estimate & No \\
\bottomrule
\end{tabular}
\end{table}

\end{landscape}

\section{CoTS Model Convergence}
\label{sec:convergence}
% \begin{figure}[H]
%     \centering
%     \includegraphics[width=0.8\textwidth]{../Figures/founder_to_terminal_evolution.png}
%     \caption{Evolution of model performance across generations. The plot shows the progression of objective values from an initial successful individual to the final individual in the lineage (either due to process termination or culling). Single dots represent individuals which were not improved upon in any subsequent iterations.}
%     \label{fig:evolution}
%     \end{figure}

    
\subsection{Model Evolution and Convergence}
The evolutionary process exhibited consistent refinement across generations, with measurable improvements in model performance. On average, populations reached their best-performing individual within 6.9 generations, and the mean improvement frequency across all populations was 38.0\%. Figure \ref{fig:status_distribution} shows the distribution of successful, culled, and broken models across generations. Notably, two populations achieved convergence below the target threshold, representing 9.5\% of all populations.
Performance varied significantly across populations. The fastest-converging population reached an optimal objective value of 0.0035 in just 3 generations, while others required up to 13 generations. This population also demonstrated a high improvement rate of -0.655 and a consistent improvement frequency of 50\%. In contrast, several populations showed minimal or no improvement, with some failing to converge within the allotted iterations.
\begin{figure}[H]
\centering
\includegraphics[width=0.8\textwidth]{../Figures/success_frequency}
\caption{Evolution of model performance during the genetic algorithm optimization process. Each generation represents an iteration of model development, where models are evaluated and classified into three categories: the best performers according to the NMSE objective value (kept, green), those that are numerically stable but outcompeted (culled, blue), and those that failed due to numerical instability, data leakage, or syntax errors (broken, orange). The vertical axis shows the count of new models in each category per generation, while rows represent independent replicates using different LLM configurations. Gemini-2.5-Pro was included in the analysis but did not produce successful runs for some populations.}
\label{fig:status_distribution}
\end{figure}

\subsection{Numerical Stability and Optimization}
Numerical stability varied across LLM configurations, with runtime and generation time metrics reflecting differences in optimization efficiency. The GPT-5 configuration showed moderate efficiency, with an average generation time of 12.0 minutes (SD = 13.0). The Claude Sonnet 4.5 configuration had longer generation times, averaging 71.2 minutes (SD = 155.2), though this includes variability from a small number of outlier populations. In contrast, the Gemini-2.5-Pro configuration demonstrated the fastest generation cycles, averaging 4.1 minutes per generation (SD = 0.54), though it exhibited lower convergence rates and higher instability in some cases.
Figure \ref{fig:iterations_by_llm} illustrates the distribution of iteration counts required for successful model convergence across LLMs. Most models converged within 4 to 7 iterations, with some outliers requiring up to 11 iterations.
\begin{figure}[H]
\centering
\includegraphics[width=0.8\textwidth]{../Figures/iterations_by_llm}
\caption{Distribution of iteration counts for successful model instances by LLM configuration. The boxplot excludes cases that reached maximum iterations or remained numerically unstable.}
\label{fig:iterations_by_llm}
\end{figure}


\section{Comparative Analysis of Best-Performing Models}
\label{sec:model_comparison}

Before presenting the full code for each model, we analyze the key differences between the best-performing models to understand their ecological approaches and mathematical structures.


\subsection{Key Parameter Comparison}
\label{subsec:parameter_comparison}

Table \ref{tab:parameter_comparison} presents a detailed comparison of key parameters across the five best-performing models and the human-generated model. These parameters represent fundamental ecological processes and reveal different modelling approaches to COTS-coral dynamics.

\begin{landscape}
\begin{table}[H]
\centering
\begin{footnotesize}
\caption{Comparison of key parameters across best-performing models}
\label{tab:parameter_comparison}
\begin{tabular}{p{3.2cm}p{2.8cm}p{2.8cm}p{2.8cm}p{2.8cm}p{2.8cm}p{2.8cm}}
 
\textbf{Parameter} & \textbf{Human Model} & \textbf{o3 mini} & \textbf{Claude Sonnet 3.6} & \textbf{Claude Sonnet 3.7} & \textbf{o4 mini} & \textbf{gpt 4.1} \\
 
COTS growth rate (yr$^{-1}$) & Beverton-Holt (h=0.5) & exp(log\_growth\_rate) & 0.8 & 0.8 & 0.5 & 0.5 \\
 
COTS mortality (yr$^{-1}$) & Mcots = 2.3 & exp(log\_decline\_rate) & -- & 0.4 & 0.3 & 0.37 \\
 
COTS carrying capacity & Derived from R0=1.0 & -- & 2.0 & 2.5 & 50 & 0.61 \\
 
Slow coral growth (yr$^{-1}$) & rm = 0.1 & 0.1 (fixed) & 0.2 & 0.1 & 0.05 (fixed) & 0.37 \\
 
Fast coral growth (yr$^{-1}$) & rf = 0.5 & 0.2 (fixed) & 0.4 & 0.3 & 0.1 (fixed) & 0.61 \\
 
Coral carrying capacity & K = 3000 (shared) & -- & 0.8 & K\_slow = 30, K\_fast = 50 & -- & K\_slow = 20.1, K\_fast = 33.1 \\
 
Fast coral optimal temp (°C) & SST0\_f = 26 & -- & -- & -- & -- & -- \\
 
Slow coral optimal temp (°C) & SST0\_m = 27 & -- & -- & -- & -- & -- \\
 
COTS optimal temp (°C) & Implicit & -- & 28 & 28 & -- & -- \\
 
Attack rate (fast coral) & p1f = 0.15 & 0.4 & 0.1 & 0.2 & 0.05 & 0.14 \\
 
Attack rate (slow coral) & p1m = 0.06 & 0.6 & 0.05 & 0.05 & 0.05 & 0.08 \\
 
Predation switching & switchSlope = 5 & -- & -- & -- & -- & -- \\
 
Functional response & Sigmoid switching & Logistic with quadratic adjustment & Type II & Type II & Type III & Type II \\
 
\end{tabular}
\end{footnotesize}
\end{table}
\end{landscape}

\subsection{Model Structure Comparison}
\label{subsec:structure_comparison}

Table \ref{tab:equation_comparison} presents a detailed comparison of the key equations used in each model, highlighting the different mathematical approaches to representing COTS-coral dynamics.


\begin{longtable}{p{2cm}p{13cm}}
\caption{Comparison of key equations across models}\label{tab:equation_comparison} \\

\textbf{Model} & \textbf{Key Equations} \\
 
\endfirsthead

\multicolumn{2}{c}%
{{\bfseries \tablename\ \thetable{} -- continued from previous page}} \\

\textbf{Model} & \textbf{Key Equations} \\
 
\endhead

\multicolumn{2}{r}{{Continued on next page}} \\
\endfoot


\endlastfoot
Human Model &
\begin{tabular}[t]{p{12.5cm}}
\textbf{COTS dynamics:} \\
Age-structured model with three age classes (0, 1, 2+) \\
$N(yr+1,1) = N(yr,0) \cdot \exp(-1 \cdot M\_CoTS\_age(0))$ \\
$N(yr+1,2) = N(yr,1) \cdot \exp(-f \cdot M\_CoTS\_age(1)) + N(yr,2) \cdot \exp(-f \cdot M\_CoTS\_age(2))$ \\
$Rcots(yr+1) = \frac{\alpha \cdot (N(yr+1,2)/Kots\_sp)}{\beta + (N(yr+1,2)/Kots\_sp)}$ \\
$N(yr+1,0) = (Rcots(yr+1) + Imm\_CoTS) \cdot \exp(Imm\_res(yr+1) + \sigma_{CoTS}^2/2)$ \\
\\
\textbf{Coral dynamics:} \\
$Cf(yr+1) = Cf(yr) \cdot (1.0 + \rho_{SST\_F} \cdot rf \cdot (1-(Cf(yr) + Cm(yr))/K)) - Qf - M\_ble\_f$ \\
$Cm(yr+1) = Cm(yr) \cdot (1.0 + \rho_{SST\_M} \cdot rm \cdot (1-(Cf(yr) + Cm(yr))/K)) - Qm - M\_ble\_m$ \\
\\
\textbf{Predation:} \\
$\rho = \exp(-switchSlope \cdot Cf(yr)/K)$ \\
$Qf = Cf(yr) \cdot (1.0-\rho) \cdot p1f \cdot \frac{N(yr,1)+N(yr,2)}{1.0+\exp(-(N(yr,1)+N(yr,2))/p2f)}$ \\
$Qm = Cm(yr) \cdot \rho \cdot p1m \cdot \frac{N(yr,1)+N(yr,2)}{1.0+\exp(-(N(yr,1)+N(yr,2))/p2m)}$ \\
\\
\textbf{Temperature effects:} \\
$\rho_{SST\_F} = \exp(-\frac{(SST-SST0\_f)^2}{2 \cdot SST\_sig\_f^2})$ \\
$\rho_{SST\_M} = \exp(-\frac{(SST-SST0\_m)^2}{2 \cdot SST\_sig\_m^2})$ \\
$M\_ble\_f = Cf(yr) \cdot \frac{1.0}{1.0 + \exp(-Eta\_f \cdot (SST-M\_SST50\_f))}$ \\
$M\_ble\_m = Cm(yr) \cdot \frac{1.0}{1.0 + \exp(-Eta\_m \cdot (SST-M\_SST50\_m))}$
\end{tabular} \\
 
o3 mini &
\begin{tabular}[t]{p{12.5cm}}
\textbf{COTS dynamics:} \\
$logistic\_factor = \frac{1}{1 + \exp(-outbreak\_steepness \cdot (resource\_limitation - threshold))}$ \\
$quadratic\_adjustment = \begin{cases}
1 + poly\_coeff \cdot (resource\_limitation - threshold)^2 & \text{if } resource\_limitation > threshold \\
1 & \text{otherwise}
\end{cases}$ \\
$outbreak\_factor = logistic\_factor \cdot quadratic\_adjustment$ \\
$temperature\_factor = 1 + effect\_sst \cdot sst\_dat(t-1)$ \\
$cots\_pred[t] = cots\_pred[t-1] + (growth\_rate \cdot cots\_pred[t-1] \cdot outbreak\_factor \cdot temperature\_factor - decline\_rate \cdot cots\_pred[t-1]) \cdot dt$ \\
\\
\textbf{Coral dynamics:} \\
$fast\_pred[t] = fast\_pred[t-1] + dt \cdot (fast\_growth\_rate \cdot fast\_pred[t-1] \cdot (1 - fast\_pred[t-1] / fast\_cap) - efficiency\_fast \cdot cots\_pred[t-1] \cdot fast\_pred[t-1])$ \\
$mod\_eff\_slow = efficiency\_slow \cdot (1 + temp\_mod\_eff\_slow \cdot sst\_dat(t-1))$ \\
$slow\_pred[t] = slow\_pred[t-1] + dt \cdot (slow\_growth\_rate \cdot slow\_pred[t-1] \cdot (1 - slow\_pred[t-1] / slow\_cap) - mod\_eff\_slow \cdot cots\_pred[t-1] \cdot slow\_pred[t-1])$
\end{tabular} \\
 
Claude Sonnet 3.6 &
\begin{tabular}[t]{p{12.5cm}}
\textbf{COTS dynamics:} \\
$temp\_effect = \exp(-0.5 \cdot \frac{(sst\_dat(t-1) - temp\_opt)^2}{temp\_range^2})$ \\
$resource\_limit = \frac{total\_coral}{total\_coral + \epsilon}$ \\
$recruitment = cotsimm\_dat(t-1) \cdot temp\_effect$ \\
$cots\_pred(t) = cots\_pred(t-1) \cdot (1 + r\_cots \cdot resource\_limit \cdot (1 - \frac{cots\_pred(t-1)}{K\_cots})) + recruitment$ \\
\\
\textbf{Coral dynamics:} \\
$coral\_space = \max(0, 1 - \frac{fast\_pred(t-1) + slow\_pred(t-1)}{100 \cdot coral\_limit})$ \\
$fast\_growth = r\_fast \cdot fast\_pred(t-1) \cdot coral\_space$ \\
$fast\_pred\_loss = grazing\_fast \cdot cots\_pred(t-1) \cdot fast\_pred(t-1)$ \\
$fast\_pred(t) = fast\_pred(t-1) + fast\_growth - fast\_pred\_loss$ \\
$slow\_growth = r\_slow \cdot slow\_pred(t-1) \cdot coral\_space$ \\
$slow\_pred\_loss = grazing\_slow \cdot cots\_pred(t-1) \cdot slow\_pred(t-1)$ \\
$slow\_pred(t) = slow\_pred(t-1) + slow\_growth - slow\_pred\_loss$
\end{tabular} \\
 
Claude Sonnet 3.7 &
\begin{tabular}[t]{p{12.5cm}}
\textbf{COTS dynamics:} \\
$temp\_effect = \exp(-0.5 \cdot \frac{(sst - temp\_opt)^2}{temp\_width^2})$ \\
$pred\_fast = \frac{a\_fast \cdot fast\_t0 \cdot cots\_t0}{1.0 + a\_fast \cdot h\_fast \cdot fast\_t0 + a\_slow \cdot h\_slow \cdot slow\_t0 + \epsilon}$ \\
$pred\_slow = \frac{a\_slow \cdot slow\_t0 \cdot cots\_t0}{1.0 + a\_fast \cdot h\_fast \cdot fast\_t0 + a\_slow \cdot h\_slow \cdot slow\_t0 + \epsilon}$ \\
$bleach\_effect = \frac{1.0}{1.0 + \exp(-2.0 \cdot (sst - bleach\_threshold))}$ \\
$cots\_growth = r\_cots \cdot cots\_t0 \cdot (1.0 - \frac{cots\_t0}{K\_cots}) \cdot temp\_effect$ \\
$imm\_term = \frac{imm\_effect \cdot cotsimm}{1.0 + cotsimm + \epsilon}$ \\
$food\_limitation = m\_cots \cdot (1.0 + \frac{1.0}{fast\_t0 + slow\_t0 + \epsilon})$ \\
$cots\_pred(t) = cots\_t0 + cots\_growth - food\_limitation \cdot cots\_t0 + imm\_term$ \\
\\
\textbf{Coral dynamics:} \\
$fast\_growth = r\_fast \cdot fast\_t0 \cdot (1.0 - \frac{fast\_t0 + competition \cdot slow\_t0}{K\_fast})$ \\
$fast\_bleaching = bleach\_mortality\_fast \cdot bleach\_effect \cdot fast\_t0$ \\
$fast\_pred(t) = fast\_t0 + fast\_growth - pred\_fast - fast\_bleaching$ \\
$slow\_growth = r\_slow \cdot slow\_t0 \cdot (1.0 - \frac{slow\_t0 + competition \cdot fast\_t0}{K\_slow})$ \\
$slow\_bleaching = bleach\_mortality\_slow \cdot bleach\_effect \cdot slow\_t0$ \\
$slow\_pred(t) = slow\_t0 + slow\_growth - pred\_slow - slow\_bleaching$
\end{tabular} \\
 
o4 mini &
\begin{tabular}[t]{p{12.5cm}}
\textbf{COTS dynamics:} \\
$coral\_availability = \frac{fast\_pred[t-1] + slow\_pred[t-1]}{200}$ \\
$resource\_factor = \frac{coral\_availability + coral\_saturation\_coefficient \cdot coral\_availability^2}{0.5 + coral\_availability + coral\_saturation\_coefficient \cdot coral\_availability^2}$ \\
$growth = growth\_rate\_cots \cdot cots\_pred[t-1] \cdot (1 - \frac{cots\_pred[t-1]}{carrying\_capacity + \epsilon}) \cdot (1 + resource\_limitation\_strength \cdot (resource\_factor - 0.5))$ \\
$effective\_sharpness = outbreak\_sharpness \cdot environmental\_modifier \cdot (1 + extreme\_outbreak\_modifier \cdot (environmental\_modifier - 1))$ \\
$raw\_trigger = \frac{1}{1 + \exp(- effective\_sharpness \cdot (cots\_pred[t-1]^{outbreak\_shape} + outbreak\_nonlinearity \cdot cots\_pred[t-1]^2 - (outbreak\_threshold \cdot carrying\_capacity)^{outbreak\_shape}))}$ \\
$outbreak\_trigger = raw\_trigger + outbreak\_hysteresis \cdot raw\_trigger \cdot (1 - raw\_trigger)$ \\
$decline = decay\_rate\_cots \cdot cots\_pred[t-1]^{outbreak\_decline\_exponent} \cdot outbreak\_trigger$ \\
$cots\_pred[t] = cots\_pred[t-1] + growth - decline$ \\
\\
\textbf{Coral dynamics:} \\
$fast\_pred[t] = fast\_pred[t-1] + 0.1 \cdot coral\_recovery\_modifier \cdot coral\_recovery\_environmental\_modifier \cdot (100 - fast\_pred[t-1]) \cdot (1 - coral\_recovery\_inhibition \cdot \frac{cots\_pred[t-1]}{carrying\_capacity + \epsilon}) - \frac{cots\_pred[t-1] \cdot coral\_predation\_efficiency \cdot fast\_pred[t-1] \cdot (\frac{fast\_pred[t-1]}{fast\_pred[t-1] + predation\_scaler})^{predation\_efficiency\_exponent}}{1 + handling\_time \cdot fast\_pred[t-1]}$ \\
$slow\_pred[t] = slow\_pred[t-1] + 0.05 \cdot coral\_recovery\_environmental\_modifier \cdot (100 - slow\_pred[t-1]) \cdot (1 - coral\_recovery\_inhibition \cdot \frac{cots\_pred[t-1]}{carrying\_capacity + \epsilon}) - \frac{cots\_pred[t-1] \cdot coral\_predation\_efficiency \cdot slow\_pred[t-1] \cdot (\frac{slow\_pred[t-1]}{slow\_pred[t-1] + predation\_scaler})^{predation\_efficiency\_exponent}}{1 + handling\_time \cdot slow\_pred[t-1]}$
\end{tabular} \\

gpt 4.1 &
\begin{tabular}[t]{p{12.5cm}}
\textbf{COTS dynamics:} \\
$coral\_effect = \frac{fast\_prev \cdot e\_fast + slow\_prev \cdot e\_slow}{K\_fast \cdot e\_fast + K\_slow \cdot e\_slow + \epsilon}$ \\
$sst\_effect = 1.0 + theta\_sst \cdot (sst\_dat(t) - 27.0)$ \\
$immig\_effect = immig\_scale \cdot cotsimm\_dat(t)$ \\
$outbreak\_boost = 1.0 + phi\_outbreak \cdot (coral\_effect - 0.5)$ \\
$cots\_growth = r\_cots \cdot cots\_prev \cdot (1.0 - \frac{cots\_prev}{K\_cots + \epsilon}) \cdot coral\_effect \cdot sst\_effect \cdot outbreak\_boost$ \\
$cots\_mortality = m\_cots \cdot cots\_prev$ \\
$cots\_next = cots\_prev + cots\_growth - cots\_mortality + immig\_effect$ \\
\\
\textbf{Coral dynamics:} \\
$pred\_fast = \frac{\alpha\_fast \cdot cots\_prev \cdot fast\_prev}{fast\_prev + slow\_prev + \epsilon}$ \\
$pred\_slow = \frac{\alpha\_slow \cdot cots\_prev \cdot slow\_prev}{fast\_prev + slow\_prev + \epsilon}$ \\
$fast\_growth = r\_fast \cdot fast\_prev \cdot (1.0 - \frac{fast\_prev}{K\_fast + \epsilon})$ \\
$fast\_mortality = m\_fast \cdot fast\_prev$ \\
$fast\_next = fast\_prev + fast\_growth - pred\_fast - fast\_mortality$ \\
$slow\_growth = r\_slow \cdot slow\_prev \cdot (1.0 - \frac{slow\_prev}{K\_slow + \epsilon})$ \\
$slow\_mortality = m\_slow \cdot slow\_prev$ \\
$slow\_next = slow\_prev + slow\_growth - pred\_slow - slow\_mortality$
\end{tabular}
\end{longtable}

\subsection{Detailed Ecological Mechanisms}
\label{subsec:ecological_mechanisms}

The models employ distinctly different approaches to represent key ecological processes:

\subsubsection{Temperature Dependency}
\label{subsubsec:temperature_dependency}

\paragraph{Human Model:} Implements temperature effects through two distinct mechanisms: (1) Gaussian functions modifying coral growth rates and (2) a logistic bleaching mortality function with explicit temperature thresholds (M\_SST50\_f, M\_SST50\_m).

\paragraph{o3 mini:} Employs an asymmetric Gaussian temperature response with a skew parameter, allowing for non-symmetric responses to temperature deviations.

\paragraph{Claude Sonnet 3.6:} Uses a standard Gaussian temperature effect (temp\_opt = 28°C) similar to the human model but without the explicit bleaching threshold.

\paragraph{Claude Sonnet 3.7:} Implements a Gaussian temperature response with temp\_opt = 28°C and temp\_width = 2°C, affecting COTS recruitment. Also includes a bleaching effect with a threshold of 30°C.

\paragraph{o4 mini:} Does not include explicit temperature dependency for COTS in its core equations, focusing instead on resource limitation and outbreak dynamics.

\paragraph{gpt 4.1:} Implements a linear temperature effect where SST modifies growth (centered at 27°C) through the parameter theta\_sst.

\subsubsection{Predation Formulations}
\label{subsubsec:predation_formulations}

\paragraph{Human Model:} Features an explicit prey-switching function where COTS predation preference between fast and slow corals depends on the relative abundance of fast-growing coral, with separate parameters for predation intensity (p1f, p1m) and density-dependence (p2f, p2m).

\paragraph{o3 mini:} Implements direct predation with efficiency factors of 0.4 for fast coral and 0.6 for slow coral, with temperature modifying the predation efficiency on slow coral.

\paragraph{Claude Sonnet 3.6:} Uses a simple Type II functional response with grazing rates of 0.1 for fast coral and 0.05 for slow coral, creating a saturating predation effect at high prey densities.

\paragraph{Claude Sonnet 3.7:} Employs a Holling Type II functional response with attack rates of 0.2 for fast coral and 0.05 for slow coral, with handling times for both coral types.

\paragraph{o4 mini:} Implements a Type III functional response with predation efficiency of 0.05, creating a sigmoidal functional response that reduces predation at low prey densities.

\paragraph{gpt 4.1:} Uses a Type II functional response with attack rates of 0.14 for fast coral and 0.08 for slow coral, with predation partitioned by coral type.

\subsubsection{Population Structure}
\label{subsubsec:population_structure}

\paragraph{Human Model:} Implements an age-structured COTS population with explicit age classes (age-0, age-1, and age-2+), each with age-dependent mortality rates, and uses a Beverton-Holt stock-recruitment relationship.

\paragraph{AI Models:} Generally employ simpler, unstructured population approaches with single-state variables for COTS abundance. The models use various forms of logistic growth (o3 mini, Claude Sonnet 3.7, o4 mini, gpt 4.1) or temperature-modified reproduction functions (Claude Sonnet 3.6, Claude Sonnet 3.7, gpt 4.1).

\subsection{Comparison with Human Model}
\label{subsec:human_comparison}

The human-generated model provides an important reference point for evaluating the AI-generated models. This expert-developed model incorporates several ecological mechanisms that differ from the AI approaches.

\paragraph{Population structure:}
Unlike the AI models, the human-generated model implements an age-structured COTS population with explicit age classes (age-0, age-1, and age-2+), each with age-dependent mortality rates. This contrasts with the simpler, unstructured population approaches in the AI models, which generally use single-state variables for COTS abundance.

\paragraph{Stock-recruitment relationship:}
The human model employs a Beverton-Holt stock-recruitment relationship for COTS reproduction, with parameters derived from unexploited population characteristics. This mechanistic approach differs from the AI models, which typically use simpler logistic growth or temperature-modified reproduction functions.

\paragraph{Prey switching:}
A distinctive feature of the human model is its explicit prey-switching function, where COTS predation preference between fast and slow corals depends on the relative abundance of fast-growing coral. This creates a dynamic feedback mechanism not fully captured in most AI models, though the o3 mini model implements a somewhat similar approach with its complex feedback mechanisms. The gpt 4.1 model also implements a form of prey partitioning based on relative coral abundance.

\paragraph{Temperature effects:}
The human model implements temperature effects through two distinct mechanisms: (1) Gaussian functions modifying coral growth rates, similar to Claude Sonnet 3.6 and Claude Sonnet 3.7, and (2) a logistic bleaching mortality function with temperature thresholds, which is also implemented in Claude Sonnet 3.7 with its bleach\_threshold parameter of 30°C.

\paragraph{Parameter differences:}
The human model uses different parameterization approaches, including:
\begin{itemize}
\item Direct parameterization of carrying capacity (K = 3000) rather than log-transformed values used in Claude Sonnet 3.6, Claude Sonnet 3.7, and gpt 4.1
\item Separate parameters for predation intensity (p1f = 0.15, p1m = 0.06) and density-dependence (p2f, p2m)
\item Explicit bleaching threshold parameters (M\_SST50\_f, M\_SST50\_m) compared to the single bleach\_threshold in Claude Sonnet 3.7
\item Age-dependent mortality components for COTS (Mcots = 2.3) compared to simpler mortality formulations in the AI models (e.g., m\_cots = 0.4 in Claude Sonnet 3.7, 0.3 in o4 mini, and 0.37 in gpt 4.1)
\item Explicit optimal temperatures for both coral types (SST0\_f = 26°C, SST0\_m = 27°C) which are not specified in the AI models
\end{itemize}

\subsection{Carrying Capacity and Growth Rate Comparison}
\label{subsec:carrying_capacity_comparison}

The models show variation in their parameterization of carrying capacity and growth rates:

\paragraph{COTS carrying capacity:} Values range from 0.61 individuals/m² (gpt 4.1) to 50 individuals/m² (o4 mini), with Claude Sonnet 3.6 at 2.0 and Claude Sonnet 3.7 at 2.5. This order-of-magnitude difference reflects fundamentally different assumptions about ecosystem capacity.

\paragraph{COTS growth rate:} More consistency is observed in growth rates, with values of 0.8 per year (Claude Sonnet 3.6 and Claude Sonnet 3.7), 0.5 per year (o4 mini and gpt 4.1), compared to the Beverton-Holt approach in the human model.

\paragraph{Coral growth rates:} The models show variation in coral growth parameters, with fast coral growth rates ranging from 0.1 per year (o4 mini) to 0.61 per year (gpt 4.1), and slow coral growth rates from 0.05 per year (o4 mini) to 0.37 per year (gpt 4.1).

\paragraph{Coral carrying capacity:} The models use different approaches to coral carrying capacity, from the shared K = 3000 in the human model to separate values for fast and slow coral in Claude Sonnet 3.7 (K\_fast = 50, K\_slow = 30) and gpt 4.1 (K\_fast = 33.1, K\_slow = 20.1).

% %========================================================
% Readable Equations + Parameter Tables for Best Models
%========================================================

\section{Best Performing Models for CoTS Case Study}
\label{sec:best_models_cots}

\subsection{openrouter:openai/gpt-5 Model (CoTS)}
\paragraph{States and data.}
Adults \(C_t\) (\si{ind.m^{-2}}), juveniles \(J_t\) (\si{ind.m^{-2}}), fast coral \(F_t\) (\%), slow coral \(S_t\) (\%), sea-surface temperature \(\mathrm{SST}_t\) (\si{\degreeC}), immigration \(I^{\text{cots}}_t\) (\si{ind.m^{-2}.yr^{-1}}).

\paragraph{Process model (discrete yearly time, using \(t-1\) drivers).}
\begin{align}
f_{\text{Allee}}(C) &= \frac{1}{1 + \exp\!\big[-k_{\text{allee}}(C - C_{\text{allee}})\big]} \\
f_{T,\text{rec}}(\mathrm{SST}) &= \exp\!\big[-\beta_{\text{rec}}(\mathrm{SST} - T_{\text{opt,rec}})^2\big] \\
\text{stock}(C) &= \frac{C^{\phi}}{1 + C/C_{\text{sat,rec}}} \\
f_{\text{food}}(\mathrm{food}) &= 
\begin{cases}
\displaystyle \frac{\mathrm{food}}{K_{\text{food}} + \mathrm{food}}, & \text{if driver available}\\[6pt]
1, & \text{otherwise}
\end{cases} \\[4pt]
\mathrm{Rec}_t &= \alpha_{\text{rec}}\cdot \text{stock}(C_{t-1}) \cdot f_{\text{Allee}}(C_{t-1})\cdot f_{T,\text{rec}}(\mathrm{SST}_{t-1})\cdot f_{\text{food}}(\mathrm{food}_{t-1}) + I^{\text{cots}}_{t-1} \\[4pt]
\mathrm{Mort}^{\text{adult}}_{t} &= (\mu_C + \gamma_C\, C_{t-1})\, C_{t-1} \\
\mathrm{Mat}_t &= m_J\, J_{t-1}, \qquad \mathrm{Mort}^{\text{juv}}_t = \mu_J\, J_{t-1} \\[4pt]
C_t &= \max\!\big\{0,\; C_{t-1} + \mathrm{Mat}_t - \mathrm{Mort}^{\text{adult}}_{t}\big\} \\
J_t &= \max\!\big\{0,\; J_{t-1} + \mathrm{Rec}_t - \mathrm{Mat}_t - \mathrm{Mort}^{\text{juv}}_{t}\big\} \\[4pt]
\text{heat}_t &= \max\!\big\{0,\; \mathrm{SST}_{t-1} - T_{\text{opt,bleach}}\big\}, \quad
\text{growth}_{\text{mod},t} = \exp(-\beta_{\text{bleach}}\cdot \text{heat}_t) \\[4pt]
\text{space}_t &= 1 - \frac{F_{t-1}+S_{t-1}}{K_{\text{tot}}} \\[4pt]
G^{\text{fast}}_t &= r_F\, F_{t-1}\cdot \text{space}_t \cdot \text{growth}_{\text{mod},t}, \qquad
G^{\text{slow}}_t = r_S\, S_{t-1}\cdot \text{space}_t \cdot \text{growth}_{\text{mod},t} \\[4pt]
B^{\text{fast}}_t &= m_{\text{bleach}F}\cdot \text{heat}_t \cdot F_{t-1}, \qquad
B^{\text{slow}}_t = m_{\text{bleach}S}\cdot \text{heat}_t \cdot S_{t-1} \\[4pt]
\text{denom}_t &= 1 + h\big(a_F F_{t-1}^{\eta_F} + a_S S_{t-1}^{\eta_S}\big) \\
\mathrm{Cons}^{\text{fast}}_t &= q_F \frac{a_F F_{t-1}^{\eta_F} C_{t-1}}{\text{denom}_t}, \qquad
\mathrm{Cons}^{\text{slow}}_t = q_S \frac{a_S S_{t-1}^{\eta_S} C_{t-1}}{\text{denom}_t} \\[4pt]
F_t &= \mathrm{clamp}_{[0,100]}\!\big(F_{t-1} + G^{\text{fast}}_t - \mathrm{Cons}^{\text{fast}}_t - B^{\text{fast}}_t\big) \\
S_t &= \mathrm{clamp}_{[0,100]}\!\big(S_{t-1} + G^{\text{slow}}_t - \mathrm{Cons}^{\text{slow}}_t - B^{\text{slow}}_t\big)
\end{align}

\paragraph{Observation model.}
\begin{align}
\log C^{\text{obs}}_t &\sim \mathcal{N}\!\big(\log C_t,\; \sigma_{\text{cots}}^2\big), \quad \text{with Jacobian } +\log C^{\text{obs}}_t \\
\mathrm{logit}\big(F^{\text{obs}}_t\big) &\sim \mathcal{N}\!\big(\mathrm{logit}(F_t),\; \sigma_{\text{fast}}^2\big) \\
\mathrm{logit}\big(S^{\text{obs}}_t\big) &\sim \mathcal{N}\!\big(\mathrm{logit}(S_t),\; \sigma_{\text{slow}}^2\big)
\end{align}

\paragraph{Parameters.}
\begin{longtable}{@{}l l l X l l@{}}
\toprule
\textbf{Symbol} & \textbf{Name} & \textbf{Units} & \textbf{Description} & \textbf{Value} & \textbf{Bounds / Source}\\
\midrule
\endhead
$C_0$ & Initial adult COTS & \si{ind.m^{-2}} & Initial condition at \(t=0\) & 0.1 & [0, 50] (init est.)\\
$J_0$ & Initial juveniles & \si{ind.m^{-2}} & Initial juvenile pool & 0.1 & [0, 50] (init est.)\\
$F_0$ & Initial fast coral & \% & Initial fast coral cover & 30.0 & [0, 100] (init est.)\\
$S_0$ & Initial slow coral & \% & Initial slow coral cover & 30.0 & [0, 100] (init est.)\\
$\alpha_{\text{rec}}$ & Recruitment scaling & \si{ind.m^{-2}.yr^{-1}} & Productivity scaling to juveniles & 1.0 & [0, 10] (init est.)\\
$\phi$ & Fecundity exponent & -- & Density exponent in fecundity & 1.5 & [1, 3] (init est.)\\
$k_{\text{allee}}$ & Allee steepness & \si{m^{2}.ind^{-1}} & Logistic steepness of Allee effect & 2.0 & [0.01, 20] (init est.)\\
$C_{\text{allee}}$ & Allee threshold & \si{ind.m^{-2}} & Density threshold for mating success & 0.2 & [0, 5] (init est.)\\
$C_{\text{sat,rec}}$ & Stock–recruit taper & \si{ind.m^{-2}} & Beverton–Holt scale (taper) & 2.0 & [0.01, 50] (improvement)\\
$\mu_C$ & Adult baseline mort. & \si{yr^{-1}} & Baseline adult mortality & 0.6 & [0, 3] (init est.)\\
$\gamma_C$ & Density mort. coeff. & \si{m^{2}.ind^{-1}.yr^{-1}} & Density-dependent mortality & 0.5 & [0, 10] (init est.)\\
$m_J$ & Maturation fraction & \si{yr^{-1}} & Annual juvenile maturation & 0.5 & [0, 1] (init est.)\\
$\mu_J$ & Juvenile mort. & \si{yr^{-1}} & Juvenile mortality & 0.5 & [0, 1] (init est.)\\
$T_{\text{opt,rec}}$ & Opt. SST (recruit) & \si{\degreeC} & Optimal SST for recruitment & 26.5 & [20, 34] (lit.)\\
$\beta_{\text{rec}}$ & Temp curvature & \si{\degreeC^{-2}} & Gaussian curvature of temp effect & 0.2 & [0, 2] (init est.)\\
$K_{\text{food}}$ & Food half-sat. & (driver units) & Monod half-saturation (if activated) & 1.0 & [0.001, 100] (scaffold)\\
$T_{\text{opt,bleach}}$ & Bleach onset SST & \si{\degreeC} & Threshold for bleaching stress & 32.65 & [31.0, 34.3] (lit.)\\
$\beta_{\text{bleach}}$ & Growth reduction & -- & Growth reduction under heat & 0.5 & [0, 5] (init est.)\\
$m_{\text{bleach}F}$ & Fast bleaching loss & \si{yr^{-1}.\degreeC^{-1}} & Proportional loss per \si{\degreeC} & 0.2 & [0, 2] (init est.)\\
$m_{\text{bleach}S}$ & Slow bleaching loss & \si{yr^{-1}.\degreeC^{-1}} & Proportional loss per \si{\degreeC} & 0.1 & [0, 2] (init est.)\\
$r_F$ & Fast coral regrowth & \si{yr^{-1}} & Intrinsic regrowth (fast) & 0.5 & [0, 2] (lit./init)\\
$r_S$ & Slow coral regrowth & \si{yr^{-1}} & Intrinsic regrowth (slow) & 0.2 & [0, 2] (lit./init)\\
$K_{\text{tot}}$ & Total coral capacity & \% & Carrying capacity (F+S) & 70.0 & [10, 100] (lit./init)\\
$a_F,a_S$ & Attack params & see code & Encounter/attack on corals & 0.02; 0.01 & [0, 1] (init est.)\\
$\eta_F,\eta_S$ & Shape exponents & -- & Type-III blend exponents & 1.5; 1.2 & [1, 3] (init est.)\\
$h$ & Handling scaler & \si{yr.\%^{-1}} & Satiation scaler & 0.02 & [0, 1] (init est.)\\
$q_F,q_S$ & Conversion efficiencies & -- (0–1) & % cover loss per feeding & 0.8; 0.5 & [0, 1] (lit./init)\\
$\sigma_{\text{cots}}$ & COTS log-SD & log-space SD & Observation SD (COTS) & 0.5 & [0.01, 2] (init est.)\\
$\sigma_{\text{fast}}$ & Fast logit-SD & logit SD & Observation SD (fast coral) & 0.3 & [0.01, 2] (init est.)\\
$\sigma_{\text{slow}}$ & Slow logit-SD & logit SD & Observation SD (slow coral) & 0.3 & [0.01, 2] (init est.)\\
\bottomrule
\end{longtable}

\medskip
\noindent\emph{Notes: All equations and parameter values above are taken directly from your file’s model and parameter listings.} 

\subsection{openrouter:anthropic/claude-sonnet-4.5 Model (CoTS)}
\paragraph{Key features.}
Type-II predation on corals, Allee effects, immigration, logistic adult growth, and \textbf{unimodal substrate–food favorability} for recruitment peaking at intermediate total coral cover.

\paragraph{Process model (selected equations).}
\begin{align}
\text{Total coral: } T_{t-1} &= F_{t-1} + S_{t-1} + \varepsilon \\
\text{Temp effect: } f_T(\mathrm{SST}_t) &= \exp\!\left[-\tfrac12\left(\frac{\mathrm{SST}_t - T_{\text{opt}}}{\text{width}}\right)^2\right] \\[4pt]
\text{Substrate–food favorability: } f_{\text{SF}}(T) &= \frac{T}{T_{\text{opt,coral}}}\,\exp\!\left(1 - \frac{T}{T_{\text{opt,coral}}}\right), \quad \text{bounded to }[0,1] \\[4pt]
\text{Composite favorability: } FAV_t &= f_T(\mathrm{SST}_t)\cdot \frac{I^{\text{cots}}_t}{\max I^{\text{cots}}}\cdot f_{\text{SF}}(T_{t-1}) \\[4pt]
\text{Recruit activation: } A_t &= \frac{1}{1+\exp\big[-20\,(FAV_t - \tau)\big]} \\[4pt]
\text{Recruit pulse: } R^{\text{pulse}}_{t} &= R_{\max}\cdot A_t\cdot FAV_t \\[4pt]
\text{Adult growth: } \Delta C_t &= r_C\,C_{t-1}\cdot f_{\text{Allee}}(C_{t-1})\cdot f_T(\mathrm{SST}_t)\cdot (1 - C_{t-1}/K_C)
+ \eta_{\text{conv}}\cdot \text{(total consumption)} \\
\text{Mortality: } M_t &= (\mu_0 + \mu_{\text{dd}} C_{t-1})\cdot \big[1 + 2\exp(-T_{t-1}/5)\big]
\end{align}
\(\text{Type-II consumption on fast/slow corals and coral logistic dynamics with temperature stress are as in the implementation.}\)

\paragraph{Observation model.}
\begin{align}
\log C^{\text{obs}}_t &\sim \mathcal{N}\!\big(\log C_t,\; \sigma_{\text{cots}}^2\big),\qquad
F^{\text{obs}}_t \sim \mathcal{N}(F_t,\; \sigma_{\text{fast}}^2),\qquad
S^{\text{obs}}_t \sim \mathcal{N}(S_t,\; \sigma_{\text{slow}}^2).
\end{align}

\paragraph{Parameters (selected; log-parameters shown on natural meaning).}
\begin{longtable}{@{}l l l X l l@{}}
\toprule
\textbf{Symbol} & \textbf{Name} & \textbf{Units} & \textbf{Description} & \textbf{Value} & \textbf{Bounds / Source}\\
\midrule
\endhead
$r_C$ & Intrinsic COTS growth & \si{yr^{-1}} & Adults’ logistic growth (unlogged) & \(\exp(\text{log\_r\_cots}) \approx 0.5\) & log in file (lit.)\\
$K_C$ & COTS carrying capacity & \si{ind.m^{-2}} & Adults’ carrying capacity & \(\exp(\text{log\_K\_cots}) \approx 5\) & log in file (lit.)\\
$C_{\text{allee}}$ & Allee threshold & \si{ind.m^{-2}} & Half-saturation for Allee & \(\exp(\text{log\_allee\_threshold}) \approx 0.2\) & log in file (lit.)\\
$\mu_0$ & Baseline mortality & \si{yr^{-1}} & Baseline adult mortality & \(\exp(\text{log\_mort\_base}) \approx 0.3\) & log bounds (lit.)\\
$\mu_{\text{dd}}$ & Density mortality coeff. & \si{m^{2}.ind^{-1}.yr^{-1}} & Density-dependent mortality & \(\exp(\text{log\_mort\_density}) \approx 0.2\) & log bounds (lit.)\\
$T_{\text{opt}}$ & Opt. recruitment SST & \si{\degreeC} & Optimal temperature & \(\exp(\text{log\_temp\_opt}) \approx 28\) & log bounds (lit.)\\
$\text{width}$ & Temp tolerance & \si{\degreeC} & Gaussian width & \(\exp(\text{log\_temp\_width}) \approx 2\) & (lit.)\\
$R_{\max}$ & Max pulse recruitment & \si{ind.m^{-2}.yr^{-1}} & Peak pulse flux & \(\exp(\text{log\_recruit\_max}) \approx 1\) & (init est.)\\
$\tau$ & Pulse threshold & -- & Favorability threshold & 0.6 & [0.4, 0.8] (init est.)\\
$T_{\text{opt,coral}}$ & Opt. total coral & \% & Peak of unimodal substrate–food & \(\exp(\text{log\_optimal\_coral}) \approx 30\) & [12–50\%] (lit.)\\
\multicolumn{6}{l}{\emph{Predation, handling and observation SD parameters as listed in the file (all log-transformed where applicable).}}\\
\bottomrule
\end{longtable}

\subsection{openrouter:google/gemini-2.5-pro Model (CoTS)}
\paragraph{Process and observation (compact).}
\begin{align}
\text{Predation (Type II):}\quad
\mathrm{Cons}^{\text{fast}}_t &= \frac{a_F\,F_{t-1}\,C_{t-1}}{1 + a_F h F_{t-1} + \varepsilon},\quad
\mathrm{Cons}^{\text{slow}}_t = \frac{a_S\,S_{t-1}\,C_{t-1}}{1 + a_S h S_{t-1} + \varepsilon} \\[4pt]
\text{Bleaching mortality:}\quad
\mathrm{Bleach}_X &= \frac{m_X}{1 + \exp[-k_{\text{bleach}}(\mathrm{SST}_t - T_{\text{bleach},X})]}\cdot X_{t-1},\quad X\in\{F,S\} \\[4pt]
\text{Corals:}\quad F_t &= F_{t-1} + r_F F_{t-1}\Big(1 - \frac{F_{t-1}+S_{t-1}}{K_{\text{coral}}}\Big) - \mathrm{Cons}^{\text{fast}}_t - \mathrm{Bleach}_F \\
S_t &= S_{t-1} + r_S S_{t-1}\Big(1 - \frac{F_{t-1}+S_{t-1}}{K_{\text{coral}}}\Big) - \mathrm{Cons}^{\text{slow}}_t - \mathrm{Bleach}_S \\[4pt]
\text{Temperature-modulated reproduction:}\quad
f_T(\mathrm{SST}_t) &= \exp\!\Big[-\frac{(\mathrm{SST}_t - T_{\text{opt,cots}})^2}{2T_{\sigma,\text{cots}}^2}\Big] \\[4pt]
\text{COTS:}\quad
C_t &= C_{t-1} + \Big[e_F\,\mathrm{Cons}^{\text{fast}}_t + e_S\,\mathrm{Cons}^{\text{slow}}_t\Big]\cdot \frac{C_{t-1}}{C_{t-1}+A_{\text{allee}}}\cdot f_T(\mathrm{SST}_t) \\
&\quad - \Big(\underbrace{\frac{m_{\max}C_{t-1}}{1 + C_{t-1}/A_{\text{mort,s}}}}_{\text{survival Allee mort.}} + m_{\text{dd}} C_{t-1}^2\Big) + I^{\text{cots}}_t
\end{align}
Observations are lognormal for COTS and normal for coral cover, as implemented.

\paragraph{Parameter table (selected).}
\begin{longtable}{@{}l l l X l l@{}}
\toprule
Symbol & Name & Units & Description & Value & Bounds/Source\\
\midrule
\endhead
$T_{\text{opt,cots}}$ & Opt. temp (COTS repro) & \si{\degreeC} & Gaussian peak temperature & 28.0 & [25,32] (init)\\
$T_{\sigma,\text{cots}}$ & Thermal width & \si{\degreeC} & Gaussian SD & 2.0 & [0.5,5] (init)\\
$r_F,r_S$ & Coral growth rates & \si{yr^{-1}} & Intrinsic growth & \(\exp(\text{log\_r\_fast}), \exp(\text{log\_r\_slow})\) & (lit.)\\
$K_{\text{coral}}$ & Coral capacity & \% & Total coral carrying capacity & \(\exp(\text{log\_K\_coral})\) & (lit.)\\
$a_F,a_S,h$ & Attack/handling & see code & Type II functional response & logs in file & (lit./init)\\
$e_F,e_S$ & Conversion eff. & -- & Coral-to-COTS conversion & logs in file & (lit./init)\\
\bottomrule
\end{longtable}

\clearpage
\section{Best Performing Models for NPZ Case Study}
\label{sec:best_models_npz}

\subsection{openrouter:openai/gpt-5 Model (NPZ)}
\paragraph{States.} Nutrient \(N\), phytoplankton \(P\), zooplankton \(Z\), detritus \(D\) (all \(\si{g\ C\ m^{-3}}\)); time in days.

\paragraph{Limitation and rates.}
\begin{align}
f_T &= q10^{(T_C - T_{\text{ref}})/10},\quad
f_I = \frac{I}{K_I + I},\quad
f_N = \frac{N}{K_N + N} \\
\mu &= \mu_{\max} f_T f_I f_N,\qquad
g = g_{\max} f_T \frac{P^h}{K_G^h + P^h}
\end{align}

\paragraph{Fluxes (per day).}
\begin{align}
U &= \mu P,& G &= g Z,& Z_g &= e_Z G,& R_{g\to D} &= (1-e_Z)G\\
R_{P\to N} &= r_P m_P P,& R_{P\to D} &= (1-r_P)m_P P\\
R_{Z\to N} &= r_Z m_Z Z,& R_{Z\to D} &= (1-r_Z)m_Z Z\\
\mathrm{Ex} &= e_x Z,& \mathrm{Mx} &= k_{\text{mix}}(N_\star - N),\\
\mathrm{Rem} &= k_{\text{rem}} D,& \mathrm{Snk} &= k_{\text{sink}} D
\end{align}

\paragraph{Dynamics (Euler forward).}
\begin{align}
\frac{dN}{dt} &= -U + R_{P\to N} + R_{Z\to N} + \mathrm{Ex} + \mathrm{Mx} + \mathrm{Rem} \\
\frac{dP}{dt} &= U - G - m_P P \\
\frac{dZ}{dt} &= Z_g - m_Z Z - \gamma_Z Z^2 \\
\frac{dD}{dt} &= R_{g\to D} + R_{P\to D} + R_{Z\to D} - \mathrm{Rem} - \mathrm{Snk}
\end{align}

\paragraph{Observation model (lognormal).}
\begin{align}
\log N^{\text{obs}}_i &\sim \mathcal{N}\big(\log N_i,\; \sigma_N^2\big),\quad
\log P^{\text{obs}}_i \sim \mathcal{N}\big(\log P_i,\; \sigma_P^2\big),\quad
\log Z^{\text{obs}}_i \sim \mathcal{N}\big(\log Z_i,\; \sigma_Z^2\big).
\end{align}

\paragraph{Parameters.}
\begin{longtable}{@{}l l l X l l@{}}
\toprule
Symbol & Name & Units & Description & Value & Bounds / Source\\
\midrule
\endhead
$\log\mu_{\max}$ & Max phyto growth (log) & \si{d^{-1}} (log) & At reference conditions & $-0.0204$ & [\(-0.2231, 0.1823\)] (lit.)\\
$\log K_N$ & Nutrient half-sat (log) & \si{g\ C\ m^{-3}} (log) & Uptake half-saturation & $-2.9957$ & [\(-6.9078, 0\)]\\
$I$ & Irradiance proxy & \si{W m^{-2}} & Mean photosynthetic light & 150 & [0, 500] (init)\\
$\log K_I$ & Light half-sat (log) & \si{W m^{-2}} (log) & Light saturation & 4.3175 & [0, 5.7038]\\
$\log g_{\max}$ & Max grazing (log) & \si{d^{-1}} (log) & Per Z biomass & $-0.6931$ & [\(-2.9957, 0.6931\)]\\
$\log K_G$ & Grazing half-sat (log) & \si{g\ C\ m^{-3}} (log) & Holling III half-sat & $-2.3026$ & [\(-6.9078, 0\)]\\
$h$ & Holling exponent & -- & Type III shape & 2.0 & [1, 3] (lit.)\\
$\mathrm{logit}\,e_Z$ & Assimilation (logit) & -- & Zooplankton assimilation & 0 & \(e_Z=0.5\) (lit.)\\
$\log m_P$ & Phyto mortality (log) & \si{d^{-1}} (log) & Linear mortality & $-2.9957$ & [\(-6.9078, -1.2040\)]\\
$\log m_Z$ & Zoo mortality (log) & \si{d^{-1}} (log) & Linear mortality & $-3.5066$ & [\(-6.9078, -1.2040\)]\\
$\log\gamma_Z$ & Zoo self-limit (log) & \((\si{g\ C\ m^{-3}})^{-1}\si{d^{-1}}\) (log) & Quadratic limitation & $-4.6052$ & [\(-9.2103, -1.6094\)]\\
$\mathrm{logit}\,r_P$ & P mort.\,\(\to N\) frac (logit) & -- & Split to nutrient & 0.8473 & (lit.)\\
$\mathrm{logit}\,r_Z$ & Z mort.\,\(\to N\) frac (logit) & -- & Split to nutrient & 0.8473 & (lit.)\\
$\log e_x$ & Excretion (log) & \si{d^{-1}} (log) & Z excretion to N & $-4.6052$ & [\(-13.8155, -1.6094\)]\\
$\log k_{\text{mix}}$ & Mixing rate (log) & \si{d^{-1}} (log) & Nutrient supply & $-3.9120$ & [\(-13.8155, -0.6931\)]\\
$N_\star$ & Deep/source nutrient & \si{g\ C\ m^{-3}} & Mixing target & 0.3 & [0, 2] (init)\\
$\log q10$ & Q10 (log) & -- & Temperature coefficient & 0.659 & [0.606, 0.711] (lit.)\\
$T_C,T_{\text{ref}}$ & Ambient/reference T & \si{\degreeC} & For Q10 scaling & 15, 15 & [0, 35] (init/lit.)\\
$\log k_{\text{rem}}$ & Remineralization (log) & \si{d^{-1}} (log) & Detritus \(\to N\) & $-2.3026$ & [\(-4.6052, 0\)] (added)\\
$\log k_{\text{sink}}$ & Sinking (log) & \si{d^{-1}} (log) & D export & $-4.6052$ & [\(-13.8155, 0\)] (added)\\
$\log\sigma_{N,P,Z}$ & Obs.\ SDs (log) & log-SD & Lognormal errors & $-2.3026$ each & [\(-5, 2\)] (init)\\
\bottomrule
\end{longtable}

\subsection{openrouter:anthropic/claude-sonnet-4.5 Model (NPZ)}
\paragraph{Core processes with variable assimilation efficiency.}
\begin{align}
\text{Seasonal surface light: } I_0(t) &= I_{0,\text{mean}}\Big[1 + A_{I}\sin\!\Big(\frac{2\pi (t - t_{\text{phase}})}{365}\Big)\Big] \\[2pt]
\text{Depth-avg light: } I_{\text{avg}} &= \frac{I_0(t)\,\big(1-e^{-(k_w + k_c P)z_{\text{mix}}}\big)}{(k_w + k_c P)z_{\text{mix}}} \\
\text{Limitations: } \ell_I &= \frac{I_{\text{avg}}}{K_I + I_{\text{avg}}},\quad \ell_N = \frac{N}{K_N + N} \\
\text{Q10 factors: } f_{\text{phy}} &= Q10_{\text{phyto}}^{(T - T_{\text{ref}})/10},\quad
f_{\text{zoo}} = Q10_{\text{zoo}}^{(T - T_{\text{ref}})/10},\quad
f_{\text{rem}} = Q10_{\text{rem}}^{(T - T_{\text{ref}})/10} \\[4pt]
\text{Phyto growth: } G_P &= r_{\max}\, f_{\text{phy}}\, \ell_N\, \ell_I\, P \\
\text{Grazing (Type II): } \mathrm{Graze} &= g_{\max}\, f_{\text{zoo}}\, \frac{P}{K_P + P}\, Z \\
\text{Variable assimilation: } \varepsilon_{\text{eff}} &= \varepsilon_{\min} + (\varepsilon_{\max} - \varepsilon_{\min})\,\ell_N \\[4pt]
\frac{dN}{dt} &= N_{\text{input}} - G_P + \gamma_P f_{\text{rem}}\, m_P P + \gamma_Z f_{\text{rem}}\, m_Z Z + (1-\varepsilon_{\text{eff}})\,\mathrm{Graze} \\
\frac{dP}{dt} &= G_P - m_P f_{\text{rem}}\, P - \mathrm{Graze} \\
\frac{dZ}{dt} &= \varepsilon_{\text{eff}}\, \mathrm{Graze} - \big(m_Z f_{\text{zoo}} Z + m_{Z2} Z^2\big)
\end{align}

\paragraph{Observation model.}
Normal errors on \(N,P,Z\) as specified (see file).

\paragraph{Parameter table (selected).}
\begin{longtable}{@{}l l l X l l@{}}
\toprule
Symbol & Name & Units & Description & Value & Bounds / Source\\
\midrule
\endhead
$r_{\max}$ & Max phyto growth & \si{d^{-1}} & Growth at reference \(T\) & 1.0 & [0.8, 1.2] (lit.)\\
$K_N$ & Nutrient half-sat & \si{g\ C\ m^{-3}} & Uptake saturation & 0.1 & [0, 5] (lit.)\\
$m_P$ & Phyto mortality & \si{d^{-1}} & Non-grazing losses & 0.05 & [0, 1] (lit.)\\
$g_{\max}$ & Max grazing & \si{d^{-1}} & Zoo grazing at ref \(T\) & 0.5 & [0, 5] (lit.)\\
$K_P$ & Grazing half-sat & \si{g\ C\ m^{-3}} & Holling II half-sat & 0.1 & [0, 5] (lit.)\\
\(\varepsilon_{\min},\varepsilon_{\max}\) & Assimilation bounds & -- & Food-quality dependent & 0.15; 0.45 & [0.05–0.40; 0.20–0.70] (lit.)\\
$m_Z,m_{Z2}$ & Zoo mort. (lin/quad) & \si{d^{-1}}, \si{d^{-1}(g\ C\ m^{-3})^{-1}} & Linear + quadratic & 0.05; 0.1 & (lit.)\\
$\gamma_P,\gamma_Z$ & Recycling eff. & -- & Fractions to \(N\) & 0.5; 0.3 & (lit.)\\
$N_{\text{input}}$ & External input & \si{g\ C\ m^{-3}\ d^{-1}} & Upwelling/mixing/depo & 0.01 & (init)\\
$I_{0,\text{mean}},A_{I},t_{\text{phase}}$ & Light params & see units & Seasonal PAR & 443; 0.4; 172 & (lit.)\\
$K_I,k_w,k_c,z_{\text{mix}}$ & Light limitation & see units & Half-sat \& attenuation & 50; 0.04; 0.03; 20 & (lit.)\\
$Q10_{\text{phyto}},Q10_{\text{zoo}},Q10_{\text{rem}}$ & Temperature Q10 & -- & Q10 coefficients & 2.7; 2.5; 3.0 & (lit.)\\
$T_{\text{ref}}$ & Reference temperature & \si{\degreeC} & Q10 reference & 15 & (lit.)\\
\bottomrule
\end{longtable}

\newpage
\section*{COTS--Coral Model: Parameter and Equation Tables}

\subsection*{Implementation details (optimization stability)}
\begin{itemize}
  \item \textbf{Soft bounds}: Parameter ranges are enforced via a smooth quadratic penalty with weight $w_{\text{pen}}=10^{-3}$ (not hard constraints).
  \item \textbf{Non-negativity and \% clamping}: States use a smooth positive-part function for $x_+ \approx \max(0,x)$; coral \% is smoothly clamped to $[0,100]$.
  \item \textbf{Logit transform safety}: The logit of \% cover uses a small $\epsilon$ to avoid 0/100 singularities.
  \item \textbf{Likelihood SD floors}: Observation SDs use a floor ($\ge 0.05$) for numerical stability.
  \item \textbf{Food term default}: If no external driver is provided, $f_{\text{food}}=1$ (neutral), leaving recruitment unaffected by food.
\end{itemize}

\begin{landscape}
\subsection*{Parameter summary}
\begin{table}[ht]
\centering
\scriptsize
\setlength{\tabcolsep}{4pt}
\begin{tabularx}{1.0\linewidth}{l p{3cm} p{7cm} c l l l}
\toprule
Symbol & Units & Meaning & Init. value & Bounds & Source & Citation \\
C0 & ind m$^{-2}$ & Initial adult COTS density at first time step & 0.1 & [0.0, 50.0] & initial estimate & No \\
J0 & ind m$^{-2}$ & Initial juvenile COTS pool at first time step & 0.1 & [0.0, 50.0] & initial estimate & No \\
F0 & \% cover & Initial fast coral (Acropora) cover at first time step & 30.0 & [0.0, 100.0] & initial estimate & No \\
S0 & \% cover & Initial slow coral (Faviidae/Porites) cover at first time step & 30.0 & [0.0, 100.0] & initial estimate & No \\
alpha\_rec & ind m$^{-2}$ yr$^{-1}$ & Recruitment productivity scaling controlling outbreak potential & 1.0 & [0.0, 10.0] & initial estimate & No \\
phi & dimensionless & Fecundity density exponent shaping recruitment curvature & 1.5 & [1.0, 3.0] & initial estimate & No \\
k\_allee & m$^{2}$ ind$^{-1}$ & Steepness of smooth Allee effect on recruitment & 2.0 & [0.01, 20.0] & initial estimate & No \\
C\_allee & ind m$^{-2}$ & Allee density where mating success increases rapidly & 0.2 & [0.0, 5.0] & initial estimate & No \\
C\_sat\_rec & ind m$^{-2}$ & Adult density scale for stock–recruitment taper (Beverton–Holt) & 2.0 & [0.01, 50.0] & improvement & No \\
muC & yr$^{-1}$ & Baseline adult COTS mortality rate & 0.6 & [0.0, 3.0] & initial estimate & No \\
gammaC & m$^{2}$ ind$^{-1}$ yr$^{-1}$ & Density-dependent mortality coefficient & 0.5 & [0.0, 10.0] & initial estimate & No \\
mJ & yr$^{-1}$ & Annual maturation fraction from juvenile to adult & 0.5 & [0.0, 1.0] & initial estimate & No \\
muJ & yr$^{-1}$ & Annual proportional mortality of juvenile COTS & 0.5 & [0.0, 1.0] & initial estimate & No \\
T\_opt\_rec & degC & Optimal SST for COTS recruitment success & 26.5 & [20.0, 34.0] & literature & Yes \citep{Pratchett_Caballes_Wilmes_Matthews_Mellin_Sweatman_Nadler_Brodie_Thompson_Hoey_et_al_2017, morello2014model, Condie_Anthony_Babcock_Baird_Beeden_Fletcher_Gorton_Harrison_Hobday_Plaganyi_et_al_2021} \\
beta\_rec & degC$^{-2}$ & Curvature of Gaussian temperature effect on recruitment & 0.2 & [0.0, 2.0] & initial estimate & No \\
K\_food & units of food\_dat & Half-saturation constant for larval food limitation on recruitment & 1.0 & [0.001, 1000.0] & improvement & No \\
T\_opt\_bleach & degC & SST threshold where bleaching stress starts impacting coral & 32.65 & [20.0, 34.0] & literature & Yes \citep{Rogers_Plaganyi_2022, Condie_Anthony_Babcock_Baird_Beeden_Fletcher_Gorton_Harrison_Hobday_Plaganyi_et_al_2021} \\
beta\_bleach & dimensionless & Multiplier controlling growth reduction under heat stress & 0.5 & [0.0, 5.0] & initial estimate & No \\
m\_bleachF & yr$^{-1}$ degC$^{-1}$ & Additional proportional loss of fast coral per °C above threshold & 0.2 & [0.0, 2.0] & initial estimate & No \\
m\_bleachS & yr$^{-1}$ degC$^{-1}$ & Additional proportional loss of slow coral per °C above threshold & 0.1 & [0.0, 2.0] & initial estimate & No \\
rF & yr$^{-1}$ & Intrinsic regrowth rate of fast coral & 0.5 & [0.0, 2.0] & literature & Yes \citep{Pratchett_Caballes_Wilmes_Matthews_Mellin_Sweatman_Nadler_Brodie_Thompson_Hoey_et_al_2017, Condie_Anthony_Babcock_Baird_Beeden_Fletcher_Gorton_Harrison_Hobday_Plaganyi_et_al_2021} \\
rS & yr$^{-1}$ & Intrinsic regrowth rate of slow coral & 0.2 & [0.0, 2.0] & literature & Yes \citep{Pratchett_Caballes_Wilmes_Matthews_Mellin_Sweatman_Nadler_Brodie_Thompson_Hoey_et_al_2017, Condie_Anthony_Babcock_Baird_Beeden_Fletcher_Gorton_Harrison_Hobday_Plaganyi_et_al_2021} \\
K\_tot & \% cover & Total carrying capacity for combined coral cover & 70.0 & [10.0, 100.0] & literature & Yes \citep{Condie_Anthony_Babcock_Baird_Beeden_Fletcher_Gorton_Harrison_Hobday_Plaganyi_et_al_2021, Rogers_Plaganyi_2022} \\
aF & yr$^{-1}$ \%$^{-\eta_F}$ m$^{2}$ ind$^{-1}$ & Encounter/attack parameter on fast coral & 0.02 & [0.0, 1.0] & initial estimate & No \\
aS & yr$^{-1}$ \%$^{-\eta_S}$ m$^{2}$ ind$^{-1}$ & Encounter/attack parameter on slow coral & 0.01 & [0.0, 1.0] & initial estimate & No \\
etaF & dimensionless & Shape exponent for fast coral & 1.5 & [1.0, 3.0] & initial estimate & No \\
etaS & dimensionless & Shape exponent for slow coral & 1.2 & [1.0, 3.0] & initial estimate & No \\
h & yr \%$^{-1}$ & Handling/satiation scaler controlling saturation & 0.02 & [0.0, 1.0] & initial estimate & No \\
qF & dimensionless & Efficiency converting fast coral feeding into \% cover loss & 0.8 & [0.0, 1.0] & literature & Yes \citep{Rogers_Plaganyi_2022} \\
qS & dimensionless & Efficiency converting slow coral feeding into \% cover loss & 0.5 & [0.0, 1.0] & literature & Yes \citep{Rogers_Plaganyi_2022} \\
sigma\_cots & log-space SD & Observation/process error SD for COTS & 0.5 & [0.01, 2.0] & initial estimate & No \\
sigma\_fast & logit-space SD & Observation/process error SD for fast coral & 0.3 & [0.01, 2.0] & initial estimate & No \\
sigma\_slow & logit-space SD & Observation/process error SD for slow coral & 0.3 & [0.01, 2.0] & initial estimate & No \\
\bottomrule
\end{tabularx}
\end{table}
\end{landscape}

\newpage
\begin{landscape}
\subsection*{Process and observation equations}

\begin{table}[ht]
  \centering
  \scriptsize
  % Tighten intercolumn spacing a bit (optional):
  \setlength{\tabcolsep}{4pt}
  \begin{tabularx}{1.0\linewidth}{l p{10cm} p{10cm}}
    \toprule
    Name & Equation & Description \\
    \midrule
Allee effect & $f_{\text{Allee}}(C)=\frac{1}{1+\exp\{-k_{\text{allee}}(C-C_{\text{allee}})\}}$ & Smooth logistic Allee effect on recruitment \\ 
Recruitment temperature & $f_{T,\text{rec}}(\mathrm{SST})=\exp\{-\beta_{\text{rec}}(\mathrm{SST}-T_{\text{opt,rec}})^2\}$ & Gaussian peak temperature modifier \\ 
Food modifier & $f_{\text{food}}=\frac{\text{food}}{K_{\text{food}}+\text{food}}\;(=1\text{ if no driver})$ & Monod saturation; neutral if driver missing \\ 
Stock term & $s(C)=\frac{C^{\phi}}{1+C/C_{\text{sat,rec}}}$ & Beverton--Holt taper at high adult density \\ 
Juvenile recruitment & $\mathrm{Rec}_{t}=\alpha_{\text{rec}}\,s(C_{t-1})\,f_{\text{Allee}}(C_{t-1})\,f_{T,\text{rec}}(\mathrm{SST}_{t-1})\,f_{\text{food}}+\mathrm{IMM}_{t-1}$ & Recruitment with environmental modifiers and immigration \\ 
Adult mortality & $\mathrm{Mort}_{\text{adult}}=(\mu_{C}+\gamma_{C}C)\,C$ & Baseline + density-dependent adult mortality \\ 
Juvenile flows & $\mathrm{Mat}=m_J J,\;\; \mathrm{Mort}_{\text{juv}}=\mu_J J$ & Maturation and juvenile mortality \\ 
Adult update & $C_t=\max\{0,\; C_{t-1}+\mathrm{Mat}-\mathrm{Mort}_{\text{adult}}\}$ & Nonnegative adult state update \\ 
Juvenile update & $J_t=\max\{0,\; J_{t-1}+\mathrm{Rec}_t-\mathrm{Mat}-\mathrm{Mort}_{\text{juv}}\}$ & Nonnegative juvenile state update \\ 
Heat stress & $\mathrm{heat}=\max\{0,\; \mathrm{SST}-T_{\text{opt,bleach}}\}$ & Bleaching temperature exceedance \\ 
Bleach growth mod & $g_{\text{bleach}}=\exp\{-\beta_{\text{bleach}}\,\mathrm{heat}\}$ & Growth reduction under heat stress \\ 
Space limitation & $\mathrm{space}=1-\frac{F+S}{K_{\text{tot}}}$ & Shared coral space carrying capacity \\ 
Coral growth & $G_F=r_F F\,\mathrm{space}\,g_{\text{bleach}},\;\; G_S=r_S S\,\mathrm{space}\,g_{\text{bleach}}$ & Fast/slow coral intrinsic regrowth \\ 
Bleaching losses & $B_F=m_{\text{bleachF}}\,\mathrm{heat}\,F,\;\; B_S=m_{\text{bleachS}}\,\mathrm{heat}\,S$ & Additional proportional bleaching losses \\ 
Functional response denom & $\mathrm{denom}=1+h( a_F F^{\eta_F}+a_S S^{\eta_S})$ & Type II/III multi-prey denominator \\ 
Feeding losses & $\mathrm{Cons}_F=q_F\frac{a_F F^{\eta_F} C}{\mathrm{denom}},\;\; \mathrm{Cons}_S=q_S\frac{a_S S^{\eta_S} C}{\mathrm{denom}}$ & COTS consumption of fast/slow coral \\ 
Coral updates & $F_t=F+G_F-\mathrm{Cons}_F-B_F,\;\; S_t=S+G_S-\mathrm{Cons}_S-B_S$ & Fast/slow coral cover updates \\ 
Observation: COTS & $\log Y^{(C)}\sim\mathcal{N}(\log C,\, \sigma_{\text{cots}}^2)$ & Lognormal observation with Jacobian in NLL \\ 
Observation: fast & $\operatorname{logit}_{\%}(Y^{(F)})\sim\mathcal{N}(\operatorname{logit}_{\%}(F),\, \sigma_{\text{fast}}^2)$ & Normal on logit-\% cover \\ 
Observation: slow & $\operatorname{logit}_{\%}(Y^{(S)})\sim\mathcal{N}(\operatorname{logit}_{\%}(S),\, \sigma_{\text{slow}}^2)$ & Normal on logit-\% cover \\ 
\bottomrule
\end{tabularx}
\end{table}
\end{landscape}


