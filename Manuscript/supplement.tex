\clearpage
\section*{Supplementary Information: An AI-Driven Framework for Automated Generation of Marine Ecosystem Models}

\setcounter{section}{0}
\renewcommand{\thesection}{S\arabic{section}}

\section{Curated Literature Collection}
\label{subsec:curated_literature}

The local document collection used in this case study was carefully curated to provide comprehensive coverage of marine ecosystem modeling approaches, with particular focus on COTS-coral dynamics and management interventions. The collection encompasses several key research areas:

\begin{itemize}
\item Ecosystem Modeling Frameworks: \cite{Plaganyi_2007} established foundational principles for ecosystem approaches to fisheries, while \cite{Plaganyi_Punt_Hillary_Morello_Thebaud_Hutton_Pillans_Thorson_Fulton_Smith_et_al_2014} introduced Models of Intermediate Complexity for Ecosystem assessments (MICE). \cite{Collie_Botsford_Hastings_Kaplan_Largier_Livingston_Plaganyi_Rose_Wells_Werner_2016} explored optimal model complexity levels.

\item COTS Management and Ecology: \cite{Pratchett_Caballes_Wilmes_Matthews_Mellin_Sweatman_Nadler_Brodie_Thompson_Hoey_et_al_2017} provided a comprehensive thirty-year review of COTS research. \cite{Morello_Plaganyi_Babcock_Sweatman_Hillary_Punt_2014} developed models for COTS outbreak management, while \cite{Rogers_Plaganyi_2022} analyzed corallivore culling impacts under bleaching scenarios.

\item Ecological Regime Shifts: \cite{Blamey_Plaganyi_Branch_2014} investigated predator-driven regime shifts in marine ecosystems. \cite{Plaganyi_Ellis_Blamey_Morello_Norman-Lopez_Robinson_Sporcic_Sweatman_2014} provided insights into ecological tipping points through ecosystem modeling.

\item Management Interventions: \cite{Condie_Anthony_Babcock_Baird_Beeden_Fletcher_Gorton_Harrison_Hobday_Plaganyi_et_al_2021} examined large-scale interventions on the Great Barrier Reef. \cite{Punt_MacCall_Essington_Francis_Hurtado-Ferro_Johnson_Kaplan_Koehn_Levin_Sydeman_2016} explored harvest control implications using MICE models.

\item Model Application Guidelines: \cite{Essington_Plaganyi_2014} provided critical guidelines for adapting ecosystem models to new applications. \cite{Gamble_Link_2009} demonstrated multispecies production model applications for analyzing ecological and fishing effects.

\item Integrated Systems: \cite{Hadley_Wild-Allen_Johnson_Macleod_2015} and \cite{Oca_Cremades_Jimenez_Pintado_Masalo_2019} explored integrated multi-trophic aquaculture modeling, providing insights into coupled biological systems. \cite{Spillias_Cottrell_2024} analyzed trade-offs in seaweed farming between food production, livelihoods, marine biodiversity, and carbon sequestration benefits.
\end{itemize}

These papers were selected based on their direct relevance to COTS population dynamics, coral reef ecology, and ecosystem modeling approaches. The collection provided both specific parameter values and broader ecological context for model development.

\section{RAG Architecture Implementation}
\label{subsec:rag_architecture}

The Retrieval-Augmented Generation (RAG) system facilitates parameter search and extraction from scientific literature. The system employs two primary search strategies: a local search of user-curated documents and a comprehensive web search. For local search, the system uses ChromaDB as a persistent vector store to maintain an indexed collection of scientific papers and technical documents specifically curated by research teams for their ecological systems. These documents are processed into semantic chunks of approximately 512 tokens with small overlaps to preserve context while enabling precise retrieval of relevant information.

The parameter search process begins with the generation of enhanced semantic descriptions for each parameter. These descriptions are crafted to improve search relevance by capturing the ecological and mathematical context in which the parameters are used. The system first searches the user-curated local documents using embeddings generated through Azure OpenAI's embedding service. When necessary, it extends to web-based sources through two channels: querying the Semantic Scholar database for highly-cited papers in biology, mathematics, and environmental science, and conducting broader literature searches through the Serper API to capture additional relevant sources.

The search results from both local and web sources are processed through an LLM to extract numerical values. The system applies consistent validation across both search pathways, identifying minimum and maximum bounds, ensuring unit consistency, and validating source reliability. When direct parameter values are not found in either the local collection or web sources, the system defaults to the initial estimates from the coding LLM. All extracted information, including parameter values, valid ranges, and complete citation details, is stored in a structured JSON database for reproducibility and future reference.

The RAG system implements automatic retry mechanisms when initial searches fail to yield usable results. Each retry attempt follows a structured progression: first accessing the curated local collection through ChromaDB queries, then expanding to Semantic Scholar for peer-reviewed literature, and finally utilizing Serper API for broader scientific content. This progressive broadening of scope, while maintaining focus on ecologically relevant sources, ensures robust parameter estimation even in cases where direct measurements are sparse in the literature.

\section{AI Prompts Used in Model Development}
\label{sec:ai_prompts}

The development of the model relied on several carefully crafted prompts to guide the artificial intelligence system. These prompts were designed to ensure numerical stability, proper likelihood calculation, and clear model structure. The following sections detail the exact prompts used at each stage of model development.

\subsection{Initial Model Creation}
\label{subsec:initial_model_prompt}

The initial model creation utilized a comprehensive prompt that emphasized three key aspects of model development. The prompt used for model initialization was:

\begin{lstlisting}
Please create a Template Model Builder model for the following topic:[PROJECT_TOPIC]. Start by writing intention.txt, in which you provide a concise summary of the ecological functioning of the model. In model.cpp, write your TMB model with the following important considerations:

1. NUMERICAL STABILITY:
- Always use small constants (e.g., Type(1e-8)) to prevent division by zero
- Use smooth transitions instead of hard cutoffs in equations
- Bound parameters within biologically meaningful ranges using smooth penalties rather than hard constraints

2. LIKELIHOOD CALCULATION:
- Always include observations in the likelihood calculation, don't skip any based on conditions
- Use fixed minimum standard deviations to prevent numerical issues when data values are small
- Consider log-transforming data if it spans multiple orders of magnitude
- Use appropriate error distributions (e.g., lognormal for strictly positive data)

3. MODEL STRUCTURE:
- Include comments after each line explaining the parameters (including their units and how to determine their values)
- Provide a numbered list of descriptions for the equations
- Ensure all important variables are included in the reporting section
- Use `_pred' suffix for model predictions corresponding to `_dat' observations
\end{lstlisting}

\subsection{Parameter Enhancement}
\label{subsec:parameter_enhancement_prompt}

To enhance parameter descriptions for improved semantic search capabilities, the following prompt was employed:

\begin{lstlisting}
Given a mathematical model about [PROJECT_TOPIC], enhance the semantic descriptions of these parameters to be more detailed and searchable. The model code shows these parameters are used in the following way:

[MODEL_CONTENT]

For each parameter below, create an enhanced semantic search, no longer than 10 words, that can be used for RAG search or semantic scholar search.
\end{lstlisting}

\subsection{Model Improvement}
\label{subsec:model_improvement_prompt}

For iterative model improvements, the system utilized this prompt:

\begin{lstlisting}
Improve the fit of the following ecological model by modifying the equations in this TMB script. Only make ONE discrete change most likely to improve the fit. Do not add stochasticity, but you may add other ecological relevant factors that may not be present here already.

You may add additional parameters if necessary, and if so, add them to parameters.json. Please concisely describe your ecological improvement in intention.txt and then provide the improved model.cpp and parameters.json content.

\end{lstlisting}

\subsection{Error Handling Prompts}
\label{subsec:error_handling_prompt}

For compilation errors, the system used this prompt:

\begin{lstlisting}
model.cpp failed to compile. Here's the error information:

[ERROR_INFO]

Do not suggest how to compile the script
\end{lstlisting}

For data leakage issues, the system employed this detailed prompt:

\begin{lstlisting}
Data leakage detected in model equations. The following response variables cannot be used to predict themselves:

To fix this:
1. Response variables ([RESPONSE_VARS]) must be predicted using only:
   - External forcing variables ([FORCING_VARS])
   - Other response variables' predictions (_pred variables)
   - Parameters and constants
2. Each response variable must have a corresponding prediction equation
3. Use ecological relationships to determine how variables affect each other

For example, instead of:
  slow_pred(i) = slow * growth_rate;
Use:
  slow_pred(i) = slow_pred(i-1) * growth_rate * (1 - impact_rate * cots_pred(i-1));

Please revise the model equations to avoid using response variables to predict themselves.
\end{lstlisting}

For numerical instabilities, the system used an adaptive prompt that became progressively more focused on simplification after multiple attempts:

\begin{lstlisting}
The model compiled but numerical instabilities occurred. Here's the error information:

[ERROR_INFO]

[After 2+ attempts: Consider making a much simpler model that we can iteratively improve later.]
Do not suggest how to compile the script
\end{lstlisting}

\subsection{NPZ Case Study - Recovering Equations}
\label{subsec:npz_evaluation_prompt}

The model implementation can be compared to the original NPZ equations from \cite{edwards1999zooplankton}:

\begin{align*}
\frac{dN}{dt} &= \underbrace{-\frac{V_m N P}{k_s + N}}_{\text{nutrient uptake}} + \underbrace{\gamma(1-\alpha)\frac{g P^2 Z}{k_g + P^2} + \mu_P P + \mu_Z Z^2}_{\text{recycling}} + \underbrace{S(N_0 - N)}_{\text{mixing}} \\
\frac{dP}{dt} &= \underbrace{\frac{V_m N P}{k_s + N}}_{\text{growth}} - \underbrace{\frac{g P^2 Z}{k_g + P^2} - \mu_P P - S P}_{\text{losses}} \\
\frac{dZ}{dt} &= \underbrace{\alpha\frac{g P^2 Z}{k_g + P^2} - \mu_Z Z^2 - S Z}_{\text{growth and mortality}}
\end{align*}


Our generated model captures several key ecological processes from the original system:
\begin{enumerate}
\item Nutrient uptake by phytoplankton following Michaelis-Menten kinetics
\item Quadratic zooplankton mortality
\item Nutrient recycling through zooplankton excretion
\item Environmental mixing effects
\end{enumerate}

For evaluating the ecological characteristics of generated models against the NPZ reference model, the system used this prompt. The prompt used for all evaluations was:

\begin{lstlisting}
Compare this C++ model against the following criteria that should be present in the NPZ model equation by equation.
The mathematical structure should be identical, even if variable names differ.

For each equation (dN/dt, dP/dt, dZ/dt), check these components:
- nutrient_equation_uptake: In dN/dt: Nutrient uptake by phytoplankton with Michaelis-Menten kinetics (N/(e+N)) and self-shading (a/(b+c*P))
- nutrient_equation_recycling: In dN/dt: Nutrient recycling from zooplankton via predation (beta*lambda*P^2/(mu^2+P^2)*Z) and excretion (gamma*q*Z)
- nutrient_equation_mixing: In dN/dt: Environmental mixing term (k*(N0-N))
- phytoplankton_equation_growth: In dP/dt: Phytoplankton growth through nutrient uptake (N/(e+N))*(a/(b+c*P))*P
- phytoplankton_equation_loss: In dP/dt: Phytoplankton losses through mortality (r*P), predation (lambda*P^2/(mu^2+P^2)*Z), and mixing ((s+k)*P)
- zooplankton_equation: In dZ/dt: Zooplankton growth through predation (alpha*lambda*P^2/(mu^2+P^2)*Z) and mortality (q*Z)

For each characteristic:
1. Score 1.0 if the mathematical structure is equivalent, regardless of:
   - Variable names (e.g., 'N' vs 'nutrients' vs 'N[0]')
   - Programming syntax (e.g., 'pow(x,2)' vs 'x^2')
   - Code organization
2. Verify the mathematical operations are in the same order
3. Confirm all coefficients are present (even if named differently)

Provide your response in JSON format:
{
    "qualitative_description": "Overall description of how well the model matches the criteria",
    "characteristic_scores": {
        "characteristic_name": {
            "score": 0.0 to 1.0,
            "explanation": "How it is implemented"
        }
    }
}
\end{lstlisting}

To validate the scoring approach, we first tested the evaluation system on a model containing the exact equations from the original NPZ model. The system correctly identified and scored all characteristics with perfect accuracy (score of 1.0 for each component), as shown in this test evaluation output:

\begin{lstlisting}
{
    "qualitative_description": "A nutrient-phytoplankton-zooplankton (NPZ) model with detailed interactions representing nutrient uptake, predation, mortality, and environmental mixing",
    "characteristic_scores": {
        "nutrient_equation_uptake": {
            "score": 1.0,
            "explanation": "Exact Michaelis-Menten kinetics with N/(e+N) and self-shading term a/(b+c*P), implemented identically"
        },
        "nutrient_equation_recycling": {
            "score": 1.0,
            "explanation": "Includes both predation-based recycling (beta*lambda*P^2/(mu^2+P^2)*Z) and zooplankton excretion (gamma*q*Z)"
        },
        "nutrient_equation_mixing": {
            "score": 1.0,
            "explanation": "Environmental mixing term k*(N0-N) present with correct implementation"
        },
        "phytoplankton_equation_growth": {
            "score": 1.0,
            "explanation": "Growth term matches exactly: (N/(e+N))*(a/(b+c*P))*P"
        },
        "phytoplankton_equation_loss": {
            "score": 1.0,
            "explanation": "Includes mortality (r*P), predation (lambda*P^2/(mu^2+P^2)*Z), and mixing ((s+k)*P)"
        },
        "zooplankton_equation": {
            "score": 1.0,
            "explanation": "Zooplankton growth through predation (alpha*lambda*P^2/(mu^2+P^2)*Z) and mortality (q*Z)"
        }
    }
}
\end{lstlisting}

This validation test confirmed that the evaluation system could correctly identify and score ecological characteristics when present.

\section{NPZ Validation}
\label{sec:npz_validation}

The NPZ validation study evaluated AIME's ability to recover known ecological relationships from synthetic data. Figure~\ref{fig:ecological_characteristics} shows the relationship between model performance (objective value) and ecological accuracy scores for each characteristic of the NPZ model. The negative correlations across multiple characteristics suggest that improvements in model fit were achieved through discovery of correct ecological mechanisms rather than overfitting.

\begin{figure}[H]
\centering
\includegraphics[width=\textwidth]{../Figures/ecological_characteristics_vs_objective}
\caption{Relationship between ecological accuracy scores and model performance for each NPZ model characteristic. Each panel shows how well models recovered a specific ecological mechanism (score from 0-1) versus their predictive accuracy (objective value). Lower objective values indicate better model fit. Two-sided Pearson's product-moment correlation coefficients (r) and their associated p-values are shown for each characteristic.}
\label{fig:ecological_characteristics}
\end{figure}

\subsection{Best Performing NPZ Model}

\subsubsection{Model Description}
The following model represents our framework's attempt to recover the NPZ dynamics from \cite{edwards1999zooplankton}. The model aims to capture three key components:
\begin{itemize}
\item Nutrient uptake and recycling
\item Phytoplankton growth and mortality
\item Zooplankton predation and dynamics
\end{itemize}

\subsubsection{Model Intention}
\begin{lstlisting}
\section{Ecological Intention}

A key modification was made to incorporate direct nutrient recycling from zooplankton grazing activity. In marine systems, zooplankton feeding is often inefficient, with a significant portion of consumed phytoplankton being released as dissolved nutrients rather than being assimilated into biomass or entering the detritus pool. This "sloppy feeding" process creates an important feedback loop where grazing can stimulate new primary production through rapid nutrient recycling.

The recycling efficiency is temperature-dependent, reflecting how metabolic rates and feeding mechanics vary with temperature. This creates an adaptive feedback where warmer conditions lead to both increased grazing pressure and faster nutrient recycling, better capturing the coupled nature of predator-prey interactions in planktonic systems.

The modification introduces a direct pathway from grazing to dissolved nutrients, complementing the slower recycling through the detritus pool. This better represents the multiple timescales of nutrient cycling in marine food webs and helps explain how high productivity can be maintained even under intense grazing pressure.
\end{lstlisting}

\subsubsection{Model Implementation}
\newpage
\section*{NPZ Model: Parameter and Equation Tables}

\begin{landscape}
\subsection*{Parameter summary}

\begin{table}[ht]
\centering
\scriptsize
\setlength{\tabcolsep}{4pt}
\begin{tabular}{l p{3cm} p{10cm} c l l l}
\toprule
Symbol & Units & Meaning & Init.\ value & Bounds & Source & Literature (citekey) \\
\midrule
log\_mu\_max & day$^{-1}$ (log scale) & Log of maximum phytoplankton growth rate at reference conditions (day$^{-1}$). & -0.02 & [-0.22, 0.18] & literature & Yes (LitNPZ\_log\_mu\_max) \\
log\_K\_N & g C m$^{-3}$ (log scale) & Log of half-saturation constant for nutrient uptake (g C m$^{-3}$). & -3.00 & [-6.91, 0.00] & literature & Yes (LitNPZ\_log\_K\_N) \\
I & W m$^{-2}$ & Mean photosynthetically active irradiance proxy over the modeled period. & 150.00 & [0.00, 500.00] & initial estimate & No \\
log\_K\_I & W m$^{-2}$ (log scale) & Log of light half-saturation constant for photosynthesis (W m$^{-2}$). & 4.32 & [0.00, 5.70] & literature & Yes (LitNPZ\_log\_K\_I) \\
log\_g\_max & day$^{-1}$ (log scale) & Log of maximum zooplankton grazing rate per unit Z biomass (day$^{-1}$). & -0.69 & [-3.00, 0.69] & literature & Yes (LitNPZ\_log\_g\_max) \\
log\_K\_G & g C m$^{-3}$ (log scale) & Log of P half-saturation constant for grazing functional response (g C m$^{-3}$). & -2.30 & [-6.91, 0.00] & literature & Yes (LitNPZ\_log\_K\_G) \\
h\_grazing & dimensionless & Holling type III shape exponent (h $\ge$ 1). & 2.00 & [1.00, 3.00] & literature & Yes (LitNPZ\_h\_grazing) \\
logit\_e\_Z & dimensionless (logit scale) & Logit of zooplankton assimilation efficiency ($e_Z \in (0,1)$); $e_Z = 0.5$ at value 0. & 0.00 & \textemdash & literature & Yes (LitNPZ\_logit\_e\_Z) \\
log\_m\_P & day$^{-1}$ (log scale) & Log of phytoplankton linear mortality rate (day$^{-1}$). & -3.00 & [-6.91, -1.20] & literature & Yes (LitNPZ\_log\_m\_P) \\
log\_m\_Z & day$^{-1}$ (log scale) & Log of zooplankton linear mortality rate (day$^{-1}$). & -3.51 & [-6.91, -1.20] & literature & Yes (LitNPZ\_log\_m\_Z) \\
log\_gamma\_Z & (g C m$^{-3}$)$^{-1}$ day$^{-1}$ (log scale) & Log of zooplankton quadratic self-limitation coefficient ((g C m$^{-3}$)$^{-1}$ day$^{-1}$). & -4.61 & [-9.21, -1.61] & initial estimate & No \\
logit\_r\_P & dimensionless (logit scale) & Logit of fraction of P mortality that is remineralized to N (0..1). & 0.85 & \textemdash & literature & Yes (LitNPZ\_logit\_r\_P) \\
logit\_r\_Z & dimensionless (logit scale) & Logit of fraction of Z mortality that is remineralized to N (0..1). & 0.85 & \textemdash & literature & Yes (LitNPZ\_logit\_r\_Z) \\
log\_ex\_Z & day$^{-1}$ (log scale) & Log of zooplankton excretion rate to nutrients (day$^{-1}$). & -4.61 & [-13.82, -1.61] & initial estimate & No \\
log\_k\_mix & day$^{-1}$ (log scale) & Log of vertical mixing rate driving nutrients toward $N_\star$ (day$^{-1}$). & -3.91 & [-13.82, -0.69] & initial estimate & No \\
$N_\star$ & g C m$^{-3}$ & Deep/source nutrient concentration towards which mixing relaxes the system. & 0.30 & [0.00, 2.00] & initial estimate & No \\
log\_q10 & dimensionless (log scale) & Log of Q10 temperature scaling factor (dimensionless), typical $Q10 \approx 2$. & 0.66 & [0.61, 0.71] & literature & Yes (LitNPZ\_log\_q10) \\
T\_C & deg C & Ambient temperature used for Q10 scaling (deg C). & 15.00 & [0.00, 35.00] & initial estimate & No \\
T\_ref & deg C & Reference temperature for Q10 scaling (deg C). & 15.00 & [0.00, 35.00] & literature & Yes (LitNPZ\_T\_ref) \\
log\_k\_rem & day$^{-1}$ (log scale) & Log of detritus remineralization rate to nutrients (day$^{-1}$). & -2.30 & [-4.61, 0.00] & conceptual addition & No \\
log\_k\_sink & day$^{-1}$ (log scale) & Log of detritus sinking/export rate out of mixed layer (day$^{-1}$). & -4.61 & [-13.82, 0.00] & conceptual addition & No \\
log\_sigma\_N & log-scale SD & Log of observation SD for N on the log scale. & -2.30 & [-5.00, 2.00] & initial estimate & No \\
log\_sigma\_P & log-scale SD & Log of observation SD for P on the log scale. & -2.30 & [-5.00, 2.00] & initial estimate & No \\
log\_sigma\_Z & log-scale SD & Log of observation SD for Z on the log scale. & -2.30 & [-5.00, 2.00] & initial estimate & No \\
\bottomrule
\end{tabular}
\end{table}

\end{landscape}

\section{CoTS Model Convergence}
\label{sec:convergence}
% \begin{figure}[H]
%     \centering
%     \includegraphics[width=0.8\textwidth]{../Figures/founder_to_terminal_evolution.png}
%     \caption{Evolution of model performance across generations. The plot shows the progression of objective values from an initial successful individual to the final individual in the lineage (either due to process termination or culling). Single dots represent individuals which were not improved upon in any subsequent iterations.}
%     \label{fig:evolution}
%     \end{figure}

    
\subsection{Model Evolution and Convergence}

The evolutionary process demonstrated systematic improvement across generations, with clear patterns of model refinement and selection. The mean time to reach best performance was 5.8 generations, with an average improvement frequency of 41.2\% across generations. Figure \ref{fig:status_distribution} illustrates the distribution of successful, culled, and numerically unstable models across generations, with half of all populations (50\%) achieving convergence below the target threshold. 

Generation-by-generation analysis showed varying rates of improvement across populations. The fastest-converging population reached optimal performance in just four generations, while others required up to 10 generations for refinement. The best-performing population demonstrated particularly efficient optimization, achieving an objective value of 0.427 within 5 generations and maintaining consistent improvement with a 75\% improvement frequency across generations.

\begin{figure}[H]
\centering
\includegraphics[width=0.8\textwidth]{../Figures/success_frequency}
\caption{Evolution of model performance during the genetic algorithm optimization process. Each generation represents an iteration of model development, where models are evaluated and classified into three categories: the best performers according to the NMSE objective value (kept, green), those that are numerically stable, but which are outcompeted by the best performers (culled, blue), and those whose scripts threw errors while running, either due to numerical instability, data leakage, or improper TMB syntax (broken, orange). The vertical axis shows the count of new models in each category per generation, while rows represent independent replicates of the optimization process using different language model configurations (columns). Gemini-2.5-Pro is not shown here, but was run unsuccessfully for five generations.}
\label{fig:status_distribution}
\end{figure}
  

\subsection{Numerical Stability and Optimization}

The optimization process demonstrated robust numerical stability characteristics with distinct patterns across LLM configurations. The o3-mini configuration showed efficient optimization with a mean runtime of 40.7 minutes and average generation time of 6.0 minutes (SD = 0.86). In contrast, the Sonnet 3.5 configuration required longer computation times, averaging 99.4 minutes total runtime with 9.9 minutes per generation (SD = 1.33).

The error rates differed across base LLMs, with some requiring more sub-iterations to create a numerically stable and error-free model than others (Figure \ref{fig:iterations_by_llm}). 

\begin{figure}[H]
\centering
\includegraphics[width=0.8\textwidth]{../Figures/iterations_by_llm}
\caption{Distribution of iteration counts for successful model instances by LLM configuration. The boxplot shows the number of iterations required for convergence, excluding cases that reached maximum iterations or remained numerically unstable.}
\label{fig:iterations_by_llm}
\end{figure}


\section{Comparative Analysis of Best-Performing Models}
\label{sec:model_comparison}

Before presenting the full code for each model, we analyze the key differences between the best-performing models to understand their ecological approaches and mathematical structures.


\subsection{Key Parameter Comparison}
\label{subsec:parameter_comparison}

Table \ref{tab:parameter_comparison} presents a detailed comparison of key parameters across the five best-performing models and the human-generated model. These parameters represent fundamental ecological processes and reveal different modelling approaches to COTS-coral dynamics.

\begin{landscape}
\begin{table}[H]
\centering
\begin{footnotesize}
\caption{Comparison of key parameters across best-performing models}
\label{tab:parameter_comparison}
\begin{tabular}{p{3.2cm}p{2.8cm}p{2.8cm}p{2.8cm}p{2.8cm}p{2.8cm}p{2.8cm}}
 
\textbf{Parameter} & \textbf{Human Model} & \textbf{o3 mini} & \textbf{Claude Sonnet 3.6} & \textbf{Claude Sonnet 3.7} & \textbf{o4 mini} & \textbf{gpt 4.1} \\
 
COTS growth rate (yr$^{-1}$) & Beverton-Holt (h=0.5) & exp(log\_growth\_rate) & 0.8 & 0.8 & 0.5 & 0.5 \\
 
COTS mortality (yr$^{-1}$) & Mcots = 2.3 & exp(log\_decline\_rate) & -- & 0.4 & 0.3 & 0.37 \\
 
COTS carrying capacity & Derived from R0=1.0 & -- & 2.0 & 2.5 & 50 & 0.61 \\
 
Slow coral growth (yr$^{-1}$) & rm = 0.1 & 0.1 (fixed) & 0.2 & 0.1 & 0.05 (fixed) & 0.37 \\
 
Fast coral growth (yr$^{-1}$) & rf = 0.5 & 0.2 (fixed) & 0.4 & 0.3 & 0.1 (fixed) & 0.61 \\
 
Coral carrying capacity & K = 3000 (shared) & -- & 0.8 & K\_slow = 30, K\_fast = 50 & -- & K\_slow = 20.1, K\_fast = 33.1 \\
 
Fast coral optimal temp (°C) & SST0\_f = 26 & -- & -- & -- & -- & -- \\
 
Slow coral optimal temp (°C) & SST0\_m = 27 & -- & -- & -- & -- & -- \\
 
COTS optimal temp (°C) & Implicit & -- & 28 & 28 & -- & -- \\
 
Attack rate (fast coral) & p1f = 0.15 & 0.4 & 0.1 & 0.2 & 0.05 & 0.14 \\
 
Attack rate (slow coral) & p1m = 0.06 & 0.6 & 0.05 & 0.05 & 0.05 & 0.08 \\
 
Predation switching & switchSlope = 5 & -- & -- & -- & -- & -- \\
 
Functional response & Sigmoid switching & Logistic with quadratic adjustment & Type II & Type II & Type III & Type II \\
 
\end{tabular}
\end{footnotesize}
\end{table}
\end{landscape}

\subsection{Model Structure Comparison}
\label{subsec:structure_comparison}

Table \ref{tab:equation_comparison} presents a detailed comparison of the key equations used in each model, highlighting the different mathematical approaches to representing COTS-coral dynamics.


\begin{longtable}{p{2cm}p{13cm}}
\caption{Comparison of key equations across models}\label{tab:equation_comparison} \\

\textbf{Model} & \textbf{Key Equations} \\
 
\endfirsthead

\multicolumn{2}{c}%
{{\bfseries \tablename\ \thetable{} -- continued from previous page}} \\

\textbf{Model} & \textbf{Key Equations} \\
 
\endhead

\multicolumn{2}{r}{{Continued on next page}} \\
\endfoot


\endlastfoot
Human Model &
\begin{tabular}[t]{p{12.5cm}}
\textbf{COTS dynamics:} \\
Age-structured model with three age classes (0, 1, 2+) \\
$N(yr+1,1) = N(yr,0) \cdot \exp(-1 \cdot M\_CoTS\_age(0))$ \\
$N(yr+1,2) = N(yr,1) \cdot \exp(-f \cdot M\_CoTS\_age(1)) + N(yr,2) \cdot \exp(-f \cdot M\_CoTS\_age(2))$ \\
$Rcots(yr+1) = \frac{\alpha \cdot (N(yr+1,2)/Kots\_sp)}{\beta + (N(yr+1,2)/Kots\_sp)}$ \\
$N(yr+1,0) = (Rcots(yr+1) + Imm\_CoTS) \cdot \exp(Imm\_res(yr+1) + \sigma_{CoTS}^2/2)$ \\
\\
\textbf{Coral dynamics:} \\
$Cf(yr+1) = Cf(yr) \cdot (1.0 + \rho_{SST\_F} \cdot rf \cdot (1-(Cf(yr) + Cm(yr))/K)) - Qf - M\_ble\_f$ \\
$Cm(yr+1) = Cm(yr) \cdot (1.0 + \rho_{SST\_M} \cdot rm \cdot (1-(Cf(yr) + Cm(yr))/K)) - Qm - M\_ble\_m$ \\
\\
\textbf{Predation:} \\
$\rho = \exp(-switchSlope \cdot Cf(yr)/K)$ \\
$Qf = Cf(yr) \cdot (1.0-\rho) \cdot p1f \cdot \frac{N(yr,1)+N(yr,2)}{1.0+\exp(-(N(yr,1)+N(yr,2))/p2f)}$ \\
$Qm = Cm(yr) \cdot \rho \cdot p1m \cdot \frac{N(yr,1)+N(yr,2)}{1.0+\exp(-(N(yr,1)+N(yr,2))/p2m)}$ \\
\\
\textbf{Temperature effects:} \\
$\rho_{SST\_F} = \exp(-\frac{(SST-SST0\_f)^2}{2 \cdot SST\_sig\_f^2})$ \\
$\rho_{SST\_M} = \exp(-\frac{(SST-SST0\_m)^2}{2 \cdot SST\_sig\_m^2})$ \\
$M\_ble\_f = Cf(yr) \cdot \frac{1.0}{1.0 + \exp(-Eta\_f \cdot (SST-M\_SST50\_f))}$ \\
$M\_ble\_m = Cm(yr) \cdot \frac{1.0}{1.0 + \exp(-Eta\_m \cdot (SST-M\_SST50\_m))}$
\end{tabular} \\
 
o3 mini &
\begin{tabular}[t]{p{12.5cm}}
\textbf{COTS dynamics:} \\
$logistic\_factor = \frac{1}{1 + \exp(-outbreak\_steepness \cdot (resource\_limitation - threshold))}$ \\
$quadratic\_adjustment = \begin{cases}
1 + poly\_coeff \cdot (resource\_limitation - threshold)^2 & \text{if } resource\_limitation > threshold \\
1 & \text{otherwise}
\end{cases}$ \\
$outbreak\_factor = logistic\_factor \cdot quadratic\_adjustment$ \\
$temperature\_factor = 1 + effect\_sst \cdot sst\_dat(t-1)$ \\
$cots\_pred[t] = cots\_pred[t-1] + (growth\_rate \cdot cots\_pred[t-1] \cdot outbreak\_factor \cdot temperature\_factor - decline\_rate \cdot cots\_pred[t-1]) \cdot dt$ \\
\\
\textbf{Coral dynamics:} \\
$fast\_pred[t] = fast\_pred[t-1] + dt \cdot (fast\_growth\_rate \cdot fast\_pred[t-1] \cdot (1 - fast\_pred[t-1] / fast\_cap) - efficiency\_fast \cdot cots\_pred[t-1] \cdot fast\_pred[t-1])$ \\
$mod\_eff\_slow = efficiency\_slow \cdot (1 + temp\_mod\_eff\_slow \cdot sst\_dat(t-1))$ \\
$slow\_pred[t] = slow\_pred[t-1] + dt \cdot (slow\_growth\_rate \cdot slow\_pred[t-1] \cdot (1 - slow\_pred[t-1] / slow\_cap) - mod\_eff\_slow \cdot cots\_pred[t-1] \cdot slow\_pred[t-1])$
\end{tabular} \\
 
Claude Sonnet 3.6 &
\begin{tabular}[t]{p{12.5cm}}
\textbf{COTS dynamics:} \\
$temp\_effect = \exp(-0.5 \cdot \frac{(sst\_dat(t-1) - temp\_opt)^2}{temp\_range^2})$ \\
$resource\_limit = \frac{total\_coral}{total\_coral + \epsilon}$ \\
$recruitment = cotsimm\_dat(t-1) \cdot temp\_effect$ \\
$cots\_pred(t) = cots\_pred(t-1) \cdot (1 + r\_cots \cdot resource\_limit \cdot (1 - \frac{cots\_pred(t-1)}{K\_cots})) + recruitment$ \\
\\
\textbf{Coral dynamics:} \\
$coral\_space = \max(0, 1 - \frac{fast\_pred(t-1) + slow\_pred(t-1)}{100 \cdot coral\_limit})$ \\
$fast\_growth = r\_fast \cdot fast\_pred(t-1) \cdot coral\_space$ \\
$fast\_pred\_loss = grazing\_fast \cdot cots\_pred(t-1) \cdot fast\_pred(t-1)$ \\
$fast\_pred(t) = fast\_pred(t-1) + fast\_growth - fast\_pred\_loss$ \\
$slow\_growth = r\_slow \cdot slow\_pred(t-1) \cdot coral\_space$ \\
$slow\_pred\_loss = grazing\_slow \cdot cots\_pred(t-1) \cdot slow\_pred(t-1)$ \\
$slow\_pred(t) = slow\_pred(t-1) + slow\_growth - slow\_pred\_loss$
\end{tabular} \\
 
Claude Sonnet 3.7 &
\begin{tabular}[t]{p{12.5cm}}
\textbf{COTS dynamics:} \\
$temp\_effect = \exp(-0.5 \cdot \frac{(sst - temp\_opt)^2}{temp\_width^2})$ \\
$pred\_fast = \frac{a\_fast \cdot fast\_t0 \cdot cots\_t0}{1.0 + a\_fast \cdot h\_fast \cdot fast\_t0 + a\_slow \cdot h\_slow \cdot slow\_t0 + \epsilon}$ \\
$pred\_slow = \frac{a\_slow \cdot slow\_t0 \cdot cots\_t0}{1.0 + a\_fast \cdot h\_fast \cdot fast\_t0 + a\_slow \cdot h\_slow \cdot slow\_t0 + \epsilon}$ \\
$bleach\_effect = \frac{1.0}{1.0 + \exp(-2.0 \cdot (sst - bleach\_threshold))}$ \\
$cots\_growth = r\_cots \cdot cots\_t0 \cdot (1.0 - \frac{cots\_t0}{K\_cots}) \cdot temp\_effect$ \\
$imm\_term = \frac{imm\_effect \cdot cotsimm}{1.0 + cotsimm + \epsilon}$ \\
$food\_limitation = m\_cots \cdot (1.0 + \frac{1.0}{fast\_t0 + slow\_t0 + \epsilon})$ \\
$cots\_pred(t) = cots\_t0 + cots\_growth - food\_limitation \cdot cots\_t0 + imm\_term$ \\
\\
\textbf{Coral dynamics:} \\
$fast\_growth = r\_fast \cdot fast\_t0 \cdot (1.0 - \frac{fast\_t0 + competition \cdot slow\_t0}{K\_fast})$ \\
$fast\_bleaching = bleach\_mortality\_fast \cdot bleach\_effect \cdot fast\_t0$ \\
$fast\_pred(t) = fast\_t0 + fast\_growth - pred\_fast - fast\_bleaching$ \\
$slow\_growth = r\_slow \cdot slow\_t0 \cdot (1.0 - \frac{slow\_t0 + competition \cdot fast\_t0}{K\_slow})$ \\
$slow\_bleaching = bleach\_mortality\_slow \cdot bleach\_effect \cdot slow\_t0$ \\
$slow\_pred(t) = slow\_t0 + slow\_growth - pred\_slow - slow\_bleaching$
\end{tabular} \\
 
o4 mini &
\begin{tabular}[t]{p{12.5cm}}
\textbf{COTS dynamics:} \\
$coral\_availability = \frac{fast\_pred[t-1] + slow\_pred[t-1]}{200}$ \\
$resource\_factor = \frac{coral\_availability + coral\_saturation\_coefficient \cdot coral\_availability^2}{0.5 + coral\_availability + coral\_saturation\_coefficient \cdot coral\_availability^2}$ \\
$growth = growth\_rate\_cots \cdot cots\_pred[t-1] \cdot (1 - \frac{cots\_pred[t-1]}{carrying\_capacity + \epsilon}) \cdot (1 + resource\_limitation\_strength \cdot (resource\_factor - 0.5))$ \\
$effective\_sharpness = outbreak\_sharpness \cdot environmental\_modifier \cdot (1 + extreme\_outbreak\_modifier \cdot (environmental\_modifier - 1))$ \\
$raw\_trigger = \frac{1}{1 + \exp(- effective\_sharpness \cdot (cots\_pred[t-1]^{outbreak\_shape} + outbreak\_nonlinearity \cdot cots\_pred[t-1]^2 - (outbreak\_threshold \cdot carrying\_capacity)^{outbreak\_shape}))}$ \\
$outbreak\_trigger = raw\_trigger + outbreak\_hysteresis \cdot raw\_trigger \cdot (1 - raw\_trigger)$ \\
$decline = decay\_rate\_cots \cdot cots\_pred[t-1]^{outbreak\_decline\_exponent} \cdot outbreak\_trigger$ \\
$cots\_pred[t] = cots\_pred[t-1] + growth - decline$ \\
\\
\textbf{Coral dynamics:} \\
$fast\_pred[t] = fast\_pred[t-1] + 0.1 \cdot coral\_recovery\_modifier \cdot coral\_recovery\_environmental\_modifier \cdot (100 - fast\_pred[t-1]) \cdot (1 - coral\_recovery\_inhibition \cdot \frac{cots\_pred[t-1]}{carrying\_capacity + \epsilon}) - \frac{cots\_pred[t-1] \cdot coral\_predation\_efficiency \cdot fast\_pred[t-1] \cdot (\frac{fast\_pred[t-1]}{fast\_pred[t-1] + predation\_scaler})^{predation\_efficiency\_exponent}}{1 + handling\_time \cdot fast\_pred[t-1]}$ \\
$slow\_pred[t] = slow\_pred[t-1] + 0.05 \cdot coral\_recovery\_environmental\_modifier \cdot (100 - slow\_pred[t-1]) \cdot (1 - coral\_recovery\_inhibition \cdot \frac{cots\_pred[t-1]}{carrying\_capacity + \epsilon}) - \frac{cots\_pred[t-1] \cdot coral\_predation\_efficiency \cdot slow\_pred[t-1] \cdot (\frac{slow\_pred[t-1]}{slow\_pred[t-1] + predation\_scaler})^{predation\_efficiency\_exponent}}{1 + handling\_time \cdot slow\_pred[t-1]}$
\end{tabular} \\

gpt 4.1 &
\begin{tabular}[t]{p{12.5cm}}
\textbf{COTS dynamics:} \\
$coral\_effect = \frac{fast\_prev \cdot e\_fast + slow\_prev \cdot e\_slow}{K\_fast \cdot e\_fast + K\_slow \cdot e\_slow + \epsilon}$ \\
$sst\_effect = 1.0 + theta\_sst \cdot (sst\_dat(t) - 27.0)$ \\
$immig\_effect = immig\_scale \cdot cotsimm\_dat(t)$ \\
$outbreak\_boost = 1.0 + phi\_outbreak \cdot (coral\_effect - 0.5)$ \\
$cots\_growth = r\_cots \cdot cots\_prev \cdot (1.0 - \frac{cots\_prev}{K\_cots + \epsilon}) \cdot coral\_effect \cdot sst\_effect \cdot outbreak\_boost$ \\
$cots\_mortality = m\_cots \cdot cots\_prev$ \\
$cots\_next = cots\_prev + cots\_growth - cots\_mortality + immig\_effect$ \\
\\
\textbf{Coral dynamics:} \\
$pred\_fast = \frac{\alpha\_fast \cdot cots\_prev \cdot fast\_prev}{fast\_prev + slow\_prev + \epsilon}$ \\
$pred\_slow = \frac{\alpha\_slow \cdot cots\_prev \cdot slow\_prev}{fast\_prev + slow\_prev + \epsilon}$ \\
$fast\_growth = r\_fast \cdot fast\_prev \cdot (1.0 - \frac{fast\_prev}{K\_fast + \epsilon})$ \\
$fast\_mortality = m\_fast \cdot fast\_prev$ \\
$fast\_next = fast\_prev + fast\_growth - pred\_fast - fast\_mortality$ \\
$slow\_growth = r\_slow \cdot slow\_prev \cdot (1.0 - \frac{slow\_prev}{K\_slow + \epsilon})$ \\
$slow\_mortality = m\_slow \cdot slow\_prev$ \\
$slow\_next = slow\_prev + slow\_growth - pred\_slow - slow\_mortality$
\end{tabular}
\end{longtable}

\subsection{Detailed Ecological Mechanisms}
\label{subsec:ecological_mechanisms}

The models employ distinctly different approaches to represent key ecological processes:

\subsubsection{Temperature Dependency}
\label{subsubsec:temperature_dependency}

\paragraph{Human Model:} Implements temperature effects through two distinct mechanisms: (1) Gaussian functions modifying coral growth rates and (2) a logistic bleaching mortality function with explicit temperature thresholds (M\_SST50\_f, M\_SST50\_m).

\paragraph{o3 mini:} Employs an asymmetric Gaussian temperature response with a skew parameter, allowing for non-symmetric responses to temperature deviations.

\paragraph{Claude Sonnet 3.6:} Uses a standard Gaussian temperature effect (temp\_opt = 28°C) similar to the human model but without the explicit bleaching threshold.

\paragraph{Claude Sonnet 3.7:} Implements a Gaussian temperature response with temp\_opt = 28°C and temp\_width = 2°C, affecting COTS recruitment. Also includes a bleaching effect with a threshold of 30°C.

\paragraph{o4 mini:} Does not include explicit temperature dependency for COTS in its core equations, focusing instead on resource limitation and outbreak dynamics.

\paragraph{gpt 4.1:} Implements a linear temperature effect where SST modifies growth (centered at 27°C) through the parameter theta\_sst.

\subsubsection{Predation Formulations}
\label{subsubsec:predation_formulations}

\paragraph{Human Model:} Features an explicit prey-switching function where COTS predation preference between fast and slow corals depends on the relative abundance of fast-growing coral, with separate parameters for predation intensity (p1f, p1m) and density-dependence (p2f, p2m).

\paragraph{o3 mini:} Implements direct predation with efficiency factors of 0.4 for fast coral and 0.6 for slow coral, with temperature modifying the predation efficiency on slow coral.

\paragraph{Claude Sonnet 3.6:} Uses a simple Type II functional response with grazing rates of 0.1 for fast coral and 0.05 for slow coral, creating a saturating predation effect at high prey densities.

\paragraph{Claude Sonnet 3.7:} Employs a Holling Type II functional response with attack rates of 0.2 for fast coral and 0.05 for slow coral, with handling times for both coral types.

\paragraph{o4 mini:} Implements a Type III functional response with predation efficiency of 0.05, creating a sigmoidal functional response that reduces predation at low prey densities.

\paragraph{gpt 4.1:} Uses a Type II functional response with attack rates of 0.14 for fast coral and 0.08 for slow coral, with predation partitioned by coral type.

\subsubsection{Population Structure}
\label{subsubsec:population_structure}

\paragraph{Human Model:} Implements an age-structured COTS population with explicit age classes (age-0, age-1, and age-2+), each with age-dependent mortality rates, and uses a Beverton-Holt stock-recruitment relationship.

\paragraph{AI Models:} Generally employ simpler, unstructured population approaches with single-state variables for COTS abundance. The models use various forms of logistic growth (o3 mini, Claude Sonnet 3.7, o4 mini, gpt 4.1) or temperature-modified reproduction functions (Claude Sonnet 3.6, Claude Sonnet 3.7, gpt 4.1).

\subsection{Comparison with Human Model}
\label{subsec:human_comparison}

The human-generated model provides an important reference point for evaluating the AI-generated models. This expert-developed model incorporates several ecological mechanisms that differ from the AI approaches.

\paragraph{Population structure:}
Unlike the AI models, the human-generated model implements an age-structured COTS population with explicit age classes (age-0, age-1, and age-2+), each with age-dependent mortality rates. This contrasts with the simpler, unstructured population approaches in the AI models, which generally use single-state variables for COTS abundance.

\paragraph{Stock-recruitment relationship:}
The human model employs a Beverton-Holt stock-recruitment relationship for COTS reproduction, with parameters derived from unexploited population characteristics. This mechanistic approach differs from the AI models, which typically use simpler logistic growth or temperature-modified reproduction functions.

\paragraph{Prey switching:}
A distinctive feature of the human model is its explicit prey-switching function, where COTS predation preference between fast and slow corals depends on the relative abundance of fast-growing coral. This creates a dynamic feedback mechanism not fully captured in most AI models, though the o3 mini model implements a somewhat similar approach with its complex feedback mechanisms. The gpt 4.1 model also implements a form of prey partitioning based on relative coral abundance.

\paragraph{Temperature effects:}
The human model implements temperature effects through two distinct mechanisms: (1) Gaussian functions modifying coral growth rates, similar to Claude Sonnet 3.6 and Claude Sonnet 3.7, and (2) a logistic bleaching mortality function with temperature thresholds, which is also implemented in Claude Sonnet 3.7 with its bleach\_threshold parameter of 30°C.

\paragraph{Parameter differences:}
The human model uses different parameterization approaches, including:
\begin{itemize}
\item Direct parameterization of carrying capacity (K = 3000) rather than log-transformed values used in Claude Sonnet 3.6, Claude Sonnet 3.7, and gpt 4.1
\item Separate parameters for predation intensity (p1f = 0.15, p1m = 0.06) and density-dependence (p2f, p2m)
\item Explicit bleaching threshold parameters (M\_SST50\_f, M\_SST50\_m) compared to the single bleach\_threshold in Claude Sonnet 3.7
\item Age-dependent mortality components for COTS (Mcots = 2.3) compared to simpler mortality formulations in the AI models (e.g., m\_cots = 0.4 in Claude Sonnet 3.7, 0.3 in o4 mini, and 0.37 in gpt 4.1)
\item Explicit optimal temperatures for both coral types (SST0\_f = 26°C, SST0\_m = 27°C) which are not specified in the AI models
\end{itemize}

\subsection{Carrying Capacity and Growth Rate Comparison}
\label{subsec:carrying_capacity_comparison}

The models show variation in their parameterization of carrying capacity and growth rates:

\paragraph{COTS carrying capacity:} Values range from 0.61 individuals/m² (gpt 4.1) to 50 individuals/m² (o4 mini), with Claude Sonnet 3.6 at 2.0 and Claude Sonnet 3.7 at 2.5. This order-of-magnitude difference reflects fundamentally different assumptions about ecosystem capacity.

\paragraph{COTS growth rate:} More consistency is observed in growth rates, with values of 0.8 per year (Claude Sonnet 3.6 and Claude Sonnet 3.7), 0.5 per year (o4 mini and gpt 4.1), compared to the Beverton-Holt approach in the human model.

\paragraph{Coral growth rates:} The models show variation in coral growth parameters, with fast coral growth rates ranging from 0.1 per year (o4 mini) to 0.61 per year (gpt 4.1), and slow coral growth rates from 0.05 per year (o4 mini) to 0.37 per year (gpt 4.1).

\paragraph{Coral carrying capacity:} The models use different approaches to coral carrying capacity, from the shared K = 3000 in the human model to separate values for fast and slow coral in Claude Sonnet 3.7 (K\_fast = 50, K\_slow = 30) and gpt 4.1 (K\_fast = 33.1, K\_slow = 20.1).

\section{Best Performing Models for CoTS Case Study}
\label{sec:best_models}
This section presents the 5 best performing models from different configurations for the Crown of Thorns Starfish case study.

\subsection{o3 mini Model}
This model achieved an objective value of 0.5606.

\subsubsection{Model Intention}
\begin{lstlisting}
Model Assessment and Ecological Improvement Rationale:

Assessment:
- The current model captures boom-bust cycles but the logistic outbreak trigger appears too smooth.
- Residual analysis indicates that the model underestimates rapid outbreak escalation when coral resource availability is high.
- While the resource limitation function accounts for saturation, it may not fully capture synergistic effects at high coral cover.

Chosen Improvement:
- Introduce a quadratic adjustment factor to the outbreak trigger. When the resource availability (reflected by coral coverage) exceeds the threshold, the quadratic term amplifies the outbreak growth.
- This additional term (scaled by the new parameter poly_coeff) is ecologically justified by the abrupt and intense outbreak episodes observed in the Great Barrier Reef.
\end{lstlisting}

\subsubsection{Model Implementation}
\begin{lstlisting}
#include <TMB.hpp>

// 1. Data:
//    - Year: Time steps from the data file.
//    - sst_dat: Sea Surface Temperature data (°C).
//    - cotsimm_dat: Crown-of-thorns larval immigration rate (individuals/m2/year).
//    - cots_dat: Adult COTS abundance (individuals/m2).
//    - fast_dat: Fast-growing coral cover (Acropora spp.) in %.
//    - slow_dat: Slow-growing coral cover (Faviidae spp. and Porities spp.) in %.
//
// 2. Parameters and equations:
//    (1) COTS outbreak dynamics:
//        cots_pred[t] = cots_pred[t-1] +
//          [ growth_rate * cots_pred[t-1] * ( (fast_dat[t-1]+slow_dat[t-1])/(fast_dat[t-1]+slow_dat[t-1]+saturation) ) 
//            - decline_rate * cots_pred[t-1] ] * dt
//    (2) Environmental modification through sea surface temperature is embedded in the outbreak growth.
//    (3) Smooth transitions and small constants (e.g., 1e-8) are used to avoid division by zero.
//    (4) Only previous time step values are used in predictions to avoid data leakage.
//
// 3. Likelihood:
//    - Observations (cots_dat) are assumed to follow a lognormal distribution around the predictions.
//    - A fixed minimum standard deviation is used for numerical stability.
template<class Type>
Type objective_function<Type>::operator() ()
{
  // Data inputs from file
  DATA_VECTOR(Year);                // Time (years)
  DATA_VECTOR(sst_dat);             // Sea Surface Temperature (°C)
  DATA_VECTOR(cotsimm_dat);         // COTS larval immigration rate (individuals/m2/year)
  DATA_VECTOR(cots_dat);            // Adult COTS abundance (individuals/m2)
  DATA_VECTOR(fast_dat);            // Fast-growing coral cover (Acropora spp.) in %
  DATA_VECTOR(slow_dat);            // Slow-growing coral cover (Faviidae spp. and Porities spp.) in %

  int n = Year.size();              // Number of time steps

  // Parameters (log-scale parameters ensure positivity)
  PARAMETER(log_growth_rate);       // Log of intrinsic outbreak growth rate (log(year^-1))
  PARAMETER(log_decline_rate);      // Log of outbreak decline rate (log(year^-1))
  PARAMETER(log_threshold);         // Log of threshold resource level triggering outbreak (log(units))
  PARAMETER(efficiency_fast);       // Efficiency factor for predation on fast-growing corals (unitless)
  PARAMETER(efficiency_slow);       // Efficiency factor for predation on slow-growing corals (unitless)
  PARAMETER(temp_mod_eff_slow);     // Temperature modifier for predation efficiency on slow-growing coral (unitless)
  PARAMETER(effect_sst);            // Effect of sea surface temperature on outbreak progression (per °C)
  PARAMETER(log_saturation);        // Log of saturation constant for resource limitation (log(units))
  PARAMETER(log_outbreak_steepness);  // Log of outbreak steepness coefficient controlling outbreak trigger sensitivity
  PARAMETER(poly_coeff);            // Quadratic adjustment parameter for outbreak triggering

  // Parameter transformations to ensure positivity where applicable
  Type growth_rate = exp(log_growth_rate);    // Intrinsic growth rate (year^-1)
  Type decline_rate = exp(log_decline_rate);    // Decline rate during bust (year^-1)
  Type threshold    = exp(log_threshold);       // Threshold resource level (units)
  Type saturation   = exp(log_saturation);      // Saturation constant (units)
  Type outbreak_steepness = exp(log_outbreak_steepness);

  // Likelihood accumulation
  Type nll = 0.0;

  // Predicted state vectors for adult COTS numbers and coral covers
  vector<Type> cots_pred(n);
  vector<Type> fast_pred(n);
  vector<Type> slow_pred(n);
  cots_pred[0] = cots_dat(0);   // Initialize COTS with first observed value
  fast_pred[0] = fast_dat(0);    // Initialize fast-growing coral with first observed value
  slow_pred[0] = slow_dat(0);    // Initialize slow-growing coral with first observed value

  for(int t = 1; t < n; t++){
    // 1. Calculate total coral cover from previous time step with a small constant to prevent division by zero.
    Type coral_total = fast_dat(t-1) + slow_dat(t-1) + Type(1e-8);
    // 2. Resource limitation modeled as a saturating function.
    Type resource_limitation = coral_total / (coral_total + saturation);

    // 3. Time difference between measurements
    Type dt = Year(t) - Year(t-1);
    // Introduce environmental modification: Sea Surface Temperature effect modulates outbreak growth rate.
    Type temperature_factor = 1 + effect_sst * sst_dat(t-1);

    // 4. COTS outbreak dynamics with non-linear outbreak threshold:
    //    Growth is modified by a logistic outbreak factor that captures triggering of outbreak events when
    //    resource availability (reflected by resource_limitation) exceeds the threshold. The steepness of this trigger
    //    is controlled by outbreak_steepness. The temperature_factor further modulates the growth rate.
    {
      Type logistic_factor = 1 / (Type(1) + exp(-outbreak_steepness * (resource_limitation - threshold)));
      Type quadratic_adjustment = CppAD::CondExpGt(
          resource_limitation,
          threshold,
          Type(1) + poly_coeff * pow(resource_limitation - threshold, 2),
          Type(1)
      );
      Type outbreak_factor = logistic_factor * quadratic_adjustment;
      cots_pred[t] = cots_pred[t-1] +
                     (growth_rate * cots_pred[t-1] * outbreak_factor * temperature_factor - decline_rate * cots_pred[t-1]) * dt;
    }

    // 5. Ensure numerical stability by preventing negative predictions:
    cots_pred[t] = CppAD::CondExpGt(cots_pred[t], Type(1e-8), cots_pred[t], Type(1e-8));
    
    // 5a. Coral dynamics:
    // Fast-growing coral dynamics
    Type fast_growth_rate = Type(0.2);  // (year^-1), assumed constant growth rate
    Type fast_cap = Type(100.0);        // (%) maximum coral cover
    fast_pred[t] = fast_pred[t-1] + dt * ( fast_growth_rate * fast_pred[t-1] * (1 - fast_pred[t-1] / fast_cap)
                         - efficiency_fast * cots_pred[t-1] * fast_pred[t-1] );
                         
    // Slow-growing coral dynamics
    Type slow_growth_rate = Type(0.1);  // (year^-1), assumed constant growth rate
    Type slow_cap = Type(100.0);        // (%) maximum coral cover
    Type mod_eff_slow = efficiency_slow * (1 + temp_mod_eff_slow * sst_dat(t-1));
    slow_pred[t] = slow_pred[t-1] + dt * ( slow_growth_rate * slow_pred[t-1] * (1 - slow_pred[t-1] / slow_cap)
                         - mod_eff_slow * cots_pred[t-1] * slow_pred[t-1] );

    // 6. Likelihood contribution:
    //    Observations cots_dat are assumed lognormally distributed about predictions.
    //    A fixed standard deviation sigma = 0.1 is used for numerical robustness.
    Type sigma = 0.1;
    nll -= dnorm(log(cots_dat(t) + Type(1e-8)), log(cots_pred[t] + Type(1e-8)), sigma, true);
    
    // Likelihood contributions for coral observations
    Type sigma_coral = 0.1;
    nll -= dnorm(log(fast_dat(t) + Type(1e-8)), log(fast_pred[t] + Type(1e-8)), sigma_coral, true);
    nll -= dnorm(log(slow_dat(t) + Type(1e-8)), log(slow_pred[t] + Type(1e-8)), sigma_coral, true);
  }

  // Report predictions for external diagnostics.
  REPORT(cots_pred);
  REPORT(fast_pred);
  REPORT(slow_pred);

  return nll;
}
\end{lstlisting}

\subsubsection{Model Parameters}
\begin{lstlisting}
{
    "parameters": [
        {
            "parameter": "log_growth_rate",
            "value": 0.0,
            "description": "Logarithm of the intrinsic growth rate of COTS outbreaks (log(year^-1)).",
            "source": "initial estimate",
            "import_type": "PARAMETER",
            "priority": 1,
            "enhanced_semantic_description": "Crown-of-thorns starfish population exponential growth potential rate",
            "processed": true
        },
        {
            "parameter": "log_decline_rate",
            "value": 0.0,
            "description": "Logarithm of the outbreak decline rate (log(year^-1)).",
            "source": "initial estimate",
            "import_type": "PARAMETER",
            "priority": 1,
            "enhanced_semantic_description": "COTS population decay and mortality dynamics in marine ecosystems",
            "processed": true
        },
        {
            "parameter": "log_threshold",
            "value": 0.0,
            "description": "Logarithm of the threshold resource level triggering outbreak (log(units)).",
            "source": "initial estimate",
            "import_type": "PARAMETER",
            "priority": 2,
            "enhanced_semantic_description": "Ecological tipping point for triggering massive Crown-of-thorns starfish outbreak",
            "processed": true
        },
        {
            "parameter": "efficiency_fast",
            "value": 0.4,
            "description": "Efficiency factor for predation on fast-growing coral communities (unitless). Lower value indicates higher resilience.",
            "source": "expert opinion",
            "import_type": "PARAMETER",
            "priority": 2,
            "enhanced_semantic_description": "Predation impact coefficient of starfish on fast-growing Acropora species",
            "processed": true
        },
        {
            "parameter": "efficiency_slow",
            "value": 0.6,
            "description": "Efficiency factor for predation on slow-growing coral communities (unitless). Higher value reflects increased vulnerability.",
            "source": "expert opinion",
            "import_type": "PARAMETER",
            "priority": 2,
            "enhanced_semantic_description": "Predation impact coefficient of starfish on slow-growing Faviidae and Porites species",
            "processed": true
        },
        {
            "parameter": "effect_sst",
            "value": 0.0,
            "description": "Effect of sea surface temperature on outbreak progression (per \u00b0C).",
            "source": "expert opinion",
            "import_type": "PARAMETER",
            "priority": 3,
            "enhanced_semantic_description": "Sea surface temperature's influence on marine predator population dynamics",
            "processed": true
        },
        {
            "parameter": "log_saturation",
            "value": 0.0,
            "description": "Logarithm of the saturation constant for resource limitation (log(units)).",
            "source": "initial estimate",
            "import_type": "PARAMETER",
            "priority": 2,
            "enhanced_semantic_description": "Resource limitation and carrying capacity for marine predator populations",
            "processed": true
        },
        {
            "parameter": "log_outbreak_steepness",
            "value": 0.0,
            "description": "Logarithm of the outbreak steepness coefficient controlling sensitivity of outbreak triggering to resource limitation (log(unitless)).",
            "source": "model enhancement",
            "import_type": "PARAMETER",
            "priority": 3,
            "enhanced_semantic_description": "Determines how sharply outbreak growth is triggered as coral resource availability exceeds the threshold",
            "processed": true
        },
        {
            "parameter": "temp_mod_eff_slow",
            "value": 0.0,
            "description": "Temperature modifier for predation efficiency on slow-growing coral. Increasing temperatures may amplify predation pressure due to reduced coral resilience.",
            "source": "model enhancement",
            "import_type": "PARAMETER",
            "priority": 3,
            "enhanced_semantic_description": "Modifies the effectiveness of starfish predation on slow-growing corals in response to sea surface temperature",
            "processed": true
        },
        {
            "parameter": "poly_coeff",
            "value": 0.0,
            "description": "Coefficient for quadratic adjustment to outbreak triggering when coral cover exceeds threshold. A positive value amplifies outbreak growth, enhancing boom dynamics.",
            "source": "model enhancement",
            "import_type": "PARAMETER",
            "priority": 3,
            "enhanced_semantic_description": "Higher-order response to excess coral resources driving rapid outbreak escalation",
            "processed": true
        }
    ]
}
\end{lstlisting}
\clearpage
\subsection{Claude 3.6 Sonnet Model}
This model achieved an objective value of 0.6599.

\subsubsection{Model Intention}
\begin{lstlisting}
This model simulates the population dynamics of Crown of Thorns Starfish (COTS) and their interactions with coral communities on the Great Barrier Reef. The model captures:

1. Temperature-dependent COTS larval survival and recruitment
2. Density-dependent COTS population growth
3. Differential predation on fast-growing (Acropora) vs slow-growing (Faviidae/Porites) corals
4. Coral growth and recovery dynamics
5. Environmental thresholds that trigger outbreak events

Key mechanisms:
- COTS outbreaks are triggered by enhanced larval survival at optimal temperatures
- Population growth is limited by coral availability (resource limitation)
- Coral recovery is modeled with logistic growth, modified by COTS predation
- System includes feedback loops where coral depletion eventually limits COTS population growth

The model uses time series data of COTS abundance, coral cover (fast/slow-growing), sea surface temperature, and larval immigration rates to predict outbreak dynamics.
\end{lstlisting}

\subsubsection{Model Implementation}
\begin{lstlisting}
#include <TMB.hpp>
template<class Type>
Type objective_function<Type>::operator() ()
{
  // Data
  DATA_VECTOR(Year);                  // Time vector (years)
  DATA_VECTOR(sst_dat);              // Sea surface temperature (°C)
  DATA_VECTOR(cotsimm_dat);          // COTS larval immigration rate (individuals/m2/year)
  DATA_VECTOR(cots_dat);             // Observed COTS density (individuals/m2)
  DATA_VECTOR(fast_dat);             // Observed fast-growing coral cover (%)
  DATA_VECTOR(slow_dat);             // Observed slow-growing coral cover (%)
  
  // Parameters
  PARAMETER(log_r_cots);             // COTS population growth rate
  PARAMETER(log_K_cots);             // COTS carrying capacity
  PARAMETER(log_temp_opt);           // Optimal temperature for COTS survival
  PARAMETER(log_temp_range);         // Temperature tolerance range
  PARAMETER(log_grazing_fast);       // Grazing rate on fast corals
  PARAMETER(log_grazing_slow);       // Grazing rate on slow corals
  PARAMETER(log_r_fast);             // Fast coral growth rate
  PARAMETER(log_r_slow);             // Slow coral growth rate
  PARAMETER(logit_coral_limit);      // Total coral cover limit
  PARAMETER(log_obs_sd_cots);        // Observation error SD for COTS
  PARAMETER(log_obs_sd_fast);        // Observation error SD for fast coral
  PARAMETER(log_obs_sd_slow);        // Observation error SD for slow coral
  
  // Transform parameters
  Type r_cots = exp(log_r_cots);
  Type K_cots = exp(log_K_cots);
  Type temp_opt = exp(log_temp_opt);
  Type temp_range = exp(log_temp_range);
  Type grazing_fast = exp(log_grazing_fast);
  Type grazing_slow = exp(log_grazing_slow);
  Type r_fast = exp(log_r_fast);
  Type r_slow = exp(log_r_slow);
  Type coral_limit = invlogit(logit_coral_limit);
  Type obs_sd_cots = exp(log_obs_sd_cots);
  Type obs_sd_fast = exp(log_obs_sd_fast);
  Type obs_sd_slow = exp(log_obs_sd_slow);
  
  // Initialize predicted vectors
  int n = Year.size();
  vector<Type> cots_pred(n);
  vector<Type> fast_pred(n);
  vector<Type> slow_pred(n);
  
  // Initialize negative log-likelihood
  Type nll = 0.0;
  
  // Small constant to prevent division by zero
  Type eps = Type(1e-8);
  
  // Set initial conditions
  cots_pred(0) = cots_dat(0);
  fast_pred(0) = fast_dat(0);
  slow_pred(0) = slow_dat(0);
  
  // Time series predictions
  for(int t = 1; t < n; t++) {
    // 1. Temperature effect on COTS survival (Gaussian response)
    Type temp_effect = exp(-0.5 * pow((sst_dat(t-1) - temp_opt) / temp_range, 2));
    
    // 2. Resource limitation based on total coral cover
    Type total_coral = (fast_pred(t-1) + slow_pred(t-1)) / 100.0; // Convert to proportion
    Type resource_limit = total_coral / (total_coral + eps);
    
    // 3. COTS population dynamics with temperature-dependent recruitment
    Type recruitment = cotsimm_dat(t-1) * temp_effect;
    cots_pred(t) = cots_pred(t-1) * (1 + r_cots * resource_limit * (1 - cots_pred(t-1)/K_cots)) + recruitment;
    cots_pred(t) = CppAD::CondExpGt(cots_pred(t), Type(0), cots_pred(t), Type(0));
    
    // 4. Coral dynamics with COTS predation
    Type coral_space = (1 - (fast_pred(t-1) + slow_pred(t-1))/100.0/coral_limit);
    coral_space = CppAD::CondExpGt(coral_space, Type(0), coral_space, Type(0));
    
    // Fast-growing coral
    Type fast_growth = r_fast * fast_pred(t-1) * coral_space;
    Type fast_pred_loss = grazing_fast * cots_pred(t-1) * fast_pred(t-1);
    fast_pred(t) = fast_pred(t-1) + fast_growth - fast_pred_loss;
    fast_pred(t) = CppAD::CondExpGt(fast_pred(t), Type(0), fast_pred(t), Type(0));
    
    // Slow-growing coral
    Type slow_growth = r_slow * slow_pred(t-1) * coral_space;
    Type slow_pred_loss = grazing_slow * cots_pred(t-1) * slow_pred(t-1);
    slow_pred(t) = slow_pred(t-1) + slow_growth - slow_pred_loss;
    slow_pred(t) = CppAD::CondExpGt(slow_pred(t), Type(0), slow_pred(t), Type(0));
  }
  
  // Observation model using log-normal distribution
  for(int t = 0; t < n; t++) {
    // Add small constant to prevent log(0)
    nll -= dnorm(log(cots_dat(t) + eps), log(cots_pred(t) + eps), obs_sd_cots, true);
    nll -= dnorm(log(fast_dat(t) + eps), log(fast_pred(t) + eps), obs_sd_fast, true);
    nll -= dnorm(log(slow_dat(t) + eps), log(slow_pred(t) + eps), obs_sd_slow, true);
  }
  
  // Report predictions
  REPORT(cots_pred);
  REPORT(fast_pred);
  REPORT(slow_pred);
  
  return nll;
}
\end{lstlisting}

\subsubsection{Model Parameters}
\begin{lstlisting}
{
    "parameters": [
        {
            "parameter": "log_r_cots",
            "value": -0.223,
            "description": "Log of COTS population growth rate (year^-1)",
            "source": "literature",
            "import_type": "PARAMETER",
            "priority": 1,
            "enhanced_semantic_description": "Crown-of-thorns starfish population exponential growth potential",
            "citations": [
                "https://www.mdpi.com/1424-2818/9/1/1",
                "https://www.nature.com/articles/s41598-023-46749-x",
                "https://royalsocietypublishing.org/doi/pdf/10.1098/rspb.2020.1052"
            ],
            "processed": true
        },
        {
            "parameter": "log_K_cots",
            "value": 0.693,
            "description": "Log of COTS carrying capacity (individuals/m2)",
            "source": "expert opinion",
            "import_type": "PARAMETER",
            "priority": 2,
            "enhanced_semantic_description": "Maximum sustainable Crown-of-thorns starfish population density",
            "processed": true
        },
        {
            "parameter": "log_temp_opt",
            "value": 3.332,
            "description": "Log of optimal temperature for COTS survival (\u00b0C)",
            "source": "literature",
            "import_type": "PARAMETER",
            "priority": 1,
            "enhanced_semantic_description": "Optimal sea temperature for Crown-of-thorns starfish survival",
            "citations": [
                "https://link.springer.com/article/10.1007/s00227-022-04027-w",
                "https://www.researchgate.net/publication/358986050_Effects_of_elevated_temperature_on_the_performance_and_survival_of_pacific_crown-of-thorns_starfish_Acanthaster_cf_solaris",
                "https://www.sciencedirect.com/science/article/abs/pii/S0141113620305328"
            ],
            "processed": true,
            "found_value": 1.41,
            "found_min": 1.38,
            "found_max": 1.46
        },
        {
            "parameter": "log_temp_range",
            "value": 1.099,
            "description": "Log of temperature tolerance range (\u00b0C)",
            "source": "literature",
            "import_type": "PARAMETER",
            "priority": 2,
            "enhanced_semantic_description": "Thermal tolerance range for Crown-of-thorns starfish population",
            "citations": [
                "https://pubmed.ncbi.nlm.nih.gov/29281400/",
                "https://link.springer.com/article/10.1007/s00338-013-1112-3",
                "https://link.springer.com/article/10.1007/s00227-022-04027-w"
            ],
            "processed": true,
            "found_value": 1.39,
            "found_min": 1.26,
            "found_max": 1.49
        },
        {
            "parameter": "log_grazing_fast",
            "value": -2.303,
            "description": "Log of grazing rate on fast-growing corals (m2/individual/year)",
            "source": "literature",
            "import_type": "PARAMETER",
            "priority": 1,
            "enhanced_semantic_description": "Predation impact on fast-growing coral species by starfish",
            "citations": [
                "https://www.sciencedirect.com/science/article/pii/S0048969724028389",
                "https://pubmed.ncbi.nlm.nih.gov/38663591/",
                "https://www.aims.gov.au/research-topics/marine-life/crown-thorns-starfish"
            ],
            "processed": true
        },
        {
            "parameter": "log_grazing_slow",
            "value": -2.996,
            "description": "Log of grazing rate on slow-growing corals (m2/individual/year)",
            "source": "literature",
            "import_type": "PARAMETER",
            "priority": 1,
            "enhanced_semantic_description": "Predation impact on slow-growing coral species by starfish",
            "citations": [
                "https://www.sciencedirect.com/science/article/abs/pii/S0022519384801381",
                "https://onlinelibrary.wiley.com/doi/10.1111/j.1461-0248.2004.00593.x",
                "https://www.int-res.com/articles/theme/m512p167.pdf"
            ],
            "processed": true
        },
        {
            "parameter": "log_r_fast",
            "value": -0.916,
            "description": "Log of fast coral growth rate (year^-1)",
            "source": "literature",
            "import_type": "PARAMETER",
            "priority": 2,
            "enhanced_semantic_description": "Rapid coral species recovery and regeneration potential",
            "citations": [
                "https://www.nature.com/articles/s41598-025-93531-2",
                "https://www.sciencedirect.com/science/article/pii/S0960982224001519",
                "https://news.mongabay.com/short-article/corals-recover-faster-on-artificial-structures-than-on-natural-reefs-study-finds/"
            ],
            "processed": true
        },
        {
            "parameter": "log_r_slow",
            "value": -1.609,
            "description": "Log of slow coral growth rate (year^-1)",
            "source": "literature",
            "import_type": "PARAMETER",
            "priority": 2,
            "enhanced_semantic_description": "Slow coral species recovery and regeneration potential",
            "citations": [
                "https://esajournals.onlinelibrary.wiley.com/doi/10.1002/ecs2.4915",
                "https://besjournals.onlinelibrary.wiley.com/doi/full/10.1111/1365-2664.13141",
                "https://www.sciencedirect.com/science/article/pii/S0141113618305002"
            ],
            "processed": true
        },
        {
            "parameter": "logit_coral_limit",
            "value": 1.386,
            "description": "Logit of maximum total coral cover as proportion",
            "source": "expert opinion",
            "import_type": "PARAMETER",
            "priority": 3,
            "enhanced_semantic_description": "Maximum ecological carrying capacity for total coral coverage",
            "processed": true
        },
        {
            "parameter": "log_obs_sd_cots",
            "value": -1.609,
            "description": "Log of observation error SD for COTS abundance",
            "source": "initial estimate",
            "import_type": "PARAMETER",
            "priority": 3,
            "enhanced_semantic_description": "Measurement uncertainty in Crown-of-thorns starfish population estimates",
            "processed": true
        },
        {
            "parameter": "log_obs_sd_fast",
            "value": -1.609,
            "description": "Log of observation error SD for fast coral cover",
            "source": "initial estimate",
            "import_type": "PARAMETER",
            "priority": 3,
            "enhanced_semantic_description": "Measurement uncertainty in fast-growing coral cover estimates",
            "processed": true
        },
        {
            "parameter": "log_obs_sd_slow",
            "value": -1.609,
            "description": "Log of observation error SD for slow coral cover",
            "source": "initial estimate",
            "import_type": "PARAMETER",
            "priority": 3,
            "enhanced_semantic_description": "Measurement uncertainty in slow-growing coral cover estimates",
            "processed": true
        }
    ]
}
\end{lstlisting}
\clearpage
\subsection{Claude 3.7 Sonnet Model}
This model achieved an objective value of 0.5204.

\subsubsection{Model Intention}
\begin{lstlisting}
This model simulates the population dynamics of Crown of Thorns Starfish (COTS) and their impact on coral communities in the Great Barrier Reef. The model captures the characteristic boom-bust cycles of COTS outbreaks while accounting for their selective predation on different coral functional groups.

Key ecological processes included:
1. COTS population dynamics with density-dependent mortality and recruitment influenced by sea surface temperature
2. Differential predation by COTS on fast-growing (Acropora spp.) and slow-growing (Faviidae and Porites spp.) corals
3. Coral growth and recovery dynamics with competition between coral functional groups
4. Environmental forcing through sea surface temperature effects on COTS recruitment and coral mortality
5. Episodic COTS larval immigration events that can trigger outbreaks

The model uses a system of coupled differential equations discretized to an annual time step, with parameters estimated from time series data. Functional responses include saturating predation rates, temperature-dependent recruitment, and density-dependent mortality. The model aims to predict the timing, magnitude, and duration of COTS outbreaks and their cascading effects on coral community composition.
\end{lstlisting}

\subsubsection{Model Implementation}
\begin{lstlisting}
#include <TMB.hpp>

template<class Type>
Type objective_function<Type>::operator() ()
{
  // DATA
  DATA_VECTOR(Year);                  // Years of observation
  DATA_VECTOR(cots_dat);              // Observed COTS abundance (individuals/m^2)
  DATA_VECTOR(fast_dat);              // Observed fast-growing coral cover (%)
  DATA_VECTOR(slow_dat);              // Observed slow-growing coral cover (%)
  DATA_VECTOR(sst_dat);               // Sea surface temperature (°C)
  DATA_VECTOR(cotsimm_dat);           // COTS larval immigration rate (individuals/m^2/year)
  
  // PARAMETERS
  PARAMETER(r_cots);                  // Intrinsic growth rate of COTS population (year^-1)
  PARAMETER(K_cots);                  // Carrying capacity of COTS population (individuals/m^2)
  PARAMETER(m_cots);                  // Natural mortality rate of COTS (year^-1)
  PARAMETER(r_fast);                  // Intrinsic growth rate of fast-growing coral (year^-1)
  PARAMETER(K_fast);                  // Maximum cover of fast-growing coral (%)
  PARAMETER(r_slow);                  // Intrinsic growth rate of slow-growing coral (year^-1)
  PARAMETER(K_slow);                  // Maximum cover of slow-growing coral (%)
  PARAMETER(a_fast);                  // Attack rate of COTS on fast-growing coral (m^2/individual/year)
  PARAMETER(a_slow);                  // Attack rate of COTS on slow-growing coral (m^2/individual/year)
  PARAMETER(h_fast);                  // Handling time for COTS feeding on fast-growing coral (% cover)
  PARAMETER(h_slow);                  // Handling time for COTS feeding on slow-growing coral (% cover)
  PARAMETER(temp_opt);                // Optimal temperature for COTS recruitment (°C)
  PARAMETER(temp_width);              // Temperature range width for COTS recruitment (°C)
  PARAMETER(imm_effect);              // Effect of larval immigration on COTS recruitment (dimensionless)
  PARAMETER(competition);             // Competition coefficient between coral types (dimensionless)
  PARAMETER(bleach_threshold);        // Temperature threshold for coral bleaching (°C)
  PARAMETER(bleach_mortality_fast);   // Mortality rate of fast-growing coral during bleaching (year^-1)
  PARAMETER(bleach_mortality_slow);   // Mortality rate of slow-growing coral during bleaching (year^-1)
  PARAMETER(sigma_cots);              // Observation error standard deviation for COTS abundance (log scale)
  PARAMETER(sigma_fast);              // Observation error standard deviation for fast-growing coral cover (log scale)
  PARAMETER(sigma_slow);              // Observation error standard deviation for slow-growing coral cover (log scale)
  
  // Initialize negative log-likelihood
  Type nll = 0.0;
  
  // Small constant to prevent division by zero
  Type eps = Type(1e-8);
  
  // Number of time steps
  int n_years = Year.size();
  
  // Vectors to store model predictions
  vector<Type> cots_pred(n_years);
  vector<Type> fast_pred(n_years);
  vector<Type> slow_pred(n_years);
  
  // Initialize with first year's data
  cots_pred(0) = cots_dat(0);
  fast_pred(0) = fast_dat(0);
  slow_pred(0) = slow_dat(0);
  
  // Minimum standard deviations to prevent numerical issues
  Type min_sigma = Type(0.01);
  Type sigma_cots_adj = sigma_cots + min_sigma;
  Type sigma_fast_adj = sigma_fast + min_sigma;
  Type sigma_slow_adj = sigma_slow + min_sigma;
  
  // Time series simulation
  for (int t = 1; t < n_years; t++) {
    // Previous time step values
    Type cots_t0 = cots_pred(t-1);
    Type fast_t0 = fast_pred(t-1);
    Type slow_t0 = slow_pred(t-1);
    Type sst = sst_dat(t-1);
    Type cotsimm = cotsimm_dat(t-1);
    
    // 1. Temperature effect on COTS recruitment
    // Gaussian response curve for temperature effect on COTS recruitment
    Type temp_effect = exp(-0.5 * pow((sst - temp_opt) / temp_width, 2));
    
    // 2. COTS functional response (Type II) for predation on corals
    // Holling Type II functional response for COTS predation on fast-growing coral
    Type pred_fast = (a_fast * fast_t0 * cots_t0) / (1.0 + a_fast * h_fast * fast_t0 + a_slow * h_slow * slow_t0 + eps);
    
    // Holling Type II functional response for COTS predation on slow-growing coral
    Type pred_slow = (a_slow * slow_t0 * cots_t0) / (1.0 + a_fast * h_fast * fast_t0 + a_slow * h_slow * slow_t0 + eps);
    
    // 3. Bleaching effect on corals
    // Smooth transition function for bleaching effect
    Type bleach_effect = 1.0 / (1.0 + exp(-2.0 * (sst - bleach_threshold)));
    
    // 4. COTS population dynamics
    // COTS population growth with density dependence, temperature effect on recruitment, and immigration
    Type cots_growth = r_cots * cots_t0 * (1.0 - cots_t0 / K_cots) * temp_effect;
    
    // Immigration effect with smooth transition
    Type imm_term = imm_effect * cotsimm / (1.0 + cotsimm + eps);
    
    // Food limitation effect (COTS mortality increases when coral cover is low)
    Type food_limitation = m_cots * (1.0 + 1.0 / (fast_t0 + slow_t0 + eps));
    
    // Update COTS abundance
    cots_pred(t) = cots_t0 + cots_growth - food_limitation * cots_t0 + imm_term;
    cots_pred(t) = cots_pred(t) < eps ? eps : cots_pred(t); // Ensure positive values
    
    // 5. Coral dynamics
    // Fast-growing coral dynamics with logistic growth, competition, predation, and bleaching
    Type fast_growth = r_fast * fast_t0 * (1.0 - (fast_t0 + competition * slow_t0) / K_fast);
    Type fast_bleaching = bleach_mortality_fast * bleach_effect * fast_t0;
    
    // Update fast-growing coral cover
    fast_pred(t) = fast_t0 + fast_growth - pred_fast - fast_bleaching;
    fast_pred(t) = fast_pred(t) < eps ? eps : fast_pred(t); // Ensure positive values
    
    // Slow-growing coral dynamics with logistic growth, competition, predation, and bleaching
    Type slow_growth = r_slow * slow_t0 * (1.0 - (slow_t0 + competition * fast_t0) / K_slow);
    Type slow_bleaching = bleach_mortality_slow * bleach_effect * slow_t0;
    
    // Update slow-growing coral cover
    slow_pred(t) = slow_t0 + slow_growth - pred_slow - slow_bleaching;
    slow_pred(t) = slow_pred(t) < eps ? eps : slow_pred(t); // Ensure positive values
  }
  
  // Calculate negative log-likelihood using lognormal distribution
  for (int t = 0; t < n_years; t++) {
    // Add small constant to data and predictions to handle zeros
    Type cots_obs = cots_dat(t) + eps;
    Type cots_mod = cots_pred(t) + eps;
    Type fast_obs = fast_dat(t) + eps;
    Type fast_mod = fast_pred(t) + eps;
    Type slow_obs = slow_dat(t) + eps;
    Type slow_mod = slow_pred(t) + eps;
    
    // Log-normal likelihood for COTS abundance
    nll -= dnorm(log(cots_obs), log(cots_mod), sigma_cots_adj, true);
    
    // Log-normal likelihood for fast-growing coral cover
    nll -= dnorm(log(fast_obs), log(fast_mod), sigma_fast_adj, true);
    
    // Log-normal likelihood for slow-growing coral cover
    nll -= dnorm(log(slow_obs), log(slow_mod), sigma_slow_adj, true);
  }
  
  // Add penalties for biologically implausible parameter values
  // Smooth penalty for growth rates to be positive
  nll += Type(10.0) / (Type(1.0) + exp(Type(5.0) * r_cots));
  nll += Type(10.0) / (Type(1.0) + exp(Type(5.0) * r_fast));
  nll += Type(10.0) / (Type(1.0) + exp(Type(5.0) * r_slow));
  
  // Report predictions
  REPORT(cots_pred);
  REPORT(fast_pred);
  REPORT(slow_pred);
  
  return nll;
}
\end{lstlisting}

\subsubsection{Model Parameters}
\begin{lstlisting}
{
    "parameters": [
        {
            "parameter": "r_cots",
            "value": 0.8,
            "description": "Intrinsic growth rate of COTS population (year^-1)",
            "source": "literature",
            "import_type": "PARAMETER",
            "priority": 1,
            "enhanced_semantic_description": "Crown-of-thorns starfish population exponential growth potential",
            "citations": [
                "https://www.mdpi.com/1424-2818/9/1/1",
                "https://www.nature.com/articles/s41598-023-46749-x",
                "https://royalsocietypublishing.org/doi/pdf/10.1098/rspb.2020.1052"
            ],
            "processed": true
        },
        {
            "parameter": "K_cots",
            "value": 2.5,
            "description": "Carrying capacity of COTS population (individuals/m^2)",
            "source": "literature",
            "import_type": "PARAMETER",
            "priority": 2,
            "enhanced_semantic_description": "Maximum sustainable population density for crown-of-thorns starfish",
            "citations": [
                "https://www.nature.com/articles/s41598-023-46749-x",
                "https://www.sciencedirect.com/science/article/abs/pii/S0964569112002669",
                "https://researchonline.jcu.edu.au/24119/2/02whole.pdf"
            ],
            "processed": true,
            "found_value": 0.002075,
            "found_min": 0.00015,
            "found_max": 0.004
        },
        {
            "parameter": "m_cots",
            "value": 0.4,
            "description": "Natural mortality rate of COTS (year^-1)",
            "source": "literature",
            "import_type": "PARAMETER",
            "priority": 3,
            "enhanced_semantic_description": "Natural death rate of crown-of-thorns starfish population",
            "citations": [
                "https://www.sciencedirect.com/science/article/pii/S0048969724054329",
                "https://www.researchgate.net/publication/324765841_Mortality_rates_of_small_juvenile_crown-of-thorns_starfish_Acanthaster_planci_on_the_Great_Barrier_Reef_Implications_for_population_size_and_larval_settlement_thresholds_for_outbreaks",
                "https://en.wikipedia.org/wiki/Crown-of-thorns_starfish"
            ],
            "processed": true,
            "found_value": 6.0,
            "found_min": 2.49,
            "found_max": 9.49
        },
        {
            "parameter": "r_fast",
            "value": 0.3,
            "description": "Intrinsic growth rate of fast-growing coral (year^-1)",
            "source": "literature",
            "import_type": "PARAMETER",
            "priority": 1,
            "enhanced_semantic_description": "Rapid coral growth rate for fast-colonizing species",
            "citations": [
                "https://besjournals.onlinelibrary.wiley.com/doi/full/10.1111/2041-210X.13388",
                "https://coralcavern.com/how-fast-does-coral-grow-in-colonies/",
                "https://link.springer.com/article/10.1007/s00227-024-04511-5"
            ],
            "processed": true,
            "found_value": 0.72,
            "found_min": 0.06,
            "found_max": 2.0
        },
        {
            "parameter": "K_fast",
            "value": 50.0,
            "description": "Maximum cover of fast-growing coral (%)",
            "source": "literature",
            "import_type": "PARAMETER",
            "priority": 2,
            "enhanced_semantic_description": "Maximum percentage coverage for fast-growing coral species",
            "citations": [
                "https://www.researchgate.net/figure/Percentage-coverage-of-coral-families_tbl1_329554775",
                "https://www.epd.gov.hk/eia/register/report/eiareport/eia_1482008/EIA/html/Text/S11_Marine%20Ecology.htm",
                "https://link.springer.com/article/10.1007/s00227-024-04511-5"
            ],
            "processed": true,
            "found_value": 41.8,
            "found_min": 10.3,
            "found_max": 75.0
        },
        {
            "parameter": "r_slow",
            "value": 0.1,
            "description": "Intrinsic growth rate of slow-growing coral (year^-1)",
            "source": "literature",
            "import_type": "PARAMETER",
            "priority": 1,
            "enhanced_semantic_description": "Gradual coral growth rate for slow-colonizing species",
            "citations": [
                "https://www.frontiersin.org/journals/marine-science/articles/10.3389/fmars.2020.00483/full",
                "https://besjournals.onlinelibrary.wiley.com/doi/full/10.1111/2041-210X.13388",
                "https://www.int-res.com/articles/meps/126/m126p145.pdf"
            ],
            "processed": true
        },
        {
            "parameter": "K_slow",
            "value": 30.0,
            "description": "Maximum cover of slow-growing coral (%)",
            "source": "literature",
            "import_type": "PARAMETER",
            "priority": 2,
            "enhanced_semantic_description": "Maximum percentage coverage for slow-growing coral species",
            "citations": [
                "https://www.researchgate.net/figure/Percentage-coverage-of-coral-families_tbl1_329554775",
                "https://www.sciencedirect.com/science/article/abs/pii/S0006320799000671",
                "https://link.springer.com/article/10.1007/s00338-024-02602-9"
            ],
            "processed": true,
            "found_value": 15.0,
            "found_min": 0.0,
            "found_max": 40.0
        },
        {
            "parameter": "a_fast",
            "value": 0.2,
            "description": "Attack rate of COTS on fast-growing coral (m^2/individual/year)",
            "source": "expert opinion",
            "import_type": "PARAMETER",
            "priority": 1,
            "enhanced_semantic_description": "Predation intensity of starfish on fast-growing coral",
            "processed": true
        },
        {
            "parameter": "a_slow",
            "value": 0.05,
            "description": "Attack rate of COTS on slow-growing coral (m^2/individual/year)",
            "source": "expert opinion",
            "import_type": "PARAMETER",
            "priority": 1,
            "enhanced_semantic_description": "Predation intensity of starfish on slow-growing coral",
            "processed": true
        },
        {
            "parameter": "h_fast",
            "value": 10.0,
            "description": "Handling time for COTS feeding on fast-growing coral (% cover)",
            "source": "initial estimate",
            "import_type": "PARAMETER",
            "priority": 3,
            "enhanced_semantic_description": "Feeding consumption time for fast-growing coral species",
            "processed": true
        },
        {
            "parameter": "h_slow",
            "value": 15.0,
            "description": "Handling time for COTS feeding on slow-growing coral (% cover)",
            "source": "initial estimate",
            "import_type": "PARAMETER",
            "priority": 3,
            "enhanced_semantic_description": "Feeding consumption time for slow-growing coral species",
            "processed": true
        },
        {
            "parameter": "temp_opt",
            "value": 28.0,
            "description": "Optimal temperature for COTS recruitment (\u00b0C)",
            "source": "literature",
            "import_type": "PARAMETER",
            "priority": 2,
            "enhanced_semantic_description": "Ideal temperature range for crown-of-thorns starfish recruitment",
            "citations": [
                "https://www.researchgate.net/publication/358986050_Effects_of_elevated_temperature_on_the_performance_and_survival_of_pacific_crown-of-thorns_starfish_Acanthaster_cf_solaris",
                "https://www.jcu.edu.au/news/releases/2022/december/coral-eating-starfish-another-victim-of-climate-change",
                "https://pmc.ncbi.nlm.nih.gov/articles/PMC4325318/"
            ],
            "processed": true,
            "found_value": 28.0,
            "found_min": 26.0,
            "found_max": 30.0
        },
        {
            "parameter": "temp_width",
            "value": 2.0,
            "description": "Temperature range width for COTS recruitment (\u00b0C)",
            "source": "literature",
            "import_type": "PARAMETER",
            "priority": 3,
            "enhanced_semantic_description": "Temperature tolerance range for starfish population dynamics",
            "citations": [
                "https://link.springer.com/article/10.1007/s00227-022-04027-w",
                "https://www.sciencedirect.com/science/article/abs/pii/S0022098101002982",
                "https://www.researchgate.net/publication/358986050_Effects_of_elevated_temperature_on_the_performance_and_survival_of_pacific_crown-of-thorns_starfish_Acanthaster_cf_solaris"
            ],
            "processed": true,
            "found_value": 28.0,
            "found_min": 26.0,
            "found_max": 30.0
        },
        {
            "parameter": "imm_effect",
            "value": 0.8,
            "description": "Effect of larval immigration on COTS recruitment (dimensionless)",
            "source": "expert opinion",
            "import_type": "PARAMETER",
            "priority": 1,
            "enhanced_semantic_description": "Larval immigration impact on starfish population recruitment",
            "processed": true
        },
        {
            "parameter": "competition",
            "value": 0.3,
            "description": "Competition coefficient between coral types (dimensionless)",
            "source": "initial estimate",
            "import_type": "PARAMETER",
            "priority": 3,
            "enhanced_semantic_description": "Inter-species competitive interaction between different coral types",
            "processed": true
        },
        {
            "parameter": "bleach_threshold",
            "value": 30.0,
            "description": "Temperature threshold for coral bleaching (\u00b0C)",
            "source": "literature",
            "import_type": "PARAMETER",
            "priority": 2,
            "enhanced_semantic_description": "Critical temperature triggering widespread coral bleaching event",
            "citations": [
                "https://www.downtoearth.org.in/climate-change/scientists-may-take-at-least-a-year-to-grasp-full-scale-of-worlds-largest-coral-bleaching-event",
                "https://www.coris.noaa.gov/activities/reef_managers_guide/reef_managers_guide_ch4.pdf",
                "https://www.nature.com/articles/s43247-025-02195-3"
            ],
            "processed": true,
            "found_value": 1.5,
            "found_min": 1.0,
            "found_max": 2.0
        },
        {
            "parameter": "bleach_mortality_fast",
            "value": 0.8,
            "description": "Mortality rate of fast-growing coral during bleaching (year^-1)",
            "source": "literature",
            "import_type": "PARAMETER",
            "priority": 2,
            "enhanced_semantic_description": "Mortality rate of fast-growing corals during bleaching stress",
            "citations": [
                "https://journals.plos.org/plosone/article?id=10.1371/journal.pone.0322636",
                "https://pmc.ncbi.nlm.nih.gov/articles/PMC6899343/",
                "https://www.sciencedirect.com/science/article/abs/pii/S0025326X17305829"
            ],
            "processed": true,
            "found_value": 0.585,
            "found_min": 0.114,
            "found_max": 0.8
        },
        {
            "parameter": "bleach_mortality_slow",
            "value": 0.3,
            "description": "Mortality rate of slow-growing coral during bleaching (year^-1)",
            "source": "literature",
            "import_type": "PARAMETER",
            "priority": 2,
            "enhanced_semantic_description": "Mortality rate of slow-growing corals during bleaching stress",
            "citations": [
                "https://www.sciencedirect.com/science/article/abs/pii/S0048969722083310",
                "https://pmc.ncbi.nlm.nih.gov/articles/PMC6899343/",
                "https://onlinelibrary.wiley.com/doi/10.1111/eva.13500"
            ],
            "processed": true,
            "found_value": 0.185,
            "found_min": 0.04,
            "found_max": 0.415
        },
        {
            "parameter": "sigma_cots",
            "value": 0.2,
            "description": "Observation error standard deviation for COTS abundance (log scale)",
            "source": "initial estimate",
            "import_type": "PARAMETER",
            "priority": 3,
            "enhanced_semantic_description": "Measurement uncertainty for crown-of-thorns starfish population abundance",
            "processed": true
        },
        {
            "parameter": "sigma_fast",
            "value": 0.3,
            "description": "Observation error standard deviation for fast-growing coral cover (log scale)",
            "source": "initial estimate",
            "import_type": "PARAMETER",
            "priority": 3,
            "enhanced_semantic_description": "Measurement uncertainty for fast-growing coral species coverage",
            "processed": true
        },
        {
            "parameter": "sigma_slow",
            "value": 0.3,
            "description": "Observation error standard deviation for slow-growing coral cover (log scale)",
            "source": "initial estimate",
            "import_type": "PARAMETER",
            "priority": 3,
            "enhanced_semantic_description": "Measurement uncertainty for slow-growing coral species coverage",
            "processed": true
        }
    ]
}
\end{lstlisting}
\clearpage
\subsection{o4 mini Model}
This model achieved an objective value of 0.3486.

\subsubsection{Model Intention}
\begin{lstlisting}
\end{lstlisting}

\subsubsection{Model Implementation}
\begin{lstlisting}
#include <TMB.hpp> // TMB header: provides templated functions and macros for the model

// Model Overview:
// 1. COTS Dynamics: Logistic growth modified by a smooth outbreak-triggered decline.
// 2. Fast-growing Coral (Acropora spp.) Dynamics: Recovery towards full cover minus losses due to COTS predation.
// 3. Slow-growing Coral (Faviidae spp. and Porites spp.) Dynamics: Similar recovery with slower dynamics and losses by COTS.
// Each parameter is commented with its units, origin, and role in the ecological processes.

template<class Type>
Type objective_function<Type>::operator() () {
    // --- DATA INPUTS ---
    // time: vector of years (as provided in the first column of the CSV)
    DATA_VECTOR(time);                   // (years)
    DATA_VECTOR(cots_dat);               // Observed COTS abundance (individuals/m2)
    DATA_VECTOR(fast_dat);               // Observed fast-growing coral cover (%) for Acropora spp.
    DATA_VECTOR(slow_dat);               // Observed slow-growing coral cover (%) for Faviidae spp. & Porites spp.
    
    // --- PARAMETERS ---
    // growth_rate_cots: Intrinsic growth rate of COTS (year^-1)
    // decay_rate_cots: Decline rate of COTS post-outbreak (year^-1)
    // coral_predation_efficiency: Efficiency of COTS predation on coral communities (per individual/m2)
    // carrying_capacity: Ecosystem carrying capacity for COTS (individuals/m2)
    // observed_sd: Standard deviation for lognormal observation errors
    PARAMETER(growth_rate_cots);         // (year^-1), literature/expert opinion
    PARAMETER(decay_rate_cots);          // (year^-1), literature
    PARAMETER(coral_predation_efficiency); // (m2/individual), expert opinion
    PARAMETER(carrying_capacity);        // (individuals/m2), literature
    PARAMETER(observed_sd);              // (log-scale units), initial estimate
    PARAMETER(outbreak_sharpness);       // (unitless), governs the steepness of the outbreak trigger function
    PARAMETER(handling_time);             // (time units), handling time for saturating predation response (Holling Type II)
    PARAMETER(outbreak_threshold);       // (unitless), fraction of carrying capacity for outbreak trigger
    PARAMETER(outbreak_shape);           // (unitless), exponent controlling non-linear outbreak trigger sensitivity - values >1 increase outbreak threshold sensitivity.
    PARAMETER(outbreak_hysteresis);      // (unitless), captures hysteresis in outbreak dynamics to model delayed decline post-outbreak.
    PARAMETER(outbreak_nonlinearity);    // (unitless), additional non-linear amplification factor in the outbreak trigger function.
    PARAMETER(outbreak_decline_exponent);       // (unitless), exponent for non-linear outbreak decline dynamics. Values > 1 intensify the decline during outbreak.
    PARAMETER(resource_limitation_strength);       // (unitless), scaling factor representing effect of coral availability on COTS growth
    PARAMETER(environmental_modifier);
    PARAMETER(predation_scaler);
    PARAMETER(coral_recovery_modifier);
    PARAMETER(coral_recovery_inhibition);
    PARAMETER(coral_recovery_environmental_modifier);
    PARAMETER(predation_efficiency_exponent);
    PARAMETER(extreme_outbreak_modifier);
    PARAMETER(coral_saturation_coefficient);
    
    // --- NUMERICAL STABILITY ---
    Type eps = Type(1e-8); // small constant to avoid division by zero
    
    int n = cots_dat.size();
    // Vectors to hold predictions (suffix _pred corresponds to observation names)
    vector<Type> cots_pred(n);
    vector<Type> fast_pred(n);
    vector<Type> slow_pred(n);
    
    // --- INITIAL CONDITIONS ---
    cots_pred[0] = cots_dat[0];  // Use first observation as initial state
    fast_pred[0] = fast_dat[0];
    slow_pred[0] = slow_dat[0];
    
    // --- MODEL EQUATIONS (loop over time steps; t uses previous state only) ---
    for(int t = 1; t < n; t++){
        // Equation 1: COTS Dynamics
        // Incorporate resource limitation by scaling growth with available coral cover (sum of fast and slow predictions).
        Type coral_availability = (fast_pred[t-1] + slow_pred[t-1]) / Type(200);
        // Introduce a saturating function with a quadratic term for coral availability to capture diminishing returns.
        Type resource_factor = (coral_availability + coral_saturation_coefficient * pow(coral_availability, 2))
                                / (Type(0.5) + coral_availability + coral_saturation_coefficient * pow(coral_availability, 2));
        Type growth = growth_rate_cots * cots_pred[t-1] * (1 - cots_pred[t-1] / (carrying_capacity + eps)) * (Type(1) + resource_limitation_strength * (resource_factor - Type(0.5)));
        Type effective_sharpness = outbreak_sharpness * environmental_modifier * (Type(1) + extreme_outbreak_modifier * (environmental_modifier - Type(1)));
        Type raw_trigger = 1 / (Type(1) + exp(- effective_sharpness * ( pow(cots_pred[t-1], outbreak_shape) + outbreak_nonlinearity * pow(cots_pred[t-1], 2) - pow(outbreak_threshold * carrying_capacity, outbreak_shape) )));
        Type outbreak_trigger = raw_trigger + outbreak_hysteresis * raw_trigger * (Type(1) - raw_trigger);
        Type decline = decay_rate_cots * pow(cots_pred[t-1], outbreak_decline_exponent) * outbreak_trigger;
        cots_pred[t] = cots_pred[t-1] + growth - decline; // Updated COTS population
        
        // Equation 2: Fast-growing Coral Dynamics (Acropora spp.)
        //  - Recovery: Proportional to the gap to maximum cover (assumed 100%) modified by inhibition from high COTS levels
        //  - Decline: Losses due to predation by COTS with saturation at low coral cover
        fast_pred[t] = fast_pred[t-1] + Type(0.1) * coral_recovery_modifier * coral_recovery_environmental_modifier * (Type(100) - fast_pred[t-1]) * (Type(1) - coral_recovery_inhibition * cots_pred[t-1] / (carrying_capacity + eps))
                     - (cots_pred[t-1] * coral_predation_efficiency * fast_pred[t-1] *
                        pow((fast_pred[t-1] / (fast_pred[t-1] + predation_scaler)), predation_efficiency_exponent)) / (Type(1) + handling_time * fast_pred[t-1]);
        
        // Equation 3: Slow-growing Coral Dynamics (Faviidae/Porites spp.)
        //  - Recovery: Slower than fast-growing coral, modified by inhibition from high COTS levels
        //  - Decline: Affected by COTS predation with saturation at low coral cover
        slow_pred[t] = slow_pred[t-1] + Type(0.05) * coral_recovery_environmental_modifier * (Type(100) - slow_pred[t-1]) * (Type(1) - coral_recovery_inhibition * cots_pred[t-1] / (carrying_capacity + eps))
                     - (cots_pred[t-1] * coral_predation_efficiency * slow_pred[t-1] *
                        pow((slow_pred[t-1] / (slow_pred[t-1] + predation_scaler)), predation_efficiency_exponent)) / (Type(1) + handling_time * slow_pred[t-1]);
    }
    
    // --- LIKELIHOOD CALCULATION ---
    // Use lognormal distributions (log-transformed data) for strictly positive observations.
    // A fixed small standard deviation is enforced via observed_sd.
    Type jnll = 0.0;
    for(int t = 0; t < n; t++){
        jnll -= dnorm(log(cots_dat[t] + eps), log(cots_pred[t] + eps), observed_sd, true);
        jnll -= dnorm(log(fast_dat[t] + eps), log(fast_pred[t] + eps), observed_sd, true);
        jnll -= dnorm(log(slow_dat[t] + eps), log(slow_pred[t] + eps), observed_sd, true);
    }
    
    // --- REPORTING ---
    // Report all predicted variables with the _pred suffix as required.
    REPORT(cots_pred);  // Predicted COTS
    REPORT(fast_pred);  // Predicted Fast-growing Coral Cover
    REPORT(slow_pred);  // Predicted Slow-growing Coral Cover
    
    return jnll;
}
\end{lstlisting}

\subsubsection{Model Parameters}
\begin{lstlisting}
{
    "parameters": [
        {
            "parameter": "growth_rate_cots",
            "value": 0.5,
            "description": "Intrinsic outbreak growth rate of COTS (year^-1). Literature/expert estimate.",
            "source": "literature",
            "import_type": "PARAMETER",
            "priority": 1,
            "enhanced_semantic_description": "Crown-of-thorns starfish population exponential growth dynamics",
            "citations": [
                "https://www.mdpi.com/1424-2818/9/1/1",
                "https://www.nature.com/articles/s41598-023-46749-x",
                "https://royalsocietypublishing.org/doi/pdf/10.1098/rspb.2020.1052"
            ],
            "processed": true
        },
        {
            "parameter": "decay_rate_cots",
            "value": 0.3,
            "description": "Decay rate of COTS post-outbreak (year^-1). Influences the bust phase.",
            "source": "literature",
            "import_type": "PARAMETER",
            "priority": 1,
            "enhanced_semantic_description": "Population collapse mechanism in marine ecosystem outbreak cycles",
            "citations": [
                "https://www.sciencedirect.com/science/article/abs/pii/S0169534707003552",
                "https://royalsocietypublishing.org/doi/10.1098/rspb.2017.2841",
                "https://esajournals.onlinelibrary.wiley.com/doi/10.1002/ecs2.4580"
            ],
            "processed": true
        },
        {
            "parameter": "coral_predation_efficiency",
            "value": 0.05,
            "description": "Predation efficiency of COTS on coral species (m2/individual). Based on expert opinion.",
            "source": "expert opinion",
            "import_type": "PARAMETER",
            "priority": 2,
            "enhanced_semantic_description": "Quantitative measure of starfish destructive impact on coral reefs",
            "processed": true
        },
        {
            "parameter": "carrying_capacity",
            "value": 50,
            "description": "Carrying capacity for COTS (individuals/m2). Determines saturation level in the ecosystem.",
            "source": "literature",
            "import_type": "PARAMETER",
            "priority": 1,
            "enhanced_semantic_description": "Maximum sustainable population density for marine predator species",
            "citations": [
                "https://iopscience.iop.org/article/10.1088/1742-6596/1280/2/022036/pdf",
                "https://www.sciencedirect.com/science/article/abs/pii/S0303264718304532",
                "https://www.researchgate.net/publication/238373500_Maximum_sustainable_yield_and_species_extinction_in_ecosystems"
            ],
            "processed": true
        },
        {
            "parameter": "observed_sd",
            "value": 0.1,
            "description": "Standard deviation for lognormal observation error (log-scale). Fixed minimum to ensure numerical stability.",
            "source": "initial estimate",
            "import_type": "PARAMETER",
            "priority": 1,
            "enhanced_semantic_description": "Statistical uncertainty and variability in ecological population measurements",
            "processed": true
        },
        {
            "parameter": "outbreak_sharpness",
            "value": 100,
            "description": "Controls outbreak trigger steepness; higher values yield a more threshold-like response capturing outbreak initiation dynamics.",
            "source": "model refinement based on ecological feedback",
            "import_type": "PARAMETER",
            "priority": 2,
            "enhanced_semantic_description": "Sensitivity of outbreak initiation relative to predator density",
            "processed": true
        },
        {
            "parameter": "handling_time",
            "value": 0.01,
            "description": "Handling time for coral predation: accounts for the saturating functional response in COTS predation on coral. A lower value implies quicker handling, leading to less saturation.",
            "source": "model refinement based on ecological feedback",
            "import_type": "PARAMETER",
            "priority": 2,
            "enhanced_semantic_description": "Non-linear saturation in predation dynamics due to handling time",
            "processed": true
        },
        {
            "parameter": "environmental_modifier",
            "value": 1.0,
            "description": "Modifier for outbreak trigger steepness representing environmental influences (e.g. temperature anomalies) on outbreak timing.",
            "source": "model refinement with ecological feedback",
            "import_type": "PARAMETER",
            "priority": 2,
            "enhanced_semantic_description": "Multiplicative factor modulating the outbreak trigger based on environmental conditions",
            "processed": true
        },
        {
            "parameter": "predation_scaler",
            "value": 0.5,
            "description": "Scales coral predation efficiency to reflect reduced effectiveness at low coral cover, representing diminishing returns in prey detection or availability.",
            "source": "ecological model refinement",
            "import_type": "PARAMETER",
            "priority": 2,
            "enhanced_semantic_description": "Non-linear modulation of coral predation efficiency based on coral cover",
            "processed": true
        },
        {
            "parameter": "coral_recovery_modifier",
            "value": 1.0,
            "description": "Modifier scaling coral recovery rates based on favorable environmental conditions (e.g., water quality, temperature).",
            "source": "model refinement based on ecological feedback",
            "import_type": "PARAMETER",
            "priority": 2,
            "enhanced_semantic_description": "Environmental influence on coral recovery dynamics, capturing variable ecosystem resilience",
            "processed": true
        },
        {
            "parameter": "coral_recovery_inhibition",
            "value": 0.2,
            "description": "Inhibition factor on coral recovery due to sustained high COTS pressure. This factor reduces the recovery rate of corals when high COTS numbers persist, capturing delayed ecosystem resilience.",
            "source": "model refinement based on ecological feedback",
            "import_type": "PARAMETER",
            "priority": 2,
            "enhanced_semantic_description": "Modulates coral recovery in response to prolonged high predation pressure",
            "processed": true
        },
        {
            "parameter": "coral_recovery_environmental_modifier",
            "value": 1.0,
            "description": "Scaling factor that modulates coral recovery rates based on environmental conditions (e.g., water temperature, nutrients). It captures how favorable or unfavorable conditions affect coral resilience.",
            "source": "model refinement based on ecological feedback",
            "import_type": "PARAMETER",
            "priority": 2,
            "enhanced_semantic_description": "Dynamic adjustment of coral recovery rates in response to environmental variability",
            "processed": true
        },
        {
            "parameter": "outbreak_threshold",
            "value": 0.5,
            "description": "Threshold fraction of carrying capacity at which an outbreak is triggered. This parameter allows tuning the outbreak onset in response to variable ecological conditions.",
            "source": "model refinement with ecological feedback",
            "import_type": "PARAMETER",
            "priority": 2,
            "enhanced_semantic_description": "Flexible threshold for triggering COTS outbreaks based on relative carrying capacity",
            "processed": true
        },
        {
            "parameter": "resource_limitation_strength",
            "value": 1.0,
            "description": "Scaling factor that modulates the effect of coral availability on COTS growth through a saturating functional response, capturing diminishing returns as coral cover increases.",
            "source": "model refinement based on ecological feedback",
            "import_type": "PARAMETER",
            "priority": 3,
            "enhanced_semantic_description": "Modulates COTS growth via a Michaelis\u2013Menten type resource limitation effect",
            "processed": true
        },
        {
            "parameter": "outbreak_decline_exponent",
            "value": 1.0,
            "description": "Exponent for non-linear outbreak decline. Values > 1 intensify the population collapse during outbreak phases, capturing stronger density-dependent effects.",
            "source": "model refinement based on ecological feedback",
            "import_type": "PARAMETER",
            "priority": 2,
            "enhanced_semantic_description": "Non-linear amplification of outbreak decline dynamics",
            "processed": true
        },
        {
            "parameter": "predation_efficiency_exponent",
            "value": 1.0,
            "description": "Exponent for non-linear scaling of coral predation efficiency. Values >1 amplify the impact of coral cover, capturing threshold dynamics in COTS predation.",
            "source": "model refinement with ecological feedback",
            "import_type": "PARAMETER",
            "priority": 2,
            "enhanced_semantic_description": "Modulates the non-linear impact of coral cover on COTS predation rates",
            "processed": true
        },
        {
            "parameter": "outbreak_shape",
            "value": 2.0,
            "description": "Exponent controlling non-linear outbreak trigger sensitivity. Values > 1 increase the sharpness of outbreak onset.",
            "source": "model refinement with ecological feedback",
            "import_type": "PARAMETER",
            "priority": 2,
            "enhanced_semantic_description": "Reflects the sensitivity of outbreak dynamics to surpassing the critical threshold, leading to a rapid outbreak once exceeded",
            "processed": true
        },
        {
            "parameter": "extreme_outbreak_modifier",
            "value": 0.2,
            "description": "Additional multiplier on outbreak sharpness when environmental conditions are extreme, capturing non-linear amplification of outbreak triggers under atypical conditions.",
            "source": "model refinement",
            "import_type": "PARAMETER",
            "priority": 2,
            "enhanced_semantic_description": "Multiplier for amplifying outbreak trigger sensitivity during extreme environmental conditions",
            "processed": true
        },
        {
            "parameter": "coral_saturation_coefficient",
            "value": 0.1,
            "description": "Coefficient for the quadratic term in the saturating function of coral availability. This parameter captures the non-linear diminishing returns of increased coral cover on COTS resource uptake.",
            "source": "model refinement with ecological feedback",
            "import_type": "PARAMETER",
            "priority": 2,
            "enhanced_semantic_description": "Non-linear modulation of resource limitation by coral availability",
            "processed": true
        },
        {
            "parameter": "outbreak_hysteresis",
            "value": 0.1,
            "description": "Hysteresis effect in outbreak dynamics to capture delayed decline post-outbreak by maintaining outbreak conditions longer.",
            "source": "model refinement incorporating delayed feedback mechanisms",
            "import_type": "PARAMETER",
            "priority": 2,
            "enhanced_semantic_description": "Captures persistence in outbreak conditions due to inertia in system response",
            "processed": true
        },
        {
            "parameter": "outbreak_nonlinearity",
            "value": 0.1,
            "description": "Additional non-linear amplification factor in the outbreak trigger. Captures accelerated outbreak initiation at high COTS densities.",
            "source": "model improvement",
            "import_type": "PARAMETER",
            "priority": 2,
            "enhanced_semantic_description": "Non-linear acceleration factor for outbreak trigger function modeling abrupt outbreak dynamics",
            "processed": true
        }
    ]
}
\end{lstlisting}
\clearpage
\subsection{gpt 4.1 Model}
This model achieved an objective value of 0.3488.

\subsubsection{Model Intention}
\begin{lstlisting}
This model simulates the population dynamics of Crown-of-Thorns starfish (COTS) outbreaks on the Great Barrier Reef, with explicit representation of their interactions with fast-growing (Acropora spp.) and slow-growing (Faviidae and Porites spp.) coral communities. The model captures the characteristic boom-bust cycles of COTS by incorporating resource limitation, selective predation, and environmental drivers (e.g., sea-surface temperature, larval immigration). COTS population growth is regulated by coral availability (with saturating functional response), environmental conditions, and density-dependent feedbacks. Outbreaks are triggered by episodic increases in larval supply or favorable environmental conditions, and sustained by high coral cover, but collapse as coral is depleted or other limiting factors intensify. The model predicts the timing, magnitude, and duration of COTS outbreaks, as well as the resulting impacts on coral community structure, by simulating annual time steps and accounting for both direct and indirect ecological feedbacks.
\end{lstlisting}

\subsubsection{Model Implementation}
\begin{lstlisting}
#include <TMB.hpp>

// 1. Model equations are described at the end of this file.

template<class Type>
Type objective_function<Type>::operator() ()
{
  // --- DATA INPUTS ---
  DATA_VECTOR(Year); // Observation year (integer)
  DATA_VECTOR(cots_dat); // Observed adult COTS density (indiv/m2)
  DATA_VECTOR(fast_dat); // Observed fast-growing coral cover (%) (Acropora)
  DATA_VECTOR(slow_dat); // Observed slow-growing coral cover (%) (Faviidae, Porites)
  DATA_VECTOR(sst_dat); // Sea-surface temperature (deg C)
  DATA_VECTOR(cotsimm_dat); // COTS larval immigration (indiv/m2/year)

  int n = Year.size();

  // --- PARAMETERS ---
  PARAMETER(log_r_cots); // log intrinsic COTS growth rate (year^-1)
  PARAMETER(log_K_cots); // log COTS carrying capacity (indiv/m2)
  PARAMETER(log_alpha_fast); // log COTS attack rate on fast coral (m2/indiv/year)
  PARAMETER(log_alpha_slow); // log COTS attack rate on slow coral (m2/indiv/year)
  PARAMETER(log_e_fast); // log assimilation efficiency from fast coral (unitless)
  PARAMETER(log_e_slow); // log assimilation efficiency from slow coral (unitless)
  PARAMETER(log_m_cots); // log natural mortality rate of COTS (year^-1)
  PARAMETER(logit_theta_sst); // logit SST sensitivity (unitless, 0-1)
  PARAMETER(log_sigma_cots); // log SD for COTS obs (lognormal)
  PARAMETER(log_sigma_fast); // log SD for fast coral obs (lognormal)
  PARAMETER(log_sigma_slow); // log SD for slow coral obs (lognormal)
  PARAMETER(log_r_fast); // log growth rate of fast coral (year^-1)
  PARAMETER(log_r_slow); // log growth rate of slow coral (year^-1)
  PARAMETER(log_K_fast); // log carrying capacity fast coral (%)
  PARAMETER(log_K_slow); // log carrying capacity slow coral (%)
  PARAMETER(log_m_fast); // log background mortality fast coral (year^-1)
  PARAMETER(log_m_slow); // log background mortality slow coral (year^-1)
  PARAMETER(logit_phi_outbreak); // logit outbreak threshold (unitless, 0-1)
  PARAMETER(log_immig_scale); // log scale for larval immigration effect

  // --- TRANSFORM PARAMETERS ---
  Type r_cots = exp(log_r_cots); // COTS intrinsic growth rate
  Type K_cots = exp(log_K_cots); // COTS carrying capacity
  Type alpha_fast = exp(log_alpha_fast); // COTS attack rate on fast coral
  Type alpha_slow = exp(log_alpha_slow); // COTS attack rate on slow coral
  Type e_fast = exp(log_e_fast); // Assimilation efficiency fast coral
  Type e_slow = exp(log_e_slow); // Assimilation efficiency slow coral
  Type m_cots = exp(log_m_cots); // COTS natural mortality
  Type theta_sst = Type(1)/(Type(1)+exp(-logit_theta_sst)); // SST sensitivity [0,1]
  Type sigma_cots = exp(log_sigma_cots); // SD for COTS obs
  Type sigma_fast = exp(log_sigma_fast); // SD for fast coral obs
  Type sigma_slow = exp(log_sigma_slow); // SD for slow coral obs
  Type r_fast = exp(log_r_fast); // Fast coral growth rate
  Type r_slow = exp(log_r_slow); // Slow coral growth rate
  Type K_fast = exp(log_K_fast); // Fast coral carrying capacity
  Type K_slow = exp(log_K_slow); // Slow coral carrying capacity
  Type m_fast = exp(log_m_fast); // Fast coral background mortality
  Type m_slow = exp(log_m_slow); // Slow coral background mortality
  Type phi_outbreak = Type(1)/(Type(1)+exp(-logit_phi_outbreak)); // Outbreak threshold [0,1]
  Type immig_scale = exp(log_immig_scale); // Immigration effect scale

  // --- INITIAL STATES ---
  Type cots_prev = cots_dat(0); // Initial COTS density (indiv/m2)
  Type fast_prev = fast_dat(0); // Initial fast coral cover (%)
  Type slow_prev = slow_dat(0); // Initial slow coral cover (%)

  // --- SMALL CONSTANTS FOR NUMERICAL STABILITY ---
  Type eps = Type(1e-8);

  // --- OUTPUT VECTORS ---
  vector<Type> cots_pred(n);
  vector<Type> fast_pred(n);
  vector<Type> slow_pred(n);

  // --- PROCESS MODEL ---
  for(int t=1; t<n; t++){
    // 1. Resource limitation: total available coral (sum of fast and slow)
    Type total_coral_prev = fast_prev + slow_prev + eps;

    // 2. Functional response: COTS predation on coral (Holling Type II)
    Type pred_fast = alpha_fast * cots_prev * fast_prev / (fast_prev + slow_prev + eps); // predation on fast coral
    Type pred_slow = alpha_slow * cots_prev * slow_prev / (fast_prev + slow_prev + eps); // predation on slow coral

    // 3. COTS population growth (logistic, modified by coral availability and SST)
    Type coral_effect = (fast_prev * e_fast + slow_prev * e_slow) / (K_fast * e_fast + K_slow * e_slow + eps); // resource effect (0-1)
    Type sst_effect = 1.0 + theta_sst * (sst_dat(t) - 27.0); // SST modifies growth (centered at 27C)
    Type immig_effect = immig_scale * cotsimm_dat(t); // immigration pulse

    // Outbreak trigger: smooth threshold on coral_effect
    Type outbreak_boost = 1.0 + phi_outbreak * (coral_effect - 0.5);

    Type cots_growth = r_cots * cots_prev * (1.0 - cots_prev / (K_cots + eps)) * coral_effect * sst_effect * outbreak_boost;
    Type cots_mortality = m_cots * cots_prev;

    Type cots_next = cots_prev + cots_growth - cots_mortality + immig_effect;
    cots_next = CppAD::CondExpGt(cots_next, eps, cots_next, eps); // Bound to >= eps

    // 4. Coral dynamics (logistic growth minus COTS predation and background mortality)
    Type fast_growth = r_fast * fast_prev * (1.0 - fast_prev / (K_fast + eps));
    Type fast_mortality = m_fast * fast_prev;
    Type fast_next = fast_prev + fast_growth - pred_fast - fast_mortality;
    fast_next = CppAD::CondExpGt(fast_next, eps, fast_next, eps);

    Type slow_growth = r_slow * slow_prev * (1.0 - slow_prev / (K_slow + eps));
    Type slow_mortality = m_slow * slow_prev;
    Type slow_next = slow_prev + slow_growth - pred_slow - slow_mortality;
    slow_next = CppAD::CondExpGt(slow_next, eps, slow_next, eps);

    // Store predictions
    cots_pred(t) = cots_next;
    fast_pred(t) = fast_next;
    slow_pred(t) = slow_next;

    // Update for next step
    cots_prev = cots_next;
    fast_prev = fast_next;
    slow_prev = slow_next;
  }

  // Set initial predictions to observed initial values
  cots_pred(0) = cots_dat(0);
  fast_pred(0) = fast_dat(0);
  slow_pred(0) = slow_dat(0);

  // --- LIKELIHOOD ---
  Type nll = 0.0;
  for(int t=0; t<n; t++){
    // Lognormal likelihood for strictly positive data
    nll -= dnorm(log(cots_dat(t) + eps), log(cots_pred(t) + eps), sigma_cots + eps, true);
    nll -= dnorm(log(fast_dat(t) + eps), log(fast_pred(t) + eps), sigma_fast + eps, true);
    nll -= dnorm(log(slow_dat(t) + eps), log(slow_pred(t) + eps), sigma_slow + eps, true);
  }

  // --- REPORTING ---
  REPORT(cots_pred);
  REPORT(fast_pred);
  REPORT(slow_pred);

  // --- EQUATION DESCRIPTIONS ---
  /*
  1. COTS predation on coral: Holling Type II functional response, partitioned by coral type.
  2. COTS population growth: Logistic, modified by coral availability, SST, and outbreak threshold.
  3. Coral growth: Logistic, minus COTS predation and background mortality.
  4. Outbreaks triggered by high coral cover and/or larval immigration.
  5. All rates and effects are bounded and smoothed for numerical stability.
  */

  return nll;
}
\end{lstlisting}

\subsubsection{Model Parameters}
\begin{lstlisting}
{
    "parameters": [
        {
            "parameter": "log_r_cots",
            "value": -0.7,
            "description": "Log intrinsic growth rate of COTS (year^-1)",
            "source": "literature",
            "import_type": "PARAMETER",
            "priority": 1,
            "enhanced_semantic_description": "Crown of Thorns starfish population intrinsic growth potential",
            "citations": [
                "https://www.researchgate.net/figure/Intrinsic-growth-rates-following-different-disturbance-types-The-intrinsic-growth-rate-r_fig7_50597254",
                "https://link.springer.com/article/10.1007/s00338-024-02560-2",
                "https://www.sciencedirect.com/science/article/pii/S0304380023001746"
            ],
            "processed": true
        },
        {
            "parameter": "log_K_cots",
            "value": -0.5,
            "description": "Log carrying capacity for COTS (indiv/m2)",
            "source": "expert opinion",
            "import_type": "PARAMETER",
            "priority": 2,
            "enhanced_semantic_description": "Maximum sustainable Crown of Thorns starfish population density",
            "processed": true
        },
        {
            "parameter": "log_alpha_fast",
            "value": -2.0,
            "description": "Log attack rate of COTS on fast coral (m2/indiv/year)",
            "source": "literature",
            "import_type": "PARAMETER",
            "priority": 2,
            "enhanced_semantic_description": "Predation rate of COTS on fast-growing branching coral species",
            "citations": [
                "https://www.sciencedirect.com/science/article/pii/S0048969724028389",
                "https://www.sciencedirect.com/science/article/pii/S0141113624003167",
                "https://link.springer.com/article/10.1007/s00338-024-02560-2"
            ],
            "processed": true
        },
        {
            "parameter": "log_alpha_slow",
            "value": -2.5,
            "description": "Log attack rate of COTS on slow coral (m2/indiv/year)",
            "source": "literature",
            "import_type": "PARAMETER",
            "priority": 2,
            "enhanced_semantic_description": "Predation rate of COTS on slow-growing massive coral species",
            "citations": [
                "https://www.sciencedirect.com/science/article/pii/S0048969724028389",
                "https://link.springer.com/article/10.1007/s00338-024-02560-2",
                "https://www.int-res.com/articles/theme/m512p167.pdf"
            ],
            "processed": true
        },
        {
            "parameter": "log_e_fast",
            "value": -0.1,
            "description": "Log assimilation efficiency from fast coral (unitless)",
            "source": "expert opinion",
            "import_type": "PARAMETER",
            "priority": 3,
            "enhanced_semantic_description": "Nutritional efficiency of fast-growing coral for COTS",
            "processed": true
        },
        {
            "parameter": "log_e_slow",
            "value": -0.2,
            "description": "Log assimilation efficiency from slow coral (unitless)",
            "source": "expert opinion",
            "import_type": "PARAMETER",
            "priority": 3,
            "enhanced_semantic_description": "Nutritional efficiency of slow-growing coral for COTS",
            "processed": true
        },
        {
            "parameter": "log_m_cots",
            "value": -1.0,
            "description": "Log natural mortality rate of COTS (year^-1)",
            "source": "literature",
            "import_type": "PARAMETER",
            "priority": 2,
            "enhanced_semantic_description": "Natural mortality rate of Crown of Thorns starfish population",
            "citations": [
                "https://www.sciencedirect.com/science/article/pii/S0048969724054329",
                "https://www.researchgate.net/publication/324765841_Mortality_rates_of_small_juvenile_crown-of-thorns_starfish_Acanthaster_planci_on_the_Great_Barrier_Reef_Implications_for_population_size_and_larval_settlement_thresholds_for_outbreaks",
                "https://pmc.ncbi.nlm.nih.gov/articles/PMC9023020/"
            ],
            "processed": true,
            "found_value": 2.25,
            "found_min": 2.25,
            "found_max": 2.25
        },
        {
            "parameter": "logit_theta_sst",
            "value": 0.0,
            "description": "Logit SST sensitivity (unitless, 0-1)",
            "source": "expert opinion",
            "import_type": "PARAMETER",
            "priority": 3,
            "enhanced_semantic_description": "Sea surface temperature sensitivity for COTS population dynamics",
            "processed": true
        },
        {
            "parameter": "log_sigma_cots",
            "value": -1.0,
            "description": "Log SD for COTS obs (lognormal)",
            "source": "initial estimate",
            "import_type": "PARAMETER",
            "priority": 4,
            "enhanced_semantic_description": "Measurement uncertainty in Crown of Thorns starfish population estimates",
            "processed": true
        },
        {
            "parameter": "log_sigma_fast",
            "value": -1.0,
            "description": "Log SD for fast coral obs (lognormal)",
            "source": "initial estimate",
            "import_type": "PARAMETER",
            "priority": 4,
            "enhanced_semantic_description": "Measurement uncertainty in fast-growing coral cover estimates",
            "processed": true
        },
        {
            "parameter": "log_sigma_slow",
            "value": -1.0,
            "description": "Log SD for slow coral obs (lognormal)",
            "source": "initial estimate",
            "import_type": "PARAMETER",
            "priority": 4,
            "enhanced_semantic_description": "Measurement uncertainty in slow-growing coral cover estimates",
            "processed": true
        },
        {
            "parameter": "log_r_fast",
            "value": -0.5,
            "description": "Log growth rate of fast coral (year^-1)",
            "source": "literature",
            "import_type": "PARAMETER",
            "priority": 2,
            "enhanced_semantic_description": "Growth rate of fast-growing branching coral species",
            "citations": [
                "https://coralcavern.com/how-fast-does-coral-grow-in-colonies/",
                "https://www.nature.com/articles/s41598-017-03085-1",
                "https://www.sciencedirect.com/science/article/abs/pii/S0025322719302452"
            ],
            "processed": true,
            "found_value": 3.77,
            "found_min": 2.3,
            "found_max": 4.61
        },
        {
            "parameter": "log_r_slow",
            "value": -1.0,
            "description": "Log growth rate of slow coral (year^-1)",
            "source": "literature",
            "import_type": "PARAMETER",
            "priority": 2,
            "enhanced_semantic_description": "Growth rate of slow-growing massive coral species",
            "citations": [
                "https://www.sciencedirect.com/science/article/abs/pii/S0025322719302452",
                "https://www.sciencedirect.com/science/article/pii/S0925857418303094",
                "https://www.livingoceansfoundation.org/wp-content/uploads/2015/04/U9-Coral-Growth-Background.pdf"
            ],
            "processed": true,
            "found_value": -0.35,
            "found_min": -0.7,
            "found_max": 0.0
        },
        {
            "parameter": "log_K_fast",
            "value": 3.5,
            "description": "Log carrying capacity for fast coral (%)",
            "source": "literature",
            "import_type": "PARAMETER",
            "priority": 3,
            "enhanced_semantic_description": "Maximum sustainable cover for fast-growing coral species",
            "citations": [
                "https://www.sciencedirect.com/science/article/pii/S0025326X23001522",
                "https://www.straitstimes.com/asia/australianz/great-barrier-reef-shows-big-increase-in-coral-cover-but-future-uncertain-report",
                "https://www.nature.com/articles/s41467-022-30234-6"
            ],
            "processed": true
        },
        {
            "parameter": "log_K_slow",
            "value": 3.0,
            "description": "Log carrying capacity for slow coral (%)",
            "source": "literature",
            "import_type": "PARAMETER",
            "priority": 3,
            "enhanced_semantic_description": "Maximum sustainable cover for slow-growing coral species",
            "citations": [
                "https://www.sciencedirect.com/science/article/pii/S0025326X23001522",
                "https://link.springer.com/article/10.1007/s00338-024-02560-2",
                "https://cosmosmagazine.com/earth/great-barrier-reef-coral-cover/"
            ],
            "processed": true,
            "found_value": 52.5,
            "found_min": 40.0,
            "found_max": 65.0
        },
        {
            "parameter": "log_m_fast",
            "value": -2.0,
            "description": "Log background mortality rate of fast coral (year^-1)",
            "source": "expert opinion",
            "import_type": "PARAMETER",
            "priority": 3,
            "enhanced_semantic_description": "Background mortality rate for fast-growing coral species",
            "processed": true
        },
        {
            "parameter": "log_m_slow",
            "value": -2.0,
            "description": "Log background mortality rate of slow coral (year^-1)",
            "source": "expert opinion",
            "import_type": "PARAMETER",
            "priority": 3,
            "enhanced_semantic_description": "Background mortality rate for slow-growing coral species",
            "processed": true
        },
        {
            "parameter": "logit_phi_outbreak",
            "value": 0.0,
            "description": "Logit outbreak threshold (unitless, 0-1)",
            "source": "expert opinion",
            "import_type": "PARAMETER",
            "priority": 2,
            "enhanced_semantic_description": "Threshold probability for Crown of Thorns starfish population outbreak",
            "processed": true
        },
        {
            "parameter": "log_immig_scale",
            "value": -1.0,
            "description": "Log scale for larval immigration effect",
            "source": "expert opinion",
            "import_type": "PARAMETER",
            "priority": 2,
            "enhanced_semantic_description": "Magnitude of larval immigration impact on COTS population",
            "processed": true
        }
    ]
}
\end{lstlisting}
\clearpage
\section{Best Out-of-Sample Test Model}
\label{sec:best_out_of_sample}
This section presents the best performing out-of-sample test model.

\subsection{Model Intention}
\begin{lstlisting}

Assessment:
- The current model includes resource limitation, environmental drivers, and feedbacks between COTS and coral, but does not fully capture the episodic, explosive nature of COTS outbreaks.
- The outbreak threshold is a smooth function of coral cover, but there is no explicit positive feedback in COTS recruitment based on previous COTS abundance, which is a key ecological process in outbreak dynamics.
- Field studies indicate that high COTS densities can amplify recruitment in subsequent years, leading to rapid population booms.

Improvement:
- Add a lagged positive feedback mechanism to COTS recruitment, where high COTS abundance in the previous year increases recruitment in the current year.
- This is implemented as a saturating Hill-type function of previous COTS abundance, controlled by a new parameter (phi_cots).
- This change is ecologically justified as it allows the model to generate sharper, more episodic outbreaks, better matching observed boom-bust cycles and the timing/magnitude of population explosions.
\end{lstlisting}

\subsection{Model Implementation}
\begin{lstlisting}
#include <TMB.hpp>

// 1. Model equations describe the coupled dynamics of COTS, fast coral, and slow coral.
// 2. Resource limitation is modeled with saturating and threshold functions.
// 3. Environmental drivers (SST, larval immigration) modulate COTS recruitment.
// 4. Feedbacks: COTS reduce coral, coral depletion limits COTS, coral recovers after COTS decline.
// 5. All _pred variables are reported and correspond to _dat observations.
// 6. No current time step values of _dat variables are used in predictions (no data leakage).

template<class Type>
Type objective_function<Type>::operator() ()
{
  // --- DATA INPUTS ---
  DATA_VECTOR(Year); // Observation year
  DATA_VECTOR(cots_dat); // Adult COTS abundance (indiv/m2)
  DATA_VECTOR(fast_dat); // Fast coral cover (%)
  DATA_VECTOR(slow_dat); // Slow coral cover (%)
  DATA_VECTOR(sst_dat); // Sea-surface temperature (deg C)
  DATA_VECTOR(cotsimm_dat); // COTS larval immigration (indiv/m2/year)

  int n = Year.size();

  // --- GUARD AGAINST EMPTY INPUT ---
  if(n == 0) {
    // Return large penalty if no data
    return Type(1e10);
  }

  // --- PARAMETERS ---
  PARAMETER(log_r_cots); // log intrinsic COTS recruitment rate (year^-1)
  PARAMETER(log_K_cots); // log COTS carrying capacity (indiv/m2)
  PARAMETER(log_K_cots_half); // log coral cover at which COTS K is half-maximal
  PARAMETER(log_alpha_fast); // log COTS predation rate on fast coral (m2/indiv/year)
  PARAMETER(log_alpha_slow); // log COTS predation rate on slow coral (m2/indiv/year)
  PARAMETER(log_r_fast); // log fast coral regrowth rate (year^-1)
  PARAMETER(log_r_slow); // log slow coral regrowth rate (year^-1)
  PARAMETER(log_K_fast); // log fast coral max cover (%)
  PARAMETER(log_K_slow); // log slow coral max cover (%)
  PARAMETER(log_beta_sst); // log SST effect on COTS recruitment (unitless)
  PARAMETER(log_imm_eff); // log efficiency of larval immigration (unitless)
  PARAMETER(log_sigma_cots); // log obs SD for COTS (lognormal)
  PARAMETER(log_sigma_fast); // log obs SD for fast coral (lognormal)
  PARAMETER(log_sigma_slow); // log obs SD for slow coral (lognormal)

  // --- TRANSFORM PARAMETERS ---
  Type r_cots = exp(log_r_cots); // Intrinsic COTS recruitment rate (year^-1)
  Type K_cots = exp(log_K_cots); // COTS carrying capacity (indiv/m2)
  Type K_cots_half = exp(log_K_cots_half); // Coral cover at which COTS K is half-maximal
  Type alpha_fast = exp(log_alpha_fast); // COTS predation rate on fast coral (m2/indiv/year)
  Type alpha_slow = exp(log_alpha_slow); // COTS predation rate on slow coral (m2/indiv/year)
  Type r_fast = exp(log_r_fast); // Fast coral regrowth rate (year^-1)
  Type r_slow = exp(log_r_slow); // Slow coral regrowth rate (year^-1)
  Type K_fast = exp(log_K_fast); // Fast coral max cover (%)
  Type K_slow = exp(log_K_slow); // Slow coral max cover (%)
  Type beta_sst = exp(log_beta_sst); // SST effect on COTS recruitment (unitless)
  Type imm_eff = exp(log_imm_eff); // Efficiency of larval immigration (unitless)
  Type sigma_cots = exp(log_sigma_cots); // Obs SD for COTS (lognormal)
  Type sigma_fast = exp(log_sigma_fast); // Obs SD for fast coral (lognormal)
  Type sigma_slow = exp(log_sigma_slow); // Obs SD for slow coral (lognormal)
  PARAMETER(log_gamma_cots); // log COTS interference strength (density-dependent reduction in per capita predation)
  Type gamma_cots = exp(log_gamma_cots); // COTS interference strength

  PARAMETER(log_phi_cots); // log positive feedback strength in COTS recruitment (lagged autocatalytic effect)
  Type phi_cots = exp(log_phi_cots); // Positive feedback strength

  // --- INITIAL STATES ---
  Type cots_prev = cots_dat(0); // Initial COTS abundance (indiv/m2)
  Type fast_prev = fast_dat(0); // Initial fast coral cover (%)
  Type slow_prev = slow_dat(0); // Initial slow coral cover (%)

  // --- OUTPUT VECTORS ---
  vector<Type> cots_pred(n);
  vector<Type> fast_pred(n);
  vector<Type> slow_pred(n);

  // --- SMALL CONSTANT FOR NUMERICAL STABILITY ---
  Type eps = Type(1e-8);

  // --- INITIALIZE PREDICTIONS ---
  cots_pred(0) = cots_prev;
  fast_pred(0) = fast_prev;
  slow_pred(0) = slow_prev;

  // --- PROCESS MODEL ---
  for(int t=1; t<n; t++) {
    // 1. COTS recruitment: logistic growth, modulated by SST, larval immigration, and lagged positive feedback
    Type env_mod = 1 + beta_sst * (sst_dat(t-1) - Type(27.0)); // SST effect (centered at 27C)
    Type immig = imm_eff * cotsimm_dat(t-1); // Immigration effect

    // Resource limitation: carrying capacity depends on coral cover (saturating, Michaelis-Menten)
    Type coral_sum = fast_prev + slow_prev + eps;
    Type K_cots_eff = K_cots * (coral_sum/(K_cots_half + coral_sum + eps)); // COTS K saturates with total coral

    // Outbreak threshold: smooth sigmoid on COTS recruitment (triggers outbreak when coral is high)
    Type outbreak_mod = 1/(1 + exp(-5*(coral_sum - 10))); // Outbreak more likely if coral >10%

    // Lagged positive feedback in COTS recruitment (Hill-type function of previous COTS abundance)
    Type feedback_mod = 1 + phi_cots * cots_prev / (1 + phi_cots * cots_prev);

    // COTS predation on corals (Type II functional response) with density-dependent interference
    Type interference = exp(-gamma_cots * cots_prev); // Reduces per capita predation at high COTS density
    Type pred_fast = alpha_fast * cots_prev * fast_prev / (fast_prev + Type(5.0) + eps) * interference; // Fast coral eaten
    Type pred_slow = alpha_slow * cots_prev * slow_prev / (slow_prev + Type(10.0) + eps) * interference; // Slow coral eaten

    // COTS population update
    Type cots_growth = r_cots * cots_prev * (1 - cots_prev/(K_cots_eff+eps)) * env_mod * outbreak_mod * feedback_mod;
    Type cots_next = cots_prev + cots_growth + immig - pred_fast*0.05 - pred_slow*0.02; // Small mortality from feeding inefficiency

    // Bound COTS to positive values
    cots_next = CppAD::CondExpGt(cots_next, eps, cots_next, eps);

    // Fast coral update: logistic regrowth minus COTS predation
    Type fast_growth = r_fast * fast_prev * (1 - fast_prev/(K_fast+eps));
    Type fast_next = fast_prev + fast_growth - pred_fast;

    fast_next = CppAD::CondExpGt(fast_next, eps, fast_next, eps);

    // Slow coral update: logistic regrowth minus COTS predation
    Type slow_growth = r_slow * slow_prev * (1 - slow_prev/(K_slow+eps));
    Type slow_next = slow_prev + slow_growth - pred_slow;

    slow_next = CppAD::CondExpGt(slow_next, eps, slow_next, eps);

    // Save predictions
    cots_pred(t) = cots_next;
    fast_pred(t) = fast_next;
    slow_pred(t) = slow_next;

    // Advance state
    cots_prev = cots_next;
    fast_prev = fast_next;
    slow_prev = slow_next;
  }

  // --- LIKELIHOOD ---
  Type nll = 0.0;
  for(int t=0; t<n; t++) {
    // Lognormal likelihood, fixed minimum SD for stability
    Type sd_cots = sqrt(sigma_cots*sigma_cots + eps);
    Type sd_fast = sqrt(sigma_fast*sigma_fast + eps);
    Type sd_slow = sqrt(sigma_slow*sigma_slow + eps);

    nll -= dnorm(log(cots_dat(t)+eps), log(cots_pred(t)+eps), sd_cots, true);
    nll -= dnorm(log(fast_dat(t)+eps), log(fast_pred(t)+eps), sd_fast, true);
    nll -= dnorm(log(slow_dat(t)+eps), log(slow_pred(t)+eps), sd_slow, true);
  }

  // --- SMOOTH PENALTIES FOR PARAMETER BOUNDS ---
  // Example: penalize negative growth rates, unreasonably high K, etc.
  nll += pow(CppAD::CondExpLt(r_cots, Type(0.01), r_cots-Type(0.01), Type(0)), 2) * 10.0;
  nll += pow(CppAD::CondExpGt(K_cots, Type(10.0), K_cots-Type(10.0), Type(0)), 2) * 10.0;
  nll += pow(CppAD::CondExpLt(r_fast, Type(0.01), r_fast-Type(0.01), Type(0)), 2) * 10.0;
  nll += pow(CppAD::CondExpLt(r_slow, Type(0.01), r_slow-Type(0.01), Type(0)), 2) * 10.0;

  // --- REPORTING ---
  REPORT(cots_pred); // Predicted COTS abundance (indiv/m2)
  REPORT(fast_pred); // Predicted fast coral cover (%)
  REPORT(slow_pred); // Predicted slow coral cover (%)

  return nll;
}
\end{lstlisting}

\subsection{Model Parameters}
\begin{lstlisting}
{
    "parameters": [
        {
            "parameter": "log_r_cots",
            "value": -0.7,
            "description": "Log intrinsic COTS recruitment rate (year^-1)",
            "source": "literature",
            "import_type": "PARAMETER",
            "priority": 1,
            "enhanced_semantic_description": "Crown-of-thorns starfish population growth potential rate",
            "citations": [
                "https://www.sciencedirect.com/science/article/pii/S0048969724054329",
                "https://pmc.ncbi.nlm.nih.gov/articles/PMC9023020/",
                "https://www.reefresilience.org/pdf/COTS_Nov2003.pdf"
            ],
            "processed": true
        },
        {
            "parameter": "log_K_cots",
            "value": -1.0,
            "description": "Log COTS carrying capacity (indiv/m2), scales with coral cover",
            "source": "expert opinion",
            "import_type": "PARAMETER",
            "priority": 1,
            "enhanced_semantic_description": "Maximum sustainable Crown-of-thorns starfish population density",
            "processed": true
        },
        {
            "parameter": "log_K_cots_half",
            "value": 2.5,
            "description": "Log coral cover at which COTS carrying capacity is half its maximum (saturating resource limitation)",
            "source": "ecological reasoning",
            "import_type": "PARAMETER",
            "priority": 2,
            "enhanced_semantic_description": "Coral cover at which COTS carrying capacity is half-maximal; controls saturation of resource limitation",
            "processed": true
        },
        {
            "parameter": "log_alpha_fast",
            "value": -2.0,
            "description": "Log COTS predation rate on fast coral (m2/indiv/year)",
            "source": "literature",
            "import_type": "PARAMETER",
            "priority": 2,
            "enhanced_semantic_description": "Predation impact of COTS on fast-growing coral species",
            "citations": [
                "https://www.sciencedirect.com/science/article/pii/S0048969724028389",
                "https://link.springer.com/article/10.1007/s00338-024-02560-2",
                "https://reefbites.com/2023/04/26/crown-of-thorns-starfish-cots-the-complicated-story-of-a-natural-predator-on-coral-reefs/"
            ],
            "processed": true
        },
        {
            "parameter": "log_alpha_slow",
            "value": -3.0,
            "description": "Log COTS predation rate on slow coral (m2/indiv/year)",
            "source": "literature",
            "import_type": "PARAMETER",
            "priority": 2,
            "enhanced_semantic_description": "Predation impact of COTS on slow-growing coral species",
            "citations": [
                "https://www.sciencedirect.com/science/article/pii/S0048969724028389",
                "https://www.nature.com/articles/s41467-021-26786-8",
                "https://link.springer.com/article/10.1007/s00338-024-02560-2"
            ],
            "processed": true
        },
        {
            "parameter": "log_r_fast",
            "value": -0.5,
            "description": "Log fast coral regrowth rate (year^-1)",
            "source": "literature",
            "import_type": "PARAMETER",
            "priority": 2,
            "enhanced_semantic_description": "Rapid coral recovery and regeneration potential rate",
            "citations": [
                "https://www.sciencedirect.com/science/article/pii/S0960982224001519",
                "https://www.cell.com/current-biology/fulltext/S0960-9822(24)00151-9?rss=yes",
                "https://pmc.ncbi.nlm.nih.gov/articles/PMC9331011/"
            ],
            "processed": true
        },
        {
            "parameter": "log_r_slow",
            "value": -1.0,
            "description": "Log slow coral regrowth rate (year^-1)",
            "source": "literature",
            "import_type": "PARAMETER",
            "priority": 2,
            "enhanced_semantic_description": "Slow coral recovery and regeneration potential rate",
            "citations": [
                "https://esajournals.onlinelibrary.wiley.com/doi/10.1002/ecy.4510",
                "https://esajournals.onlinelibrary.wiley.com/doi/10.1002/ecs2.4915",
                "https://www.sciencedirect.com/science/article/abs/pii/S0025326X15001940"
            ],
            "processed": true
        },
        {
            "parameter": "log_K_fast",
            "value": 3.0,
            "description": "Log fast coral max cover (%)",
            "source": "expert opinion",
            "import_type": "PARAMETER",
            "priority": 3,
            "enhanced_semantic_description": "Maximum sustainable coverage for fast-growing coral species",
            "processed": true
        },
        {
            "parameter": "log_K_slow",
            "value": 3.0,
            "description": "Log slow coral max cover (%)",
            "source": "expert opinion",
            "import_type": "PARAMETER",
            "priority": 3,
            "enhanced_semantic_description": "Maximum sustainable coverage for slow-growing coral species",
            "processed": true
        },
        {
            "parameter": "log_beta_sst",
            "value": -2.0,
            "description": "Log SST effect on COTS recruitment (unitless)",
            "source": "initial estimate",
            "import_type": "PARAMETER",
            "priority": 3,
            "enhanced_semantic_description": "Sea surface temperature influence on COTS population dynamics",
            "processed": true
        },
        {
            "parameter": "log_imm_eff",
            "value": -2.0,
            "description": "Log efficiency of larval immigration (unitless)",
            "source": "initial estimate",
            "import_type": "PARAMETER",
            "priority": 3,
            "enhanced_semantic_description": "Larval immigration effectiveness in COTS population expansion",
            "processed": true
        },
        {
            "parameter": "log_sigma_cots",
            "value": -1.0,
            "description": "Log obs SD for COTS (lognormal)",
            "source": "initial estimate",
            "import_type": "PARAMETER",
            "priority": 1,
            "enhanced_semantic_description": "Measurement uncertainty for Crown-of-thorns starfish population",
            "processed": true
        },
        {
            "parameter": "log_sigma_fast",
            "value": -1.0,
            "description": "Log obs SD for fast coral (lognormal)",
            "source": "initial estimate",
            "import_type": "PARAMETER",
            "priority": 1,
            "enhanced_semantic_description": "Measurement uncertainty for fast-growing coral species coverage",
            "processed": true
        },
        {
            "parameter": "log_sigma_slow",
            "value": -1.0,
            "description": "Log obs SD for slow coral (lognormal)",
            "source": "initial estimate",
            "import_type": "PARAMETER",
            "priority": 1,
            "enhanced_semantic_description": "Measurement uncertainty for slow-growing coral species coverage",
            "processed": true
        },
        {
            "parameter": "log_gamma_cots",
            "value": -2.0,
            "description": "Log COTS interference strength (density-dependent reduction in per capita predation)",
            "source": "ecological reasoning",
            "import_type": "PARAMETER",
            "priority": 2,
            "enhanced_semantic_description": "Strength of intraspecific interference among COTS; higher values reduce per capita predation at high COTS densities",
            "processed": true
        },
        {
            "parameter": "log_phi_cots",
            "value": -2.0,
            "description": "Log positive feedback strength in COTS recruitment (lagged autocatalytic effect)",
            "source": "ecological reasoning",
            "import_type": "PARAMETER",
            "priority": 2,
            "enhanced_semantic_description": "Strength of lagged positive feedback in COTS recruitment; higher values increase the likelihood and magnitude of outbreak events when previous COTS abundance is high",
            "processed": true
        }
    ]
}
\end{lstlisting}
\clearpage

