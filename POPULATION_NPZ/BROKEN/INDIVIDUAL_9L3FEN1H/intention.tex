This modification implements the Droop model of phytoplankton growth, which separates nutrient uptake from growth through an internal nutrient quota mechanism. This better represents how phytoplankton actually process nutrients in several ways:

1. Luxury consumption: When nutrients are abundant, cells can store excess nutrients beyond their immediate growth requirements. This stored pool helps sustain growth when external nutrients become scarce.

2. Variable stoichiometry: The internal nutrient:carbon ratio (Q) can vary between a minimum (Q_min) required for basic cell function and a maximum (Q_max) representing storage capacity. This reflects real physiological constraints.

3. Growth-uptake decoupling: Nutrient uptake rate (V_N) depends on external concentration and current quota, while growth rate depends on internal quota. This creates realistic time lags between nutrient acquisition and biomass increase.

4. More realistic nutrient limitation: Growth limitation occurs gradually as internal quota approaches Q_min, rather than being purely dependent on external concentration. This better captures how cells manage resource stress.

The model should now better reproduce:
- Continued growth after external nutrients are depleted
- More realistic bloom timing and magnitude
- Smoother transitions between nutrient-limited and nutrient-replete conditions
