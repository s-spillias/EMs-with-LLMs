\section{Ecological Mechanisms}

We introduced nutrient-dependent phytoplankton growth efficiency (phi_N) to better represent how phytoplankton adapt their physiology to nutrient availability. This mechanism captures the observation that phytoplankton can modify their internal resource allocation and metabolic efficiency based on nutrient conditions.

The efficiency multiplier phi_N increases with nutrient concentration following a saturating response, representing:
1. Enhanced photosynthetic efficiency under good nutrient conditions
2. More efficient resource allocation when nutrients are abundant
3. Reduced metabolic costs when cells are not nutrient-stressed

This addition helps explain the observed higher phytoplankton growth rates during nutrient-replete conditions while maintaining realistic dynamics during nutrient limitation. The mechanism provides an important feedback between resource availability and population growth that better matches the biological reality of phytoplankton physiology.
