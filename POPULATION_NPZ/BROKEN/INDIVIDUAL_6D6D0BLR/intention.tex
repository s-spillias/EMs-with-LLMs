\section{Ecological Rationale for Internal Nutrient Quota}

The original model assumed that phytoplankton growth was directly coupled to external nutrient concentrations through Monod kinetics. However, this simplification ignores the well-documented ability of phytoplankton to store nutrients internally and continue growing even when external nutrients become scarce (luxury consumption).

By separating nutrient uptake from growth and tracking internal nutrient quotas:

\begin{itemize}
    \item Uptake rate now depends on both external nutrient availability and current storage status (regulated by quota)
    \item Growth rate depends on internal rather than external nutrients, better reflecting actual physiology
    \item Cells can continue growing using stored nutrients even when external concentrations drop
    \item Nutrient stress effects on mortality and sinking now depend on internal status rather than external conditions
\end{itemize}

This mechanism helps explain the observed patterns where phytoplankton biomass peaks higher than predicted by simple Monod kinetics, as cells can accumulate and utilize stored nutrients. The quota-dependent growth also provides a more realistic representation of nutrient limitation, potentially improving the timing and magnitude of population dynamics.

The chosen parameter values reflect typical ranges for marine phytoplankton, with maximum uptake rate (v\_max) higher than maximum growth rate (r\_max) to allow for luxury consumption, and quota ranges based on observed cellular N:C ratios.
