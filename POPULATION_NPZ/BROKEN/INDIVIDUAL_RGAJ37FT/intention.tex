This modification implements variable internal nutrient quotas for phytoplankton, representing "luxury uptake" - a well-documented ecological mechanism where cells can store excess nutrients when available and use them for growth when external nutrients become scarce.

The key equations added are:

1. Quota-dependent growth limitation: (Q - Q_min)/(Q + ε)
   - Growth stops when internal quota reaches minimum
   - Approaches 1 as quota increases
   
2. Uptake regulation: (Q_max - Q)/(Q_max - Q_min)
   - Uptake slows as cells approach maximum quota
   - Provides negative feedback on nutrient storage

3. Quota dynamics: dQ/dt = V/P - μQ
   - First term is nutrient uptake per unit biomass
   - Second term accounts for quota dilution during growth

This better represents:
- Temporal decoupling between nutrient uptake and growth
- Storage effects that allow continued growth in low nutrients
- More realistic nutrient limitation through internal rather than external concentrations
- Competitive dynamics through variable nutrient storage strategies

The mechanism is based on Droop's cell quota model, a foundational framework in phytoplankton ecology that has been extensively validated experimentally.
