\section{Model Enhancement: Temperature-Dependent Grazing Selectivity}

We enhance the ecological model by incorporating temperature-dependent grazing selectivity in zooplankton feeding behavior. This mechanism is based on empirical evidence that zooplankton become more selective in their feeding at higher temperatures, which affects their grazing efficiency and ultimately influences phytoplankton bloom dynamics.

The enhanced grazing function now includes a temperature-dependent selectivity factor that modifies the half-saturation constant for grazing (K_P). As temperature increases, zooplankton become more selective (higher effective K_P), requiring higher phytoplankton densities for efficient grazing. This represents a more realistic predator-prey interaction where metabolic demands and feeding behavior are coupled with environmental conditions.

This modification helps explain:
1. The timing and magnitude of phytoplankton blooms
2. The phase relationship between predator and prey populations
3. The temperature dependence of trophic transfer efficiency

The temperature dependence is implemented through a modified Arrhenius equation, with the selectivity parameter θ controlling the strength of this effect.
