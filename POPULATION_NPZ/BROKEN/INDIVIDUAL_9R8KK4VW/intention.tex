\section{Density-Dependent Grazing Efficiency}

We incorporate density-dependent zooplankton grazing efficiency to better represent how predator-prey interactions change with prey abundance. At high phytoplankton densities, zooplankton feeding efficiency tends to decrease due to:

\begin{itemize}
\item Handling time limitations as feeding apparatus become saturated
\item Increased energy costs of processing excess food
\item Potential interference between prey items
\end{itemize}

This is implemented using a modified assimilation efficiency term that smoothly transitions between maximum efficiency at optimal prey density and reduced efficiency at higher or lower densities:

\[ \alpha_{eff} = \alpha_N \cdot (0.5 + 0.5 \cdot \exp(-\frac{|P - P_{opt}|}{P_{opt}})) \]

where $\alpha_N$ is the nutrient-dependent base efficiency, $P_{opt}$ is the optimal prey density, and $P$ is the current phytoplankton density. This exponential form ensures a smooth peak in efficiency at the optimal prey density while maintaining numerical stability.

This mechanism helps explain the observed higher phytoplankton peaks in the data compared to the original model, as reduced grazing efficiency at high densities allows phytoplankton to reach higher populations before being controlled by predation.
