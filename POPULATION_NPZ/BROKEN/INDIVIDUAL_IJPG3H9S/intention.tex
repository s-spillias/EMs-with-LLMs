This model incorporates nutrient-dependent phytoplankton uptake efficiency to better represent adaptive strategies in nutrient-limited conditions. When nutrients become scarce, phytoplankton can upregulate their uptake mechanisms and become more efficient at acquiring available nutrients. This is represented through a sigmoid response function that increases uptake efficiency as nutrient concentrations decrease below a critical threshold.

The enhanced uptake efficiency (eta_N) is calculated as:
eta_N = eta_base + (eta_max - eta_base) / (1 + exp(-k_eta * (N - N_crit)))

Where:
- eta_base is the baseline uptake efficiency
- eta_max is the maximum efficiency under nutrient stress
- k_eta controls how sharply efficiency changes around N_crit
- N_crit is the nutrient concentration threshold that triggers efficiency changes

This mechanism helps capture the ability of phytoplankton to maintain growth under nutrient limitation, leading to more realistic bloom dynamics and nutrient cycling patterns. The sigmoid function provides a smooth transition between efficiency states while maintaining biological realism in the bounds of the response.
