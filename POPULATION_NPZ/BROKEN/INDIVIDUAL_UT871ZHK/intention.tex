\section{Variable Internal Nutrient Quota Model}

This model enhancement incorporates dynamic phytoplankton stoichiometry through an internal nutrient quota mechanism. The key ecological principles captured are:

\begin{itemize}
\item Phytoplankton can partially decouple nutrient uptake from carbon fixation, allowing "luxury" nutrient storage when available
\item Growth rate depends on internal rather than external nutrient concentrations (Droop model)
\item Nutrient-stressed cells have lower nutritional quality for zooplankton
\item Variable stoichiometry creates a lag between nutrient availability and biomass accumulation
\end{itemize}

This better represents how phytoplankton adapt to fluctuating nutrient conditions and how their physiological state affects trophic transfer efficiency. The internal quota ($Q$) varies between physiological limits ($Q_{min}$ to $Q_{max}$) and modifies both growth rate and predator-prey interactions.
