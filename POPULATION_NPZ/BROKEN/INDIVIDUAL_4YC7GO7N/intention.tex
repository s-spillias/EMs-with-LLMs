\section{Model Description}

This model incorporates variable nutrient quota dynamics in phytoplankton to better represent how nutrient limitation affects growth and mortality. Phytoplankton can store excess nutrients when they are abundant and continue growing even when external nutrients become scarce, until their internal stores are depleted. This mechanism is well-documented in marine phytoplankton and helps explain the timing and magnitude of bloom dynamics.

The internal nutrient quota (Q) represents the nutrient:carbon ratio within phytoplankton cells. Growth rate depends on this internal quota rather than external nutrients directly, following Droop kinetics. When external nutrients are high, phytoplankton can accumulate luxury nutrients (up to Q_max). As external nutrients decline, they draw on these stores to maintain growth until reaching their minimum quota (Q_min).

This addition helps capture:
1. The lag between nutrient depletion and growth limitation
2. More realistic bloom dynamics as cells can continue dividing using stored nutrients
3. Variable nutrient uptake rates based on cell quota status
4. More mechanistic representation of nutrient stress mortality
