This model enhancement implements luxury uptake in phytoplankton - their ability to store excess nutrients internally when available and use them later during nutrient-poor conditions. This is represented through a variable internal nutrient quota (Q) that affects both nutrient uptake and growth rates.

The key ecological mechanisms are:

1. Nutrient uptake rate depends on both external nutrient concentration and current internal quota, approaching zero as Q reaches Q_max (maximum storage capacity)
2. Growth rate depends on internal quota rather than external nutrients, following Droop kinetics where growth increases with Q but plateaus at high quotas
3. A minimum quota (Q_min) represents basic cellular requirements - growth stops when Q falls below this level

This better represents how phytoplankton actually respond to nutrient availability in nature, where internal storage acts as a buffer against environmental variability. The mechanism helps explain both rapid bloom formation when nutrients become available (due to enhanced uptake capacity) and sustained growth even after external nutrients are depleted (drawing on stored reserves).
