This model enhancement implements variable internal nutrient storage (luxury uptake) in phytoplankton, which is a well-documented physiological mechanism in marine ecosystems. The key changes are:

1. Separation of nutrient uptake and growth processes:
   - Uptake rate (V_N) depends on external nutrient concentration and current quota
   - Growth rate (mu) follows Droop equation, depending on internal quota
   
2. Quota dynamics:
   - Q represents internal nutrient:carbon ratio
   - Bounded between Q_min (subsistence quota) and Q_max (storage capacity)
   - Feedback on uptake through f_Q term (reduces uptake as quota fills)
   
3. Modified stress responses:
   - Mortality and sinking now depend on internal quota rather than external nutrients
   - Better represents physiological stress timing

This implementation allows phytoplankton to:
- Buffer against nutrient fluctuations through storage
- Continue growing when external nutrients are depleted
- Regulate uptake based on internal status

These mechanisms are particularly important for capturing:
- Bloom dynamics and timing
- Nutrient cycling patterns
- Population resilience to environmental variability

The changes are based on classic work by Droop (1968) and subsequent empirical studies of phytoplankton physiology.
