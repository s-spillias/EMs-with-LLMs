This model implements nutrient-dependent phytoplankton growth efficiency to better capture the dynamics of nutrient-phytoplankton-zooplankton interactions. 

The key addition is a flexible nutrient uptake efficiency term (eta_N) that allows phytoplankton to modify their nutrient acquisition strategy based on ambient nutrient concentrations. This represents physiological adaptations such as:

1. Upregulation of nutrient transporters under low nutrient conditions
2. Changes in cell surface area to volume ratios
3. Switching between different nutrient acquisition mechanisms

The efficiency term follows a sigmoidal function that transitions between a baseline efficiency (eta_base) and maximum efficiency (eta_max) as nutrient concentrations change. The steepness parameter (k_eta) controls how quickly this transition occurs, while N_crit represents the nutrient concentration at which the transition is centered.

This mechanism allows phytoplankton to maintain growth under varying nutrient conditions, which better reflects their observed ecological strategies and helps explain the timing and magnitude of population dynamics seen in the data.
