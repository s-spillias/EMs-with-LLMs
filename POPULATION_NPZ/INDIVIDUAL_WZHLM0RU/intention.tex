This model incorporates nutrient-dependent phytoplankton growth efficiency to better represent adaptive strategies in nutrient-limited conditions. When nutrients become scarce, phytoplankton can increase their uptake efficiency through various physiological adaptations like:

1. Upregulation of nutrient transporters
2. Changes in cell surface area to volume ratios
3. Enhanced enzyme production for nutrient acquisition

The uptake efficiency (η) follows a sigmoid response to nutrient concentration:
η = η_base + (η_max - η_base) * (K_N^n / (N^n + K_N^n))

Where:
- η_base is the baseline efficiency
- η_max is the maximum efficiency under nutrient stress
- n is the Hill coefficient controlling response steepness
- K_N is the half-saturation constant
- N is the nutrient concentration

This formulation allows phytoplankton to maintain growth at lower nutrient concentrations through increased uptake efficiency, while operating at baseline efficiency when nutrients are abundant. The Hill function provides a more realistic sigmoid response compared to simpler saturating functions.
