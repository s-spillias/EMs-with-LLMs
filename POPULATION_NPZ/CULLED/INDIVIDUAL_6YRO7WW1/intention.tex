\section{Ecological Mechanism: Nutrient-Dependent Zooplankton Mortality}

We enhanced the model's representation of nutrient limitation effects on zooplankton mortality by incorporating food quality dynamics. The new formulation considers the N:P ratio as a proxy for food quality, which directly impacts zooplankton survival. This mechanism is based on ecological evidence that nutrient limitation affects not just the quantity but also the quality of phytoplankton as food for zooplankton.

The modified equation:
\[
\text{Z}_{\text{mortality}} = m_Z \cdot Z + m_{Z_N} \cdot e^{-\frac{N/P}{K_N}} \cdot Z^2
\]

Key ecological justifications:
\begin{itemize}
\item The N:P ratio (N/P) serves as an indicator of food quality, with lower ratios indicating poorer quality food
\item The exponential term creates a stronger mortality effect when food quality is poor
\item The quadratic Z term (Z^2) represents density-dependent effects that become more pronounced under poor food conditions
\item This formulation captures the observation that nutrient limitation can lead to cascading effects through the food web
\end{itemize}

This mechanism better represents the complex relationship between nutrient availability, food quality, and zooplankton survival, which is particularly important during the early bloom period when nutrient dynamics are most dynamic.
