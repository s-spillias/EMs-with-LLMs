This model implementation adds dynamic nutrient storage effects on phytoplankton growth efficiency. The key addition is a "luxury consumption" mechanism where growth efficiency depends not just on current nutrient quota (Q) but also on the rate of change in quota (dQ/dt).

The modified growth equation includes a new efficiency term η_g that increases when cells are building up nutrient reserves (positive dQ/dt) and decreases when depleting reserves (negative dQ/dt). This better represents how real phytoplankton can:
1. Maintain growth temporarily even as external nutrients decline by using stored nutrients
2. Avoid excessive growth when nutrients suddenly become available, as cells need time to rebuild internal reserves
3. Adjust their growth efficiency based on nutrient status trends, not just instantaneous levels

The mechanism is implemented using a hyperbolic function that scales growth efficiency based on both absolute quota level and its rate of change. The new parameter q_sens controls how sensitively growth responds to quota dynamics.

This addition creates more realistic nutrient-growth coupling while maintaining model parsimony. It captures an important biological mechanism (luxury consumption) that helps explain the observed phase relationships between nutrients and plankton populations.
