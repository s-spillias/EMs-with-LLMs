The model has been enhanced to include zooplankton prey switching behavior, a well-documented ecological mechanism where grazers modify their feeding preferences based on relative prey abundances. This addition better represents how zooplankton optimize their foraging strategy in response to changing prey conditions.

The grazing function now includes a prey switching term that modifies the effective half-saturation constant (K_P) based on the ratio of phytoplankton to detritus concentrations. This creates a more realistic representation of zooplankton foraging behavior, where they become more efficient at capturing abundant prey types.

The mathematical formulation uses a sigmoid function to smoothly transition between different grazing preferences, controlled by the prey switching steepness parameter (h_sw). This implementation is based on empirical observations of copepod feeding behavior and similar formulations in the literature.

This mechanism creates an important feedback loop: as phytoplankton become more abundant, zooplankton become more efficient at grazing them, which helps prevent unrealistic phytoplankton blooms and contributes to the system's natural oscillatory behavior.
