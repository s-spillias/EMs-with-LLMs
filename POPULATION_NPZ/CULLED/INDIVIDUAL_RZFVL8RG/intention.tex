This modification introduces a dynamic nutrient recycling mechanism to better represent the rapid nutrient cycling that occurs in marine planktonic food webs. The key ecological insights driving this change are:

1. Zooplankton excretion provides a direct pathway for nutrient regeneration that can help sustain phytoplankton blooms
2. The efficiency of this recycling depends on food quality, represented by the P:N ratio
3. This creates an important feedback loop where efficient recycling during blooms helps maintain elevated nutrient levels

The recycling efficiency (ε) is modeled using a saturating function of the P:N ratio:
ε = ε_max * (P/N)/(P/N + K_recyc)

This formulation captures how:
- Recycling is more efficient when food quality is high (optimal P:N ratio)
- There are physiological limits to recycling efficiency (ε_max)
- The response saturates at high P:N ratios

The recycled nutrients are added directly back to the dissolved nutrient pool, creating a fast-cycling pathway that better represents the tight coupling between zooplankton and nutrient regeneration observed in marine systems.
