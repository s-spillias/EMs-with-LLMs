This model enhancement implements variable internal nutrient storage (luxury uptake) in phytoplankton, a well-documented physiological mechanism that allows cells to decouple nutrient uptake from growth. The key ecological implications are:

1. Nutrient Storage Buffer: Phytoplankton can store excess nutrients when available, continuing growth even when external nutrients become scarce. This better represents the observed phase relationships between nutrients and biomass.

2. Uptake-Growth Decoupling: Separating nutrient uptake from growth reflects real physiological constraints. Uptake depends on external nutrients and is downregulated as internal stores fill up. Growth depends on internal quota, allowing for continued production even when external nutrients are depleted.

3. Realistic Stress Response: Nutrient stress is now based on internal quota rather than external concentration, better representing cellular physiology. This affects both mortality and sinking rates, as nutrient-starved cells are more likely to die or sink out of the surface layer.

The implementation uses the Droop quota model framework, with:
- Uptake rate = V_max * (Q_max - Q)/(Q_max - Q_min) * N/(K_N + N)
- Growth rate = r_max * (Q - Q_min)/Q * f(light,temp)

This mechanistic representation should improve the model's ability to capture:
- Bloom timing and magnitude
- Nutrient drawdown patterns
- Population resilience to nutrient fluctuations
