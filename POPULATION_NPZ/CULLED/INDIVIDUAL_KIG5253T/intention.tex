\section{Model Description}

This model incorporates nutrient-dependent phytoplankton growth efficiency to better represent adaptive strategies in marine ecosystems. When nutrients are scarce, phytoplankton can increase their uptake efficiency through various physiological mechanisms such as:

\begin{itemize}
\item Upregulation of nutrient transporters
\item Changes in cell size or surface area to volume ratios
\item Modification of cellular nutrient quotas
\end{itemize}

The growth efficiency term $\eta_N$ varies sigmoidally with nutrient concentration:

\[ \eta_N = \eta_{base} + \frac{\eta_{max} - \eta_{base}}{1 + e^{-k_{\eta}(N - N_{crit})}} \]

where:
\begin{itemize}
\item $\eta_{base}$ is the baseline efficiency
\item $\eta_{max}$ is the maximum efficiency under optimal conditions
\item $k_{\eta}$ controls how sharply efficiency changes around $N_{crit}$
\item $N_{crit}$ is the nutrient concentration threshold
\end{itemize}

This formulation allows phytoplankton to maintain growth at lower nutrient concentrations through increased uptake efficiency, while preventing unrealistic growth rates when nutrients are abundant. The sigmoid function provides a smooth transition between efficiency states, reflecting gradual physiological adjustments.
