\section{Model Improvements: Phytoplankton Self-Shading}

We have enhanced the ecological model by incorporating phytoplankton self-shading effects on light availability. This is a crucial feedback mechanism in aquatic ecosystems where high phytoplankton concentrations can significantly reduce light penetration through the water column, thereby affecting their own growth rates.

The modified light limitation includes an exponential decay of light intensity based on phytoplankton biomass:

\[ I_{effective} = I \cdot e^{-k_w P} \]

where:
\begin{itemize}
\item $I_{effective}$ is the light intensity after accounting for self-shading
\item $k_w$ is the light attenuation coefficient by phytoplankton
\item $P$ is the phytoplankton concentration
\end{itemize}

This mechanism creates a negative feedback loop that can help explain:
\begin{itemize}
\item The limitation of phytoplankton bloom magnitudes
\item More realistic bloom dynamics through self-regulation
\item Improved representation of the vertical light environment
\end{itemize}

The addition of self-shading is particularly important for capturing the observed patterns in the historical data, where we see limitations on maximum phytoplankton concentrations that cannot be explained by nutrient limitation or grazing pressure alone.
