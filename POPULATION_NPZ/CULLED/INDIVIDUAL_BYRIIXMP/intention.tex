\section{Ecological Mechanism: Variable Internal Nutrient Storage}

The model has been enhanced to include variable internal nutrient storage in phytoplankton, implementing the widely-accepted Droop model of nutrient-limited growth. This mechanism better represents how phytoplankton respond to nutrient availability in natural systems:

\begin{itemize}
    \item Phytoplankton can store excess nutrients when they are abundant, continuing to grow even when external nutrients become scarce
    \item Nutrient uptake is regulated by internal stores - cells reduce uptake when well-supplied
    \item Growth rate depends on internal nutrient status rather than external concentration
    \item Minimum quota (Q_min) represents basic cellular requirements
    \item Maximum quota (Q_max) represents storage capacity
\end{itemize}

This change allows for more realistic time delays between nutrient depletion and growth limitation, which may help explain the observed lag patterns in the historical data. The mechanism is particularly important for capturing dynamics during transitions between nutrient-rich and nutrient-poor conditions.
