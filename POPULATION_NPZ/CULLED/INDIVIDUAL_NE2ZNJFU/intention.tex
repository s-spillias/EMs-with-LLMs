This model enhancement implements variable internal nutrient storage (luxury uptake) in phytoplankton. The key ecological reasoning is:

1. Phytoplankton can store excess nutrients when they are abundant, continuing growth even when external nutrients become scarce
2. Growth rate depends on internal rather than external nutrient concentrations, better reflecting true physiological limitations
3. Nutrient uptake and growth are decoupled, allowing for more realistic dynamics especially during nutrient pulses

This mechanism is implemented through:
- A new state variable Q representing the internal nutrient quota (cellular nutrient content)
- Uptake rate that depends on external nutrients but is also down-regulated as Q approaches maximum capacity
- Growth rate that depends on internal quota rather than external nutrients
- Separate parameters for maximum quota (Q_max) and minimum quota (Q_min) that bound the physiological ranges

This better captures:
- More dynamic phytoplankton response to nutrient pulses
- Continued growth during periods of nutrient scarcity
- More realistic nutrient limitation effects
