This model incorporates variable nutrient storage by phytoplankton through an enhanced uptake efficiency term that accounts for both external nutrient concentrations and internal nutrient status. The storage mechanism is represented by:

1. A modified uptake efficiency (eta_N) that increases when nutrients are abundant and internal stores are depleted
2. A storage saturation term (Q) that reduces uptake as internal stores fill
3. A nutrient quota effect on growth that allows continued growth even when external nutrients decline

This better captures the ecological reality that phytoplankton can decouple nutrient uptake from growth through internal storage, leading to:
- More realistic bloom dynamics as stored nutrients support continued growth
- Better representation of the temporal offset between nutrient uptake and biomass increase
- More accurate prediction of peak bloom magnitudes

The mathematical form uses a Droop-style quota model modified by external nutrient availability, representing how cells opportunistically acquire and store nutrients when available while maintaining growth through internal reserves when external nutrients become scarce.
