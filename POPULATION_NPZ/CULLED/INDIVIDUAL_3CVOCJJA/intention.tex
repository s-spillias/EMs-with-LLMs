\section{Ecological Mechanisms}

The model incorporates nutrient-dependent phytoplankton growth efficiency to better represent physiological adaptations to resource limitation. When nutrients become scarce, phytoplankton can upregulate their nutrient transporters and alter their cellular stoichiometry to maintain growth under suboptimal conditions. This is represented by a dynamic efficiency term:

\[ \gamma = 1 + \gamma_{max} \frac{K_N}{N + K_N} \frac{N}{N + K_\gamma} \]

where $\gamma_{max}$ is the maximum efficiency enhancement factor and $K_\gamma$ is the half-saturation constant for this response. This formulation increases efficiency under moderate nutrient limitation but reduces the effect at very low nutrient concentrations where physiological constraints become limiting.

This mechanism helps explain the observed peaks in phytoplankton biomass despite apparent nutrient limitation, as cells can temporarily maintain higher growth rates through increased resource use efficiency.
