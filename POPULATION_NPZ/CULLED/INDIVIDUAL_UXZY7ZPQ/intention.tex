This model enhancement incorporates photoacclimation - the physiological adjustment of phytoplankton to changing light conditions. 

Phytoplankton can modify their cellular chlorophyll content and photosynthetic machinery in response to light history. This process occurs over a characteristic timescale τ_I (typically 3-7 days) and allows cells to optimize their light harvesting capacity.

The modified light limitation term now includes a dynamic light adaptation state (I_acc) that tracks the light history experienced by the population. This creates a more realistic representation of how phytoplankton respond to seasonal light variations:

- Under prolonged low light, cells increase their photosynthetic efficiency
- Under prolonged high light, cells decrease their efficiency to avoid photodamage
- The response has an appropriate lag determined by τ_I

This mechanism helps explain the timing and magnitude of spring blooms, as populations can maintain higher growth rates during the transition from winter to spring by adapting their photosynthetic apparatus.
