This modification introduces nutrient storage effects in phytoplankton uptake dynamics. Many phytoplankton species can store excess nutrients when they are abundant, allowing them to maintain higher growth rates even as external nutrient concentrations begin to decline. This storage capability is represented through a storage_factor term that enhances nutrient uptake efficiency when ambient nutrients are high.

The storage effect is modeled using a Monod-type function with half-saturation constant K_store, which determines how quickly phytoplankton can achieve maximum storage-enhanced uptake as nutrient levels increase. This creates a more realistic representation of phytoplankton growth dynamics by:

1. Allowing faster nutrient uptake during high-nutrient conditions, better capturing observed bloom dynamics
2. Creating a physiological "memory" effect where past nutrient conditions influence current growth rates
3. Providing a mechanistic basis for the lag between nutrient depletion and growth rate reduction

This mechanism is particularly important for capturing the timing and magnitude of phytoplankton blooms, as it allows populations to better exploit transient periods of high nutrient availability.
