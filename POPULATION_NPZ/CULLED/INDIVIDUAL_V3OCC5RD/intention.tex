\section{Ecological Mechanisms}

A key addition to this model is the incorporation of prey-quality dependent grazing efficiency. In marine ecosystems, phytoplankton nutritional quality varies with their physiological state, particularly under nutrient stress. Nutrient-limited phytoplankton often have higher C:N ratios and lower nutritional value for zooplankton grazers.

This mechanism is implemented through a modified grazing efficiency term that depends on both ambient nutrient concentrations and phytoplankton physiological state. The grazing efficiency multiplier (q_P) is calculated as:

q_P = q_base + (1 - q_base) * (N / (N + K_q))

where:
- q_base is the baseline grazing efficiency under severe nutrient limitation
- K_q is the half-saturation constant for nutrient-dependent grazing quality
- N is the ambient nutrient concentration

This formulation captures how nutrient limitation affects both phytoplankton quality and zooplankton feeding efficiency, creating an important feedback between nutrient availability and trophic transfer efficiency. When nutrients are scarce, reduced grazing efficiency leads to lower zooplankton growth and potentially allows phytoplankton populations to persist at higher levels despite nutrient limitation.
