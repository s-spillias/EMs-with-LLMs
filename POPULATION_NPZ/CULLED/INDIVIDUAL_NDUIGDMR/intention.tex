\section{Ecological Justification for Luxury Uptake}

The model has been enhanced to include variable internal nutrient storage by phytoplankton, known as luxury uptake. This is an important ecological mechanism where phytoplankton can accumulate nutrients beyond their immediate growth requirements when external nutrients are abundant, creating a buffer against future nutrient limitation.

The key modifications include:

\begin{itemize}
    \item Introduction of internal nutrient quota (Q) bounded by physiological limits (Q\_min, Q\_max)
    \item Separation of nutrient uptake and growth processes
    \item Uptake rate that decreases as internal stores fill up
    \item Growth rate dependent on internal rather than external nutrient status
    \item Stress responses (mortality, sinking) now based on internal quota rather than external nutrients
\end{itemize}

This better represents the biological reality that phytoplankton can maintain growth during periods of nutrient scarcity by drawing on internal reserves. The mechanism helps explain the observed phase relationships between nutrients and biomass, and the ability of phytoplankton to achieve higher peak abundances than predicted by instantaneous nutrient availability alone.

The implementation follows the Droop model framework, which has strong empirical support in both laboratory and field studies of phytoplankton physiology.
