\section{Ecological Mechanism: Prey Quality-Dependent Grazing}

We modified the zooplankton grazing efficiency (α) to depend on the N:P ratio of their phytoplankton prey rather than just ambient nutrient concentrations. This change better represents the ecological reality that zooplankton feeding efficiency is influenced by prey nutritional quality.

The new formulation uses N:P ratio as a proxy for prey quality, with:
α = α_base + α_max * (N:P / (N:P + K_α))

This mechanism captures several important ecological processes:
1. Zooplankton feed more efficiently on nutrient-replete phytoplankton
2. Low N:P ratios indicate poor quality prey, reducing assimilation efficiency
3. The response saturates at high N:P ratios, reflecting physiological limits

This change creates an indirect feedback where nutrient availability affects zooplankton not just through phytoplankton abundance, but also through prey quality. This better represents the complex trophic interactions observed in marine systems.
