\section{Ecological Mechanism: Density-Dependent Grazing Refuge}

We have modified the zooplankton grazing term to include a density-dependent refuge effect for phytoplankton. This represents an important ecological mechanism where phytoplankton can escape predation at low densities through various mechanisms:

1. Spatial refuges: At low densities, phytoplankton cells become more dispersed, making it harder for zooplankton to efficiently locate and consume them.

2. Defensive strategies: Many phytoplankton species can alter their morphology, chemistry, or behavior when under intense grazing pressure.

3. Predator-prey encounter rates: The probability of predator-prey encounters naturally decreases at lower prey densities.

The modified grazing term now includes a quadratic refuge function:
\[ P_{refuge} = \frac{P^2}{P_r + P} \]

Where:
- $P$ is phytoplankton density
- $P_r$ is the half-saturation constant for the refuge effect

This formulation means that:
- At low P, grazing pressure increases quadratically with density (P²)
- At high P, the refuge effect saturates and grazing becomes more linear
- The parameter $P_r$ controls how quickly the refuge effect diminishes with increasing density

This mechanism should help better capture:
- The observed higher phytoplankton peaks in the data
- More realistic predator-prey dynamics at low densities
- The persistence of phytoplankton populations under strong grazing pressure
