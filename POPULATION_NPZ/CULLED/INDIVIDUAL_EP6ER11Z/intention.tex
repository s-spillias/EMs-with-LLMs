\section{Ecological Mechanism: Zooplankton Interference Competition}

We introduced density-dependent interference competition into the zooplankton grazing function to better represent how zooplankton feeding efficiency decreases at high population densities. The modified grazing term includes an interference factor:

\[ \text{interference} = \frac{1}{1 + c_Z Z} \]

where $c_Z$ is the interference competition coefficient and $Z$ is the zooplankton density.

This mechanism captures several important ecological processes:

1. At low zooplankton densities (Z → 0), interference → 1, so grazing proceeds at the basic rate
2. As zooplankton density increases, interference increases, reducing per-capita grazing efficiency
3. The strength of interference (c_Z) represents physical contact, behavioral modifications, and indirect effects that reduce feeding effectiveness

This addition better represents the self-limiting nature of zooplankton populations and should improve the model's ability to capture the observed population dynamics, particularly during periods of high zooplankton abundance.
