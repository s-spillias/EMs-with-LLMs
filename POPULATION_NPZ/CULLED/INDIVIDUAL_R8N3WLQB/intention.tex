The model has been enhanced to include prey handling time limitations in the zooplankton grazing functional response. This modification better represents the mechanical and physiological constraints that limit maximum ingestion rates at high prey densities.

The original Holling Type II response (P/(K_P + P)) assumes instantaneous prey handling. The enhanced formulation (P/(K_P + P + g_max * h_time * P)) explicitly accounts for the time predators spend handling each prey item, which reduces their ability to capture new prey. This creates a more realistic saturating response where grazing efficiency decreases at high phytoplankton concentrations.

This mechanism is particularly important for capturing:
1. Reduced grazing pressure at high phytoplankton densities, allowing bloom formation
2. More gradual transitions between resource-limited and saturation-limited grazing regimes
3. Better representation of the predator-prey time lags that drive population cycles

The handling time parameter (h_time) is based on empirical measurements of zooplankton feeding mechanics and provides a mechanistic constraint on maximum ingestion rates.
