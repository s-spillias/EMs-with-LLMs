\section{Prey-Dependent Predator Efficiency}

The model has been enhanced to include prey-dependent predator efficiency, where zooplankton feeding efficiency increases with phytoplankton density. This represents a common ecological phenomenon where predators become more efficient at capturing prey when prey are abundant, due to:

\begin{itemize}
\item Improved prey encounter rates leading to more efficient capture techniques
\item Reduced handling time per prey item when prey are plentiful
\item Behavioral adaptations that optimize feeding at higher prey densities
\end{itemize}

This is implemented through a Holling Type III-like response in the assimilation efficiency term:

\[ \alpha_{total} = \alpha_{base} + \alpha_{max} \cdot \frac{N}{N + K_{\alpha}} \cdot \frac{P^2}{P^2 + K_{P\alpha}^2} \]

where $K_{P\alpha}$ is the half-saturation constant for prey-dependent efficiency. This formulation allows zooplankton to achieve higher growth efficiencies during phytoplankton blooms, better capturing the observed population dynamics.
