\section{Ecological Mechanism: Nutrient-Dependent Zooplankton Mortality}

We enhanced the model by incorporating nutrient-dependent zooplankton mortality to better represent how nutrient limitation affects zooplankton population dynamics through prey quality effects. Under low nutrient conditions, phytoplankton have higher C:N ratios and lower nutritional value, which can increase zooplankton mortality through:

1. Increased metabolic costs of processing low-quality food
2. Reduced energy available for maintenance and immune function
3. Physiological stress from nutrient deficiency

The modified mortality term now includes both density-dependent mortality (m_Z * Z) and a nutrient stress component (m_Z_N * K_N/(N + K_N) * Z). The nutrient stress follows a Monod-type response, increasing mortality when ambient nutrient concentrations are low relative to the half-saturation constant K_N.

This mechanism creates an important feedback where nutrient limitation affects zooplankton both directly through mortality and indirectly through reduced phytoplankton quality and abundance. This better captures the complex trophic interactions observed in marine systems under nutrient stress.
