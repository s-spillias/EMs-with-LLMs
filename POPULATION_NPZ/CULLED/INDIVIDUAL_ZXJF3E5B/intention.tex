This model enhancement introduces nutrient-dependent zooplankton mortality to better represent resource limitation effects on zooplankton populations. When nutrient levels are low, zooplankton experience increased mortality due to:

1. Reduced food quality - Low nutrient environments lead to poor-quality phytoplankton prey
2. Metabolic stress - Limited resources force increased energy expenditure for food searching
3. Reduced immune function - Nutritional stress can compromise zooplankton health

The mortality rate is modeled using a Monod-type function where mortality increases as nutrient concentration decreases, with K_M controlling the nutrient concentration at which mortality effects become significant.

This mechanism creates an important feedback loop: low nutrients → increased zooplankton mortality → reduced grazing pressure → opportunity for phytoplankton recovery → nutrient depletion, which helps explain the oscillatory patterns observed in the data.
