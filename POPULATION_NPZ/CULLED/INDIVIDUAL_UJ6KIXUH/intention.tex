The model has been enhanced to include temperature dependence in nutrient uptake efficiency. This modification reflects the biological reality that temperature affects both membrane fluidity and protein-mediated transport processes in phytoplankton cells. Warmer temperatures typically increase membrane fluidity, which can enhance the diffusion of nutrients across cell membranes. Additionally, the activity of nutrient transporter proteins often increases with temperature due to faster protein conformational changes and higher enzymatic activity rates.

The new formulation multiplies the existing nutrient-dependent uptake efficiency by a temperature scaling factor (1 + θ_η(T_scale - 1)), where θ_η represents the temperature sensitivity of nutrient uptake. This creates an important feedback mechanism where seasonal temperature changes can modify the efficiency of nutrient acquisition, potentially leading to more realistic bloom dynamics.

This mechanism is particularly relevant for capturing the timing and magnitude of phytoplankton blooms, as it couples two key environmental drivers (temperature and nutrient availability) in determining resource acquisition efficiency. The temperature dependence of uptake efficiency may help explain why some blooms occur more rapidly than would be predicted by nutrient availability alone.
