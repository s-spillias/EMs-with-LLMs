\section{Ecological Mechanisms}

The model has been enhanced to include temperature-dependent nutrient storage capacity in phytoplankton. This modification is based on the ecological principle that cellular nutrient storage capabilities are constrained by metabolic processes, which are temperature-dependent.

At higher temperatures, maintaining large nutrient reserves becomes more metabolically costly due to:
\begin{itemize}
\item Increased maintenance respiration costs
\item Higher rates of protein turnover
\item Reduced efficiency of storage mechanisms
\end{itemize}

The maximum quota (Q\_max) now varies with temperature according to:
\[ Q_{max}(T) = Q_{max} \cdot (1 - Q_{temp\_sens} \cdot (f(T) - 1)) \]
where f(T) is the temperature scaling function and Q\_temp\_sens controls the strength of the temperature effect.

This mechanism creates an important trade-off: while warmer temperatures increase growth potential, they simultaneously reduce storage capacity. This better represents the physiological constraints faced by phytoplankton and should improve the model's ability to capture nutrient-phytoplankton dynamics, particularly during temperature fluctuations.
