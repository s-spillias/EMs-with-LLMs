The model has been enhanced to better represent zooplankton mortality under resource-limited conditions. The previous formulation only considered nutrient stress, but zooplankton survival is realistically affected by both prey abundance and prey quality (which depends on nutrient availability).

The new mortality term combines two limitation factors:
1. prey_limitation: Increases mortality when prey (P) is scarce relative to the half-saturation constant K_P
2. nutrient_limitation: Increases mortality when nutrients (N) are scarce relative to K_N

This multiplicative effect captures the synergistic impact of poor feeding conditions (low P) and poor food quality (low N) on zooplankton survival. When either resources or nutrients are abundant, the respective limitation term approaches zero, reducing the additional mortality. This better represents the ecological reality that zooplankton can partially compensate for one type of limitation but face severe consequences when both food quantity and quality are poor.

The functional form uses the same half-saturation constants as grazing (K_P) and nutrient uptake (K_N) to maintain parameter parsimony while capturing the relevant resource thresholds that affect zooplankton physiology.
