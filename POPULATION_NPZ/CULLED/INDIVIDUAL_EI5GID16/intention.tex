\section{Model Enhancement: Phytoplankton Self-Shading}

We have enhanced the ecological model by incorporating phytoplankton self-shading effects on light availability. This mechanism represents how increasing phytoplankton biomass reduces light penetration in the water column, creating a negative feedback on phytoplankton growth.

The modified light limitation equation now includes an exponential attenuation term:

\[ I_{effective} = I_0 \exp(-k_w P) \]

where:
\begin{itemize}
\item $I_0$ is the surface light intensity
\item $k_w$ is the light attenuation coefficient due to phytoplankton
\item $P$ is the phytoplankton concentration
\end{itemize}

This addition captures an important density-dependent feedback mechanism in aquatic ecosystems. As phytoplankton populations grow, they increasingly shade themselves, which can help explain the observed limitations on peak biomass and contribute to population stability. This mechanism is particularly relevant in productive waters where self-shading can become a major factor limiting phytoplankton growth.

The implementation uses a light attenuation coefficient ($k_w$) based on literature values for typical marine phytoplankton communities. This modification should improve the model's ability to capture the observed peak concentrations and subsequent decline patterns seen in the empirical data.
