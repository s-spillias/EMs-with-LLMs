\section{Model Enhancement: Nutrient-Dependent Growth Efficiency}

This model enhancement introduces nutrient-dependent phytoplankton growth efficiency to better represent how phytoplankton optimize their resource utilization under varying nutrient conditions. The key additions are:

\begin{itemize}
    \item A new parameter $\eta_{max}$ representing the maximum additional growth efficiency possible under nutrient-replete conditions
    \item A dynamic efficiency multiplier $\eta_N = 1 + \eta_{max} \frac{N}{N + K_N}$ that scales growth rate
\end{itemize}

\subsection{Ecological Justification}

Phytoplankton can adjust their metabolic efficiency based on nutrient availability. When nutrients are abundant, cells can:
\begin{itemize}
    \item Optimize their nutrient uptake systems
    \item Allocate more energy to growth rather than maintenance
    \item Maintain optimal internal stoichiometry
\end{itemize}

This enhancement allows the model to capture higher growth rates during nutrient-rich conditions while maintaining appropriate limitation effects when nutrients are scarce. The efficiency factor approaches 1 (base efficiency) when nutrients are limiting and $(1 + \eta_{max})$ when nutrients are abundant.

The half-saturation constant $K_N$ serves double duty - controlling both basic nutrient limitation and the transition in efficiency, reflecting how these processes are fundamentally linked to nutrient availability.
