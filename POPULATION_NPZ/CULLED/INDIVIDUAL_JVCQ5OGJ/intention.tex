\section{Temperature-Dependent Nutrient Recycling}

The model has been enhanced to better represent temperature-dependent nutrient recycling through bacterial decomposition of detritus. In marine systems, bacterial activity typically increases more rapidly with temperature than other metabolic processes, leading to faster nutrient recycling in warmer conditions. This is implemented through an additional temperature sensitivity term (E_r) that modifies the base temperature scaling of remineralization.

The enhanced remineralization equation is:
\[ remin = r_D \cdot temp\_scale \cdot (1 + E_r \cdot (temp\_scale - 1)) \cdot D \]

where:
\begin{itemize}
\item $r_D$ is the base remineralization rate
\item $temp\_scale$ is the Arrhenius temperature scaling
\item $E_r$ is the additional temperature sensitivity
\item $D$ is the detritus concentration
\end{itemize}

This modification creates a stronger positive feedback between temperature and nutrient availability through enhanced recycling, which can affect the timing and magnitude of phytoplankton blooms. The mechanism is particularly important in capturing the system's response to seasonal temperature changes and potential climate change impacts.
