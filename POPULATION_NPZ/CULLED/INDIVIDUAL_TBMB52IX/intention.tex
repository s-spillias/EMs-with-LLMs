This model enhancement introduces luxury nutrient uptake by phytoplankton, a well-documented ecological process where cells can store excess nutrients when they are abundant. The key additions are:

1. Internal nutrient quota (Q) tracking for phytoplankton
2. Decoupling of nutrient uptake from growth
3. Growth rate dependent on internal rather than external nutrient concentration
4. Maximum storage capacity (q_max) limiting uptake

This mechanism creates a buffer against nutrient limitation, allowing phytoplankton to maintain growth even when external nutrients become scarce. It better represents the physiological adaptations of phytoplankton to variable nutrient environments and helps explain the observed higher biomass peaks in the data.

The implementation uses a Droop-style quota model where growth depends on internal nutrient status rather than external concentration. This captures the reality that phytoplankton growth can continue using stored nutrients even when external nutrients are depleted.
