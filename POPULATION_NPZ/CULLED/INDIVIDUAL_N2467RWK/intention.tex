\section{Model Description}

This model implements nutrient-dependent phytoplankton growth efficiency to better capture adaptive responses to resource limitation. When nutrients become scarce, phytoplankton can upregulate their nutrient transporters and increase their uptake efficiency. This represents a realistic physiological adaptation observed in many phytoplankton species.

The uptake efficiency ($\eta$) increases as nutrient concentrations decrease below a critical threshold:

\[ \eta = \eta_{base} + \frac{\eta_{max} - \eta_{base}}{1 + e^{k_\eta(N - N_{crit})}} \]

where:
- $\eta_{base}$ is the baseline uptake efficiency
- $\eta_{max}$ is the maximum efficiency under nutrient stress
- $k_\eta$ controls how sharply efficiency changes around $N_{crit}$
- $N_{crit}$ is the nutrient threshold where efficiency begins to increase

This mechanism allows phytoplankton to maintain growth under nutrient limitation, potentially explaining the higher observed phytoplankton peaks compared to the basic model. The sigmoid function provides a smooth transition between efficiency states while remaining biologically realistic.
