\section{Ecological Mechanism: Nutrient-Dependent Growth Efficiency}

We have incorporated a nutrient-dependent growth efficiency mechanism to better represent phytoplankton adaptation to nutrient limitation. This addition is based on empirical evidence that phytoplankton can increase their growth efficiency under nutrient stress through various physiological adaptations:

1. Metabolic restructuring to optimize resource allocation
2. Enhanced nutrient recycling within cells
3. Increased efficiency of biosynthetic pathways

The new efficiency term $\gamma_N$ takes the form:

\[ \gamma_N = 1 + (\gamma_{max} - 1) \frac{K_N}{N + K_N} \]

where $\gamma_{max}$ represents the maximum efficiency multiplier under extreme nutrient limitation. This formulation ensures that:

- Under nutrient-replete conditions $(N \gg K_N)$, $\gamma_N \approx 1$ (baseline efficiency)
- Under severe limitation $(N \ll K_N)$, $\gamma_N \approx \gamma_{max}$ (maximum efficiency)
- The transition between these states follows a saturation curve

This mechanism complements the existing uptake efficiency term $\eta_N$ by capturing a distinct physiological response: while $\eta_N$ represents the ability to acquire nutrients, $\gamma_N$ represents the efficiency of converting acquired nutrients into biomass.
