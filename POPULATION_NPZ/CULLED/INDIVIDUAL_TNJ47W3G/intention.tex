\section{Ecological Mechanism: Nutrient-Dependent Growth Efficiency}

We introduce variable nutrient use efficiency in phytoplankton growth to better represent physiological adaptations to nutrient limitation. This mechanism is based on empirical evidence that phytoplankton can increase their nutrient use efficiency when resources are scarce through various adaptations including:

\begin{itemize}
\item Increased expression of nutrient transporters
\item Optimization of cellular resource allocation
\item Reduced cellular nutrient quotas
\item Enhanced nutrient recycling within cells
\end{itemize}

The efficiency enhancement term ($1 + \frac{e_{max} K_e}{N + K_e}$) increases as nutrient concentration decreases, reaching maximum enhancement ($1 + e_{max}$) under severe limitation. This represents a compensatory mechanism that helps maintain productivity under stress, potentially explaining the higher-than-predicted initial phytoplankton bloom in the historical data.

This addition provides a more mechanistic representation of phytoplankton adaptation while maintaining model parsimony. The functional form ensures smooth transitions between efficiency states and prevents unrealistic growth enhancement when nutrients are abundant.
