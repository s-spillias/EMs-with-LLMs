\section{Ecological Mechanism: Nutrient Storage and Enhanced Growth}

We have implemented a mechanism to represent luxury nutrient uptake and storage by phytoplankton, which is a well-documented physiological adaptation. When nutrients are abundant, phytoplankton can accumulate internal nutrient stores above their immediate growth requirements. This stored nutrient pool can then support enhanced growth rates even as external nutrient concentrations decline.

The storage effect is modeled through a multiplier $\phi$ that enhances the nutrient uptake rate:

\[\phi = 1 + (\phi_{max} - 1) \frac{N}{N + k_\phi}\]

where:
\begin{itemize}
\item $\phi_{max}$ is the maximum enhancement factor (typically 1.5-2.5)
\item $k_\phi$ is the half-saturation constant for the storage effect
\item $N$ is the ambient nutrient concentration
\end{itemize}

This formulation allows for:
\begin{itemize}
\item Enhanced growth during nutrient-replete conditions ($\phi \approx \phi_{max}$ when $N \gg k_\phi$)
\item Gradual transition to baseline growth as nutrients become limiting ($\phi \approx 1$ when $N \ll k_\phi$)
\item Smooth coupling between external nutrient availability and internal storage capacity
\end{itemize}

This mechanism better represents the observed lag between nutrient depletion and decline in phytoplankton growth rates, potentially improving the model's ability to capture bloom dynamics and subsequent population trajectories.
