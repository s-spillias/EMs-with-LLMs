\section{Model Improvements}

We enhanced the NPZ (Nutrient-Phytoplankton-Zooplankton) model by incorporating temperature dependence in the nutrient recycling process. This modification is based on the well-established relationship between temperature and microbial decomposition rates in marine ecosystems.

The key changes include:

1. Implementation of the Arrhenius equation to model temperature effects on biological rates
2. Application of temperature scaling to the nutrient recycling coefficient (gamma)

This modification recognizes that nutrient regeneration through bacterial decomposition of organic matter is strongly temperature-dependent. Higher temperatures typically accelerate microbial activity and thus nutrient recycling, while lower temperatures slow these processes. This temperature dependence helps explain seasonal variations in nutrient availability and should improve the model's ability to capture observed nutrient dynamics.

The Arrhenius equation used for temperature scaling follows the form:
\[ f(T) = \exp\left(\frac{E_a}{k}\left(\frac{1}{T_{ref}} - \frac{1}{T}\right)\right) \]

where:
- $E_a$ is the activation energy
- $k$ is the Boltzmann constant
- $T_{ref}$ is the reference temperature
- $T$ is the ambient temperature

This mechanistic representation of temperature effects on nutrient cycling provides a more realistic description of the seasonal dynamics in marine ecosystems.
