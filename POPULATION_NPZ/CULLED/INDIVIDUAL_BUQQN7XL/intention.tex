\section{Ecological Motivation}

We introduce nutrient-dependent zooplankton assimilation efficiency to better represent the quality-dependent nature of predator-prey interactions in marine systems. When ambient nutrient concentrations are low, phytoplankton tend to have higher C:N ratios and lower nutritional quality, reducing the efficiency with which zooplankton can convert consumed biomass into growth.

This mechanism creates an important feedback loop: low nutrients lead to lower quality phytoplankton, which reduces zooplankton growth efficiency, potentially allowing phytoplankton populations to persist at higher levels despite apparent grazing pressure. This addition helps explain the observed higher phytoplankton peaks and subsequent zooplankton dynamics in the empirical data.

The modified assimilation efficiency ($\alpha$) is modeled as:

\[ \alpha = \alpha_{base} + \alpha_{max} \cdot \frac{N}{N + K_{\alpha}} \]

where $\alpha_{base}$ is the baseline efficiency, $\alpha_{max}$ is the maximum additional efficiency possible under optimal nutrient conditions, and $K_{\alpha}$ is the half-saturation constant for this response.
