\section{Ecological Mechanism: Nutrient-Dependent Growth Efficiency}

We introduced a competitive adaptation mechanism to better represent how phytoplankton respond to nutrient limitation. The new term $\gamma_P$ modifies nutrient uptake efficiency based on nutrient availability:

\[
\text{comp\_factor} = 1 + \gamma_P \frac{K_N}{N + K_N}
\]

This represents:
\begin{itemize}
\item Enhanced uptake efficiency when nutrients are scarce ($N \ll K_N$)
\item Return to baseline efficiency when nutrients are abundant ($N \gg K_N$)
\end{itemize}

This mechanism is ecologically justified by:
\begin{itemize}
\item Observed physiological adaptations in nutrient-limited conditions
\item Competitive pressure selecting for efficient resource use
\item Field evidence of enhanced uptake during nutrient stress
\end{itemize}

The modification should improve model fit by:
\begin{itemize}
\item Better capturing bloom dynamics when nutrients become available
\item More realistic representation of phytoplankton adaptation
\item Improved nutrient-biomass coupling in the model
\end{itemize}
