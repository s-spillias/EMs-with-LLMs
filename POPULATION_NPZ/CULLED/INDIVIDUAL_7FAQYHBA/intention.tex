\section{Ecological Model Description}

\subsection{Nutrient Storage Mechanism}

The model has been enhanced to include phytoplankton nutrient storage (luxury uptake), a well-documented ecological process where cells can accumulate nutrients beyond their immediate growth requirements. This mechanism allows phytoplankton to:

\begin{itemize}
\item Buffer against nutrient fluctuations by storing excess nutrients when available
\item Continue growth for some time after external nutrients become scarce
\item Better compete in variable nutrient environments
\end{itemize}

The implementation uses a variable internal quota (Q) bounded by physiological limits (Q_min, Q_max). Growth rate depends on the internal quota rather than external nutrients, while nutrient uptake is regulated by both external concentration and current storage status. This more realistically represents how phytoplankton interact with their nutrient environment.

This mechanism should improve model predictions by:
\begin{itemize}
\item Allowing higher phytoplankton peaks through stored nutrients
\item Creating more realistic time delays between nutrient uptake and growth
\item Better capturing competitive dynamics in variable environments
\end{itemize}
