\section{Adaptive Nutrient Use Efficiency}

This model enhancement introduces variable nutrient uptake efficiency for phytoplankton, representing their ability to adapt to nutrient-limited conditions. When nutrients become scarce, phytoplankton can upregulate their nutrient transporters and modify their metabolic pathways to increase uptake efficiency.

The efficiency boost term (e_max * K_e / (N + K_e)) increases as nutrient concentrations decrease, reaching maximum (e_max) under severe limitation and approaching zero when nutrients are abundant. This creates a compensatory mechanism that helps maintain productivity under stress while remaining energetically efficient when resources are plentiful.

This addition better captures the physiological plasticity of phytoplankton and their ability to partially compensate for resource limitation - a key ecological process missing from the original model. The functional form ensures smooth transitions between efficiency states while maintaining biological realism through bounded responses.
