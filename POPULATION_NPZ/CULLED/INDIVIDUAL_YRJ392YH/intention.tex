\section{Nutrient-Dependent Photosynthetic Efficiency}

We enhanced the model by incorporating nutrient-dependent photosynthetic efficiency, representing how phytoplankton can optimize their photosynthetic machinery based on nutrient availability. The photosynthetic efficiency term ($1 + \eta_N \frac{N}{N + K_N}$) scales the maximum growth rate, allowing for up to $(1 + \eta_N)$ times higher efficiency under nutrient-replete conditions.

This mechanism is ecologically justified because:
\begin{itemize}
\item Phytoplankton require nutrients (especially N and P) to build and maintain photosynthetic proteins and pigments
\item Under nutrient-rich conditions, cells can invest in more photosynthetic machinery
\item This creates a positive feedback where better nutrient conditions enable more efficient growth
\item The half-saturation formulation ensures a smooth transition between nutrient-limited and nutrient-replete states
\end{itemize}

This addition should improve the model's ability to capture:
\begin{itemize}
\item Higher peak phytoplankton abundances under nutrient-rich conditions
\item More realistic timing of bloom development
\item Better representation of the coupling between nutrient availability and primary production
\end{itemize}
