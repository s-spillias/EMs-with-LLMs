This model incorporates luxury nutrient uptake by phytoplankton, a well-documented ecological process where cells can store excess nutrients when they are abundant. This storage capacity is implemented through a variable internal nutrient quota mechanism, where uptake and growth are decoupled. The uptake rate increases with external nutrient concentration but decreases as internal stores approach maximum capacity (negative feedback). Growth rate then depends on the internal nutrient quota rather than external concentration, allowing continued growth even when external nutrients become scarce.

This better represents the biological reality that phytoplankton can buffer against nutrient fluctuations through internal storage, leading to more realistic bloom dynamics and predator-prey interactions. The mechanism helps explain how phytoplankton can achieve and sustain higher peak abundances during blooms despite apparent nutrient limitations in the external environment.
