\section{Ecological Mechanisms}

We have implemented internal nutrient storage (luxury uptake) in phytoplankton using the Droop model framework. This better represents how phytoplankton can decouple nutrient uptake from growth by storing excess nutrients when they are abundant and using these reserves during periods of scarcity.

Key mechanisms:
\begin{itemize}
    \item Nutrient uptake is regulated by the current internal quota (Q) relative to maximum storage capacity (Q\_max)
    \item Growth rate depends on internal quota rather than external nutrients, with a minimum quota (Q\_min) required for survival
    \item This creates a time lag between nutrient uptake and biomass production
    \item Allows phytoplankton to maintain growth during temporary nutrient limitation
\end{itemize}

This mechanism should improve model predictions by:
\begin{itemize}
    \item Better capturing the magnitude of phytoplankton blooms
    \item Creating more realistic phase relationships between nutrients and biomass
    \item Representing the observed persistence of phytoplankton during nutrient-depleted periods
\end{itemize}
