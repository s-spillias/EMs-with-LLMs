This model enhancement introduces nutrient-dependent phytoplankton growth efficiency to better capture adaptive responses to resource limitation. When nutrients are scarce, phytoplankton can increase their nutrient use efficiency through various physiological mechanisms including:

1. Upregulation of high-affinity nutrient transporters
2. Increased allocation to nutrient acquisition machinery
3. More efficient internal nutrient recycling
4. Reduced luxury nutrient consumption

The enhancement factor eta increases as nutrient concentration decreases, following a hyperbolic relationship similar to Michaelis-Menten kinetics. This allows phytoplankton to maintain higher growth rates under nutrient limitation than would be predicted by simple uptake kinetics alone.

The maximum efficiency enhancement eta_max represents the upper limit of these adaptive capabilities. Literature values suggest phytoplankton can increase their nutrient use efficiency by 20-60% under severe limitation.

This mechanism should improve model predictions particularly during nutrient-limited periods, allowing for higher phytoplankton biomass accumulation despite low ambient nutrient concentrations.
