\section{Ecological Mechanism: Adaptive Nutrient Storage}

The model has been enhanced to include internal nutrient quotas in phytoplankton cells, implementing the Droop model concept of variable internal stores. This better represents how phytoplankton can decouple nutrient uptake from growth by storing excess nutrients when available.

Key mechanisms added:
\begin{itemize}
    \item Internal quota (Q) calculated as N:P ratio
    \item Uptake rate decreases as cells approach maximum quota (Q\_max)
    \item Growth rate depends on internal stores exceeding minimum quota (Q\_min)
    \item Allows temporal separation between nutrient acquisition and biomass growth
\end{itemize}

This mechanism helps explain the higher observed biomass peaks in the data, as cells can continue growing even when external nutrients become scarce by utilizing internal stores accumulated during nutrient-replete conditions. The quota-based limitation provides more realistic nutrient-growth dynamics than simple Monod kinetics.
