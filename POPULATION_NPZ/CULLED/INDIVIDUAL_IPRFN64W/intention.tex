This modification implements the Droop model of nutrient-limited phytoplankton growth, which separates nutrient uptake from growth using an internal nutrient quota mechanism. This better represents how phytoplankton actually respond to changing nutrient conditions in several ways:

1. Nutrient Storage: Phytoplankton can accumulate nutrients beyond their immediate growth requirements when nutrients are abundant, creating a buffer against future limitation.

2. Growth-Uptake Decoupling: Growth rate depends on the internal nutrient status rather than external concentration, allowing continued growth for some time after external nutrients are depleted.

3. Uptake Regulation: Cells reduce their nutrient uptake as they approach their maximum storage capacity, representing a more realistic physiological response.

The key equations are:
- Nutrient uptake: v(N,Q) = v_max * (1 - (Q-Q_min)/(Q_max-Q_min)) * N/(K_N + N)
- Growth rate: μ(Q) = r_max * (1 - Q_min/Q)
- Quota change: dQ/dt = v(N,Q) - μ(Q)*Q

This mechanism should improve the model's ability to capture:
- More realistic time delays between nutrient availability and growth response
- Better representation of growth during nutrient transitions
- More accurate peak biomass predictions due to storage effects
