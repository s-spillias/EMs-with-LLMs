\section{Ecological Mechanism: Nutrient Storage}

The model has been enhanced to include luxury uptake of nutrients by phytoplankton, a well-documented ecological process where cells can store excess nutrients when they are abundant. This storage acts as a buffer against nutrient limitation and better represents the temporal dynamics of phytoplankton growth.

Key aspects of the implementation:

\begin{itemize}
    \item Internal nutrient quota (Q) tracks stored nutrients within cells
    \item Uptake rate decreases as cells approach their maximum storage capacity (Q_max)
    \item Growth rate depends on internal rather than external nutrient concentration
    \item Minimum quota (Q_min) represents essential structural nutrients
\end{itemize}

This mechanism creates important feedbacks:
\begin{itemize}
    \item Delayed response to nutrient limitation due to internal stores
    \item More realistic nutrient uptake dynamics that depend on cell status
    \item Better representation of the decoupling between nutrient uptake and growth
\end{itemize}

The addition of nutrient storage should improve model predictions by:
\begin{itemize}
    \item Allowing higher phytoplankton peaks through stored nutrients
    \item Creating more realistic time delays in population responses
    \item Better capturing the relationship between nutrient availability and growth
\end{itemize}
