This model incorporates variable nutrient storage in phytoplankton cells, representing luxury uptake and storage during nutrient-replete conditions and subsequent utilization during nutrient-poor periods. This mechanism is implemented through a quota-based growth model where:

1. Nutrient uptake and growth are separated processes
2. Maximum uptake rate increases when internal stores are depleted (up-regulation)
3. Growth depends on internal rather than external nutrient concentration
4. A minimum cell quota (Q_min) represents the structural nutrients required for survival
5. A maximum quota (Q_max) limits luxury consumption

This better represents the biological reality that phytoplankton can buffer against nutrient fluctuations through internal storage, rather than responding instantaneously to external nutrient levels. This mechanism helps explain the observed lag in phytoplankton decline after nutrient depletion and provides a more mechanistic basis for nutrient limitation effects.
