\section{Ecological Mechanism: Temperature-Dependent Bacterial Remineralization}

We enhanced the model's representation of nutrient cycling by implementing a Q10 temperature-dependent bacterial remineralization pathway. This better captures how warming accelerates the bacterial breakdown of detrital organic matter back into bioavailable nutrients.

The Q10 relationship is a well-established pattern in microbial ecology where bacterial metabolic rates approximately double with every 10°C increase in temperature. This temperature sensitivity is typically stronger than the Arrhenius relationship used for other biological rates in the model.

This mechanism creates an important feedback loop: warmer temperatures increase bacterial activity, which accelerates nutrient recycling, potentially fueling larger phytoplankton blooms. This addition helps explain the observed rapid nutrient regeneration and bloom dynamics in the empirical data.

The modification uses a Q10 formulation:
\[ \text{bact\_scale} = Q10^{(T - T_{ref})/10} \]
where T is the current temperature and T_{ref} is a reference temperature (15°C).

This scaling factor then modifies the base remineralization rate:
\[ \text{remin} = r_D \cdot \text{bact\_scale} \cdot D \]
