\section{Ecological Model Improvements}

\subsection{Internal Nutrient Storage}

The original model assumed instantaneous nutrient uptake and growth coupling, which is unrealistic for phytoplankton. In nature, phytoplankton can store excess nutrients internally (luxury uptake) and continue growing even when external nutrients become scarce. This storage capacity helps explain:

1. Larger bloom magnitudes - cells can accumulate nutrients during pre-bloom conditions
2. Sustained growth during nutrient depletion - stored nutrients buffer against external scarcity
3. More realistic nutrient-growth decoupling - growth depends on internal rather than external nutrients

The modified model implements this by:
- Tracking internal nutrient quota Q (g N/g C)
- Making uptake dependent on external nutrients but inversely related to current quota
- Making growth dependent on internal quota rather than external nutrients
- Maintaining quota bounds (Q_min to Q_max) based on physiological limits

This better represents the temporal dynamics of nutrient-phytoplankton interactions and should improve prediction of bloom timing and magnitude.
