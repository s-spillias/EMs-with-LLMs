\section{Model Enhancement: Temperature-Dependent Grazing Selectivity}

We enhanced the ecological model by incorporating temperature-dependent grazing selectivity, representing how temperature affects zooplankton feeding behavior and capture efficiency. This modification is based on empirical evidence that zooplankton grazing rates don't just scale metabolically with temperature, but their prey selection and capture success also vary with temperature.

The new mechanism is implemented through a selectivity coefficient that modifies the grazing function:

\[ \text{select}_T = e^{\theta_g(T - T_{ref})} \]

where:
\begin{itemize}
\item $\theta_g$ is the temperature sensitivity parameter
\item $T$ is the ambient temperature
\item $T_{ref}$ is the reference temperature (20°C)
\end{itemize}

This affects the grazing term in the model:

\[ \text{grazing} = g_{max} \cdot f(T) \cdot \text{select}_T \cdot \frac{P \cdot Z}{K_P + P} \]

The modification allows for:
\begin{itemize}
\item More efficient grazing at optimal temperatures
\item Reduced feeding efficiency at suboptimal temperatures
\item Better representation of seasonal zooplankton-phytoplankton coupling
\end{itemize}

This change is ecologically justified as many zooplankton species show temperature-dependent changes in their feeding behavior, beyond just metabolic scaling. This can include changes in:
\begin{itemize}
\item Swimming speed and encounter rates
\item Prey handling time
\item Capture success rate
\item Selective feeding behavior
\end{itemize}
