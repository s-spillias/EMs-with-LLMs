\section{Ecological Intention: Internal Nutrient Storage}

This model enhancement introduces luxury uptake and storage of nutrients by phytoplankton, a well-documented ecological mechanism. Many phytoplankton species can accumulate nutrients beyond their immediate growth requirements when nutrients are abundant, storing them for later use during periods of scarcity.

The implementation uses a quota-based approach where:
\begin{itemize}
    \item Q represents the variable internal nutrient:carbon ratio
    \item Q\_max caps maximum storage capacity
    \item Q\_min represents the minimum quota needed for survival
    \item Uptake rate decreases as internal stores fill up (quota\_limit term)
    \item Growth becomes limited when internal quota approaches Q\_min
\end{itemize}

This mechanism creates a buffer against nutrient limitation, allowing phytoplankton to maintain growth even as external nutrients decline. It better represents the physiological plasticity of phytoplankton and their ability to exploit temporal variability in nutrient availability.

The addition of internal stores introduces an important time delay between nutrient uptake and biomass production, which can help explain observed phase shifts between nutrient depletion and peak biomass in natural systems.
