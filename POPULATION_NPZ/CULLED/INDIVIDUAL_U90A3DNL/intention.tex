\section{Ecological Mechanisms}

We enhanced the model's representation of phytoplankton nutrient uptake by incorporating physiological adaptation to nutrient limitation. The new formulation adds a term that increases uptake efficiency under nutrient stress, representing mechanisms like:

\begin{itemize}
\item Upregulation of nutrient transporters
\item Changes in cell quota
\item Increased surface area to volume ratios
\item Expression of high-affinity uptake systems
\end{itemize}

The adaptation response follows a sigmoidal function that increases as nutrient concentrations decrease, with maximum effect at severe limitation. This better captures how phytoplankton optimize their resource acquisition strategy under stress, potentially explaining the higher biomass peaks observed in the data.

The mathematical form $(K_N^2)/(N^2 + K_N^2)$ ensures the adaptation effect:
\begin{itemize}
\item Approaches 1 when N << K_N (strong limitation)
\item Approaches 0 when N >> K_N (nutrient replete)
\item Has smooth transition around K_N
\end{itemize}

This mechanism is supported by extensive literature on phytoplankton physiological plasticity and provides a more realistic representation of how these organisms respond to resource limitation.
