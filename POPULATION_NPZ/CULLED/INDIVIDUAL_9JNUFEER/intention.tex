\section{Ecological Justification for Variable Internal Nutrient Quota}

The model has been enhanced to include variable internal nutrient storage by phytoplankton, known as luxury uptake. This fundamental ecological process allows phytoplankton to decouple nutrient uptake from growth, storing excess nutrients when they are abundant and using these reserves to maintain growth when external nutrients become scarce.

The implementation follows Droop's cell quota model, where:
\begin{itemize}
    \item Growth rate depends on internal rather than external nutrient concentration
    \item Nutrient uptake rate decreases as internal stores approach maximum capacity
    \item Minimum quota (Q\_min) represents the subsistence quota below which no growth occurs
    \item Maximum quota (Q\_max) represents storage capacity under luxury uptake
\end{itemize}

This mechanism better represents:
\begin{itemize}
    \item The ability of phytoplankton to buffer against nutrient fluctuations
    \item More realistic nutrient limitation of growth
    \item Competitive dynamics under variable nutrient conditions
    \item The observed lag between nutrient uptake and biomass increase
\end{itemize}

The modification should improve model performance by:
\begin{itemize}
    \item Better capturing the magnitude of phytoplankton blooms
    \item Producing more realistic phase relationships between nutrients and biomass
    \item Generating more accurate nutrient-biomass ratios
    \item Better representing adaptation to nutrient pulses
\end{itemize}
