\section{Nutrient-Dependent Growth Efficiency}

We introduce nutrient-dependent phytoplankton growth efficiency to better represent how phytoplankton cells optimize their resource utilization under different nutrient conditions. The growth efficiency term $\phi_N$ is modeled as:

\[\phi_N = 1 + \phi_{max} \frac{N}{N + K_\phi}\]

where $\phi_{max}$ is the maximum additional efficiency possible under nutrient-replete conditions, and $K_\phi$ is the half-saturation constant for this response.

This mechanism represents how phytoplankton can invest in more efficient photosynthetic and nutrient uptake machinery when resources are abundant, while maintaining minimal efficiency under nutrient stress. The base efficiency of 1 ensures some growth is still possible under nutrient limitation.

This addition helps explain:
\begin{itemize}
\item Higher growth rates during nutrient-replete conditions (initial bloom)
\item More realistic transition dynamics as nutrients become depleted
\item Better coupling between nutrient availability and biomass production
\end{itemize}

The mechanism is supported by studies showing that phytoplankton can adjust their cellular machinery and resource allocation based on nutrient availability.
