\section{Ecological Mechanisms}

A key improvement to the model is the incorporation of nutrient-dependent zooplankton mortality that accounts for food quality effects. The original model included a simple nutrient stress term, but the new formulation better represents how nutrient limitation affects zooplankton through multiple pathways:

1. Direct nutrient stress remains through the term $m_{Z_N} \frac{K_N}{N + K_N}$

2. Food quality is now explicitly modeled using the nutrient:phytoplankton ratio ($\frac{N}{P}$) as a proxy. This represents the nutritional value of phytoplankton prey, which declines under nutrient limitation.

3. The combined effect scales mortality with both ambient nutrient levels and food quality: poor food quality amplifies the negative effects of nutrient stress.

This mechanism creates an important feedback loop: nutrient limitation reduces not just phytoplankton growth but also the quality of phytoplankton as food for zooplankton. This can help explain the observed patterns in zooplankton dynamics, particularly during periods of nutrient stress.

The formulation uses a Holling Type II response to food quality, similar to the grazing function, reflecting that the negative effects of poor food quality saturate at very low nutrient:phytoplankton ratios. This is ecologically justified as zooplankton have minimum nutritional requirements and behavioral/physiological adaptations to cope with poor food quality.
