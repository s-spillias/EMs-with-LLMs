\section{Ecological Mechanism: Variable Nutrient Storage}

We have implemented a variable nutrient storage mechanism for phytoplankton using the Droop model framework. This better represents how phytoplankton cells can maintain growth even when external nutrients become scarce, through internal nutrient reserves.

The key additions are:
\begin{itemize}
    \item Internal nutrient quota (Q) tracking the N:P ratio
    \item Minimum (Q\_min) and maximum (Q\_max) quota constraints
    \item Growth rate now depends on internal quota rather than external nutrients
    \item Uptake rate still depends on external nutrients
\end{itemize}

This mechanism creates a more realistic time delay between nutrient depletion and growth reduction, as observed in natural systems. It allows phytoplankton to continue growing for some time after external nutrients are depleted, using stored reserves. This better matches the observed patterns in the comparison plots, particularly the smoother transitions in phytoplankton biomass during nutrient-limited conditions.

The quota-based approach represents a fundamental biological process in phytoplankton ecology and has strong empirical support from laboratory and field studies (Droop 1968, Klausmeier et al. 2004).
