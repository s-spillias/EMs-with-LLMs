\section{Ecological Mechanism: Density-Dependent Nutrient Uptake}

We enhanced the model by incorporating density-dependent nutrient uptake efficiency for phytoplankton. This mechanism represents how phytoplankton cells can adapt their nutrient acquisition strategies based on both nutrient availability and population density.

The key addition is the density_factor term:
$1 + \eta_{adapt} (1 - \frac{P}{P + K_P})$

This factor increases nutrient uptake efficiency when:
1. Population density (P) is low, allowing more efficient resource use during population recovery
2. Nutrient concentrations are limiting, representing physiological adaptation

This matches ecological observations where phytoplankton show enhanced per-capita nutrient uptake during early bloom stages and under resource limitation. The mechanism helps explain the rapid population growth during spring blooms while maintaining realistic nutrient dynamics during other periods.

The density dependence creates an important feedback: as populations grow, per-capita efficiency decreases, helping stabilize the dynamics. This better captures the self-limiting nature of phytoplankton blooms observed in natural systems.
