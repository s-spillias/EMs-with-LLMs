\section{Ecological Mechanisms}

\subsection{Nutrient-Dependent Grazing Efficiency}

The model has been enhanced to include nutrient-dependent zooplankton grazing efficiency. This modification reflects the ecological observation that zooplankton grazing rates are often higher when feeding on nutrient-replete phytoplankton, which are more nutritious prey items. The mechanism is implemented through a multiplier on the grazing rate that increases with ambient nutrient concentrations:

\[ g_N = 1 + g_{N_{max}} \frac{N}{N + K_N} \]

where $g_{N_{max}}$ is the maximum enhancement to grazing rate under nutrient-replete conditions, and $K_N$ is the same half-saturation constant used in nutrient uptake.

This represents a more realistic coupling between nutrient conditions and trophic transfer efficiency, as zooplankton can adjust their grazing rates based on prey quality. The mechanism helps explain the initial zooplankton peak observed in the data, as high nutrient conditions support both higher phytoplankton biomass and more efficient zooplankton grazing.
