\section{Ecological Mechanism: Internal Nutrient Storage}

The model has been enhanced to include internal nutrient storage dynamics in phytoplankton cells using the Droop model framework. This better represents how phytoplankton respond to nutrient availability in three key ways:

1. \textbf{Nutrient Storage}: Cells can now store excess nutrients when available, creating a buffer against limitation. This matches the biological reality that phytoplankton don't immediately stop growing when external nutrients become scarce.

2. \textbf{Growth-Quota Relationship}: Growth rate is now controlled by the internal nutrient quota (Q) rather than external concentration. The Droop equation $\mu = \mu_{max}(1 - Q_{min}/Q)$ captures how growth gradually declines as internal stores are depleted.

3. \textbf{Uptake Regulation}: Nutrient uptake is regulated by both external availability and internal status. The $(Q_{max} - Q)/(Q_{max} - Q_{min})$ term reduces uptake as cells become nutrient-replete, representing physiological down-regulation of uptake systems.

This mechanism creates more realistic time lags between nutrient depletion and growth limitation, which should better capture the dynamics seen in the historical data, particularly the timing and magnitude of phytoplankton peaks.
