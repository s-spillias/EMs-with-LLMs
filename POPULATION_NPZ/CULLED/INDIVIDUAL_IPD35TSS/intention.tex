\section{Model Enhancement: Nutrient-Dependent Photosynthetic Efficiency}

This model enhancement introduces nutrient-dependent photosynthetic efficiency to better capture the relationship between nutrient availability and phytoplankton growth. The modification is based on the ecological understanding that phytoplankton can adjust their photosynthetic machinery in response to nutrient conditions.

The key addition is the factor $\phi_N = 1 + \phi_{max} \frac{N}{N + K_N}$, which modulates photosynthetic efficiency based on ambient nutrient concentrations. When nutrients are scarce, $\phi_N$ approaches 1, representing baseline efficiency. As nutrients become more abundant, efficiency increases up to a maximum enhancement of $(1 + \phi_{max})$.

This mechanism represents several real ecological processes:
\begin{itemize}
    \item Enhanced chlorophyll synthesis under nutrient-replete conditions
    \item Improved light-harvesting complex assembly with sufficient nutrients
    \item More efficient electron transport chains when nutrients are available
\end{itemize}

The enhancement helps explain the observed rapid bloom development and subsequent decline seen in the data, as phytoplankton can more effectively utilize light energy when nutrients are abundant but become less efficient as nutrients are depleted.
