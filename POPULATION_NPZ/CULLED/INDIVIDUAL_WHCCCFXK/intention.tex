This modification introduces a nutrient history effect into phytoplankton nutrient uptake efficiency, representing physiological adaptation to changing nutrient conditions. The key ecological reasoning behind this change is:

1. Phytoplankton can modify their nutrient uptake machinery based on both current and recent nutrient conditions
2. This adaptation occurs over characteristic timescales (tau_N) as cells adjust their physiological state
3. The relative importance of nutrient history (w_hist) reflects the balance between maintaining existing uptake machinery and developing new capabilities

The new equations track a weighted nutrient history that exponentially decays toward current conditions:

dN_hist/dt = (N - N_hist)/tau_N

The uptake efficiency now depends on both immediate and historical nutrient levels:

eta_N = eta_base + (eta_max - eta_base) * ((1-w_hist)*current_response + w_hist*history_response)

This better represents how phytoplankton maintain their uptake capabilities even as nutrients become depleted, potentially explaining the observed higher initial bloom magnitude. The history effect also introduces a biologically realistic lag in efficiency changes that could improve the timing and magnitude of subsequent oscillations.
