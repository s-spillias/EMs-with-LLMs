\section{Ecological Mechanism: Temperature-Dependent Nutrient Recycling}

The model has been enhanced to better represent the temperature dependence of bacterial remineralization processes. This modification is based on the following ecological principles:

\begin{itemize}
    \item Bacterial metabolism, which drives nutrient recycling, typically shows stronger temperature sensitivity than photosynthetic processes
    \item This creates a positive feedback loop where warmer temperatures accelerate nutrient recycling, potentially supporting higher primary production
    \item The activation energy (E_r) for bacterial processes is set higher than that for photosynthesis (E_p) based on empirical studies
\end{itemize}

This mechanism helps explain the observed early-season dynamics where rapid nutrient recycling may support the sharp initial phytoplankton bloom. The separate temperature scaling for remineralization versus photosynthesis creates more realistic temporal patterns in nutrient availability.

The modification uses the Arrhenius equation with a higher activation energy (0.95 eV vs 0.45 eV for photosynthesis) to capture the stronger temperature response of bacterial processes, while maintaining bounded scaling to prevent numerical instabilities.
