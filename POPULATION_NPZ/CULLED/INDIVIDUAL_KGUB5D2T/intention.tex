\section{Adaptive Nutrient Uptake}

The model incorporates nutrient-dependent phytoplankton growth efficiency to better represent physiological adaptations in nutrient acquisition. Many phytoplankton species can modify their nutrient uptake mechanisms based on ambient conditions through:

\begin{itemize}
\item Increased transporter protein expression under nutrient limitation
\item Luxury uptake and storage when nutrients are abundant
\item Adjustments to cellular nutrient quotas
\end{itemize}

This adaptive capability is modeled using a sigmoid response function that increases uptake efficiency under high nutrient conditions, representing the ability to capitalize on nutrient pulses through enhanced uptake and storage. The functional form ensures smooth transitions between efficiency states while maintaining biological realism in the parameter ranges.

This mechanism helps explain the observed higher phytoplankton peaks during bloom conditions, as cells can more efficiently convert available nutrients into biomass when resources are abundant. The gradual efficiency response also contributes to more realistic bloom development and decline dynamics.
