\section{Ecological Mechanism: Adaptive Nutrient Uptake Efficiency}

We have implemented a variable nutrient uptake efficiency mechanism to better represent phytoplankton physiological adaptation to nutrient availability. This modification is based on the following ecological principles:

\begin{itemize}
    \item Phytoplankton can modify their nutrient uptake machinery in response to ambient nutrient concentrations
    \item Under low nutrient conditions, cells can increase their uptake efficiency by producing more nutrient transporters
    \item At very high nutrient concentrations, uptake becomes saturated as cellular machinery reaches maximum capacity
\end{itemize}

The new uptake efficiency term $\eta_N$ follows a sigmoid function:

\[ \eta_N = \frac{\eta_{max}}{1 + e^{-k_\eta(N - N_{crit})}} \]

where:
\begin{itemize}
    \item $\eta_{max}$ is the maximum uptake efficiency multiplier
    \item $k_\eta$ controls how sharply efficiency changes around the critical concentration
    \item $N_{crit}$ is the nutrient concentration where efficiency response is centered
\end{itemize}

This formulation allows phytoplankton to:
\begin{itemize}
    \item Maintain growth under low nutrient conditions through increased efficiency
    \item Avoid excessive nutrient uptake when concentrations are high
    \item Smoothly transition between efficiency states as conditions change
\end{itemize}

The mechanism helps explain the observed higher-than-predicted initial phytoplankton bloom in the data, as cells can achieve higher growth rates through increased uptake efficiency when nutrients are abundant.
