This model incorporates nutrient-dependent phytoplankton growth efficiency to better represent adaptive nutrient uptake strategies. When nutrients become scarce, phytoplankton can increase their uptake efficiency through various physiological adaptations like:

1. Upregulation of nutrient transporters
2. Changes in cell surface area to volume ratios
3. Enhanced enzyme production for nutrient acquisition

The uptake efficiency (eta) follows a sigmoid response to nutrient concentration:
eta = eta_base + (eta_max - eta_base)/(1 + exp(-k_eta * (N - N_crit)))

Where:
- eta_base is the baseline efficiency
- eta_max is the maximum efficiency under optimal conditions
- k_eta controls how sharply efficiency changes around N_crit
- N_crit is the nutrient concentration where the response is centered

This allows phytoplankton to maintain growth at lower nutrient concentrations through increased uptake efficiency, while preventing unrealistic growth rates at high nutrient levels. The sigmoid function provides a smooth transition between efficiency states, reflecting the gradual physiological changes that occur as nutrients become limiting.
