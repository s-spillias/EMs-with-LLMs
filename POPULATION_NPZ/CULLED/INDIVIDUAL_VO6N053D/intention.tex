\section{Ecological Justification for Adaptive Nutrient Uptake}

The model has been enhanced to include adaptive nutrient uptake efficiency by phytoplankton, representing a key physiological response to resource limitation. This modification is based on extensive empirical evidence that phytoplankton can modify their nutrient acquisition strategies under limiting conditions.

The new formulation uses a negative Hill function to model how uptake efficiency (η) increases at low nutrient concentrations:

η = η_base + (η_max - η_base) * K_N^n / (K_N^n + N^n)

where:
- η_base is the baseline efficiency
- η_max is the maximum efficiency
- n is the Hill coefficient controlling response steepness
- K_N is the half-saturation constant

This represents:
1. Increased expression of nutrient transporters under limitation
2. Changes in cellular nutrient quotas and storage
3. Physiological trade-offs in resource allocation

The Hill function form captures the saturating nature of this response and allows for a more rapid transition between efficiency states than the previous linear formulation. This better reflects the observed ability of phytoplankton to rapidly adjust their uptake machinery in response to nutrient stress.

This mechanism is particularly important for capturing:
- More realistic nutrient uptake dynamics at low concentrations
- Enhanced competitive ability under limiting conditions
- Population persistence during resource-scarce periods
