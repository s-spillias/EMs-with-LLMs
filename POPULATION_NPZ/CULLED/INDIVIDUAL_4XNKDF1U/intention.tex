\section{Model Description}

This model incorporates nutrient-dependent phytoplankton growth efficiency to better represent physiological adaptations to resource limitation. When nutrients become scarce, phytoplankton can upregulate their nutrient transporters and increase their uptake efficiency, allowing them to maintain growth rates even as resource availability declines. This adaptation is represented through a sigmoid function that increases uptake efficiency as nutrient concentrations decrease below a critical threshold.

The enhanced uptake efficiency term $\eta_N$ is calculated as:

\[ \eta_N = \eta_{base} + \frac{\eta_{max} - \eta_{base}}{1 + e^{-k_{\eta}(N - N_{crit})}} \]

where:
- $\eta_{base}$ is the baseline uptake efficiency
- $\eta_{max}$ is the maximum efficiency under nutrient limitation
- $k_{\eta}$ controls how sharply efficiency changes around $N_{crit}$
- $N_{crit}$ is the nutrient concentration threshold that triggers efficiency changes

This mechanism allows phytoplankton to respond more dynamically to nutrient limitation, potentially explaining the observed rapid bloom development and subsequent nutrient drawdown seen in the empirical data.
