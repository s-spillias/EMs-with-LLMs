The model has been enhanced to include prey-dependent predator feeding efficiency. This ecological mechanism reflects how zooplankton feeding success improves at higher phytoplankton densities due to:

1. Enhanced prey detection probability
2. Reduced handling time per prey item
3. More efficient capture techniques at higher prey densities
4. Potential for switching to more selective feeding strategies

The modification adds a prey-density dependent term to the assimilation efficiency (alpha_P), which combines with the existing nutrient-dependent efficiency (alpha_N). This better represents how predator-prey interactions change across different prey densities, a key mechanism in marine food webs documented in numerous studies.

The total efficiency is capped at 0.95 to maintain biological realism, as no trophic transfer can be 100% efficient due to fundamental thermodynamic constraints.

This change should improve the model's ability to capture peak zooplankton abundances when phytoplankton densities are high, while maintaining reasonable behavior at lower prey densities.
