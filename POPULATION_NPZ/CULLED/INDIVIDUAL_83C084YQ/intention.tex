This model incorporates luxury nutrient uptake by phytoplankton - their ability to take up and store excess nutrients when available. This better represents how phytoplankton can maintain growth even when external nutrients become scarce, by drawing on internal reserves.

The key additions are:
- Tracking internal nutrient quota (Q) in phytoplankton cells
- Separating nutrient uptake from growth (Droop kinetics)
- Making growth dependent on internal rather than external nutrients
- Adding physiological limits on minimum and maximum quotas

This mechanism is ecologically important because:
1. It creates a time lag between nutrient uptake and growth
2. It allows phytoplankton to buffer against nutrient fluctuations
3. It better represents the true physiology of nutrient-limited growth
4. It can explain continued growth after external nutrients are depleted

The Droop equation relates growth to internal quota: μ = μmax(1 - Qmin/Q)
This captures how growth approaches zero as Q approaches Qmin, and saturates at μmax.
