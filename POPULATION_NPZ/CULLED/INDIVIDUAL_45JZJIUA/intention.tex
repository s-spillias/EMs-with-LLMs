\section{Model Enhancement: Photoacclimation}

This enhancement adds photoacclimation to better capture phytoplankton growth dynamics under varying light conditions. Photoacclimation is the process by which phytoplankton adjust their photosynthetic apparatus in response to changes in light intensity.

The key additions are:

1. A new parameter tau_a (day^-1) representing the rate at which phytoplankton can adjust their optimal light level.

2. A dynamic optimal light level (I_opt_acc) that tracks the effective light intensity experienced by the phytoplankton, replacing the fixed I_opt in the light limitation calculation.

This mechanism allows phytoplankton to:
- Optimize their photosynthetic efficiency under changing light conditions
- Maintain higher growth rates when light levels change gradually
- Better capture the observed rapid growth phases seen in the data

The modification improves the ecological realism of the model by representing the well-documented ability of phytoplankton to physiologically adapt to their light environment. This is particularly important in seasonal systems where light availability changes significantly over time.
