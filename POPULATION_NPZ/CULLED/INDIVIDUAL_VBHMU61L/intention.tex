\section{Ecological Mechanism: Nutrient-Dependent Growth Efficiency}

We have enhanced the model to include physiological adaptation of phytoplankton to nutrient limitation. In natural systems, phytoplankton can modify their nutrient uptake efficiency when resources become scarce through various mechanisms including:

\begin{itemize}
\item Increasing the number or efficiency of nutrient transporters
\item Adjusting internal resource allocation
\item Modifying cell quotas and stoichiometry
\end{itemize}

This adaptation is represented by the term $\eta_N = \eta_{max} \frac{K_N}{N + K_N}$, which increases the growth rate under nutrient limitation. The form ensures that:

\begin{itemize}
\item Efficiency enhancement is strongest when nutrients are most limiting
\item The response saturates at a maximum additional efficiency $\eta_{max}$
\item The effect smoothly transitions as nutrient levels change
\end{itemize}

This mechanism allows phytoplankton to partially compensate for nutrient limitation, potentially improving the model's ability to capture the observed initial bloom dynamics and subsequent oscillations. The modification maintains mathematical tractability while incorporating an important biological feedback process.
