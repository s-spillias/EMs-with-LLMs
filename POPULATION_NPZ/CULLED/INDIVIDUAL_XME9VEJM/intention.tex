\section{Ecological Justification for Nutrient-Dependent Remineralization}

The model has been enhanced to include nutrient-dependent bacterial remineralization efficiency, representing an important feedback mechanism in marine ecosystems. When nutrients are scarce, bacterial communities become more efficient at recycling organic matter to access limiting nutrients, while under nutrient-rich conditions, they exhibit lower recycling efficiency as there is less selective pressure for conservative nutrient use.

This modification captures the adaptive behavior of marine bacterial communities and provides a stabilizing feedback: during nutrient limitation, enhanced recycling efficiency helps maintain nutrient availability, while during nutrient-rich conditions, reduced efficiency prevents excessive nutrient accumulation.

The functional form uses a sigmoid response (similar to other nutrient-dependent processes in the model) to represent the transition between low and high efficiency states, with parameters derived from empirical studies of marine bacterial communities. This addition better represents the microbial loop's role in nutrient cycling and should improve the model's ability to capture ecosystem dynamics, particularly during transitions between nutrient-limited and nutrient-replete conditions.
