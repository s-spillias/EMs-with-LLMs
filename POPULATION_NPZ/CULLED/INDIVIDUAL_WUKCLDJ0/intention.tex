\section{Ecological Mechanisms}

A key addition to the model is the incorporation of luxury nutrient uptake by phytoplankton through an internal nutrient quota mechanism. This better represents how phytoplankton actually process nutrients in nature:

\begin{itemize}
\item Phytoplankton can take up nutrients faster than they use them for growth when nutrients are abundant
\item The stored nutrients can support growth even when external nutrients become scarce
\item Uptake slows as internal stores fill up (negative feedback)
\item Growth depends on internal rather than external nutrient concentrations
\end{itemize}

This mechanism helps explain the observed dynamics by:
\begin{itemize}
\item Allowing higher peak biomass through stored nutrients
\item Creating a delay between nutrient uptake and growth
\item Providing a more realistic representation of nutrient limitation
\end{itemize}

The implementation uses the Droop model framework where:
\begin{itemize}
\item Nutrient uptake rate decreases as the internal quota approaches maximum
\item Growth rate increases with internal quota above a minimum threshold
\item The quota is bounded between physiologically realistic minimum and maximum values
\end{itemize}
