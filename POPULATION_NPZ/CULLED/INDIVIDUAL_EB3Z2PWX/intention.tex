\section{Ecological Mechanisms}

The model incorporates dynamic nutrient uptake efficiency by phytoplankton to better represent physiological adaptations to nutrient availability. Under nutrient-limited conditions, phytoplankton can upregulate their nutrient transporters and modify their cellular machinery to increase uptake efficiency. This adaptation is represented through a sigmoid response function that increases uptake efficiency as nutrient concentrations decrease below a critical threshold.

This mechanism captures an important feedback where phytoplankton become more efficient at acquiring nutrients when they are scarce, helping explain the maintenance of population levels even during periods of apparent nutrient limitation. The sigmoid response provides a smooth transition between efficiency states while maintaining biological realism in the bounds of the response.

The key parameters controlling this response are:
\begin{itemize}
\item $\eta_{max}$ - Maximum uptake efficiency multiplier
\item $k_{\eta}$ - Steepness of efficiency response
\item $N_{crit}$ - Critical nutrient concentration threshold
\item $\eta_{base}$ - Baseline uptake efficiency
\end{itemize}

This addition improves the model's ability to capture:
\begin{itemize}
\item Higher phytoplankton peaks through enhanced nutrient acquisition
\item More realistic nutrient-phytoplankton coupling
\item Physiological adaptation timescales
\end{itemize}
