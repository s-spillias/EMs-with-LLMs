This modification introduces nutrient-dependent phytoplankton growth efficiency to better capture bloom dynamics. The key ecological mechanism added is the ability of phytoplankton to enhance their nutrient use efficiency when nutrients become scarce.

In nature, phytoplankton show remarkable physiological plasticity in their nutrient acquisition and utilization strategies. When nutrients are limited, they can:
1. Increase the number or efficiency of nutrient transporters
2. Reduce their cellular nutrient quotas
3. Reallocate internal resources to maximize growth under limitation

The new efficiency term η (eta) increases as nutrient concentrations decrease, following:
η = 1 + η_max * (K_N / (N + K_N))

This formulation means that:
- Under high nutrients (N >> K_N), η ≈ 1 (baseline efficiency)
- Under severe limitation (N << K_N), η approaches 1 + η_max
- The response curve matches the timescale of physiological adaptation

This mechanism should improve the model's ability to:
- Sustain growth under nutrient limitation
- Better capture bloom magnitude and timing
- Represent the observed resilience of phytoplankton populations
