\section{Ecological Mechanism: Zooplankton Interference Competition}

We have modified the zooplankton grazing function to include density-dependent interference competition. The original Holling Type II functional response has been extended to include a term $\gamma Z$ in the denominator, where $\gamma$ is the interference coefficient and $Z$ is zooplankton density:

\[ \text{Grazing} = \frac{g_{max} \cdot T(E_a) \cdot P \cdot Z}{K_P + P + \gamma Z} \]

This modification represents the ecological process where zooplankton individuals interfere with each other's feeding at high densities. Such interference can occur through:
\begin{itemize}
\item Physical contact disrupting feeding behavior
\item Competition for optimal feeding spaces
\item Increased time spent in social interactions vs feeding
\end{itemize}

This mechanism is particularly important during bloom conditions when zooplankton reach high densities. The interference effect helps explain why zooplankton populations may not achieve theoretical maximum growth rates even when prey is abundant. This addition should improve the model's ability to capture the population dynamics during and after bloom events, particularly the recovery phase where the original model showed some discrepancies with observations.
