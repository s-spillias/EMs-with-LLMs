\section{Ecological Mechanism: Nutrient-Dependent Growth Efficiency}

We have implemented a variable nutrient use efficiency mechanism to better capture phytoplankton physiological responses to nutrient limitation. This represents an important adaptive strategy observed in natural populations where cells can enhance their nutrient uptake capabilities under limiting conditions.

The new efficiency term $\eta$ increases as nutrient concentrations decrease:

\[ \eta = 1 + \eta_{max} \frac{K_N}{N + K_N} \]

where $\eta_{max}$ is the maximum additional efficiency and $K_N$ is the same half-saturation constant used in the Monod function. This formulation means that:

\begin{itemize}
\item When nutrients are abundant ($N \gg K_N$), $\eta \approx 1$ (baseline efficiency)
\item When nutrients are scarce ($N \ll K_N$), $\eta \approx 1 + \eta_{max}$ (maximum efficiency)
\end{itemize}

This mechanism is supported by extensive empirical evidence showing that phytoplankton can upregulate nutrient transporters, modify their cellular stoichiometry, and adjust their metabolic pathways under nutrient stress. The enhancement of uptake efficiency helps explain the observed higher-than-expected phytoplankton growth rates under nutrient-limited conditions.

The modification affects the nutrient uptake term in the model:

\[ \text{uptake} = r_{max} \cdot T(E_p) \cdot L(I) \cdot \eta \cdot \frac{N}{K_N + N} \cdot P \]

where $T(E_p)$ represents temperature scaling and $L(I)$ represents light limitation.
