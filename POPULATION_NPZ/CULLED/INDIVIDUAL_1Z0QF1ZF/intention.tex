\section{Variable Nutrient Storage Model}

This model enhancement incorporates variable internal nutrient storage by phytoplankton, known as luxury uptake. In nature, phytoplankton can accumulate nutrients beyond their immediate growth requirements when nutrients are abundant, storing them for later use during periods of scarcity. This mechanism is particularly important in dynamic environments where nutrient availability fluctuates.

The model now tracks internal nutrient quotas (Q) bounded by physiological limits (Q_min to Q_max). Nutrient uptake rate increases when internal stores are depleted and decreases as they fill up, following Droop kinetics. Growth rate depends on the internal quota rather than external nutrient concentration, better reflecting the true biological constraints on phytoplankton growth.

This addition should improve the model's ability to capture:
1. Rapid initial bloom development using stored nutrients
2. Continued growth during brief periods of nutrient limitation
3. More realistic phase relationships between nutrients and biomass
4. Better representation of nutrient luxury consumption

The implementation uses established formulations from phytoplankton physiology literature, particularly building on work by Droop (1968) and Klausmeier et al. (2004).
