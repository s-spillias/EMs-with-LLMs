This model enhancement introduces nutrient-dependent phytoplankton growth efficiency to better capture luxury uptake and storage dynamics. When nutrients are abundant, phytoplankton can increase their uptake efficiency and store excess nutrients, leading to higher growth rates. This storage capacity acts as a buffer against future nutrient limitation.

The enhancement adds a dynamic efficiency term η(N) that modifies the basic Monod uptake function:
η(N) = η_base + (η_max - η_base)/(1 + exp(-k_η(N - N_crit)))

Where:
- η_base is the baseline uptake efficiency
- η_max is the maximum efficiency under optimal conditions  
- k_η controls how sharply efficiency changes with nutrient concentration
- N_crit is the nutrient concentration where efficiency response is centered

This sigmoidal response reflects the physiological constraints on nutrient uptake, with efficiency increasing as nutrients become more available until reaching a maximum. The storage effect helps explain the higher initial phytoplankton peak observed in the data, as cells can capitalize on early nutrient availability.
