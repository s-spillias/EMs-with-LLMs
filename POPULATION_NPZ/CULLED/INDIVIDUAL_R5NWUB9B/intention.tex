\section{Nutrient-Dependent Growth Efficiency}

We have enhanced the model by incorporating nutrient-dependent phytoplankton growth efficiency. This modification reflects the physiological plasticity of phytoplankton in their nutrient acquisition strategies. When nutrients are abundant, phytoplankton can increase their uptake efficiency through various mechanisms such as:

\begin{itemize}
\item Upregulation of nutrient transporters
\item Increased enzyme production for nutrient processing
\item Modified cellular quotas and storage capacity
\end{itemize}

The efficiency term $\eta$ varies between 1 and $(1 + \eta_{max})$, following a Monod-type response to nutrient availability. This captures how phytoplankton can optimize their resource acquisition under favorable conditions while maintaining baseline function under limitation.

This addition better represents the observed initial bloom dynamics, where rapid nutrient uptake and growth occur when resources are plentiful. The mechanism provides an ecologically realistic way for phytoplankton to achieve higher growth rates during optimal conditions without requiring unrealistic base growth rates.
