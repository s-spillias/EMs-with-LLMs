This model incorporates nutrient-dependent phytoplankton uptake efficiency to better represent physiological adaptations in nutrient-limited conditions. When nutrient concentrations are low relative to the half-saturation constant K_N, phytoplankton can enhance their uptake efficiency by up to a factor of (1 + eta_max). This represents real physiological mechanisms such as:

1. Upregulation of nutrient transporters
2. Increased surface area to volume ratio
3. Enhanced enzyme production for nutrient acquisition

The enhancement factor eta follows a Monod-type relationship inversely proportional to nutrient concentration, meaning efficiency is highest when nutrients are most limiting. This matches empirical observations of phytoplankton adaptation to nutrient stress.

This mechanism provides an important negative feedback that can help stabilize phytoplankton populations under nutrient limitation while also allowing for higher peak biomass when conditions improve. The parameter eta_max controls the strength of this adaptation response.
