\section{Model Description}

This model incorporates an important ecological feedback mechanism: nutrient recycling through zooplankton feeding inefficiency ("sloppy feeding"). When zooplankton graze on phytoplankton, a portion of the consumed biomass is not assimilated and is instead released back into the dissolved nutrient pool. This process is represented by the parameter $\epsilon_{feed}$ (feeding inefficiency rate).

This mechanism creates a positive feedback loop where grazing activity can stimulate phytoplankton growth by recycling nutrients, particularly important during bloom conditions. The recycling pathway is modeled as an immediate release of dissolved nutrients proportional to the grazing rate and feeding inefficiency:

$\text{Nutrient recycling} = \epsilon_{feed} \cdot \text{grazing}$

This addition better represents the tight coupling between zooplankton grazing and nutrient availability observed in marine ecosystems, potentially explaining the rapid succession of phytoplankton and zooplankton blooms seen in the data.
