\section{Model Enhancement: Temperature Dependence}

We enhanced the NPZ (Nutrient-Phytoplankton-Zooplankton) model by incorporating temperature dependence on biological rates using the Arrhenius equation. This modification reflects the fundamental influence of temperature on metabolic processes in marine ecosystems.

The temperature scaling factor is calculated as:
\[ f(T) = \exp(\frac{E_a}{k}(\frac{1}{T_{ref}} - \frac{1}{T})) \]

where:
\begin{itemize}
\item $E_a$ is the activation energy (eV)
\item $k$ is the Boltzmann constant (eV/K)
\item $T_{ref}$ is the reference temperature (K)
\item $T$ is the ambient temperature (K)
\end{itemize}

This temperature scaling affects both phytoplankton growth rates and zooplankton grazing rates, as these processes are fundamentally metabolic. The scaling captures the exponential increase in biological rates with temperature, a well-documented pattern in marine systems (Brown et al., 2004).

Currently, we use a fixed temperature for demonstration, but future versions could incorporate seasonal or spatial temperature variations to better capture environmental heterogeneity.

\textbf{References:}
Brown, J. H., et al. (2004). Toward a metabolic theory of ecology. Ecology, 85(7), 1771-1789.
