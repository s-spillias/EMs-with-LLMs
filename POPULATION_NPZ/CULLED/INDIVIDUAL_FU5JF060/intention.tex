\section{Ecological Intention}

The model has been enhanced to include nutrient-dependent phytoplankton growth efficiency. In natural systems, phytoplankton can adjust their nutrient uptake and growth efficiency based on resource availability through mechanisms like luxury uptake and internal storage. When nutrients are abundant, cells can increase their uptake efficiency and store excess nutrients for later use. Conversely, under nutrient limitation, they operate at a baseline efficiency.

This adaptive response is implemented through a sigmoid function that modulates growth efficiency based on ambient nutrient concentrations. The function includes:
\begin{itemize}
\item A baseline efficiency (eta\_base) representing minimal uptake capability
\item A maximum efficiency multiplier (eta\_max) for optimal conditions
\item A critical nutrient concentration (N\_crit) where efficiency changes most rapidly
\item A steepness parameter (k\_eta) controlling the sharpness of the transition
\end{itemize}

This formulation better captures the physiological plasticity of phytoplankton and their ability to optimize resource acquisition under varying conditions. The sigmoid form ensures a smooth transition between efficiency states while maintaining biological realism in the limits.
