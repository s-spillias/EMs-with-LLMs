\section{Model Description}

This model implements nutrient-dependent phytoplankton growth efficiency to better capture adaptive responses to resource availability. When nutrients are scarce, phytoplankton can upregulate their uptake mechanisms and increase their nutrient absorption efficiency. Conversely, when nutrients are abundant, they may reduce energy expenditure on uptake machinery.

This is represented through a sigmoid function that modifies the base uptake rate:

\[ \eta_N = \eta_{base} + \frac{\eta_{max} - \eta_{base}}{1 + e^{-k_\eta(N - N_{crit})}} \]

where:
- \eta_{base} is the baseline uptake efficiency
- \eta_{max} is the maximum efficiency under optimal conditions  
- k_\eta controls how sharply efficiency changes around N_{crit}
- N_{crit} is the nutrient concentration where efficiency response is centered

This mechanism allows phytoplankton to maintain growth under nutrient limitation while avoiding excessive energy expenditure when nutrients are plentiful. The sigmoid form creates a smooth transition between efficiency states rather than an abrupt switch.
