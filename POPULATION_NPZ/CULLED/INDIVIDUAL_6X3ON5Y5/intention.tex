\section{Ecological Intention}

The model has been enhanced to include nutrient-dependent phytoplankton growth efficiency. In natural systems, phytoplankton can adjust their nutrient uptake and growth efficiency based on resource availability through mechanisms like luxury uptake and internal storage. When nutrients are abundant, cells can increase their uptake efficiency and store excess nutrients for later use. Conversely, under nutrient limitation, they operate at baseline efficiency.

This adaptive response is implemented through a sigmoid function that modifies the uptake rate based on ambient nutrient concentrations. The function includes:
\begin{itemize}
\item A baseline efficiency (eta\_base) representing minimum uptake capability
\item A maximum efficiency multiplier (eta\_max) for optimal conditions  
\item A critical nutrient threshold (N\_crit) where efficiency changes most rapidly
\item A steepness parameter (k\_eta) controlling the sharpness of the transition
\end{itemize}

This better represents the physiological plasticity of phytoplankton and their ability to optimize resource acquisition based on environmental conditions. The sigmoid form creates a smooth transition between efficiency states while maintaining biological realism in the parameter bounds.
