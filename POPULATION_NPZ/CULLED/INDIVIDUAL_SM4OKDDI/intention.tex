\section{Model Enhancement: Temperature-Dependent Grazing Selectivity}

We enhance the ecological model by incorporating temperature-dependent grazing selectivity for zooplankton. This mechanism reflects how zooplankton feeding behavior and efficiency vary with temperature, which is well-documented in marine ecosystems (Kiørboe et al., 1982; Saiz & Calbet, 2011).

At higher temperatures, zooplankton typically:
\begin{itemize}
\item Exhibit increased swimming speeds and encounter rates with prey
\item Show enhanced prey detection and capture efficiency
\item Have higher metabolic demands requiring more selective feeding
\end{itemize}

This temperature dependence is implemented through a modified grazing term that includes a temperature-scaling factor for grazing efficiency. The scaling follows an Arrhenius-type relationship, consistent with metabolic theory, but with parameters specific to grazing behavior rather than general metabolism.

References:
\begin{itemize}
\item Kiørboe, T., Møhlenberg, F., & Hamburger, K. (1982). Bioenergetics of the planktonic copepod Acartia tonsa: relation between feeding, egg production and respiration, and composition of specific dynamic action. Marine Ecology Progress Series, 26, 85-97.
\item Saiz, E., & Calbet, A. (2011). Copepod feeding in the ocean: scaling patterns, composition of their diet and the bias of estimates due to microzooplankton grazing during incubations. Hydrobiologia, 666(1), 181-196.
\end{itemize}
