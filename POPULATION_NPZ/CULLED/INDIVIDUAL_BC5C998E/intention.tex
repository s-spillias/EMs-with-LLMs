This ecological model has been enhanced to include explicit detritus dynamics and temperature-dependent remineralization, which are crucial processes in marine nutrient cycling. The key changes are:

1. Addition of a detritus pool (D) that accumulates dead organic matter from:
   - Phytoplankton mortality
   - Zooplankton mortality
   - Unassimilated grazing

2. Temperature-dependent remineralization of detritus back to dissolved nutrients, replacing the previous instantaneous recycling fraction (gamma).

This modification better represents the delayed feedback between organic matter production and nutrient regeneration, which is particularly important for capturing:
- Seasonal nutrient dynamics
- Vertical transport of organic matter
- Bacterial decomposition processes
- Temperature effects on remineralization rates

The temperature dependence of remineralization uses the same Arrhenius relationship as other biological rates, reflecting the fundamental role of temperature in controlling bacterial activity and chemical reaction rates in aquatic systems.

This mechanistic representation should improve the model's ability to capture:
- Temporal lags in nutrient recycling
- Bottom-up effects on primary production
- Ecosystem responses to temperature change
