This model incorporates variable nutrient uptake efficiency in phytoplankton based on their internal nutrient status. The key ecological mechanism added is luxury nutrient uptake - the ability of phytoplankton to store excess nutrients when they are abundant and use these reserves during periods of scarcity.

This addition better represents how phytoplankton adapt to fluctuating nutrient conditions in natural systems. When nutrients are plentiful, cells can accumulate internal stores above their immediate growth requirements. These reserves then support continued growth when external nutrients become limiting.

The implementation uses a Droop-style quota model where uptake efficiency (eta_N) varies with both external nutrient concentration and the cell's current nutrient status (Q). The maximum storage capacity (Q_max) and minimum quota (Q_min) parameters define the physiological limits of this storage capability.

This mechanism creates an important feedback between nutrient availability and uptake dynamics. It allows phytoplankton to maintain higher growth rates through nutrient-poor periods, better matching the observed patterns in the empirical data.
