This model implements temperature dependence using the Arrhenius equation to capture seasonal effects on biological rates. The temperature scaling affects both phytoplankton growth and zooplankton grazing rates, as these processes are fundamentally controlled by biochemical reactions that depend on temperature.

The temperature variation follows a sinusoidal pattern with a period of one year, representing seasonal changes. The amplitude of 5°C around a mean of 20°C (293.15K) is typical for many marine systems. The Arrhenius equation scales biological rates based on activation energy (E_a) and a reference temperature (T_ref).

This mechanistic approach to temperature dependence should better capture:
1. Seasonal variations in population dynamics
2. Different temperature sensitivities of biological processes
3. The nonlinear nature of temperature effects on ecosystem processes

The temperature scaling particularly helps explain:
- Spring/summer phytoplankton blooms
- Seasonal changes in grazing pressure
- Annual cycles in nutrient turnover
