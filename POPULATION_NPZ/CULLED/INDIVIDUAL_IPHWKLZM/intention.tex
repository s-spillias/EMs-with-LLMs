\section{Adaptive Nutrient Use Efficiency}

We enhanced the model by incorporating variable nutrient use efficiency for phytoplankton, representing their ability to adapt to different nutrient conditions. The efficiency multiplier $\eta$ is calculated as:

\[ \eta = 1 + \eta_{max} \cdot \frac{K_N}{N + K_N} \cdot \frac{N}{N + K_\eta} \]

This formulation captures two key ecological processes:
\begin{itemize}
    \item The term $\frac{K_N}{N + K_N}$ increases efficiency when nutrients are scarce
    \item The term $\frac{N}{N + K_\eta}$ ensures the enhancement only occurs when some nutrients are available
\end{itemize}

This mechanism better represents phytoplankton's physiological adaptations to nutrient stress, allowing them to maintain higher growth rates during nutrient-limited conditions while preventing unrealistic growth at extremely low nutrient concentrations. The enhanced uptake term becomes:

\[ uptake = r_{max} \cdot T_{scale} \cdot L_{scale} \cdot \eta \cdot \frac{N \cdot P}{K_N + N} \]

This modification should improve the model's ability to capture bloom dynamics and the temporal patterns of nutrient-phytoplankton interactions.
