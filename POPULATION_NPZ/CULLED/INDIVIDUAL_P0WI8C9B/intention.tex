This model incorporates luxury uptake and storage of nutrients by phytoplankton, a key physiological mechanism in marine ecosystems. Phytoplankton can accumulate nutrients beyond their immediate growth requirements when resources are abundant, creating an internal buffer against future nutrient limitation.

The internal nutrient quota (Q) varies between a minimum subsistence quota (Q_min) and a maximum storage capacity (Q_max). Growth rate becomes a function of this internal quota rather than external nutrient concentration, while nutrient uptake remains dependent on external concentration but is also regulated by current storage levels.

This mechanism creates important feedbacks:
1. Decoupling of nutrient uptake from growth
2. Enhanced resilience to nutrient fluctuations
3. More realistic bloom dynamics through storage-mediated growth
4. Competitive advantages under variable nutrient conditions

The modified equations better represent the physiological processes controlling phytoplankton growth and nutrient dynamics in marine systems.
