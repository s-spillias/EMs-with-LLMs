\section{Ecological Justification for Internal Nutrient Quota}

The addition of internal nutrient quota mechanics represents an important biological reality in phytoplankton ecology - the ability to store nutrients in excess of immediate growth requirements. This "luxury uptake" allows cells to continue growing even when external nutrients become scarce, creating a buffer against environmental variability.

The implemented Droop-style quota model separates nutrient uptake from growth, where:
\begin{itemize}
    \item Uptake rate decreases as internal stores fill up (uptake\_regulation term)
    \item Growth rate depends on internal rather than external nutrient levels (growth\_limitation term)
    \item Minimum quota (Q\_min) represents essential cellular nutrients
    \item Maximum quota (Q\_max) represents storage capacity
\end{itemize}

This mechanism better captures the observed phase relationships between nutrients and phytoplankton biomass, as cells can maintain growth even after external nutrients are depleted. It also provides a more mechanistic basis for nutrient stress responses, as mortality and sinking rates now respond to internal nutrient status rather than external concentrations.
