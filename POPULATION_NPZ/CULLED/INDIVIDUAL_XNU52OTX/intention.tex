This model enhancement incorporates prey quality-dependent assimilation efficiency in zooplankton grazing. The modification recognizes that phytoplankton nutritional value varies with their growth conditions, not just ambient nutrient concentrations.

When phytoplankton grow rapidly under favorable conditions (high nutrients, optimal light, suitable temperature), they tend to have higher nutrient content per cell and better biochemical composition, making them more nutritious prey for zooplankton. Conversely, slowly growing phytoplankton often have lower nutritional value even if ambient nutrients are available.

The enhancement calculates a "prey quality" term based on the realized phytoplankton growth rate relative to their maximum potential rate. This quality factor then modulates the nutrient-dependent assimilation efficiency. This creates an indirect feedback where optimal phytoplankton growing conditions lead to both higher prey abundance AND higher prey quality, potentially explaining some of the observed peaks in zooplankton biomass.

This mechanism is ecologically justified by extensive literature on variable prey quality in marine systems and its effects on zooplankton nutrition. It adds biological realism without requiring additional parameters, instead leveraging existing growth rate calculations to infer prey quality.
