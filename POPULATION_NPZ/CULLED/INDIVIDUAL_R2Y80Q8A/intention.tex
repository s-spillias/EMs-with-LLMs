\section{Nutrient-Dependent Growth Efficiency}

We have enhanced the model by incorporating nutrient-dependent phytoplankton growth efficiency. This modification reflects the biological reality that phytoplankton can adjust their nutrient uptake mechanisms based on ambient nutrient concentrations.

The new efficiency term η (eta) varies between 1 and (1 + η_max), increasing with nutrient availability according to a Monod-type relationship. This represents:

1. Upregulation of nutrient transporters under favorable conditions
2. More efficient resource allocation when nutrients are abundant
3. Physiological adaptation to local nutrient conditions

This mechanism helps explain the observed higher phytoplankton growth rates during nutrient-replete conditions while maintaining realistic dynamics during nutrient-limited periods. The functional form ensures smooth transitions between efficiency states and maintains mathematical tractability.

The modification improves the model's ability to capture the initial phytoplankton bloom magnitude while preserving the overall system dynamics. This change is supported by empirical studies showing variable nutrient uptake efficiencies in marine phytoplankton communities.
