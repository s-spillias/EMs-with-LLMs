\section{Nutrient-Dependent Growth Efficiency}

We introduce nutrient-dependent growth efficiency to better capture phytoplankton adaptation to nutrient limitation. When nutrients become scarce, phytoplankton can enhance their growth efficiency through various physiological mechanisms including:

\begin{itemize}
\item Upregulation of nutrient transporters
\item Increased affinity for nutrients
\item More efficient internal nutrient recycling
\item Optimization of cellular resource allocation
\end{itemize}

This is implemented as a multiplier on the growth rate that increases under nutrient limitation:

\[ \phi = 1 + \phi_{max} \cdot \frac{K_N}{N + K_N} \]

where $\phi_{max}$ is the maximum efficiency enhancement factor and $\frac{K_N}{N + K_N}$ increases as nutrient concentration decreases.

This allows phytoplankton to partially compensate for nutrient limitation, potentially explaining the higher observed bloom magnitudes compared to the base model predictions.
