\section{Ecological Mechanisms}

The model has been enhanced to include luxury uptake of nutrients by phytoplankton, a well-documented ecological process where cells can store excess nutrients when they are abundant. This storage mechanism creates an important buffer against nutrient limitation and helps explain the timing and magnitude of phytoplankton blooms.

The implementation follows Droop's cell quota model, where:
\begin{itemize}
    \item Nutrient uptake rate is regulated by the difference between current and maximum internal quota
    \item Growth rate depends on internal rather than external nutrient concentration
    \item A minimum quota (Q\_min) is required for growth
    \item Maximum storage capacity is defined by Q\_max
\end{itemize}

This mechanism better represents the physiological reality that phytoplankton can maintain growth even when external nutrients become scarce, leading to more realistic bloom dynamics. The storage buffer also creates an important temporal decoupling between nutrient uptake and growth that is often observed in natural systems.
