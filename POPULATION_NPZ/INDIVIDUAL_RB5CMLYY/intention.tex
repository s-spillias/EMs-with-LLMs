\section{Ecological Intention}

A key modification was made to incorporate direct nutrient recycling from zooplankton grazing activity. In marine systems, zooplankton feeding is often inefficient, with a significant portion of consumed phytoplankton being released as dissolved nutrients rather than being assimilated into biomass or entering the detritus pool. This "sloppy feeding" process creates an important feedback loop where grazing can stimulate new primary production through rapid nutrient recycling.

The recycling efficiency is temperature-dependent, reflecting how metabolic rates and feeding mechanics vary with temperature. This creates an adaptive feedback where warmer conditions lead to both increased grazing pressure and faster nutrient recycling, better capturing the coupled nature of predator-prey interactions in planktonic systems.

The modification introduces a direct pathway from grazing to dissolved nutrients, complementing the slower recycling through the detritus pool. This better represents the multiple timescales of nutrient cycling in marine food webs and helps explain how high productivity can be maintained even under intense grazing pressure.
